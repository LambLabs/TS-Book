\begin{task}
\TT{Oblicz transformatę Fouriera sygnału $f(t)$ przedstawionego na rysunku.}{Compute the Fourier transform of a the $f(t)$ signal shown below.}

\begin{figure}[H]
\centering
\begin{tikzpicture}
	\draw[->] (-3.0,+0.0) -- (+5.0,+0.0) node[right] {$t$};
	\draw[->] (+0.0,-1.5) -- (+0.0,+1.5) node[above] {$f(t)$};
	%\draw[-,red, thick] (-3.5,+0.0) -- (-1.0,+0.0) -- (0.0,+1.0) -- (1.0,+0.0) -- (3.0,0.0);
	\draw[-,red, thick] (-3.0,+0.0) -- (-1.0,+0.0) -- (-1.0,+1.0) -- (0.0,+0.0) -- (1.0,1.0) -- (1.0,0.0) -- (3.0,0.0);
	\draw[-] (-1.0-0.1,-0.1)--(-1.0+0.1,0.1) node[midway, below, outer sep=5pt,align=center] {$-t_{0}$};
	\draw[-] (+1.0-0.1,-0.1)--(+1.0+0.1,0.1) node[midway, below, outer sep=5pt] {$t_{0}$};
	\draw[-] (-0.1,1.0-0.1)--(+0.1,1.0+0.1) node[midway, above left] {$A$};
\end{tikzpicture}
\end{figure}

\TT{Transformatę Fouriera obliczamy ze wzoru:}{The Fourier transform is defined as:}

\begin{equation}
F(\jmath \omega )=\int_{-\infty }^{\infty}f(t) \cdot e^{-\jmath \cdot \omega \cdot t}\cdot dt
\end{equation}

\TT{Do obliczenia całki potrzebujemy jawnej postaci równań opisujących proste na odcinkach $(-t_{0}, 0)$ oraz $(0, t_{0})$.}{In order to integrate the $f(t)$ signal, we need to describe it as a piecewise linear signal.}

\TT{Ogólne równanie prostej to:}{The simplest form of a linear function is:}

\begin{equation}
f(t) = m \cdot t + b
\end{equation} 

\TT{Dla pierwszego zakresu wartości $t$ wykres funkcji jest prostą przechodzącą przez dwa punkty: $(-t_{0},A)$ oraz $(0,0)$. Możemy więc napisać układ równań, rozwiązać go i wyznaczyć parametry prostej $m$ i $b$.}{In the first interval (e.g. $t \in (-t_0; 0)$), linear function crosses two points: $(-t_0,A)$ and $(0,0)$. So, in order to derive $m$ and $b$, the following system of the equations has to be solved.}

\begin{align*}
&\left\{\begin{matrix*}[l]
A = m\cdot (-t_{0}) +b\\ 
0 = m\cdot 0 +b
\end{matrix*}\right. \\
&\left\{\begin{matrix*}[l]
A = m\cdot (-t_{0}) +b\\ 
0 = b
\end{matrix*}\right. \\
&\left\{\begin{matrix*}[l]
A = m\cdot (-t_{0}) +0\\ 
0 = b
\end{matrix*}\right. \\
&\left\{\begin{matrix*}[l]
-\frac{A}{t_{0}} = m\\ 
0 = b
\end{matrix*}\right. \\
\end{align*}

\TT{Równianie prostej dla $t$ z zakresu  $(-t_{0},0)$ to:}{As a result we get:}

\begin{align*}
f(t) = -\frac{A}{t_{0}}\cdot t
\end{align*}

\TT{Dla drugiego zakresu wartości $t$ wykres funkcji jest prostą przechodzącą przez dwa punkty: $(0,0)$ oraz $(t_{0},A)$. Możemy więc napisać układ równań, rozwiązać go i wyznaczyć parametry prostej $m$ i $b$.}{In the second interval (e.g. $t \in (0;t_0)$), linear function crosses two points: $(0;0)$ and $(t_0,A)$. So, in order to derive $m$ and $b$, the following system of the equations has to be solved.}  

\begin{align*}
&\left\{\begin{matrix*}[l]
A = m\cdot t_{0} +b\\ 
0 = m\cdot 0 +b
\end{matrix*}\right. \\
&\left\{\begin{matrix*}[l]
A = m\cdot t_{0} +b\\ 
0 = b
\end{matrix*}\right. \\
&\left\{\begin{matrix*}[l]
A = m\cdot t_{0} +0\\ 
0 = b
\end{matrix*}\right. \\
&\left\{\begin{matrix*}[l]
\frac{A}{t_{0}} = m\\ 
0 = b
\end{matrix*}\right. \\
\end{align*}

\TT{Równianie prostej dla $t$ z zakresu  $(0,t_{0})$ to:}{As a result we get:}

\begin{align*}
f(t) = \frac{A}{t_{0}}\cdot t
\end{align*}

\TT{Podsumowując, sygnał $f(t)$ możemy opisać jako funkcję przedziałową:}{As a result the piecewise linear function is given by:}

\begin{equation}
f(t) = \left\{\begin{matrix*}[l]
0 & \TT{\text{ dla }}{\text{ for }} & t \in \left(-\infty; -t_{0}\right)\\
-\frac{A}{t_{0}} \cdot t  & \TT{\text{ dla }}{\text{ for }} & t \in \left(-t_{0}; 0\right)\\
\frac{A}{t_{0}} \cdot t & \TT{\text{ dla }}{\text{ for }} & t \in \left(0; t_{0}\right)\\
0 & \TT{\text{ dla }}{\text{ for }} & t \in \left(t_{0};\infty\right)
\end{matrix*}\right.
\end{equation}

\TT{Podstawiamy do wzoru na transformatę wzór naszej funkcji:}{For the given $f(t)$ signal we get:}

\begin{align*}
F(\jmath \omega )&=\int_{-\infty }^{\infty}f(t) \cdot e^{-\jmath \cdot \omega \cdot t}\cdot dt=\\
&=\int_{-\infty}^{-t_{0}} 0 \cdot e^{-\jmath \cdot \omega \cdot t}\cdot dt %2
+\int_{-t_{0}}^{0} \left(-\frac{A}{t_{0}} \cdot t\right) \cdot e^{-\jmath \cdot \omega \cdot t}\cdot dt=\\
&+\int_{0}^{t_{0}} \frac{A}{t_{0}} \cdot t \cdot e^{-\jmath \cdot \omega \cdot t}\cdot dt
+\int_{t_{0}}^{\infty} 0 \cdot e^{-\jmath \cdot \omega \cdot t}\cdot dt=\\
&=\int_{-\infty}^{-t_{0}} 0 \cdot dt -\int_{-t_{0}}^{0}\frac{A}{t_{0}} \cdot t \cdot e^{-\jmath \cdot \omega \cdot t}\cdot dt =\\%3
&+\int_{0}^{t_{0}} \frac{A}{t_{0}} \cdot t \cdot e^{-\jmath \cdot \omega \cdot t}\cdot dt 
+ \int_{t_{0}}^{\infty} 0 \cdot dt=\\
&= 0  - \frac{A}{t_{0}} \cdot \int_{-t_{0}}^{0} t \cdot e^{-\jmath \cdot \omega \cdot t}\cdot dt +\frac{A}{t_{0}} \cdot \int_{0}^{t_{0}} t \cdot e^{-\jmath \cdot \omega \cdot t}\cdot dt + 0=\\%4
&=\begin{Bmatrix*}[l] %5
u&=t & dv&=e^{ -\jmath \cdot \omega \cdot t} \cdot dt \\
du&=dt & v&=\frac{1}{-\jmath \cdot \omega}\cdot e^{ -\jmath \cdot \omega \cdot t}
\end{Bmatrix*}=\\
&=-\frac{A}{t_{0}}\cdot \left( \left. t \cdot \frac{1}{-\jmath \cdot \omega}\cdot e^{ -\jmath \cdot \omega \cdot t} \right|_{-t_{0}}^{0} %6
- \int_{-t_{0}}^{0} \frac{1}{-\jmath \cdot \omega}\cdot e^{ -\jmath \cdot \omega \cdot t} \cdot dt \right)=\\
&+\frac{A}{t_{0}}\cdot \left( \left. t \cdot \frac{1}{-\jmath \cdot \omega}\cdot e^{ -\jmath \cdot \omega \cdot t} \right|_{0}^{t_{0}}
- \int_{0}^{t_{0}} \frac{1}{-\jmath \cdot \omega}\cdot e^{ -\jmath \cdot \omega \cdot t} \cdot dt \right)=\\
&=-\frac{A}{t_{0}}\cdot \left( 0 \cdot e^{-\jmath \cdot \omega \cdot 0} - (-t_{0}) \cdot \frac{1}{-\jmath \cdot \omega}\cdot e^{ -\jmath \cdot \omega \cdot (-t_{0})} %7
+ \frac{1}{\jmath \cdot \omega} \left( \left. \frac{1}{-\jmath \cdot \omega}\cdot e^{ -\jmath \cdot \omega \cdot t} \right|_{-t_{0}}^{0}\right)\right)+\\
&+\frac{A}{t_{0}}\cdot \left(t_{0} \cdot \frac{1}{-\jmath \cdot \omega}\cdot e^{ -\jmath \cdot \omega \cdot t_{0}} - 0 \cdot e^{-\jmath \cdot \omega \cdot 0} 
+ \frac{1}{\jmath \cdot \omega} \left( \left. \frac{1}{-\jmath \cdot \omega}\cdot e^{ -\jmath \cdot \omega \cdot t} \right|_{0}^{t_{0}}\right)\right)=\\
&=-\frac{A}{t_{0}}\cdot \left( 0 - t_{0} \cdot \frac{1}{\jmath \cdot \omega}\cdot e^{ \jmath \cdot \omega \cdot t_{0}} %8
- \frac{1}{\jmath^{2} \cdot \omega^{2}} \left( e^{ -\jmath \cdot \omega \cdot 0} - e^{ -\jmath \cdot \omega \cdot (-t_{0})} \right)\right)+\\
&+\frac{A}{t_{0}}\cdot \left(t_{0} \cdot \frac{1}{-\jmath \cdot \omega}\cdot e^{ -\jmath \cdot \omega \cdot t_{0}} - 0 
- \frac{1}{\jmath^{2} \cdot \omega^{2}} \left(e^{ -\jmath \cdot \omega \cdot t_{0}}-e^{ -\jmath \cdot \omega \cdot 0}\right)\right)=\\
&=\frac{A}{t_{0}}\cdot \left(t_{0} \cdot \frac{1}{\jmath \cdot \omega}\cdot e^{ \jmath \cdot \omega \cdot t_{0}} %8
+ \frac{1}{\jmath^{2} \cdot \omega^{2}} \left( e^{0} - e^{ -\jmath \cdot \omega \cdot (-t_{0})} \right)\right)+\\
&+\frac{A}{t_{0}}\cdot \left(-t_{0} \cdot \frac{1}{\jmath \cdot \omega}\cdot e^{ -\jmath \cdot \omega \cdot t_{0}}
- \frac{1}{\jmath^{2} \cdot \omega^{2}} \left(e^{ -\jmath \cdot \omega \cdot t_{0}}-e^{0}\right)\right)=\\
&=\frac{A}{t_{0}}\cdot \left(t_{0} \cdot \frac{1}{\jmath \cdot \omega}\cdot e^{ \jmath \cdot \omega \cdot t_{0}} %8
+ \frac{1}{\jmath^{2} \cdot \omega^{2}} \left( 1 - e^{ -\jmath \cdot \omega \cdot (-t_{0})} \right)\right)+\\
&-\frac{A}{t_{0}}\cdot \left(t_{0} \cdot \frac{1}{\jmath \cdot \omega}\cdot e^{ -\jmath \cdot \omega \cdot t_{0}}
+ \frac{1}{\jmath^{2} \cdot \omega^{2}} \left(e^{ -\jmath \cdot \omega \cdot t_{0}}-1\right)\right)=\\
&= \frac{A}{t_{0}}\cdot t_{0} \cdot \frac{1}{\jmath \cdot \omega}\cdot e^{ \jmath \cdot \omega \cdot t_{0}} %8
+ \frac{A}{t_{0}}\cdot \frac{1}{\jmath^{2} \cdot \omega^{2}} \left( 1 - e^{ -\jmath \cdot \omega \cdot (-t_{0})} \right)+\\
& -\frac{A}{t_{0}}\cdot t_{0} \cdot \frac{1}{\jmath \cdot \omega}\cdot e^{ -\jmath \cdot \omega \cdot t_{0}}
 -\frac{A}{t_{0}}\cdot \frac{1}{\jmath^{2} \cdot \omega^{2}} \left(e^{ -\jmath \cdot \omega \cdot t_{0}}-1\right)=\\
&= \frac{A}{\jmath \cdot \omega}\cdot e^{ \jmath \cdot \omega \cdot t_{0}} %8
 -\frac{A}{\jmath \cdot \omega}\cdot e^{ -\jmath \cdot \omega \cdot t_{0}} +\\
&+ \frac{A}{t_{0}}\cdot \frac{1}{\jmath^{2} \cdot \omega^{2}} \left( 1 - e^{ -\jmath \cdot \omega \cdot (-t_{0})} \right)
-\frac{A}{t_{0}}\cdot \frac{1}{\jmath^{2} \cdot \omega^{2}} \left(e^{ -\jmath \cdot \omega \cdot t_{0}}-1\right)=\\
&= \frac{A}{\jmath \cdot \omega}\cdot \left( e^{ \jmath \cdot \omega \cdot t_{0}} - e^{ -\jmath \cdot \omega \cdot t_{0}} \right)+\\
&+ \frac{A}{t_{0}}\cdot \frac{1}{\jmath^{2} \cdot \omega^{2}}  - \frac{A}{t_{0}}\cdot \frac{1}{\jmath^{2} \cdot \omega^{2}} \cdot e^{ -\jmath \cdot \omega \cdot (-t_{0})} -\frac{A}{t_{0}}\cdot \frac{1}{\jmath^{2} \cdot \omega^{2}} \cdot e^{ -\jmath \cdot \omega \cdot t_{0}} + \frac{A}{t_{0}}\cdot \frac{1}{\jmath^{2} \cdot \omega^{2}}=\\
&= \frac{2\cdot A}{2\cdot \jmath \cdot \omega}\cdot \left( e^{ \jmath \cdot \omega \cdot t_{0}} - e^{ -\jmath \cdot \omega \cdot t_{0}} \right)+\\
&+ \frac{A}{t_{0}}\cdot \frac{1}{\jmath^{2} \cdot \omega^{2}}  + \frac{A}{t_{0}}\cdot \frac{1}{\jmath^{2} \cdot \omega^{2}}  - \frac{A}{t_{0}}\cdot \frac{1}{\jmath^{2} \cdot \omega^{2}} \cdot e^{\jmath \cdot \omega \cdot t_{0}} -\frac{A}{t_{0}}\cdot \frac{1}{\jmath^{2} \cdot \omega^{2}} \cdot e^{ -\jmath \cdot \omega \cdot t_{0}}=\\
&= \frac{2\cdot A}{\omega}\cdot \frac{e^{ \jmath \cdot \omega \cdot t_{0}} - e^{ -\jmath \cdot \omega \cdot t_{0}}}{2\cdot \jmath}+\\
&+ 2\cdot \frac{A}{t_{0}}\cdot \frac{1}{\jmath^{2} \cdot \omega^{2}}  - \frac{A}{t_{0}}\cdot \frac{1}{\jmath^{2} \cdot \omega^{2}} \cdot \left(e^{ \jmath \cdot \omega \cdot t_{0}} +e^{ -\jmath \cdot \omega \cdot t_{0}}\right)=\\
&= \frac{2\cdot A}{\omega}\cdot \frac{e^{ \jmath \cdot \omega \cdot t_{0}} - e^{ -\jmath \cdot \omega \cdot t_{0}}}{2\cdot \jmath}+\\
&+ \frac{A}{t_{0}}\cdot \frac{1}{\jmath^{2} \cdot \omega^{2}} \cdot \left(2  - \frac{2}{2}\cdot \left(e^{ \jmath \cdot \omega \cdot t_{0}} +e^{ -\jmath \cdot \omega \cdot t_{0}}\right)\right)=\\
&= \frac{2\cdot A}{\omega}\cdot \frac{e^{ \jmath \cdot \omega \cdot t_{0}} - e^{ -\jmath \cdot \omega \cdot t_{0}}}{2\cdot \jmath}+\\
&+ \frac{A}{t_{0}}\cdot \frac{1}{\jmath^{2} \cdot \omega^{2}} \cdot \left(2  - 2\cdot \frac{e^{ \jmath \cdot \omega \cdot t_{0}} +e^{ -\jmath \cdot \omega \cdot t_{0}}}{2}\right)=\\
&=\begin{Bmatrix*}[l]%15
\EulerSin\\
\EulerCos
\end{Bmatrix*}=\\
&= \frac{2\cdot A}{\omega}\cdot sin\left(\omega \cdot t_{0}\right) + \frac{A}{t_{0}}\cdot \frac{1}{\jmath^{2} \cdot \omega^{2}} \cdot \left(2  - 2\cdot cos \left( \omega \cdot t_{0}\right)\right)=\\
&= \frac{2\cdot A}{\omega}\cdot sin\left(\omega \cdot t_{0}\right) + \frac{A}{t_{0}}\cdot \frac{1}{\jmath^{2} \cdot \omega^{2}} \cdot 4 \cdot \left(\frac{1}{2}  - \frac{1}{2}\cdot cos \left( \omega \cdot t_{0}\right)\right)=\\
&=\begin{Bmatrix*}[l]%15
sin^2\left(x\right)=\frac{1}{2}-\frac{1}{2}\cdot cos\left(2\cdot x\right)
\end{Bmatrix*}=\\
&= \frac{2\cdot A}{\omega}\cdot sin\left(\omega \cdot t_{0}\right) + \frac{A}{t_{0}}\cdot \frac{4}{\jmath^{2} \cdot \omega^{2}} \cdot sin^2 \left( \frac{1}{2} \cdot \omega \cdot t_{0}\right)=\\
&= \frac{2\cdot A}{\omega}\cdot \frac{t_{0}}{t_{0}}sin\left(\omega \cdot t_{0}\right) + \frac{A}{t_{0}}\cdot \frac{4}{\jmath^{2} \cdot \omega^{2}} \cdot \frac{t_{0}}{t_{0}} \cdot sin^2 \left( \frac{1}{2} \cdot \omega \cdot t_{0}\right)=\\
&= 2\cdot A \cdot t_{0} \cdot \frac{sin\left(\omega \cdot t_{0}\right)}{t_{0} \cdot \omega} + A \cdot t_{0} \cdot \frac{4}{-1 \cdot \omega^{2} \cdot t_{0}^2} \cdot sin^2 \left( \frac{1}{2} \cdot \omega \cdot t_{0}\right)=\\
&= 2\cdot A \cdot t_{0} \cdot \frac{sin\left(\omega \cdot t_{0}\right)}{t_{0} \cdot \omega} + A \cdot t_{0} \cdot \frac{-1}{\frac{\omega^{2} \cdot t_{0}^2}{4}} \cdot sin^2 \left( \frac{1}{2} \cdot \omega \cdot t_{0}\right)=\\
&= 2\cdot A \cdot t_{0} \cdot \frac{sin\left(\omega \cdot t_{0}\right)}{t_{0} \cdot \omega} - A \cdot t_{0} \cdot \frac{sin^2 \left( \frac{1}{2} \cdot \omega \cdot t_{0}\right)}{\left(\frac{\omega \cdot t_{0}}{2}\right)^2} =\\
&= 2\cdot A \cdot t_{0} \cdot \frac{sin\left(\omega \cdot t_{0}\right)}{t_{0} \cdot \omega} - A \cdot t_{0} \cdot \left( \frac{sin \left( \frac{1}{2} \cdot \omega \cdot t_{0}\right)}{\frac{1}{2} \cdot \omega \cdot t_{0}}\right)^2 =\\
&=\begin{Bmatrix*}[l]%18
\SaDef
\end{Bmatrix*}=\\
&= 2\cdot A \cdot t_{0} \cdot Sa\left(\omega \cdot t_{0}\right) - A \cdot t_{0} \cdot Sa^2 \left( \frac{1}{2} \cdot \omega \cdot t_{0}\right)
%&= A \cdot t_{0} \cdot Sa^{2}(\frac{\omega \cdot t_{0}}{2})\\%17
\end{align*}

\TT{Transformata sygnału $f(t)$ to $F(\jmath\omega)=2\cdot A \cdot t_{0} \cdot Sa\left(\omega \cdot t_{0}\right) - A \cdot t_{0} \cdot Sa^2 \left( \frac{1}{2} \cdot \omega \cdot t_{0}\right)$.}{The Fourier transform of the $f(t)$ is equal to $F(\jmath\omega)=2\cdot A \cdot t_{0} \cdot Sa\left(\omega \cdot t_{0}\right) - A \cdot t_{0} \cdot Sa^2 \left( \frac{1}{2} \cdot \omega \cdot t_{0}\right)$.}

\end{task}
