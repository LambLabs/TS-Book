\begin{task}
Oblicz transformatę Fouriera sygnału $f(t)=Sa\left(\omega_0 \cdot t\right) \cdot sin\left(\omega_0 \cdot t\right)$ za pomocą twierdzeń, wiedząc że transformata sygnału $\Pi(t)$ jest rowna $Sa\left(\frac{\omega}{2}\right)$.

\begin{equation}
f(t) = Sa\left(\omega_0 \cdot t\right) \cdot sin\left(\omega_0 \cdot t\right)
\end{equation}

\begin{equation}
\Pi(t) \overset{F}{\rightarrow} Sa\left(\frac{\omega}{2}\right)
\end{equation}

\begin{figure}[H]
\centering
\begin{tikzpicture}
  %\draw (0,0) circle (1in);
  \draw[->] (-4.0,+0.0) -- (+4.0,+0.0) node[right] {$t$};
  \draw[->] (+0.0,-2.5) -- (+0.0,+2.5) node[above] {$f(t)$};
  \draw[scale=1.0,domain=-3.5:3.5,samples=200,smooth,variable=\x,red,thick] plot ({\x},{2*sinc(\x*3.141592*2)*sin(\x*3.141592*2 r)});%
  
  \draw[-] (-1.0-0.1,-0.1)--(-1.0+0.1,0.1) node[midway, above, outer sep=5pt,align=center] {$-\frac{2\pi}{\omega_0}$};
  \draw[-] (-2.0-0.1,-0.1)--(-2.0+0.1,0.1) node[midway, above, outer sep=5pt,align=center] {$-\frac{4\pi}{\omega_0}$};
  \draw[-] (-3.0-0.1,-0.1)--(-3.0+0.1,0.1) node[midway, above, outer sep=5pt,align=center] {$-\frac{6\pi}{\omega_0}$};
  \draw[-] (+1.0-0.1,-0.1)--(+1.0+0.1,0.1) node[midway, below, outer sep=5pt] {$\frac{2\pi}{\omega_0}$};
  \draw[-] (+2.0-0.1,-0.1)--(+2.0+0.1,0.1) node[midway, below, outer sep=5pt] {$\frac{4\pi}{\omega_0}$};
  \draw[-] (+3.0-0.1,-0.1)--(+3.0+0.1,0.1) node[midway, below, outer sep=5pt] {$\frac{6\pi}{\omega_0}$};
  \draw[-] (-0.1,+2.0-0.1)--(+0.1,+2.0+0.1) node[midway, left] {$A$};
  \draw[-] (-0.1,-2.0-0.1)--(+0.1,-2.0+0.1) node[midway, left] {$-A$};
\end{tikzpicture}
\end{figure}

W pierwszej kolejności można funkcję $f(t)$ rozpisać następująco

\begin{align*}
f(t) &= Sa\left(\omega_0 \cdot t\right) \cdot sin\left(\omega_0 \cdot t\right)=\\
&=\begin{Bmatrix}
\EulerSin
\end{Bmatrix}=\\
&= Sa\left(\omega_0 \cdot t\right) \cdot \frac{e^{\jmath \cdot \omega_0 \cdot t} - e^{-\jmath \cdot \omega_0 \cdot t}}{2 \cdot \jmath}=\\
&= \frac{1}{2 \cdot \jmath} \cdot \left( Sa\left(\omega_0 \cdot t\right) \cdot e^{\jmath \cdot \omega_0 \cdot t} - Sa\left(\omega_0 \cdot t\right) \cdot e^{-\jmath \cdot \omega_0 \cdot t} \right)=\\
&=\begin{Bmatrix}
f_1(t) &= Sa\left(\omega_0 \cdot t\right) \cdot e^{\jmath \cdot \omega_0 \cdot t} \\
f_2(t) &= Sa\left(\omega_0 \cdot t\right) \cdot e^{-\jmath \cdot \omega_0 \cdot t}
\end{Bmatrix}=\\
&=\frac{1}{2 \cdot \jmath} \cdot \left(f_1(t) - f_2(t) \right)
\end{align*}

Należy zauważyć iż funkcja $f_1(t)$ i $f_2(t)$ jest złożeniem funkcji $Sa$ i funkcji wykładniczych.

\begin{align*}
f_1(t) &= Sa\left(\omega_0 \cdot t\right) \cdot e^{\jmath \cdot \omega_0 \cdot t} = g(t)\cdot e^{\jmath \cdot \omega_0 \cdot t}\\
f_2(t) &= Sa\left(\omega_0 \cdot t\right) \cdot e^{-\jmath \cdot \omega_0 \cdot t} = g(t) \cdot e^{-\jmath \cdot \omega_0 \cdot t}
\end{align*}

Znając transformatę sygnału $g(t) = Sa\left(\omega_0 \cdot t\right)$ możemy skorzystać z twierdzenia o przesunięciu w dziedzinie częstotliwości. 

\begin{align*}
\FrequencyShiftTeorem{g}{G}{f}{F}
\end{align*}

Aby wyznaczyć transformatę sygnału $g(t)$ możemy skorzystać z twierdzenia o symetrii. Znając transformatę $H(\jmath \omega)$ sygnału $h(t)$ można wyznaczyć transformatę $G(\jmath \omega)$ sygnału $g(t)$

\begin{align*}
\SymetryTeorem{h}{H}{g}{G}
\end{align*}

Tak wiec zacznijmy od transformaty sygnału prostokątnego $h(t)=\Pi(t)$ i wyznaczymy transformatę funkcji $Sa$

\begin{align*}
h(t)=\Pi(t) &\overset{F}{\rightarrow} H(\jmath \omega) = Sa\left(\frac{\omega}{2}\right)\\
g_1(t) = H(t) = Sa\left(\frac{t}{2}\right) &\overset{F}{\rightarrow} 
G_1(\jmath \omega) = 2\pi \cdot h(- \jmath \omega) = \pi \cdot  \Pi\left(-\omega\right) = 2\pi \cdot  \Pi\left(\omega\right)
\end{align*}

Wyznaczyliśmy transformatę funkcji $g_1(t)$. Jednak funkcja $g_1(t)$ nie ma takiej samej postaci jak funkcja $g(t)$ 

\begin{align*}
g(t)&=Sa\left(\omega_0 \cdot t\right)=\\
&=Sa\left(\omega_0 \cdot t \cdot \frac{2}{2}\right)=\\
&=Sa\left(2\cdot \omega_0 \cdot \frac{t}{2}\right)=\\
&=Sa\left(\frac{2\cdot \omega_0 \cdot t}{2}\right)=\\
&=\begin{Bmatrix}
a = 2\cdot \omega_0
\end{Bmatrix}=\\
&=Sa\left(\frac{a\cdot t}{2}\right)=\\
&=g_1(a\cdot t) 
\end{align*}

Znając transformatę funkcji $g_1(t)$ możemy wyznaczyć transformatę funkcji $g(t)=g_1(a \cdot t)$ za pomocą twierdzenia o zmianie skali.

\begin{align*}
\TimeScalingTeorem{g_1}{G_1}{g}{G}
\end{align*}

Podstawiając wyznaczoną transformatę $G_1(\jmath \omega)$

\begin{align*}
G(\jmath \omega) &= \frac{1}{\left|\alpha \right|} \cdot G_1(\jmath \frac{\omega}{\alpha})=\\
&=\begin{Bmatrix}
\alpha = 2\cdot \omega_0
\end{Bmatrix}=\\
&=\frac{1}{\left| 2\cdot \omega_0 \right|} \cdot G_1( \frac{\omega}{2\cdot \omega_0})=\\
&=\begin{Bmatrix}
G_1(\jmath \omega) = 2\pi \cdot \Pi\left(\omega\right)
\end{Bmatrix}=\\
&=\frac{1}{ 2\cdot \omega_0 } \cdot 2\pi \cdot \Pi\left( \frac{\omega}{2\cdot \omega_0}\right)=\\
&=\frac{\pi}{ \omega_0 } \cdot \Pi\left( \frac{\omega}{2\cdot \omega_0}\right)
\end{align*}

Tak wiec transformata sygnału $g(t)=Sa\left(\omega_0 \cdot t\right)$ jest równa $G(\jmath \omega)=\frac{\pi}{ \omega_0 } \cdot \Pi\left( \frac{\omega}{2\cdot \omega_0}\right)$

Kolejnym krokiem jest wyznaczenie transformaty dwóch sygnałów

\begin{align*}
f_1(t)&=Sa\left(\omega_0 \cdot t\right) \cdot e^{\jmath \cdot \omega_0 \cdot t}\\
f_2(t)&=Sa\left(\omega_0 \cdot t\right) \cdot e^{-\jmath \cdot \omega_0 \cdot t}\\
\end{align*}

Korzystając z twierdzenie o przesunięciu w dziedzinie częstotliwości

\begin{align*}
\FrequencyShiftTeorem{g}{G}{f_1}{F_1}
\end{align*}

otrzymujemy wprost

\begin{align*}
F_1(\jmath \omega)&=G\left(\jmath \left(\omega -\omega_0\right)\right)=\\
&=\frac{\pi}{ \omega_0 } \cdot \Pi\left( \frac{\omega - \omega_0}{2\cdot \omega_0}\right)
\end{align*}

\begin{align*}
F_2(\jmath \omega)&=G\left(\jmath \left(\omega +\omega_0\right)\right)=\\
&=\frac{\pi}{ \omega_0 } \cdot \Pi\left( \frac{\omega + \omega_0}{2\cdot \omega_0}\right)
\end{align*}

Ostatecznie korzystając z liniowości transformaty Fouriera

\begin{align*}
\HomogeneousTeorem{f}{F}
\end{align*}

otrzymujemy

\begin{align*}
F(\jmath \omega)&=F_1(\jmath \omega)-F_2(\jmath \omega)=\\
&=\frac{1}{2 \cdot \jmath} \cdot \left( \frac{\pi}{ \omega_0 } \cdot \Pi\left( \frac{\omega - \omega_0}{2\cdot \omega_0}\right) - \frac{\pi}{ \omega_0 } \cdot \Pi\left( \frac{\omega + \omega_0}{2\cdot \omega_0}\right) \right)
\end{align*}


Transformata Fouriera sygnału $f(t)$ jest równa $F(\jmath \omega)=\frac{1}{2 \cdot \jmath} \cdot \left( \frac{\pi}{ \omega_0 } \cdot \Pi\left( \frac{\omega - \omega_0}{2\cdot \omega_0}\right) - \frac{\pi}{ \omega_0 } \cdot \Pi\left( \frac{\omega + \omega_0}{2\cdot \omega_0}\right) \right)$

\end{task}

