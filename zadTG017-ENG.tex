\begin{task}
\TT{Rozwiń poniższe sygnały }{Expand the following signals into a sum of sine and cosine functions, and a constant, by using the Euler identities.}

\begin{align*}
f_{1}(t)=sin^5(t)-sin^3(t) \\
f_{2}(t)=cos^6(t)-cos^4(t)
\end{align*}

Euler identities:
\begin{align*}
sin\left(x\right) = \frac{e^{\jmath \cdot x} - e^{-\jmath \cdot x}}{2 \cdot \jmath}\\
cos\left(x\right) = \frac{e^{\jmath \cdot x} + e^{-\jmath \cdot x}}{2}
\end{align*}

\TT{Dwumian Newtona}{Binomial theorem}:
\begin{equation}
(x+y)^{n} = \sum_{k=0}^{n} \binom{n}{k} x^{n-k} \cdot y^{k}
\end{equation}

\TT{gdzie współczynniki $\binom{n}{k}$ określone są jako:}{where $\binom{n}{k}$ are called binomial coefficients:}
\begin{equation}
\binom{n}{k} = \frac{n!}{k!(n-k)!}
\end{equation}

The binomial coefficient $\binom{n}{k}$ appears as the $kth$ entry in the $nth$ row of Pascal's triangle (counting starts at 0). Each entry is the sum of the two above it. Below, example for $n=6$ is presented:

\begin{equation}
\begin{Bsmallmatrix}
    n=0:&   &   &   &   &   &   & 1 &   &   &   &   &  & \\
    n=1:&   &   &   &   &   & 1 &   & 1 &   &   &   &  & \\
    n=2:&   &   &   &   & 1 &   & 2 &   & 1 &   &   &  & \\
    n=3:&   &   &   & 1 &   & 3 &   & 3 &   & 1 &   &  & \\
    n=4:&   &   & 1 &   & 4 &   & 6 &   & 4 &   & 1 &  & \\
    n=5:&   & 1 &   & 5 &   &10 &   & 10&   & 5 &   & 1& \\
    n=6:& 1 &   & 6 &   & 15&   & 20&   & 15&   & 6 &  & 1
\end{Bsmallmatrix}
\end{equation}

\begin{align*}
f_{1}(t)&=sin^5(t)-sin^3(t)= \\
&=\left( \frac{e^{\jmath \cdot t} - e^{-\jmath \cdot t}}{2 \cdot \jmath} \right)^5 - \left( \frac{e^{\jmath \cdot t} - e^{-\jmath \cdot t}}{2 \cdot \jmath} \right)^3= \\
&=\frac{\left(e^{\jmath \cdot t} - e^{-\jmath \cdot t}\right)^5}{\left(2 \cdot \jmath\right)^5}-\frac{\left(e^{\jmath \cdot t} - e^{-\jmath \cdot t}\right)^3}{\left(2 \cdot \jmath\right)^3}=\\
&=\frac{1 \cdot (e^{\jmath \cdot t})^5 \cdot (-e^{-\jmath \cdot t})^0 + 5 \cdot (e^{\jmath \cdot t})^4 \cdot (-e^{-\jmath \cdot t})^1 + 10 \cdot (e^{\jmath \cdot t})^3 \cdot (-e^{-\jmath \cdot t})^2}{\left(2 \cdot \jmath\right)^5} + \\
&+\frac{10 \cdot (e^{\jmath \cdot t})^2 \cdot (-e^{-\jmath \cdot t})^3 + 5 \cdot (e^{\jmath \cdot t})^1 \cdot (-e^{-\jmath \cdot t})^4 +1 \cdot (e^{\jmath \cdot t})^0 \cdot (-e^{-\jmath \cdot t})^5}{\left(2 \cdot \jmath\right)^5} - \\
&-\left(\frac{1 \cdot (e^{\jmath \cdot t})^3 \cdot (-e^{-\jmath \cdot t})^0 + 3 \cdot (e^{\jmath \cdot t})^2 \cdot (-e^{-\jmath \cdot t})^1 + 3 \cdot (e^{\jmath \cdot t})^1 \cdot (-e^{-\jmath \cdot t})^2+1 \cdot (e^{\jmath \cdot t})^0 \cdot (-e^{-\jmath \cdot t})^3}{\left(2 \cdot \jmath\right)^3}\right)=\\
&=\frac{1 \cdot e^{\jmath \cdot t \cdot 5} \cdot 1 + 5 \cdot e^{\jmath \cdot t \cdot 4} \cdot (-e^{-\jmath \cdot t \cdot 1}) + 10 \cdot e^{\jmath \cdot t \cdot 3} \cdot e^{-\jmath \cdot t \cdot 2}}{\left(2 \cdot \jmath\right)^5} + \\
&+\frac{10 \cdot e^{\jmath \cdot t \cdot 2} \cdot (-e^{-\jmath \cdot t \cdot 3}) + 5 \cdot e^{\jmath \cdot t \cdot 1} \cdot e^{-\jmath \cdot t \cdot 4} + 1 \cdot 1 \cdot (-e^{-\jmath \cdot t \cdot 5})}{\left(2 \cdot \jmath\right)^5} - \\
&-\left(\frac{1 \cdot e^{\jmath \cdot t \cdot 3} \cdot 1 + 3 \cdot e^{\jmath \cdot t \cdot 2} \cdot (-e^{-\jmath \cdot t \cdot 1}) + 3 \cdot e^{\jmath \cdot t \cdot 1} \cdot e^{-\jmath \cdot t \cdot 2} + 1 \cdot 1 \cdot (-e^{-\jmath \cdot t \cdot 3})}{\left(2 \cdot \jmath\right)^3}\right)=\\
&=\frac{e^{\jmath \cdot t \cdot 5} - 5 \cdot e^{\jmath \cdot t \cdot 3}  + 10 \cdot e^{\jmath \cdot t} - 10 \cdot e^{-\jmath \cdot t} + 5 \cdot e^{-\jmath \cdot t \cdot 3} - e^{-\jmath \cdot t \cdot 5}}{\left(2 \cdot \jmath\right)^5} - \\
&-\left(\frac{e^{\jmath \cdot t \cdot 3} - 3 \cdot e^{\jmath \cdot t} + 3 \cdot e^{-\jmath \cdot t} - e^{-\jmath \cdot t \cdot 3}}{\left(2 \cdot \jmath\right)^3}\right)=\\
&=\frac{e^{\jmath \cdot t \cdot 5} - e^{-\jmath \cdot t \cdot 5} - 5 \cdot e^{\jmath \cdot t \cdot 3} + 5 \cdot e^{-\jmath \cdot t \cdot 3} + 10 \cdot e^{\jmath \cdot t} - 10 \cdot e^{-\jmath \cdot t}}{\left(2 \cdot \jmath\right)^5} - \\
&-\left(\frac{e^{\jmath \cdot t \cdot 3} - e^{-\jmath \cdot t \cdot 3} - 3 \cdot e^{\jmath \cdot t} + 3 \cdot e^{-\jmath \cdot t}}{\left(2 \cdot \jmath\right)^3}\right)=\\
&=\frac{e^{\jmath \cdot t \cdot 5} - e^{-\jmath \cdot t \cdot 5} - 5 \cdot (e^{\jmath \cdot t \cdot 3} - e^{-\jmath \cdot t \cdot 3}) + 10 \cdot (e^{\jmath \cdot t} - e^{-\jmath \cdot t})}{\left(2 \cdot \jmath\right)^4 \cdot (2 \cdot \jmath)} - \\
&-\left(\frac{e^{\jmath \cdot t \cdot 3} - e^{-\jmath \cdot t \cdot 3} - 3 \cdot (e^{\jmath \cdot t} - \cdot e^{-\jmath \cdot t})}{\left(2 \cdot \jmath\right)^2 \cdot (2 \cdot \jmath)}\right)=\\
&=\frac{sin(5 \cdot t) - 5 \cdot sin(3 \cdot t) + 10 \cdot sin(t)}{\left(2 \cdot \jmath\right)^4} - \\
&-\left(\frac{sin(3 \cdot t) - 3 \cdot sin(t)}{\left(2 \cdot \jmath\right)^2}\right)=\\
&=\frac{sin(5 \cdot t) - 5 \cdot sin(3 \cdot t) + 10 \cdot sin(t)}{16} +\left(\frac{sin(3 \cdot t) - 3 \cdot sin(t)}{4}\right)=\\
&=\frac{sin(5 \cdot t) - 5 \cdot sin(3 \cdot t) + 10 \cdot sin(t)}{16} +\frac{4 \cdot sin(3 \cdot t) - 12 \cdot sin(t)}{16}=\\
&=\frac{sin(5 \cdot t) - sin(3 \cdot t) -2 \cdot sin(t)}{16}\\
\end{align*}

To sum up:
\begin{align*}
f_{1}(t)=sin^5(t)-sin^3(t) = \frac{sin(5 \cdot t) - sin(3 \cdot t) -2 \cdot sin(t)}{16}
\end{align*}


\begin{align*}
f_{2}(t)&=cos^6(t)-cos^4(t)= \\
&=\left( \frac{e^{\jmath \cdot t} + e^{-\jmath \cdot t}}{2} \right)^6 - \left( \frac{e^{\jmath \cdot t} + e^{-\jmath \cdot t}}{2} \right)^4= \\
&=\frac{\left(e^{\jmath \cdot t} + e^{-\jmath \cdot t}\right)^6}{2^6}-\frac{\left(e^{\jmath \cdot t} + e^{-\jmath \cdot t}\right)^4}{2^4}=\\
&=\frac{1 \cdot (e^{\jmath \cdot t})^6 \cdot (e^{-\jmath \cdot t})^0 + 6 \cdot (e^{\jmath \cdot t})^5 \cdot (e^{-\jmath \cdot t})^1 + 15 \cdot (e^{\jmath \cdot t})^4 \cdot (e^{-\jmath \cdot t})^2 + 20 \cdot (e^{\jmath \cdot t})^3 \cdot (e^{-\jmath \cdot t})^3}{2^6} + \\
&+\frac{15 \cdot (e^{\jmath \cdot t})^2 \cdot (e^{-\jmath \cdot t})^4 + 6 \cdot (e^{\jmath \cdot t})^1 \cdot (e^{-\jmath \cdot t})^5 +1 \cdot (e^{\jmath \cdot t})^0 \cdot (e^{-\jmath \cdot t})^6}{2^6} - \\
&-\left(\frac{1 \cdot (e^{\jmath \cdot t})^4 \cdot (e^{-\jmath \cdot t})^0 + 4 \cdot (e^{\jmath \cdot t})^3 \cdot (e^{-\jmath \cdot t})^1 + 6 \cdot (e^{\jmath \cdot t})^2 \cdot (e^{-\jmath \cdot t})^2 + 4 \cdot (e^{\jmath \cdot t})^1 \cdot (e^{-\jmath \cdot t})^3 + 1 \cdot (e^{\jmath \cdot t})^0 \cdot (e^{-\jmath \cdot t})^4}{2^4}\right)=\\
&=\frac{1 \cdot e^{\jmath \cdot t \cdot 6} \cdot 1 + 6 \cdot e^{\jmath \cdot t \cdot 5} \cdot e^{-\jmath \cdot t \cdot 1} + 15 \cdot e^{\jmath \cdot t \cdot 4} \cdot e^{-\jmath \cdot t \cdot 2} + 20 \cdot e^{\jmath \cdot t \cdot 3} \cdot e^{-\jmath \cdot t \cdot 3}}{2^6} + \\
&+\frac{15 \cdot e^{\jmath \cdot t \cdot 2} \cdot e^{-\jmath \cdot t \cdot 4} + 6 \cdot e^{\jmath \cdot t \cdot 1} \cdot e^{-\jmath \cdot t \cdot 5} + 1 \cdot 1 \cdot e^{-\jmath \cdot t \cdot 6}}{2^6} - \\
&-\left(\frac{1 \cdot e^{\jmath \cdot t \cdot 4} \cdot 1 + 4 \cdot e^{\jmath \cdot t \cdot 3} \cdot e^{-\jmath \cdot t \cdot 1} + 6 \cdot e^{\jmath \cdot t \cdot 2} \cdot e^{-\jmath \cdot t \cdot 2} + 4 \cdot e^{\jmath \cdot t \cdot 1} \cdot e^{-\jmath \cdot t \cdot 3} + 1 \cdot 1 \cdot e^{-\jmath \cdot t \cdot 4}}{2^4}\right)=\\
&=\frac{e^{\jmath \cdot t \cdot 6} + 6 \cdot e^{\jmath \cdot t \cdot 4}  + 15 \cdot e^{\jmath \cdot t \cdot 2} + 20 \cdot e^{\jmath \cdot t \cdot 0} + 15 \cdot e^{-\jmath \cdot t \cdot 2} + 6 \cdot e^{-\jmath \cdot t \cdot 4} + e^{-\jmath \cdot t \cdot 6}}{2^6} - \\
&-\left(\frac{e^{\jmath \cdot t \cdot 4} + 4 \cdot e^{\jmath \cdot t \cdot 2} + 6 \cdot e^{\jmath \cdot t \cdot 0} + 4 \cdot e^{-\jmath \cdot t \cdot 2} + e^{-\jmath \cdot t \cdot 4}}{2^4}\right)=\\
&=\frac{e^{\jmath \cdot t \cdot 6} + e^{-\jmath \cdot t \cdot 6} + 6 \cdot e^{\jmath \cdot t \cdot 4} + 6 \cdot e^{-\jmath \cdot t \cdot 4}  + 15 \cdot e^{\jmath \cdot t \cdot 2} + 15 \cdot e^{-\jmath \cdot t \cdot 2} + 20}{2^6} - \\
&-\left(\frac{e^{\jmath \cdot t \cdot 4} + e^{-\jmath \cdot t \cdot 4} + 4 \cdot e^{\jmath \cdot t \cdot 2} + 4 \cdot e^{-\jmath \cdot t \cdot 2} + 6}{2^4}\right)=\\
&=\frac{e^{\jmath \cdot t \cdot 6} + e^{-\jmath \cdot t \cdot 6} + 6 \cdot (e^{\jmath \cdot t \cdot 4} + e^{-\jmath \cdot t \cdot 4})  + 15 \cdot (e^{\jmath \cdot t \cdot 2} + e^{-\jmath \cdot t \cdot 2}) + 20}{2^5 \cdot 2} - \\
&-\left(\frac{e^{\jmath \cdot t \cdot 4} + e^{-\jmath \cdot t \cdot 4} + 4 \cdot (e^{\jmath \cdot t \cdot 2} + e^{-\jmath \cdot t \cdot 2}) + 6}{2^3 \cdot 2}\right)=\\
&=\frac{cos(6 \cdot t) + 6 \cdot cos(4 \cdot t) + 15 \cdot cos(2 \cdot t) + 10}{2^5} - \\
&-\left(\frac{cos(4 \cdot t) + 4 \cdot cos(2 \cdot t) + 3}{2^3}\right)=\\
&=\frac{cos(6 \cdot t) + 6 \cdot cos(4 \cdot t) + 15 \cdot cos(2 \cdot t) + 10 - 4 \cdot cos(4 \cdot t) - 16 \cdot cos(2 \cdot t) - 12}{2^5}=\\
&=\frac{cos(6 \cdot t) + 2 \cdot cos(4 \cdot t) - cos(2 \cdot t) - 2}{2^5}=\\
&=\frac{cos(6 \cdot t) + 2 \cdot cos(4 \cdot t) - cos(2 \cdot t) - 2}{32}
\end{align*}

To sum up:
\begin{align*}
f_{2}(t) =cos^6(t)-cos^4(t)=\frac{cos(6 \cdot t) + 2 \cdot cos(4 \cdot t) - cos(2 \cdot t) - 2}{32}
\end{align*}
\end{task}