\begin{task}
\TT{Wyznacz udział mocy wyższych harmonicznych ($k>1$) w całkowitej mocy okresowego sygnału $f(t)$ przedstawionego na rysunku:}{Calculate the percentage contribution of the power of higher harmonics ($k > 1$) to the total average power of the periodic signal shown below.} 

\begin{figure}[H]
    \centering
    \begin{tikzpicture}
    %\draw (0,0) circle (1in);
    \draw[->] (-3.0,+0.0) -- (+5.0,+0.0) node[right] {$t$};
    \draw[->] (+0.0,-1.5) -- (+0.0,+1.5) node[above] {$f(t)$};
    \draw[-,red, thick] (-2.5,+0.0) -- (-2.0,+0.0) -- (-2.0,+1.0) -- (-1.0,+1.0) -- (-1.0,+0.0)--(+0.0,+0.0) -- (+0.0,+1.0) -- (+1.0,+1.0) -- (+1.0,+0.0) -- (+2.0,+0.0) -- (+2.0,+1.0) -- (3.0,1.0) -- (3.0,0.0) -- (3.5,0.0);
    %\draw[-] (-1.0-0.1,-0.1)--(-1.0+0.1,0.1) node[midway, below, outer sep=10pt,align=center] {$-\frac{T}{2}$};
    \draw[-] (-1.0-0.1,-0.1)--(-1.0+0.1,0.1) node[midway, below, outer sep=5pt,align=center] {$-\frac{T}{2}$};
    \draw[-] (+1.0-0.1,-0.1)--(+1.0+0.1,0.1) node[midway, below, outer sep=5pt] {$\frac{T}{2}$};
    \draw[-] (+2.0-0.1,-0.1)--(+2.0+0.1,0.1) node[midway, below, outer sep=5pt] {$T$};
    \draw[-] (-0.1,1.0-0.1)--(+0.1,1.0+0.1) node[midway, left] {$A$};
    \end{tikzpicture}
\end{figure}

\begin{equation}
\frac{P_{>1}}{P} = ?
\end{equation}

\TT{W pierwszej kolejności należy opisać sygnał za pomocą wzoru:}{First of all, the definition of $f(t)$ signal has to be derived. This is periodic piecewise linear function, which may be describe as:}

\begin{equation}
f(x)=\begin{cases}A & t \in \left (  0+k \cdot T; \frac{T}{2}+k \cdot T \right ) \\0 & t \in \left ( \frac{T}{2}+k \cdot T; T +k \cdot T\right )\end{cases} \wedge k \in \TT{C}{Z}
\end{equation}

\TT{Moc sygnału możemy obliczyć ze wzoru:}{The total power of the signal is defined as:}

\begin{equation}
P=\frac{1}{T} \cdot \int_{T}^{}\left|f(t)\right|^2 \cdot dt
\end{equation}

\TT{Podstawiając wartości sygnału $f(t)$ do wzoru na moc otrzymujemy:}{For the period $t \in (0; T)$, i.e. $k=0$, we get:}

\begin{align*}
P &=\frac{1}{T} \cdot \int_{T}\left|f(t)\right|^2 \cdot dt=\\
&=\frac{1}{T} \cdot \left( \int_{0}^{\frac{T}{2}} \left|A\right|^2 \cdot dt + \int_{\frac{T}{2}}^{T} \left|0\right|^2 \cdot dt \right )=\\
&=\frac{1}{T} \cdot \left( A^2 \cdot \int_{0}^{\frac{T}{2}} dt + 0  \right )=\\
&=\frac{A^2}{T} \cdot \left( \left. t\right |_{0}^{\frac{T}{2}} \right )=\\
&=\frac{A^2}{T} \cdot \left( \frac{T}{2} - 0\right )=\\
&=\frac{A^2}{T} \cdot \left( \frac{T}{2} \right )=\\
&=\frac{A^2}{2} 
\end{align*}

\TT{Moc sygnału $f(t)$ równa się $\frac{A^2}{2}$.}{The total power of the $f(t)$ signal equals $\frac{A^2}{2}$.}

\TT{Na podstawie twierdzenia Parsevala, moc wyższych harmonicznych to:}{Based on Parseval theorem, power of the higher harmonics is defined as:}

\begin{equation}
P_{>1}=P - P_0 - P_1
\end{equation}

\TT{gdzie:}{where:}

\begin{align*}
P_0&=\left|F_{0}\right|^2\\
P_1&=\left|F_{1}\right|^2+\left|F_{-1}\right|^2
\end{align*}

\TT{Ponieważ sygnał $f(t)$ jest sygnałem rzeczywistym, to $\left|F_{1}\right|=\left|F_{-1}\right|$, czyli moc podstawowej harmonicznej:}{Because the $f(t) \in R$, thus $\left|F_{1}\right|=\left|F_{-1}\right|$ and the power of the fundamental harmonic may be calculated as}
 
\begin{equation}
P_1=2 \cdot \left|F_{1}\right|^2
\end{equation}

\TT{W związku z tym, należy obliczyć wartości współczynników $F_0$ i $F_1$.  Współczynniki $F_k$ wyznaczyliśmy już w zadaniu \ref{TaskKW012} is wynoszą:}{In order to calculate $P_0$ and $P_1$, the $F_0$ and $F_1$ coefficients have to be calculated. The $F_k$ coefficients have been calculated in task \ref{TaskKW012} and are equal to:}

\begin{align*}
F_0&=\frac{A}{2}\\
F_k&=\jmath \cdot \frac{A}{k\cdot 2 \pi}\cdot \left( (-1)^{k} -1 \right)\\
\end{align*}

\TT{Teraz możemy obliczyć $P_0$ oraz $P_1$:}{Now, we may calculate the $P_0$ and $P_1$:}

\begin{align*}
P_0&=\left|F_{0}\right|^2\\
&=\left|\frac{A}{2}\right|^2\\
&=\frac{A^2}{4}
\end{align*}

\begin{align*}
P_1&=2 \cdot \left|F_{1}\right|^2\\
&=2 \cdot \left|\jmath \cdot \frac{A}{1\cdot 2 \pi}\cdot \left( (-1)^{1} -1 \right)\right|^2\\
&=2 \cdot \left|\jmath \cdot \frac{A}{2 \pi}\cdot \left( -1 -1 \right)\right|^2\\
&=2 \cdot \left|\jmath \cdot \frac{A}{2 \pi}\cdot \left( -2 \right)\right|^2\\
&=2 \cdot \left|\jmath \cdot \frac{-A}{\pi}\right|^2\\
&=2 \cdot \left(\frac{A}{\pi}\right)^2\\
&=2 \cdot \frac{A^2}{\pi^2}\\
\end{align*}

\TT{Ostatecznie, moc wyższych harmonicznych to:}{Finally, power of the higher harmonics is defined as:}

\begin{align*}
P_{>1}&=P - P_0 - P_1\\
&=\frac{A^2}{2} - \frac{A^2}{4} - 2 \cdot \frac{A^2}{\pi^2}\\
&=\frac{2 \cdot A^2 \cdot \pi^2}{4 \pi^2} - \frac{A^2\cdot \pi^2}{4\pi^2} - \frac{8 \cdot A^2}{4 \pi^2}\\
&=\frac{A^2 \cdot \pi^2 - 8 \cdot A^2}{4 \pi^2}\\
&=\frac{A^2 \cdot \left(\pi^2 - 8 \right)}{4 \pi^2}
\end{align*}

\TT{Moc wyższych harmonicznych równa się $P_{>1}=\frac{A^2 \cdot \left(\pi^2 - 8 \right)}{4 \pi^2}$.}{The power of the fundamental harmonic equals $P_{>1}=\frac{A^2 \cdot \left(\pi^2 - 8 \right)}{4 \pi^2}$.}

\TT{Teraz można wyznaczyć udział mocy wyższych harmonicznych w całkowitej mocy okresowego sygnału $f(t)$:}{Finally, the percentage contribution of the higher harmonics in the total power of the $f(t)$ signal is equal to:}

\begin{equation}
\frac{P_{>1}}{P} = \frac{\frac{A^2 \cdot \left(\pi^2 - 8 \right)}{4 \pi^2}}{\frac{A^2}{2}} = \frac{A^2 \cdot \left(\pi^2 - 8 \right)}{4 \pi^2} \cdot \frac{2}{A^2}= \frac{\pi^2 - 8}{2 \pi^2} = \approx 9\%
\end{equation}


\end{task}