\begin{task}
\TT{Oblicz transformatę Fouriera sygnału $f(t)$ przedstawionego na rysunku wykorzystując twierdzenia opisujące własciwości transformacji Fouriera.
Wykorzystaj informację o tym, że $\mathcal F\{\Pi(t)\}=Sa\left(\frac{\omega}{2}\right)$ oraz $\mathcal F\{\Lambda(t)\}=Sa^2\left(\frac{\omega}{2}\right)$.}{Compute the Fourier transform of the $f(t)$ signal shown below using theorems describing the properties of Fourier transformation. Exploit the following transforms $\mathcal F\{\Pi(t)\}=Sa\left(\frac{\omega}{2}\right)$ and $\mathcal F\{\Lambda(t)\}=Sa^2\left(\frac{\omega}{2}\right)$. }

\begin{figure}[H]
  \centering
  \begin{tikzpicture}
  \draw[->] (-3.0,+0.0) -- (+5.0,+0.0) node[right] {$t$};
  \draw[->] (+0.0,-1.5) -- (+0.0,+1.5) node[above] {$f(t)$};
  %\draw[-,red, thick] (-3.5,+0.0) -- (-1.0,+0.0) -- (0.0,+1.0) -- (1.0,+0.0) -- (3.0,0.0);
  \draw[-,red, thick] (-3.0,+0.0) -- (-1.0,+0.0) -- (-1.0,+1.0) -- (0.0,+0.0) -- (1.0,1.0) -- (1.0,0.0) -- (3.0,0.0);
  \draw[-] (-1.0-0.1,-0.1)--(-1.0+0.1,0.1) node[midway, below, outer sep=5pt,align=center] {$-t_{0}$};
  \draw[-] (+1.0-0.1,-0.1)--(+1.0+0.1,0.1) node[midway, below, outer sep=5pt] {$t_{0}$};
  \draw[-] (-0.1,1.0-0.1)--(+0.1,1.0+0.1) node[midway, above left] {$A$};
  \end{tikzpicture}
\end{figure}

\TT{W pierwszej kolejności opiszmy sygnał za pomocą sygnałów elementarnych:}{First of all, describe the $f(t)$ signal using elementary signals:}

\begin{equation}
f(t) = A \cdot \left(\Pi\left(\frac{t}{2\cdot t_{0}}\right) - \Lambda\left(\frac{t}{t_{0}}\right)\right)
\end{equation}

\TT{Ponieważ transformacja Fouriera jest przekształceniem liniowym, dlatego można wyznaczyć osobno transformaty poszczególnych sygnałów elementarnych, czyli:}{Based on linearity of the Fourier transformation, we can calculate transforms for elementary signals separately:}

\begin{equation}
f(t) = A\cdot \left( f_{1}(t) - f_{2}(t) \right)
\end{equation}
\TT{gdzie:}{where:}
\begin{align*}
f_{1}(t) = \Pi\left(\frac{t}{2\cdot t_{0}}\right)\\
f_{2}(t) = \Lambda\left(\frac{t}{t_{0}}\right)
\end{align*}

\TT{Wyznaczmy transformtę sygnału $f_{1}(t)$, czyli $F_{1}(\jmath \omega)$.}{Calculate the Fourier transform $F_{1}(\jmath \omega)$ for the first signal $f_{1}(t)$.}

\TT{Z treści zadania wiemy, że:}{We know that:}
$\mathcal F \{\Pi(t)\} = Sa\left(\frac{\omega}{2}\right)$.

\TT{Wykorzystując twierdzenie o zmianie skali mamy:}{Based on the scaling theorem:}

\begin{align*}
\TimeScalingTeorem{g}{G}{f}{F}
\end{align*}

\TT{otrzymujemy:}{we get:}

\begin{align*}
\Pi(t) \xrightarrow{\mathcal F} & Sa\left(\frac{\omega}{2}\right)\\
\Pi(\frac{t}{2\cdot t_{0}}) \xrightarrow{\mathcal F} & \frac{1}{\left|\frac{1}{2 \cdot t_{0}}\right|} \cdot Sa\left(\frac{ \frac{\omega}{ \frac{1}{2\cdot t_{0}} }}{2}\right)\\
\Pi(\frac{t}{2\cdot t_{0}}) \xrightarrow{\mathcal F} & 2 \cdot t_{0} \cdot Sa\left(\frac{\omega \cdot 2 \cdot t_{0}}{2}\right)\\
\Pi(\frac{t}{2\cdot t_{0}}) \xrightarrow{\mathcal F} & 2 \cdot t_{0} \cdot Sa\left(\omega \cdot t_{0}\right)\\
\end{align*}

\TT{Transformata sygnału $f_{1}(t)$ to:}{The Fourier transform of $f_{1}(t)$ signal is equal to:}

\begin{equation}
F_{1}(\jmath \omega) = \mathcal F\{f_{1}(t)\} = 2 \cdot t_{0} \cdot Sa\left(\omega \cdot t_{0}\right)
\end{equation}

\TT{Wyznaczmy transformtę sygnału $f_{2}(t)$, czyli $F_{2}(\jmath \omega)$.}{Now, we calculate the Fourier transform $F_{2}(\jmath \omega)$ for the second signal $f_{2}(t)$.}

\TT{Z treści zadania wiemy, że:}{We know that:}
$\mathcal F \{\Lambda(t)\} = Sa^2\left(\frac{\omega}{2}\right)$.

\TT{Wykorzystując twierdzenie o zmianie skali:}{Based on the scaling theorem:}

\begin{align*}
\TimeScalingTeorem{g}{G}{f}{F}
\end{align*}

\TT{otrzymujemy:}{we get:}

\begin{align*}
\Lambda(t) \xrightarrow{\mathcal F} & Sa^2\left(\frac{\omega}{2}\right)\\
\Lambda(\frac{t}{t_{0}}) \xrightarrow{\mathcal F} & \frac{1}{\left|\frac{1}{t_{0}}\right|} \cdot Sa^2\left(\frac{ \frac{\omega}{ \frac{1}{t_{0}} }}{2}\right)\\
\Lambda(\frac{t}{t_{0}}) \xrightarrow{\mathcal F} & t_{0} \cdot Sa^2\left(\frac{\omega \cdot t_{0}}{2}\right)\\
\end{align*}

\TT{Transformata sygnału $f_{2}(t)$ to:}{The Fourier transform of $f_{2}(t)$ signal is equal to:}
\begin{equation}
F_{2}(\jmath \omega) = \mathcal F\{f_{2}(t)\} = t_{0} \cdot Sa^2\left(\frac{\omega \cdot t_{0}}{2}\right)
\end{equation}

\TT{Czyli transformata sygnału $f(t)$ to:}{Finally, the Fourier transform for $f(t)$ signal is equal to:}

\begin{align*}
F(\jmath \omega) = \mathcal F\{f(t)\} &= A \cdot \left( 2 \cdot t_{0} \cdot Sa\left(\omega \cdot t_{0}\right) - t_{0} \cdot Sa^2\left(\frac{\omega \cdot t_{0}}{2}\right)\right)
\end{align*}

\TT{Transformata sygnału $f(t) = A \cdot \Pi\left(\frac{t}{2\cdot t_{0}}\right) - A \cdot \Lambda(\frac{t}{t_{0}})$ to $F(\jmath \omega)=2 \cdot  A \cdot t_{0} \cdot Sa\left(\omega \cdot t_{0}\right) - A \cdot  t_{0} \cdot Sa^2\left(\frac{\omega \cdot t_{0}}{2}\right)$.}{The Fourier transform of $f(t) = A \cdot \Pi\left(\frac{t}{2\cdot t_{0}}\right) - A \cdot \Lambda(\frac{t}{t_{0}})$ is equal to $F(\jmath \omega)=2 \cdot  A \cdot t_{0} \cdot Sa\left(\omega \cdot t_{0}\right) - A \cdot  t_{0} \cdot Sa^2\left(\frac{\omega \cdot t_{0}}{2}\right)$.}

\end{task}
