\begin{task}
\TT{Oblicz transformatę Fouriera sygnału $f(t)$ przedstawionego na rysunku oraz narysuj jego widmo amplitudowe i fazowe.}{Compute the Fourier transform of a signal shown below. Compute and draw magnitude and phase spectra.}

\begin{figure}[H]
\centering
\begin{tikzpicture}
  %\draw (0,0) circle (1in);
  \draw[->] (-4.0,+0.0) -- (+4.0,+0.0) node[right] {$t$};
  \draw[->] (+0.0,-0.5) -- (+0.0,+3.0) node[above] {$f(t)$};
  \draw[-,red, thick] (-3.5,+0.0) -- (-2.0,0.0);
  \draw[-,red, thick] (+2.0,+0.0) -- (+3.5,0.0);
  \draw[scale=1.0,domain=-2.0:0.0,smooth,variable=\x,red,thick] plot ({\x},{1.0*\x+2});%96%3.141592
  \draw[scale=1.0,domain=0.0:2.0,smooth,variable=\x,red,thick] plot ({\x},{1.0*\x});%96%3.141592
  
  \draw[-,red, thick] (0.0,+0.0) -- (0.0,2.0);
  \draw[-,red, thick] (+2.0,+0.0) -- (+2.0,2.0);
  
  \draw[-] (-2.0-0.1,-0.1)--(-2.0+0.1,0.1) node[midway, below, outer sep=5pt,align=center] {$-t_0$};
  \draw[-] (+2.0-0.1,-0.1)--(+2.0+0.1,0.1) node[midway, below, outer sep=5pt] {$t_0$};
  \draw[-] (-0.1,+2.0-0.1)--(+0.1,+2.0+0.1) node[midway, above left] {$A$};
\end{tikzpicture}
\end{figure}

\begin{equation}
f(t)=\begin{cases}
0 & t \in \left( -\infty; -t_0 \right ) \\
\frac{A}{t_0} \cdot t + A & t \in \left( -t_0; 0 \right ) \\
\frac{A}{t_0} \cdot t & t \in \left( 0; t_0 \right ) \\
0 & t \in \left( t_0; \infty \right )
\end{cases} 
\end{equation}

\TT{Transformatę Fouriera obliczamy ze wzoru:}{The Fourier transform is defined as:}

\begin{equation}
F(\jmath \omega )=\int_{-\infty }^{\infty}f(t) \cdot e^{-\jmath \cdot \omega \cdot t}\cdot dt
\end{equation}

\TT{Podstawiamy do wzoru na transformatę wzór naszej funkcji:}{For the given $f(t)$ signal we get:}

\begin{align*}
F(\jmath \omega )&=\int_{-\infty }^{\infty}f(t) \cdot e^{-\jmath \cdot \omega \cdot t}\cdot dt=\\
&=\int_{-\infty }^{-t_0} 0 \cdot e^{-\jmath \cdot \omega \cdot t} \cdot dt + \int_{-t_0 }^{0} \left(\frac{A}{t_0} \cdot t + A \right) \cdot e^{-\jmath \cdot \omega \cdot t} \cdot dt + \int_{0}^{t_0} \frac{A}{t_0} \cdot t \cdot e^{-\jmath \cdot \omega \cdot t} \cdot dt + \int_{t_0 }^{\infty} 0 \cdot e^{-\jmath \cdot \omega \cdot t} \cdot dt = \\
&=\int_{-\infty }^{-t_0} 0 dt + \int_{-t_0 }^{0} \frac{A}{t_0} \cdot t \cdot e^{-\jmath \cdot \omega \cdot t} \cdot dt + \int_{-t_0 }^{0} A \cdot e^{-\jmath \cdot \omega \cdot t} \cdot dt + \int_{0}^{t_0} \frac{A}{t_0} \cdot t \cdot e^{-\jmath \cdot \omega \cdot t} \cdot dt + \int_{t_0 }^{\infty} 0 \cdot dt = \\
&=\frac{A}{t_0} \cdot \int_{-t_0 }^{0}  t \cdot e^{-\jmath \cdot \omega \cdot t} \cdot dt + A \cdot \int_{-t_0 }^{0} e^{-\jmath \cdot \omega \cdot t} \cdot dt + \frac{A}{t_0} \cdot \int_{0}^{t_0} t \cdot e^{-\jmath \cdot \omega \cdot t} \cdot dt = \\
&=\begin{Bmatrix}
u &= t & dv &= e^{-\jmath \cdot \omega \cdot t} \cdot dt \\
du&=dt & v & \frac{1}{-\jmath \cdot \omega} \cdot e^{-\jmath \cdot \omega \cdot t}
\end{Bmatrix}=\\
&=\frac{A}{t_0} \cdot \left(\left. t \cdot \frac{1}{-\jmath \cdot \omega} \cdot e^{-\jmath \cdot \omega \cdot t}\right|_{-t_0 }^{0} + \int_{-t_0 }^{0} \frac{1}{-\jmath \cdot \omega} \cdot e^{-\jmath \cdot \omega \cdot t} \cdot dt\right) + \\
&+ A \cdot \int_{-t_0 }^{0} e^{-\jmath \cdot \omega \cdot t} \cdot dt + \\
&+\frac{A}{t_0} \cdot \left(\left. t \cdot \frac{1}{-\jmath \cdot \omega} \cdot e^{-\jmath \cdot \omega \cdot t}\right|_{0 }^{t_0} + \int_{0 }^{t_0} \frac{1}{-\jmath \cdot \omega} \cdot e^{-\jmath \cdot \omega \cdot t} \cdot dt\right)  = \\
&=\frac{A}{t_0} \cdot \left(\left( 0 \cdot \frac{1}{-\jmath \cdot \omega} \cdot e^{-\jmath \cdot \omega \cdot 0} + t_0 \cdot \frac{1}{-\jmath \cdot \omega} \cdot e^{-\jmath \cdot \omega \cdot (-t_0)}\right) + \frac{1}{-\jmath \cdot \omega} \cdot \int_{-t_0 }^{0} e^{-\jmath \cdot \omega \cdot t} \cdot dt\right) + \\
&+ A \cdot \int_{-t_0 }^{0} e^{-\jmath \cdot \omega \cdot t} \cdot dt + \\
&+\frac{A}{t_0} \cdot \left(\left( t_0 \cdot \frac{1}{-\jmath \cdot \omega} \cdot e^{-\jmath \cdot \omega \cdot t_0} - 0 \cdot \frac{1}{-\jmath \cdot \omega} \cdot e^{-\jmath \cdot \omega \cdot 0}\right) + \frac{1}{-\jmath \cdot \omega} \cdot \int_{0 }^{t_0}  e^{-\jmath \cdot \omega \cdot t} \cdot dt\right)  = \\
&=\begin{Bmatrix}
z &= -\jmath \cdot \omega \cdot t \\
dz &= -\jmath \cdot \omega \cdot dt \\
dt &= \frac{dz}{-\jmath \cdot \omega}
\end{Bmatrix}=\\
&=\frac{A}{t_0} \cdot \left(\left( 0 + t_0 \cdot \frac{1}{-\jmath \cdot \omega} \cdot e^{\jmath \cdot \omega \cdot t_0}\right) + \frac{1}{-\jmath \cdot \omega} \cdot \int_{-t_0 }^{0} e^{z} \cdot \frac{dz}{-\jmath \cdot \omega}\right) + \\
&+ A \cdot \int_{-t_0 }^{0} e^{z} \cdot \frac{dz}{-\jmath \cdot \omega} + \\
&+\frac{A}{t_0} \cdot \left(\left( t_0 \cdot \frac{1}{-\jmath \cdot \omega} \cdot e^{-\jmath \cdot \omega \cdot t_0} - 0\right) + \frac{1}{-\jmath \cdot \omega} \cdot \int_{0 }^{t_0}  e^{z} \cdot \frac{dz}{-\jmath \cdot \omega}\right)  = \\
&=\frac{A}{t_0} \cdot \left(t_0 \cdot \frac{1}{-\jmath \cdot \omega} \cdot e^{\jmath \cdot \omega \cdot t_0} + \frac{1}{-\jmath \cdot \omega} \cdot \frac{1}{-\jmath \cdot \omega} \cdot \int_{-t_0 }^{0} e^{z} \cdot dz\right) + \\
&+ A \cdot \frac{1}{-\jmath \cdot \omega} \cdot \int_{-t_0 }^{0} e^{z} \cdot dz + \\
&+\frac{A}{t_0} \cdot \left( t_0 \cdot \frac{1}{-\jmath \cdot \omega} \cdot e^{-\jmath \cdot \omega \cdot t_0} + \frac{1}{-\jmath \cdot \omega} \cdot \frac{1}{-\jmath \cdot \omega} \cdot \int_{0 }^{t_0}  e^{z} \cdot dz\right)  = \\
&=\frac{A}{t_0} \cdot \left(t_0 \cdot \frac{1}{-\jmath \cdot \omega} \cdot e^{\jmath \cdot \omega \cdot t_0} + \frac{1}{-1 \cdot \omega^2} \cdot \left. e^{z} \right|_{-t_0 }^{0} \right) + \\
&- A \cdot \frac{1}{\jmath \cdot \omega} \cdot \left. e^{z} \right|_{-t_0 }^{0} + \\
&+\frac{A}{t_0} \cdot \left( t_0 \cdot \frac{1}{-\jmath \cdot \omega} \cdot e^{-\jmath \cdot \omega \cdot t_0} + \frac{1}{-1 \cdot \omega^2} \cdot \left. e^{z} \right|_{0 }^{t_0} \right)  = \\
&=\frac{A}{t_0} \cdot \left(t_0 \cdot \frac{1}{-\jmath \cdot \omega} \cdot e^{\jmath \cdot \omega \cdot t_0} + \frac{1}{-1 \cdot \omega^2} \cdot \left. e^{-\jmath \cdot \omega \cdot t} \right|_{-t_0 }^{0} \right) + \\
&- A \cdot \frac{1}{\jmath \cdot \omega} \cdot \left. e^{-\jmath \cdot \omega \cdot t} \right|_{-t_0 }^{0} + \\
&+\frac{A}{t_0} \cdot \left( t_0 \cdot \frac{1}{-\jmath \cdot \omega} \cdot e^{-\jmath \cdot \omega \cdot t_0} + \frac{1}{-1 \cdot \omega^2} \cdot \left. e^{-\jmath \cdot \omega \cdot t} \right|_{0 }^{t_0} \right)  = \\
&=\frac{A}{t_0} \cdot \left(t_0 \cdot \frac{1}{-\jmath \cdot \omega} \cdot e^{\jmath \cdot \omega \cdot t_0} + \frac{1}{-1 \cdot \omega^2} \cdot \left( e^{-\jmath \cdot \omega \cdot 0} - e^{-\jmath \cdot \omega \cdot (-t_0)}\right) \right) + \\
&- A \cdot \frac{1}{\jmath \cdot \omega} \cdot \left( e^{-\jmath \cdot \omega \cdot 0} - e^{-\jmath \cdot \omega \cdot (-t_0)} \right) + \\
&+\frac{A}{t_0} \cdot \left( t_0 \cdot \frac{1}{-\jmath \cdot \omega} \cdot e^{-\jmath \cdot \omega \cdot t_0} + \frac{1}{-1 \cdot \omega^2} \cdot \left( e^{-\jmath \cdot \omega \cdot t_0} - e^{-\jmath \cdot \omega \cdot 0}\right) \right)  = \\
&=\frac{A}{t_0} \cdot \left(t_0 \cdot \frac{1}{-\jmath \cdot \omega} \cdot e^{\jmath \cdot \omega \cdot t_0} + \frac{1}{-1 \cdot \omega^2} \cdot \left( e^{0} - e^{\jmath \cdot \omega \cdot t_0}\right) \right) + \\
&- A \cdot \frac{1}{\jmath \cdot \omega} \cdot \left( e^{0} - e^{\jmath \cdot \omega \cdot t_0} \right) + \\
&+\frac{A}{t_0} \cdot \left( t_0 \cdot \frac{1}{-\jmath \cdot \omega} \cdot e^{-\jmath \cdot \omega \cdot t_0} + \frac{1}{-1 \cdot \omega^2} \cdot \left( e^{-\jmath \cdot \omega \cdot t_0} - e^{0}\right) \right)  = \\
&=\frac{A}{t_0} \cdot  t_0 \cdot \frac{1}{-\jmath \cdot \omega} \cdot e^{\jmath \cdot \omega \cdot t_0} + \frac{A}{t_0} \cdot \frac{1}{-1 \cdot \omega^2} \cdot \left( 1 - e^{\jmath \cdot \omega \cdot t_0}\right) + \\
&- A \cdot \frac{1}{\jmath \cdot \omega} \cdot \left( 1 - e^{\jmath \cdot \omega \cdot t_0} \right) + \\
&+\frac{A}{t_0} \cdot t_0 \cdot \frac{1}{-\jmath \cdot \omega} \cdot e^{-\jmath \cdot \omega \cdot t_0} + \frac{A}{t_0} \cdot \frac{1}{-1 \cdot \omega^2} \cdot \left( e^{-\jmath \cdot \omega \cdot t_0} - 1\right) = \\
&= -\frac{A}{\jmath \cdot \omega} \cdot e^{\jmath \cdot \omega \cdot t_0} - \frac{A}{t_0 \cdot \omega^2} \cdot \left( 1 - e^{\jmath \cdot \omega \cdot t_0}\right) + \\
&- \frac{A}{\jmath \cdot \omega} + \frac{A}{\jmath \cdot \omega} \cdot e^{\jmath \cdot \omega \cdot t_0} + \\
&- \frac{A}{\jmath \cdot \omega} \cdot e^{-\jmath \cdot \omega \cdot t_0} - \frac{A}{t_0 \cdot \omega^2} \cdot \left( e^{-\jmath \cdot \omega \cdot t_0} - 1\right) = \\
&= -\frac{A}{t_0 \cdot \omega^2} + \frac{A}{t_0 \cdot \omega^2} \cdot  e^{\jmath \cdot \omega \cdot t_0} - \frac{A}{\jmath \cdot \omega} + \\
&- \frac{A}{\jmath \cdot \omega} \cdot e^{-\jmath \cdot \omega \cdot t_0} - \frac{A}{t_0 \cdot \omega^2} \cdot e^{-\jmath \cdot \omega \cdot t_0} + \frac{A}{t_0 \cdot \omega^2} = \\
&= \frac{A}{t_0 \cdot \omega^2} \cdot  e^{\jmath \cdot \omega \cdot t_0} - \frac{A}{t_0 \cdot \omega^2} \cdot e^{-\jmath \cdot \omega \cdot t_0} - \frac{A}{\jmath \cdot \omega} - \frac{A}{\jmath \cdot \omega} \cdot e^{-\jmath \cdot \omega \cdot t_0}  = \\
&= \frac{A}{t_0 \cdot \omega^2} \cdot  \left( e^{\jmath \cdot \omega \cdot t_0} - e^{-\jmath \cdot \omega \cdot t_0} \right) - \frac{A}{\jmath \cdot \omega} \left( 1+  e^{-\jmath \cdot \omega \cdot t_0} \right)  = \\
&= \frac{A \cdot 2 \cdot \jmath}{t_0 \cdot \omega^2} \cdot  \frac{ e^{\jmath \cdot \omega \cdot t_0} - e^{-\jmath \cdot \omega \cdot t_0} }{2\cdot \jmath} - \frac{A}{\jmath \cdot \omega} \left( 1+  e^{-\jmath \cdot \omega \cdot t_0} \right)  = \\
&=\begin{Bmatrix}
\EulerSin
\end{Bmatrix}=\\
&= \jmath \cdot \frac{2 \cdot A}{t_0 \cdot \omega^2} \cdot  sin\left(\omega \cdot t_0\right) + \jmath \cdot \frac{A}{\omega} \left( 1+  e^{-\jmath \cdot \omega \cdot t_0} \right)
\end{align*}

\TT{Transformata sygnału $f(t)$ to }{The Fourier transform of the $f(t)$ is equal to } $F(\jmath \omega)=\jmath \cdot \frac{2 \cdot A}{t_0 \cdot \omega^2} \cdot  sin\left(\omega \cdot t_0\right) + \jmath \cdot \frac{A}{\omega} \left( 1+  e^{-\jmath \cdot \omega \cdot t_0} \right)$.

\TT{Narysujmy widmo amplitudowe sygnału $f(t)$ czyli:}{Draw complex spectrum of the $f(t)$:}

\begin{equation}
M\left(\omega\right) = \left|F(\jmath \omega)\right|
\end{equation}

\TT{Podstawmy wzór na obliczoną transformatę Fouriera}{}.

\begin{align*}
M\left(\omega\right) &= \left|F(\jmath \omega)\right| =\\
&=\left| \jmath \cdot \frac{2 \cdot A}{t_0 \cdot \omega^2} \cdot  sin\left(\omega \cdot t_0\right) + \jmath \cdot \frac{A}{\omega} \left( 1+  e^{-\jmath \cdot \omega \cdot t_0} \right) \right| =\\
&=\begin{Bmatrix}
e^{\jmath \cdot x} = cos\left(x\right) + \jmath \cdot sin\left(x\right)
\end{Bmatrix}=\\
&=\left| \jmath \cdot \frac{2 \cdot A}{t_0 \cdot \omega^2} \cdot  sin\left(\omega \cdot t_0\right) + \jmath \cdot \frac{A}{\omega} \left( 1+  cos\left(- \omega \cdot t_0\right) + \jmath \cdot sin\left(- \omega \cdot t_0\right) \right) \right| =\\
&=\left| \jmath \cdot \frac{2 \cdot A}{t_0 \cdot \omega^2} \cdot  sin\left(\omega \cdot t_0\right) + \jmath \cdot \frac{A}{\omega} \left( 1+  cos\left(\omega \cdot t_0\right) - \jmath \cdot sin\left(\omega \cdot t_0\right) \right) \right| =\\
&=\left| \jmath \cdot \frac{2 \cdot A}{t_0 \cdot \omega^2} \cdot  sin\left(\omega \cdot t_0\right) + \jmath \cdot \frac{A}{\omega} + \jmath \cdot \frac{A}{\omega} \cdot  cos\left(\omega \cdot t_0\right) - \jmath \cdot \frac{A}{\omega} \cdot \jmath \cdot sin\left(\omega \cdot t_0\right) \right| =\\
&=\left| \jmath \cdot \left( \frac{2 \cdot A}{t_0 \cdot \omega^2} \cdot  sin\left(\omega \cdot t_0\right) + \frac{A}{\omega} + \frac{A}{\omega} \cdot  cos\left(\omega \cdot t_0\right)\right) + \frac{A}{\omega} \cdot sin\left(\omega \cdot t_0\right) \right| =\\
&=\left| \frac{A}{\omega} \cdot sin\left(\omega \cdot t_0\right) + \jmath \cdot \left( \frac{2 \cdot A}{t_0 \cdot \omega^2} \cdot  sin\left(\omega \cdot t_0\right) + \frac{A}{\omega} \cdot \left(1 + cos\left(\omega \cdot t_0\right)\right)\right) \right| =\\
&=\left| \frac{A}{\omega} \cdot \left( sin\left(\omega \cdot t_0\right) + \jmath \cdot \left( \frac{2}{t_0 \cdot \omega} \cdot  sin\left(\omega \cdot t_0\right) + 1 + cos\left(\omega \cdot t_0\right)\right)\right) \right| =\\
&=\left| \frac{A}{\omega}\right| \cdot \left| sin\left(\omega \cdot t_0\right) + \jmath \cdot \left( \frac{2}{t_0 \cdot \omega} \cdot  sin\left(\omega \cdot t_0\right) + 1 + cos\left(\omega \cdot t_0\right)\right) \right| =\\
&= \frac{A}{\left|\omega\right|} \cdot \sqrt{sin^2\left(\omega \cdot t_0\right) + \left( \frac{2}{t_0 \cdot \omega} \cdot  sin\left(\omega \cdot t_0\right) + 1 + cos\left(\omega \cdot t_0\right)\right)^2 }\\
\end{align*}

\TT{Wyznaczmy gdzie znajdują się miejsca zerowe modułu transformaty Fouriera}{}

\begin{align*}
M\left(w\right) = \frac{A}{\left|\omega\right|} \cdot \sqrt{sin^2\left(\omega \cdot t_0\right) + \left( \frac{2}{t_0 \cdot \omega} \cdot  sin\left(\omega \cdot t_0\right) + 1 + cos\left(\omega \cdot t_0\right)\right)^2 } &= 0  \left.  \right \| \cdot \frac{\left|\omega\right|}{A}\\
\sqrt{sin^2\left(\omega \cdot t_0\right) + \left( \frac{2}{t_0 \cdot \omega} \cdot  sin\left(\omega \cdot t_0\right) + 1 + cos\left(\omega \cdot t_0\right)\right)^2 } &= 0 \left.  \right \| (\cdot)^2\\
sin^2\left(\omega \cdot t_0\right) + \left( \frac{2}{t_0 \cdot \omega} \cdot  sin\left(\omega \cdot t_0\right) + 1 + cos\left(\omega \cdot t_0\right)\right)^2 = 0
\end{align*}

\TT{Suma kwadratów może być równa zero wtedy i tylko wtedy gdy oba czynniki sa równe zero a wiec: }

\begin{align*}
sin^2\left(\omega \cdot t_0\right) &= 0 \\
sin\left(\omega \cdot t_0\right) &= 0 \\
sin\left(\omega \cdot t_0\right) &= sin\left(\pi \cdot k\right)\\ 
\omega \cdot t_0 &= \pi \cdot k\\
\omega &= \frac{\pi}{t_0} \cdot k\\
\end{align*}

\TT{oraz}{}

\begin{align*}
\left( \frac{2}{t_0 \cdot \omega} \cdot  sin\left(\omega \cdot t_0\right) + 1 + cos\left(\omega \cdot t_0\right)\right)^2 &= 0\\
\frac{2}{t_0 \cdot \omega} \cdot  sin\left(\omega \cdot t_0\right) + 1 + cos\left(\omega \cdot t_0\right) &= 0\\
\begin{Bmatrix}
1+cos(x) = 2 \cdot cos^2\left(\frac{x}{2}\right)\\
\SaDef
\end{Bmatrix}\\
2 \cdot Sa\left(\omega \cdot t_0\right) + 2 \cdot cos^2\left(\frac{\omega \cdot t_0}{2}\right) &= 0\\
\end{align*}

\TT{Ponownie suma funkcji może być zerem wtedy i tylko wtedy gdy obie sa równe zero a wiec}{}

\begin{align*}
2 \cdot Sa\left(\omega \cdot t_0\right) &= 0 \\
Sa\left(\omega \cdot t_0\right) &= 0 \\
Sa\left(\omega \cdot t_0\right) &= Sa\left(\pi \cdot k\right) \\
\omega \cdot t_0 &= \pi \cdot k\\
\omega &= \frac{\pi}{t_0} \cdot k\\
\end{align*}

\begin{align*}
2 \cdot cos^2\left(\frac{\omega \cdot t_0}{2}\right) &= 0\\
cos^2\left(\frac{\omega \cdot t_0}{2}\right) &= 0\\
cos\left(\frac{\omega \cdot t_0}{2}\right) &= 0\\
cos\left(\frac{\omega \cdot t_0}{2}\right) &= cos\left(\frac{\pi}{2} + \pi\cdot k\right)\\
\frac{\omega \cdot t_0}{2} &= \frac{\pi}{2} + \pi\cdot k\\
\omega \cdot t_0 &= \pi + 2\cdot \pi \cdot k\\
\omega &= \frac{\pi + 2\cdot \pi \cdot k}{t_0}\\
\end{align*}

\TT{A wiec ostatecznie miejsca zerową znajdują się w miejscach: $\omega = \frac{\pi + 2\cdot \pi \cdot k}{t_0}\$}

\TT{A wiec widmo amplitudowe wygląda następująco}{}
\begin{figure}[H]
	\centering
	\begin{tikzpicture}
	\draw[->] (-7.0,+0.0) -- (+7.0,+0.0) node[right] {$\omega$};
  \draw[->] (+0.0,-0.5) -- (+0.0,+3.0) node[above] {$M(\omega)$};
  
  \draw[scale=1.0,domain=-6.0:-0.5,samples=2000,smooth,variable=\x,red,thick] plot ({\x},{abs(1.0/\x)*sqrt((sin(\x*2.0*180/3.14))^2+(2/(2.0*\x)*sin(\x*2.0*180/3.14)+1+cos(\x*2.0*180/3.14))^2)});
  \draw[scale=1.0,domain=0.5:6.0,samples=2000,smooth,variable=\x,red,thick] plot ({\x},{abs(1.0/\x)*sqrt((sin(\x*2.0*180/3.14))^2+(2/(2.0*\x)*sin(\x*2.0*180/3.14)+1+cos(\x*2.0*180/3.14))^2)});
  
  \draw[-] (-3*3.14/2.0-0.1,-0.1)--(-3*3.14/2.0+0.1,0.1) node[midway, below, outer sep=5pt] {-$\frac{3\pi}{t_0}$};
  \draw[-] (-3.14/2.0-0.1,-0.1)--(-3.14/2.0+0.1,0.1) node[midway, below, outer sep=5pt] {-$\frac{\pi}{t_0}$};
  \draw[-] (+3.14/2.0-0.1,-0.1)--(+3.14/2.0+0.1,0.1) node[midway, below, outer sep=5pt] {$\frac{\pi}{t_0}$};
  \draw[-] (+3*3.14/2.0-0.1,-0.1)--(+3*3.14/2.0+0.1,0.1) node[midway, below, outer sep=5pt] {$\frac{3\pi}{t_0}$};
  \draw[-] (-0.1,+2.0-0.1)--(+0.1,+2.0+0.1) node[midway, above right] {$A \cdot t_0$};
  		
	\end{tikzpicture}
\end{figure}

\TT{Widmo aplitudowe sygnału rzeczywistego jest zawsze parzyste.}{The magnitude spectrum of a \underline{real signal} is an even-symmetric function of $w$.}

\TT{Widmo fazowe obliczamy ze wzoru:}{The phase spectrum is defined as:}

\begin{equation}
\Phi ( \omega )=\TT{arctg}{arctan2}\left(\frac{Im\{F(\jmath \cdot \omega )\}}{Re\{F(\jmath \cdot \omega )\}}\right)
\end{equation}

%Z jakiegoś powodu nie działa
\begin{figure}[H]
  \centering
  \begin{tikzpicture}
  \draw[->] (-7.0,+0.0) -- (+7.0,+0.0) node[right] {$\omega$};
	\draw[->] (+0.0,-2.5) -- (+0.0,+3.0) node[above] {$\Phi(\omega)$};
  
  \draw[scale=1.0,domain=-6.0:-0.2,samples=2000,smooth,variable=\x,red,thick] plot ({\x},{atan2(2/(2.0*\x)*sin(\x*2.0*180/3.14)+1+cos(\x*2.0*180/3.14),sin(\x*2.0*180/3.14))/90});
  \draw[scale=1.0,domain=0.2:6.0,samples=2000,smooth,variable=\x,red,thick] plot ({\x},{atan2(2/\x*(2/(2.0*\x)*sin(\x*2.0*180/3.14)+1+cos(\x*2.0*180/3.14)),2/\x*sin(\x*2.0*180/3.14))/90});
  
  %\draw[scale=1.0,domain=0.5:6.0,samples=2000,smooth,variable=\x,red,thick] plot ({\x},{abs(1.0/\x)*sqrt((sin(\x*2.0*180/3.14))^2+(2/(2.0*\x)*sin(\x*2.0*180/3.14)+1+cos(\x*2.0*180/3.14))^2)});
  
  \draw[-] (-3*3.14/2.0-0.1,-0.1)--(-3*3.14/2.0+0.1,0.1) node[midway, below, outer sep=5pt] {-$\frac{3\pi}{t_0}$};
  \draw[-] (-3.14/2.0-0.1,-0.1)--(-3.14/2.0+0.1,0.1) node[midway, below, outer sep=5pt] {-$\frac{\pi}{t_0}$};
  \draw[-] (+3.14/2.0-0.1,-0.1)--(+3.14/2.0+0.1,0.1) node[midway, below, outer sep=5pt] {$\frac{\pi}{t_0}$};
  \draw[-] (+3*3.14/2.0-0.1,-0.1)--(+3*3.14/2.0+0.1,0.1) node[midway, below, outer sep=5pt] {$\frac{3\pi}{t_0}$};
  \draw[-] (-0.1,+2.0-0.1)--(+0.1,+2.0+0.1) node[midway, above right] {$\pi$};
  \draw[-] (-0.1,-2.0-0.1)--(+0.1,-2.0+0.1) node[midway, above right] {$-\pi$};
  
  \end{tikzpicture}
\end{figure}

\TT{Widmo fazowe sygnału rzeczywistego jest zawsze nieparzyste.}{The phase spectrum of a \underline{real signal} is an odd-symmetric function of $w$.}

\end{task}

