\begin{task}
\TT{Oblicz transformatę Fouriera sygnału $f(t)=\frac{1}{1+t^2}$ przedstawionego na rysunku wykorzystując twierdzenia opisujące własciwości transformacji Fouriera. Wykorzystaj informację o tym, że $\mathcal F\{e^{-\left|t\right|}\} = \frac{2}{1 + \omega^2}$.}{Compute the Fourier transform of the $f(t)=\frac{1}{1+t^2}$ signal shown below using theorems describing the properties of Fourier transformation. Exploit the following transform  $\mathcal F\{e^{-\left|t\right|}\} = \frac{2}{1 + \omega^2}$.}

\begin{figure}[H]
\centering
\begin{tikzpicture}
  %\draw (0,0) circle (1in);
  \draw[->] (-4.0,+0.0) -- (+4.0,+0.0) node[right] {$t$};
  \draw[->] (+0.0,-2.5) -- (+0.0,+2.5) node[above] {$f(t)$};
  
  \draw[scale=1.0,domain=-3.5:3.5,samples=1000,smooth,variable=\x,red,thick] plot ({\x},{2/(1+\x*\x)});%
  
  \draw[-,red, dashed] (-1.0,+1.0) -- (+1.0,+1.0);
  \draw[-,red, dashed] (-1.0,+0.0) -- (-1.0,+1.0);
  \draw[-,red, dashed] (+1.0,+0.0) -- (+1.0,+1.0);
  
  \draw[-] (-0.1,+2.0-0.1)--(+0.1,+2.0+0.1) node[midway, above left] {$1$};
  \draw[-] (-0.1,+1.0-0.1)--(+0.1,+1.0+0.1) node[midway, above left] {$0.5$};
  \draw[-] (-1.0-0.1,-0.1)--(-1.0+0.1,0.1) node[midway, below, outer sep=5pt,align=center] {$-1$};
  \draw[-] (+1.0-0.1,-0.1)--(+1.0+0.1,0.1) node[midway, below, outer sep=5pt,align=center] {$1$};
\end{tikzpicture}
\end{figure}

\TT{Z treści zadania wiemy, że:}{We know that:}
\begin{equation}
\mathcal F\{e^{-\left|t\right|}\} = \frac{2}{1 + \omega^2}
\end{equation}

\TT{Oznaczmy $g(t) = e^{-\left|t\right|}$ oraz $G(\jmath \omega)=\frac{2}{1+\omega^2}$.}{Let's denote $g(t) = e^{-\left|t\right|}$ and $G(\jmath \omega)=\frac{2}{1+\omega^2}.$} 

\TT{Na podstawie twierdzenia o symetrii możemy wyznaczyć następującą transformatę:}{Based on duality theorem we can derive the following transform:}

\begin{align*}
\SymetryTeorem{g}{G}{h}{H}
\end{align*}

\begin{align*}
H(\jmath \omega)&=2\pi \cdot g(-\omega)=\\
&=2\pi \cdot e^{-\left|-\omega\right|}=\\
&=2\pi \cdot e^{-\left|\omega\right|}
\end{align*}

\TT{Niestety sygnał $h(t)=\frac{2}{1+t^2}$ nie jest równy sygnałowi $f(t)=\frac{1}{1+t^2}$, ale:}{Unfortunately, the $h(t)=\frac{2}{1+t^2}$ signal in not equal to $f(t)=\frac{1}{1+t^2}$ signal. But:}

\begin{equation}
f(t) = \frac{1}{2} \cdot h(t)
\end{equation}

\TT{Z twierdzenia o liniowości transformaty możemy obliczyć:}{Based on linearity theorem we can calculate:}

\begin{align*}
h(t) &\xrightarrow{\mathcal F} H(\jmath \omega)\\
f(t)=\alpha \cdot h(t) &\xrightarrow{\mathcal F} F(\jmath \omega) = \alpha \cdot H(\jmath \omega)\
\end{align*}

\TT{Ostatecznie otrzymujemy:}{Finally, we get:}

\begin{align*}
f(t) &= \frac{1}{2} \cdot \frac{2}{1+t^2}\\
F(\jmath \omega)&=\frac{1}{2} \cdot 2\pi \cdot e^{-\left|\omega\right|}=\\
&=\pi \cdot e^{-\left|\omega\right|}
\end{align*}

\TT{Transformata Fouriera sygnału $f(t)=\frac{1}{1+t^2}$ jest równa $F(\jmath \omega)=\pi \cdot e^{-\left|\omega\right|}$.}{The Fourier transform of the $f(t)=\frac{1}{1+t^2}$ signal is equal to $F(\jmath \omega)=\pi \cdot e^{-\left|\omega\right|}$.}

\end{task}

