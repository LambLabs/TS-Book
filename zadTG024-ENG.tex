\begin{task}
\TT{Oblicz transformatę Fouriera sygnału $f(t)=e^{-\left|t\right|}$ przedstawionego na rysunku wykorzystując twierdzenia opisujące własciwości transformacji Fouriera. Wykorzystaj informację o tym, że $\mathcal F\{A \cdot \mathbb{1}(t) \cdot e^{-a\cdot t}\} = \frac{A}{ a + \jmath \cdot \omega}$.}{Compute the Fourier transform of the $f(t)=e^{-\left|t\right|}$ signal shown below using theorems describing the properties of Fourier transformation. Exploit the following transform  $\mathcal F\{A \cdot \mathbb{1}(t) \cdot e^{-a\cdot t}\}=\frac{A}{ a + \jmath \cdot \omega}$.}

\begin{figure}[H]
    \centering
    \begin{tikzpicture}
    %\draw (0,0) circle (1in);
    \draw[->] (-4.5,+0.0) -- (+4.5,+0.0) node[right] {$t$};
    \draw[->] (+0.0,-1.0) -- (+0.0,+2.0) node[above] {$f(t)$};
    
    \draw[-] (-0.1,+1.5-0.1)--(+0.1,+1.5+0.1) node[midway, left] {$1$};
    
    \draw[scale=1.0,domain=-4.0:0.0,smooth,variable=\x,red,thick] plot ({\x},{1.5*exp(\x)});
    \draw[scale=1.0,domain=0.0:4.0,smooth,variable=\x,red,thick] plot ({\x},{1.5*exp(-\x)});
    
    \end{tikzpicture}
\end{figure}

\TT{W pierwszej kolejności należy opisać sygnał za pomocą wzoru:}{The signal $f(t)$, as a piecewise function, is given by:}

\begin{equation}
f(t) = \begin{cases}
e^{t} & \TT{\text{ dla }}{\text{ for }} t \in \left(-\infty; 0\right)\\
e^{-t} & \TT{\text{ dla }}{\text{ for }} t \in \left(0; \infty\right)
\end{cases}
\end{equation}

\TT{Oznaczmy sygnał $\mathbb{1}(t) \cdot e^{-t}$ jako $g(t)$:}{Let's denote $\mathbb{1}(t) \cdot e^{-t}$ signal as $g(t)$:}

\begin{equation}
g(t) = \mathbb{1}(t) \cdot e^{-t}
\end{equation}

\begin{figure}[H]
    \centering
    \begin{tikzpicture}
    %\draw (0,0) circle (1in);
    \draw[->] (-4.5,+0.0) -- (+4.5,+0.0) node[right] {$t$};
    \draw[->] (+0.0,-1.0) -- (+0.0,+2.0) node[above] {$g(t)$};
    
    \draw[-] (-0.1,+1.5-0.1)--(+0.1,+1.5+0.1) node[midway, left] {$1$};
    
    \draw[scale=1.0,domain=-4.0:0.0,smooth,variable=\x,red,thick] plot ({\x},{0});
    \draw[scale=1.0,domain=0.0:4.0,smooth,variable=\x,red,thick] plot ({\x},{1.5*exp(-\x)});
    
    \end{tikzpicture}
\end{figure}

\TT{Teraz sygnał $f(t)$ możemy wyrazić jako liniową kombinację sygnałów $g(t)$ oraz $g(-t)$:}{Now, the $f(t)$ signal may be expressed as a linear combination of $g(t)$ and $g(-t)$ signals:}

\begin{equation}
f(t) = g(t) + g(-t)
\end{equation}

\TT{Ponieważ transformacja Fouriera jest przekształceniem liniowym, dlatego można wyznaczyć osobno transformaty poszczególnych sygnałów.}{Based on linearity of the Fourier transformation, we can calculate transforms for elementary signals separately.}

\TT{Wyznaczmy transformtę sygnału $g(t)$, czyli $G(\jmath \omega)$.}{Let's calculate the Fourier transform $G(\jmath \omega)$ for the first signal $g(t)$.}

\TT{Z treści zadania wiemy, że:}{We know that:}
\begin{equation}
\mathcal F\{A \cdot \mathbb{1}(t) \cdot e^{-a\cdot t}\}=\frac{A}{ a + \jmath \cdot \omega}
\end{equation}

\TT{Podstawiając $A=1$ oraz $a=1$ otrzymujemy wprost transformatę sygnału $g(t)$, czyli $G(\jmath \omega)$:}{Substituting $A=1$ and $a=1$ we directly get the Fourier transform $G(\jmath \omega)$ for $g(t)$ signal:}

\begin{align*}
A \cdot \mathbb{1}(t) \cdot e^{-a\cdot t} &\xrightarrow{\mathcal F} \frac{A}{ a + \jmath \cdot \omega}\\
\begin{Bmatrix}
A=1\\
a=1
\end{Bmatrix}\\
g(t) = \mathbb{1}(t) \cdot e^{-t} &\xrightarrow{\mathcal F} \frac{1}{ 1 + \jmath \cdot \omega} = G(\jmath \omega)\\
\end{align*}


\TT{Wykorzystując twierdzenie o zmianie skali, możemy obliczyć transformatę sygnału $g(-t)$:}{Based on the scaling theorem, we can derive the Fourier transform of the $g(-t)$ signal:}

\begin{align*}
\TimeScalingTeorem{g}{G}{f}{F}
\end{align*}

\begin{align*}
g(t) \xrightarrow{\mathcal F} & \frac{1}{ 1 + \jmath \cdot \omega}\\
g(-1 \cdot t) \xrightarrow{\mathcal F} & \frac{1}{\left|-1\right|} \cdot \frac{1}{ 1 + \jmath \cdot \frac{\omega}{-1}}\\
g(-t) \xrightarrow{\mathcal F} & \frac{1}{1} \cdot \frac{1}{ 1 - \jmath \cdot \omega}\\
g(-t) \xrightarrow{\mathcal F} & \frac{1}{ 1 - \jmath \cdot \omega}
\end{align*}


\TT{Czyli transformata sygnału $f(t)$ to:}{Finally, the Fourier transform for the $f(t)$ signal is equal to:}

\begin{align*}
F(\jmath \omega) = \mathcal F\{f(t)\} & = \frac{1}{ 1 + \jmath \cdot \omega} + \frac{1}{ 1 - \jmath \cdot \omega}\\
F(\jmath \omega) & = \frac{1 - \jmath \cdot \omega + 1 + \jmath \cdot \omega}{ (1 + \jmath \cdot \omega) \cdot (1 - \jmath \cdot \omega)}\\
F(\jmath \omega) & = \frac{2}{1 - \jmath^2 \cdot \omega^2}\\
F(\jmath \omega) & = \frac{2}{1 + \omega^2}
\end{align*}

\TT{Transformata sygnału $f(t)=e^{-\left|t\right|}$ to $F(\jmath \omega) = \frac{2}{1 + \omega^2}$.}{The Fourier transform of the $f(t)=e^{-\left|t\right|}$ is equal to $F(\jmath \omega) = \frac{2}{1 + \omega^2}$.}

\end{task}
