\begin{task}
Oblicz splot sygnałów $f(t)=A \cdot \Pi\left(\frac{t-T}{T}\right)$ i $h(t)=\mathbb{1}(t) \cdot e^{-a \cdot t}$

\begin{figure}[H]
\centering
\begin{tikzpicture}
  %\draw (0,0) circle (1in);
  \draw[->] (-2.0-5.0,+0.0) -- (+2.0-5.0,+0.0) node[right] {$t$};
  \draw[->] (-1.0-5.0,-1.0) -- (-1.0-5.0,+2.5) node[above] {$f(t)$};
  
  \draw[-,red, thick] (-1.0-5.0,+2.0) -- (1.0-5.0,+2.0);
  \draw[-,red, thick] (-1.0-5.0,+0.0) -- (-1.0-5.0,+2.0);
  \draw[-,red, thick] (1.0-5.0,+0.0) -- (1.0-5.0,+2.0);
  
  \draw[-] (-0.1-6.0,+2.0-0.1)--(+0.1-6.0,+2.0+0.1) node[midway, above left] {$A$};
  \draw[-] (+1.0-0.1-5.0,-0.1)--(+1.0+0.1-5.0,0.1) node[midway, below, outer sep=5pt,align=center] {$T$};
  
  
  \draw[->] (-2.0,+0.0) -- (+4.0,+0.0) node[right] {$t$};
  \draw[->] (-1.0,-1.0) -- (-1.0,+2.5) node[above] {$h(t)$};
  
  \draw[scale=1.0,domain=0.0:4.0,smooth,variable=\x,green,thick] plot ({\x-1},{2*exp(-\x)});
 
  \draw[-] (-1.0-0.1,+2.0-0.1)--(-1.0+0.1,+2.0+0.1) node[midway, above left] {$1$};
  
\end{tikzpicture}
\end{figure}

Wzór na slot sygnałów
\begin{equation}
\Convolution{y}{f}{h}
\end{equation}

Wzory sygnałów pod całką
\begin{align*}
f(\tau)&=A \cdot \Pi\left(\frac{\tau}{T}\right)\\
h(t-\tau)&=\mathbb{1}(t) \cdot e^{-a \cdot (t-\tau)}
\end{align*}

\begin{align*}
f(\tau) &= \begin{cases}
0 & \tau \in \left( -\infty; 0 \right) \\
A & \tau \in \left(0; T \right) \\
0 & \tau \in \left(T; \infty \right) \\
\end{cases}\\
h(t-\tau) &= \begin{cases}
e^{-a \cdot (t-\tau)} & \tau \in \left(-\infty; t\right)\\
0 & \tau \in \left(t; \infty \right)
\end{cases}
\end{align*}

Wykresy obu funkcji w dziedzinie $\tau$ dla różnych wartości $t$:

\begin{figure}[H]
  \centering
  \begin{animateinline}[controls,autoplay,loop,poster = 25,palindrome]{10}
    \multiframe{81}{n=-1.0+0.05}{%Number of Frames, variable = initial + increment
      \begin{tikzpicture}
      \draw[->] (-5.0,+0.0) -- (+4.5,+0.0) node[right] {$\tau$};
      \draw[->] (+0.0,-1.0) -- (+0.0,+2.5) node[above] {$f(\tau),h(t-\tau)$};
      
      \draw[-,red, thick] (0.0,+2.0) -- (2.0,+2.0);
      \draw[-,red, thick] (0.0,+0.0) -- (0.0,+2.0);
      \draw[-,red, thick] (+2.0,+0.0) -- (+2.0,+2.0);
      
      \draw[scale=1.0,domain=-4.0:0.0,smooth,variable=\x,green,thick] plot ({\x+\n},{2*exp(\x)});
      \draw[-,green, thick] (0.0+\n,+0.0) -- (0.0+\n,+2.0);
           
      \draw[-] (-0.1,+2.0-0.1)--(+0.1,+2.0+0.1) node[midway, above left] {$A$};
      \draw[-] (-0.0-0.1+\n,-0.1)--(-0.0+0.1+\n,0.1) node[midway, below, outer sep=25pt,align=center] {$t$};
      
      \draw[-] (+2.0-0.1,-0.1)--(+2.0+0.1,0.1) node[midway, below, outer sep=5pt,align=center] {$T$};
      
      \end{tikzpicture}  
    }
  \end{animateinline}
\end{figure}

Po wymnożeniu obu funkcji dla przykładowych wartości $t$ otrzymujemy:

\begin{figure}[H]
	\centering
	\begin{animateinline}[controls,autoplay,loop,poster = 25,palindrome]{10}
		\multiframe{81}{n=-1.0+0.05}{%Number of Frames, variable = initial + increment
			\begin{tikzpicture}
			
			\tikzmath{
				function fun_tg01(\x,\n,\T) {
					if (\x < 0) || (\x > 2)|| (\x > \n) then {
						return 0.0;
					} else {
						return exp(\x-\n);
					};
				};
			}
			\draw[->] (-5.0,+0.0) -- (+4.5,+0.0) node[right] {$\tau$};
			\draw[->] (+0.0,-1.0) -- (+0.0,+2.5) node[above] {$f(\tau),h(t-\tau)$};
			
			\draw[-,red, dotted] (0.0,+2.0) -- (2.0,+2.0);
			\draw[-,red, dotted] (0.0,+0.0) -- (0.0,+2.0);
			\draw[-,red, dotted] (+2.0,+0.0) -- (+2.0,+2.0);
			
			\draw[scale=1.0,domain=-4.0:0.0,smooth,variable=\x,green,dotted] plot ({\x+\n},{2*exp(\x)});
			\draw[-,green, dotted] (0.0+\n,+0.0) -- (0.0+\n,+2.0);
			
			\draw[-] (-0.1,+2.0-0.1)--(+0.1,+2.0+0.1) node[midway, above left] {$A$};
			\draw[-] (-0.0-0.1+\n,-0.1)--(-0.0+0.1+\n,0.1) node[midway, below, outer sep=25pt,align=center] {$t$};
			
			\draw[-] (+2.0-0.1,-0.1)--(+2.0+0.1,0.1) node[midway, below, outer sep=5pt,align=center] {$T$};
			
			\draw[scale=1.0,domain=-4.0:4.0,smooth,samples=200,variable=\x,red,thick] plot ({\x},{2*fun_tg01(\x,\n,2)});
			
			\end{tikzpicture}  
		}
	\end{animateinline}
\end{figure}


%Jak widać dla różnych wartości $t$ otrzymujemy różny kształt funkcji podcałkowej $f(\tau)\cdot g(t-\tau)$.
%
%\paragraph{Przedział 1}.
%
%\begin{figure}[H]
%  \centering
%  \begin{animateinline}[controls,autoplay,loop,poster = 20,palindrome]{10}
%    \multiframe{21}{n=-2.5+0.05}{%Number of Frames, variable = initial + increment
%      \begin{tikzpicture}
%      \tikzmath{
%        function fun1(\x,\n,\T) {
%          if (\x < -\T/2) || (\x < \n- \T) || (\x > \T/2) || (\x > \n + \T) then {
%            return 0.0;
%          } else {
%            if \x < \n then {
%              return \x/\T -(\n-\T)/\T;
%            } else {
%              return -\x/\T - (-\n-\T)/\T;
%            };
%          };
%        };
%      }
%      
%      %\draw (0,0) circle (1in);
%      \draw[->] (-4.5,+0.0) -- (+4.5,+0.0) node[right] {$\tau$};
%      \draw[->] (+0.0,-1.0) -- (+0.0,+2.5) node[above] {$f(\tau) \cdot g(t-\tau)$};
%      
%      \draw[-,red, dotted] (-0.5,+2.0) -- (+0.5,+2.0);
%      \draw[-,red, dotted] (-0.5,+0.0) -- (-0.5,+2.0);
%      \draw[-,red, dotted] (+0.5,+0.0) -- (+0.5,+2.0);
%      
%      \draw[-,green, dotted] (-1.0+\n,+0.0) -- (+0.0+\n,+2.0);
%      \draw[-,green, dotted] (+1.0+\n,+0.0) -- (-0.0+\n,+2.0);
%      \draw[-,green, dashed] (+0.0+\n,+0.0) -- (-0.0+\n,+2.0);
%      
%      \draw[-] (-0.1,+2.0-0.1)--(+0.1,+2.0+0.1) node[midway, above left] {$1$};
%      \draw[-] (-1.0-0.1+\n,-0.1)--(-1.0+0.1+\n,0.1) node[midway, below, outer sep=25pt,align=center] {$t-T$};
%      \draw[-] (+1.0-0.1+\n,-0.1)--(+1.0+0.1+\n,0.1) node[midway, below, outer sep=25pt,align=center] {$t+T$};
%      \draw[-] (+0.0-0.1+\n,-0.1)--(+0.0+0.1+\n,0.1) node[midway, below, outer sep=25pt,align=center] {$t$};
%      
%      \draw[-] (-0.5-0.1,-0.1)--(-0.5+0.1,0.1) node[midway, below, outer sep=5pt,align=center] {$-\frac{T}{2}$};
%      \draw[-] (+0.5-0.1,-0.1)--(+0.5+0.1,0.1) node[midway, below, outer sep=5pt,align=center] {$\frac{T}{2}$};
%      
%      %\draw[-,pink, thick] (-1.0+\n,+0.0) -- (+0.0+\n,+2.0);
%      \draw[scale=1.0,domain=-3.5:3.5,smooth,samples=200,variable=\x,red,thick] plot ({\x},{2*fun1(\x,\n,1)});
%      
%      \end{tikzpicture}  
%    }
%  \end{animateinline}
%\end{figure}
%
%Dla wartości $t$ spełniających warunek $t+T<-\frac{T}{2}$ 
%
%\begin{align*}
%t+T&<-\frac{T}{2}\\
%t&<-\frac{T}{2}-T\\
%t&<-\frac{3}{2}\cdot T\\
%\end{align*}
%
%w wyniku mnożenia otrzymyjemy $0$ a więc wartość splotu jest także równa $0$
%
%\begin{align*}
%h(t)&=\int_{-\infty}^{\infty} 0 \cdot d\tau\\
%&=0
%\end{align*}
%
%\paragraph{Przedział 2}.
%
%\begin{figure}[H]
%  \centering
%  \begin{animateinline}[controls,autoplay,loop,poster = 20,palindrome]{10}
%    \multiframe{21}{n=-1.5+0.05}{%Number of Frames, variable = initial + increment
%      \begin{tikzpicture}
%      \tikzmath{
%        function fun1(\x,\n,\T) {
%          if (\x < -\T/2) || (\x < \n- \T) || (\x > \T/2) || (\x > \n + \T) then {
%            return 0.0;
%          } else {
%            if \x < \n then {
%              return \x/\T -(\n-\T)/\T;
%            } else {
%              return -\x/\T - (-\n-\T)/\T;
%            };
%          };
%        };
%      }
%      
%      %\draw (0,0) circle (1in);
%      \draw[->] (-4.5,+0.0) -- (+4.5,+0.0) node[right] {$\tau$};
%      \draw[->] (+0.0,-1.0) -- (+0.0,+2.5) node[above] {$f(\tau) \cdot g(t-\tau)$};
%      
%      \draw[-,red, dotted] (-0.5,+2.0) -- (+0.5,+2.0);
%      \draw[-,red, dotted] (-0.5,+0.0) -- (-0.5,+2.0);
%      \draw[-,red, dotted] (+0.5,+0.0) -- (+0.5,+2.0);
%      
%      \draw[-,green, dotted] (-1.0+\n,+0.0) -- (+0.0+\n,+2.0);
%      \draw[-,green, dotted] (+1.0+\n,+0.0) -- (-0.0+\n,+2.0);
%      \draw[-,green, dashed] (+0.0+\n,+0.0) -- (-0.0+\n,+2.0);
%      
%      \draw[-] (-0.1,+2.0-0.1)--(+0.1,+2.0+0.1) node[midway, above left] {$1$};
%      \draw[-] (-1.0-0.1+\n,-0.1)--(-1.0+0.1+\n,0.1) node[midway, below, outer sep=25pt,align=center] {$t-T$};
%      \draw[-] (+1.0-0.1+\n,-0.1)--(+1.0+0.1+\n,0.1) node[midway, below, outer sep=25pt,align=center] {$t+T$};
%      \draw[-] (+0.0-0.1+\n,-0.1)--(+0.0+0.1+\n,0.1) node[midway, below, outer sep=25pt,align=center] {$t$};
%      
%      \draw[-] (-0.5-0.1,-0.1)--(-0.5+0.1,0.1) node[midway, below, outer sep=5pt,align=center] {$-\frac{T}{2}$};
%      \draw[-] (+0.5-0.1,-0.1)--(+0.5+0.1,0.1) node[midway, below, outer sep=5pt,align=center] {$\frac{T}{2}$};
%      
%      %\draw[-,pink, thick] (-1.0+\n,+0.0) -- (+0.0+\n,+2.0);
%      \draw[scale=1.0,domain=-3.5:3.5,smooth,samples=200,variable=\x,red,thick] plot ({\x},{2*fun1(\x,\n,1)});
%      
%      \end{tikzpicture}  
%    }
%  \end{animateinline}
%\end{figure}
%
%Dla wartości $t$ spełniających warunki $t+T \geq -\frac{T}{2}$ i $t<-\frac{T}{2}$
%
%\begin{align*}
%t+T& \geq -\frac{T}{2} &\wedge&&  t&<-\frac{T}{2}\\
%t&\geq-\frac{T}{2}-T  &\wedge&&  t&<-\frac{T}{2}\\
%t&\geq-\frac{3}{2}\cdot T &\wedge&&  t&<-\frac{T}{2}\\
%\end{align*}
%
%a więc $t\in \left<-\frac{3}{2}\cdot T, -\frac{T}{2} \right)$
%
%w wyniku mnożenia otrzymujemy prostą zdefiniowaną na odcinku $t \in \left(-\frac{T}{2}, t+T\right)$.
%
%\begin{align*}
%f(\tau) \cdot g(t-\tau)&=\begin{cases}
%0 & \tau \in \left(-\infty; -\frac{T}{2}\right)\\
%-\frac{1}{T}\cdot \tau - \frac{-t-T}{T} & \tau \in \left(-\frac{T}{2}; t+T\right)\\
%0 & \tau \in \left( t+T; \infty \right)\\
%\end{cases}
%\end{align*}
%
%wartość splotu wyznaczamy z ze wzoru
%
%\begin{align*}
%h(t)&=\int_{-\infty}^{\infty} f(\tau) \cdot g(t-\tau) \cdot d\tau\\
%&=\int_{-\infty}^{\frac{T}{2}} 0 \cdot d\tau + \int_{-\frac{T}{2}}^{t+T}\left( -\frac{1}{T}\cdot \tau - \frac{-t-T}{T} \right)\cdot d\tau +\int_{t+T}^{\infty} 0 \cdot d\tau\\
%&=0 - \int_{-\frac{T}{2}}^{t+T} \frac{1}{T}\cdot \tau d\tau - \int_{-\frac{T}{2}}^{t+T} \frac{-t-T}{T} \cdot d\tau +0\\
%&=- \frac{1}{T}\cdot \int_{-\frac{T}{2}}^{t+T} \tau d\tau - \frac{-t-T}{T} \cdot \int_{-\frac{T}{2}}^{t+T} d\tau\\
%&=- \frac{1}{T}\cdot \left( \frac{1}{2} \cdot \tau^2 \right)_{-\frac{T}{2}}^{t+T} - \frac{-t-T}{T} \cdot \left(\tau \right)_{-\frac{T}{2}}^{t+T}\\
%&=- \frac{1}{T}\cdot \frac{1}{2} \cdot \left(  \left(t+T\right)^2 -\left(-\frac{T}{2}\right)^2 \right) - \frac{-t-T}{T} \cdot \left(t+T - \left(-\frac{T}{2}\right) \right)\\
%&=- \frac{1}{T}\cdot \frac{1}{2} \cdot \left(  t^2 -2 \cdot t \cdot T +  T^2 + \frac{1}{4}\cdot T^2 \right) - \frac{-t-T}{T} \cdot \left(t+T + \frac{T}{2} \right)\\
%&=- \frac{1}{2 \cdot T} \cdot \left(  t^2 -2 \cdot t \cdot T +  T^2 + \frac{1}{4}\cdot T^2 \right) + \frac{1}{T} \cdot \left(t+T\right)\cdot \left(t+T + \frac{T}{2} \right)\\
%&=- \frac{1}{2 \cdot T}\cdot \left(  t^2 -2 \cdot t \cdot T +  T^2 + \frac{1}{4}\cdot T^2 \right) + \frac{1}{2\cdot T} \cdot \left(t+T\right)\cdot \left(2 \cdot t+ 2\cdot T + T \right)\\
%&=- \frac{1}{2 \cdot T}\cdot \left(  t^2 -2 \cdot t \cdot T +  T^2 + \frac{1}{4}\cdot T^2 \right) + \frac{1}{2\cdot T} \cdot \left(t+T\right)\cdot \left(2 \cdot t+ 3\cdot T \right)\\
%&=- \frac{1}{2 \cdot T}\cdot \left(  t^2 -2 \cdot t \cdot T +  T^2 + \frac{1}{4}\cdot T^2 \right) + \frac{1}{2\cdot T} \cdot \left(2 \cdot t^2 + 3\cdot t \cdot T + 2 \cdot t \cdot T + 3 \cdot T^2\right)\\
%&= \frac{1}{2\cdot T} \cdot \left( -t^2 +2 \cdot t \cdot T -  T^2 - \frac{1}{4}\cdot T^2  + 2 \cdot t^2 + 3\cdot t \cdot T + 2 \cdot t \cdot T + 3 \cdot T^2\right)\\
%&= \frac{1}{2\cdot T} \cdot \left(t^2 + 7\cdot t \cdot T + 2 \cdot T^2  - \frac{1}{4}\cdot T^2 \right)\\
%&= \frac{1}{2\cdot T} \cdot t^2 + \frac{1}{2\cdot T} \cdot 7\cdot t \cdot T + \frac{1}{2\cdot T} \cdot 2 \cdot T^2  - \frac{1}{2\cdot T} \cdot \frac{1}{4}\cdot T^2\\
%&= \frac{1}{2\cdot T} \cdot t^2 + \frac{7}{2} \cdot t + \frac{7}{8} \cdot T\\
%\end{align*}
%

\end{task}

