\begin{task}
Oblicz, jaka część energi sygnału $f(t)=A \cdot Sa\left(2 \cdot \omega_0 \cdot t\right) \cdot cos^2\left(2 \cdot \omega_0 \cdot t\right)$ przypada na wartości pulsacji 
$\left| \omega \right| < 2 \cdot \omega_0$. Wykorzystaj informację, że transformata sygnału $\Pi(t)$ jest równa $Sa\left(\frac{\omega}{2}\right)$.

\begin{equation}
f(t) = A \cdot Sa\left(2 \cdot \omega_0 \cdot t\right) \cdot cos^2\left(2 \cdot \omega_0 \cdot t\right)
\end{equation}

\begin{equation}
\Pi(t) \xrightarrow{\mathcal F} Sa\left(\frac{\omega}{2}\right)
\end{equation}

\begin{equation}
\frac{E_{\left| \omega \right| < 2 \cdot \omega_0}}{E} = ?
\end{equation}

Ponieważ musimy obliczyć energię tylko dla pewnego zakresu pulsacji, to wykorzystamy twierdzenie Parsevala:

\begin{equation}
\Parseval{F}
\end{equation}

W tym podejściu musimy obliczyć transformatę Fouriera sygnału $f(t)$, czyli $F(\jmath \omega)$.

Ponieważ możemy korzystać tylko ze znanych twierdzeń oraz wiedzy o transformacie sygnału $\Pi(t)$, to spróbujmy przekształcić sygnał $f(t)$ do postaci, w której wprost możemy zastosować twierdzenia. Zauważmy, że:
\begin{align*}
f(t) &= A \cdot Sa\left(2 \cdot \omega_0 \cdot t\right) \cdot cos^2\left(2 \cdot \omega_0 \cdot t\right)=\\
&=\begin{Bmatrix}
\EulerCos
\end{Bmatrix}=\\
&= A \cdot Sa\left(2 \cdot \omega_0 \cdot t\right) \cdot \left(\frac{e^{2 \cdot \jmath \cdot \omega_0 \cdot t}+e^{-2 \cdot \jmath \cdot \omega_0 \cdot t}}{2}\right)^2=\\
&= A \cdot Sa\left(2 \cdot \omega_0 \cdot t\right) \cdot \left(\frac{(e^{2 \cdot \jmath \cdot \omega_0 \cdot t})^2+ 2 \cdot e^{2 \cdot \jmath \cdot \omega_0 \cdot t} \cdot e^{-2 \cdot \jmath \cdot \omega_0 \cdot t}+(e^{-2 \cdot \jmath \cdot \omega_0 \cdot t})^2}{4}\right)=\\
&= A \cdot Sa\left(2 \cdot \omega_0 \cdot t\right) \cdot \left(\frac{e^{4 \cdot \jmath \cdot \omega_0 \cdot t}+ 2 \cdot e^{2 \cdot \jmath \cdot \omega_0 \cdot t -2 \cdot \jmath \cdot \omega_0 \cdot t}+e^{-4 \cdot \jmath \cdot \omega_0 \cdot t}}{4}\right)=\\
&= A \cdot Sa\left(2 \cdot \omega_0 \cdot t\right) \cdot \left(\frac{e^{4 \cdot \jmath \cdot \omega_0 \cdot t}+ 2 \cdot e^{0}+e^{-4 \cdot \jmath \cdot \omega_0 \cdot t}}{4}\right)=\\
&= \frac{A}{4} \cdot Sa\left(2 \cdot \omega_0 \cdot t\right) \cdot e^{4 \cdot \jmath \cdot \omega_0 \cdot t} + \frac{A}{2} \cdot Sa\left(2 \cdot \omega_0 \cdot t\right) +\frac{A}{4} \cdot Sa\left(2 \cdot \omega_0 \cdot t\right) \cdot e^{-4 \cdot \jmath \cdot \omega_0 \cdot t}=\\
&= f_1(t)+f_2(t)+f_3(t)
\end{align*}

Korzystając z liniowości przekształcenia Fouriera możemy niezależnie obliczyć transformaty dla sygnałów $f_1(t)$, $f_2(t)$ i $f_3(t)$, a następnie zsumować te transformaty. Zacznijmy od sygnału $f_2(t)$:

Skoro wiemy, że:

\begin{equation}
g(t) = \Pi(t) \xrightarrow{\mathcal F} Sa\left(\frac{\omega}{2}\right)
\end{equation}

to, na podstawie twierdzenia o symetrii przekształcenia Fouriera:
\begin{align*}
\SymetryTeorem{g}{G}{f_2}{F_2}
\end{align*}
otrzymujemy:
\begin{align*}
Sa\left(\frac{t}{2}\right) &\xrightarrow{\mathcal F} 2\pi \cdot \Pi(-\omega)\\
Sa\left(\frac{t}{2}\right) &\xrightarrow{\mathcal F} 2\pi \cdot \Pi(\omega)
\end{align*}

Teraz musimy przeskalować $Sa\left(\frac{t}{2}\right)$ tak, aby otrzymać $Sa\left(2 \cdot \omega_0 \cdot t\right)$. W tym celu skorzystamy z twierdzenia o zmianie skali podstawiając $\alpha=4 \cdot \omega_0$:
\begin{align*}
\TimeScalingTeorem{f}{F}{f_1}{F_1}
\end{align*}

\begin{align*}
Sa\left(\frac{t}{2}\right) &\xrightarrow{\mathcal F} 2\pi \cdot \Pi(\omega)\\
Sa\left(4 \cdot \omega_0 \cdot \frac{t}{2}\right) &\xrightarrow{\mathcal F} \frac{1}{4 \cdot \omega_0} \cdot 2\pi \cdot \Pi \left(\frac{\omega}{4 \cdot \omega_0}\right)\\
Sa\left(2 \cdot \omega_0 \cdot t\right) &\xrightarrow{\mathcal F} \frac{\pi}{2 \cdot \omega_0} \cdot \Pi \left(\frac{\omega}{4 \cdot \omega_0}\right)\\
f_2(t) = \frac{A}{2} \cdot Sa\left(2 \cdot \omega_0 \cdot t\right) &\xrightarrow{\mathcal F} \frac{A \cdot \pi}{4 \cdot \omega_0} \cdot \Pi \left(\frac{\omega}{4 \cdot \omega_0}\right) = F_2(\jmath \omega)\\
\end{align*}

Podsumowując, transformata sygnału $f_2(t)$ to $F_2(\jmath \omega)= \frac{A \cdot \pi}{4 \cdot \omega_0} \cdot \Pi \left(\frac{\omega}{4 \cdot \omega_0}\right)$.

Zauważmy, że $f_1(t)=\frac{1}{2} \cdot f_2(t) \cdot e^{4 \cdot \jmath \cdot \omega_0 \cdot t}$, czyli $f_1(t)$ to zmodulowany sygnał $f_2(t)$. Stosując twierdzenie o modulacji:
\begin{align*}
\FrequencyShiftTeorem{f_2}{F_2}{f_1}{F_1}
\end{align*}

otrzymujemy:

\begin{align*}
f_2(t) &\xrightarrow{\mathcal F} \frac{A \cdot \pi}{4 \cdot \omega_0} \cdot \Pi \left(\frac{\omega}{4 \cdot \omega_0}\right)\\
f_2(t) \cdot e^{4 \cdot \jmath \cdot \omega_0 \cdot t} &\xrightarrow{\mathcal F} \frac{A \cdot \pi}{4 \cdot \omega_0} \cdot \Pi \left(\frac{\omega - 4 \cdot \omega_0}{4 \cdot \omega_0}\right)\\
f_1(t) = \frac{1}{2} \cdot f_2(t) \cdot e^{4 \cdot \jmath \cdot \omega_0 \cdot t} &\xrightarrow{\mathcal F} \frac{A \cdot \pi}{8 \cdot \omega_0} \cdot \Pi \left(\frac{\omega - 4 \cdot \omega_0}{4 \cdot \omega_0}\right) = F_1(\jmath \omega)\\
\end{align*}

Podsumowując, transformata sygnału $f_1(t)$ to $F_1(\jmath \omega)= \frac{A \cdot \pi}{8 \cdot \omega_0} \cdot \Pi \left(\frac{\omega - 4 \cdot \omega_0}{4 \cdot \omega_0}\right)$.

Podobnie, zauważmy, że $f_3(t)=\frac{1}{2} \cdot f_2(t) \cdot e^{-4 \cdot \jmath \cdot \omega_0 \cdot t}$, czyli $f_3(t)$ to zmodulowany sygnał $f_2(t)$. Stosując twierdzenie o modulacji:
\begin{align*}
\FrequencyShiftTeorem{f_2}{F_2}{f_3}{F_3}
\end{align*}

otrzymujemy:

\begin{align*}
f_2(t) &\xrightarrow{\mathcal F} \frac{A \cdot \pi}{4 \cdot \omega_0} \cdot \Pi \left(\frac{\omega}{4 \cdot \omega_0}\right)\\
f_2(t) \cdot e^{-4 \cdot \jmath \cdot \omega_0 \cdot t} &\xrightarrow{\mathcal F} \frac{A \cdot \pi}{4 \cdot \omega_0} \cdot \Pi \left(\frac{\omega + 4 \cdot \omega_0}{4 \cdot \omega_0}\right)\\
f_3(t) = \frac{1}{2} \cdot f_2(t) \cdot e^{-4 \cdot \jmath \cdot \omega_0 \cdot t} &\xrightarrow{\mathcal F} \frac{A \cdot \pi}{8 \cdot \omega_0} \cdot \Pi \left(\frac{\omega + 4 \cdot \omega_0}{4 \cdot \omega_0}\right) = F_3(\jmath \omega)\\
\end{align*}

Podsumowując, transformata sygnału $f_3(t)$ to $F_3(\jmath \omega)= \frac{A \cdot \pi}{8 \cdot \omega_0} \cdot \Pi \left(\frac{\omega + 4 \cdot \omega_0}{4 \cdot \omega_0}\right)$.

Teraz możemy podać transformatę sygnału $f(t) = f_1(t)+f_2(t)+f_3(t)$,
\begin{align*}
F(\jmath \omega)&=F_1(\jmath \omega)+F_2(\jmath \omega)+F_3(\jmath \omega)=\\
&=\frac{A \cdot \pi}{8 \cdot \omega_0} \cdot \Pi \left(\frac{\omega - 4 \cdot \omega_0}{4 \cdot \omega_0}\right) + \frac{A \cdot \pi}{4 \cdot \omega_0} \cdot \Pi \left(\frac{\omega}{4 \cdot \omega_0}\right) + \frac{A \cdot \pi}{8 \cdot \omega_0} \cdot \Pi \left(\frac{\omega + 4 \cdot \omega_0}{4 \cdot \omega_0}\right)
\end{align*}

Narysujmy widmo amplitudowe sygnału $f(t)$, czyli $\left|F(\jmath \omega)\right|$.

\begin{figure}[H]
	\centering
	\begin{tikzpicture}
		\draw[->] (-6.0,+0.0) -- (+6.0,+0.0) node[right] {$\omega$};
		\draw[->] (+0.0,-1.0) -- (+0.0,+3.5) node[above] {$\left|F_1(\jmath \omega)\right|, \left|F_2(\jmath \omega)\right|, \left|F_3(\jmath \omega)\right|$};
		
		\draw[-,red, thick] (-1.5,+0.0) -- (-1.5,+2.0) -- (1.5,+2.0) -- (1.5,+0.0);
		\draw[-,green, thick] (-4.5,+0.0) -- (-4.5,+1.0) -- (-1.5,+1.0) -- (-1.5,+0.0);
		\draw[-,blue, thick] (1.5,+0.0) -- (1.5,+1.0) -- (4.5,+1.0) -- (4.5,+0.0);
		
		
		\draw[-] (-0.1,+2.0-0.1)--(+0.1,+2.0+0.1) node[midway, above left] {$\frac{A \cdot \pi}{4 \cdot \omega_0}$};
		\draw[-] (-0.1,+1.0-0.1)--(+0.1,+1.0+0.1) node[midway, above left] {$\frac{A \cdot \pi}{8 \cdot \omega_0}$};

		\draw[-] (-4.5-0.1,-0.1)--(-4.5+0.1,0.1) node[midway, below, outer sep=5pt,align=center] {$-6 \cdot \omega_0$};		
		\draw[-] (-3.0-0.1,-0.1)--(-3.0+0.1,0.1) node[midway, below, outer sep=5pt,align=center] {$-4 \cdot \omega_0$};		
		\draw[-] (-1.5-0.1,-0.1)--(-1.5+0.1,0.1) node[midway, below, outer sep=5pt,align=center] {$-2 \cdot \omega_0$};
		\draw[-] (1.5-0.1,-0.1)--(1.5+0.1,0.1) node[midway, below, outer sep=5pt,align=center] {$2 \cdot \omega_0$};
		\draw[-] (3.0-0.1,-0.1)--(3.0+0.1,0.1) node[midway, below, outer sep=5pt,align=center] {$4 \cdot \omega_0$};
		\draw[-] (4.5-0.1,-0.1)--(4.5+0.1,0.1) node[midway, below, outer sep=5pt,align=center] {$4 \cdot \omega_0$};
	
	\end{tikzpicture}  
\end{figure}


\begin{figure}[H]
	\centering
	\begin{tikzpicture}
	\draw[->] (-6.0,+0.0) -- (+6.0,+0.0) node[right] {$\omega$};
	\draw[->] (+0.0,-1.0) -- (+0.0,+3.5) node[above] {$\left|F(\jmath \omega)\right|$};

	\draw[-,red, thick] (-4.5,+0.0) -- (-4.5,+1.0) -- (-1.5,+1.0) -- (-1.5,+2.0) -- (1.5,+2.0) -- (1.5,+1.0) -- (4.5,+1.0) -- (4.5,+0.0);
	
	
	\draw[-] (-0.1,+2.0-0.1)--(+0.1,+2.0+0.1) node[midway, above left] {$\frac{A \cdot \pi}{4 \cdot \omega_0}$};
	\draw[-] (-0.1,+1.0-0.1)--(+0.1,+1.0+0.1) node[midway, above left] {$\frac{A \cdot \pi}{8 \cdot \omega_0}$};
	
	\draw[-] (-4.5-0.1,-0.1)--(-4.5+0.1,0.1) node[midway, below, outer sep=5pt,align=center] {$-6 \cdot \omega_0$};		
	\draw[-] (-3.0-0.1,-0.1)--(-3.0+0.1,0.1) node[midway, below, outer sep=5pt,align=center] {$-4 \cdot \omega_0$};		
	\draw[-] (-1.5-0.1,-0.1)--(-1.5+0.1,0.1) node[midway, below, outer sep=5pt,align=center] {$-2 \cdot \omega_0$};
	\draw[-] (1.5-0.1,-0.1)--(1.5+0.1,0.1) node[midway, below, outer sep=5pt,align=center] {$2 \cdot \omega_0$};
	\draw[-] (3.0-0.1,-0.1)--(3.0+0.1,0.1) node[midway, below, outer sep=5pt,align=center] {$4 \cdot \omega_0$};
	\draw[-] (4.5-0.1,-0.1)--(4.5+0.1,0.1) node[midway, below, outer sep=5pt,align=center] {$4 \cdot \omega_0$};
	
	\end{tikzpicture}  
\end{figure}



\begin{align*}
\left|F(\jmath \omega)\right| &= \begin{cases}
0 & \omega \in \left( -\infty; -6 \cdot \omega_0 \right) \\
\frac{A \cdot \pi}{8 \cdot \omega_0} & \omega \in \left(-6 \cdot \omega_0; -2 \cdot \omega_0 \right) \\
\frac{A \cdot \pi}{4 \cdot \omega_0} & \omega \in \left(-2 \cdot \omega_0; 2 \cdot \omega_0 \right) \\
\frac{A \cdot \pi}{8 \cdot \omega_0} & \omega \in \left(2 \cdot \omega_0; 6 \cdot \omega_0 \right) \\
0 & \omega \in \left(6 \cdot \omega_0; \infty \right) \\
\end{cases}\\
\end{align*}

Ponieważ energię wyznaczamy ze wzoru:
\begin{equation}
\Parseval{F}
\end{equation}

to wyznaczmy $\left|F(\jmath \omega)\right|^2$:
\begin{align*}
\left|F(\jmath \omega)\right|^2 &= \begin{cases}
0 & \omega \in \left( -\infty; -6 \cdot \omega_0 \right) \\
\left(\frac{A \cdot \pi}{8 \cdot \omega_0}\right)^2 & \omega \in \left(-6 \cdot \omega_0; -2 \cdot \omega_0 \right) \\
\left(\frac{A \cdot \pi}{4 \cdot \omega_0}\right)^2 & \omega \in \left(-2 \cdot \omega_0; 2 \cdot \omega_0 \right) \\
\left(\frac{A \cdot \pi}{8 \cdot \omega_0}\right)^2 & \omega \in \left(2 \cdot \omega_0; 6 \cdot \omega_0 \right) \\
0 & \omega \in \left(6 \cdot \omega_0; \infty \right) \\
\end{cases}\\
\end{align*}

\begin{align*}
E &= \frac{1}{2\pi} \cdot \int_{-\infty}^{\infty} \left|F(\jmath \omega)\right|^2 \cdot d\omega=\\
&=\frac{1}{2\pi} \cdot \left(\int_{-\infty}^{-6 \cdot \omega_0} 0 \cdot d\omega + \int_{-6 \cdot \omega_0}^{-2 \cdot \omega_0} \left(\frac{A \cdot \pi}{8 \cdot \omega_0}\right)^2 \cdot d\omega + \int_{-2 \cdot \omega_0}^{2 \cdot \omega_0} \left(\frac{A \cdot \pi}{4 \cdot \omega_0}\right)^2 \cdot d\omega + \int_{2 \cdot \omega_0}^{6 \cdot \omega_0} \left(\frac{A \cdot \pi}{8 \cdot \omega_0}\right)^2 \cdot d\omega + \int_{6 \cdot \omega_0}^{\infty} 0 \cdot d\omega\right)=\\
&=\frac{1}{2\pi} \cdot \left( 0 + \frac{A^2 \cdot \pi^2}{64 \cdot \omega_0^2} \cdot \int_{-6 \cdot \omega_0}^{-2 \cdot \omega_0} d\omega + \frac{A^2 \cdot \pi^2}{16 \cdot \omega_0^2} \cdot \int_{-2 \cdot \omega_0}^{2 \cdot \omega_0} d\omega + \frac{A^2 \cdot \pi^2}{64 \cdot \omega_0^2} \cdot \int_{2 \cdot \omega_0}^{6 \cdot \omega_0} d\omega + 0\right)=\\
&=\frac{1}{2\pi} \cdot \left(\frac{A^2 \cdot \pi^2}{64 \cdot \omega_0^2} \cdot \left.\omega \right|_{-6 \cdot \omega_0}^{-2 \cdot \omega_0} + \frac{A^2 \cdot \pi^2}{16 \cdot \omega_0^2} \cdot \left.\omega \right|_{-2 \cdot \omega_0}^{2 \cdot \omega_0} + \frac{A^2 \cdot \pi^2}{64 \cdot \omega_0^2} \cdot \left.\omega \right|_{2 \cdot \omega_0}^{6 \cdot \omega_0}\right)=\\
&=\frac{1}{2\pi} \cdot \left(\frac{A^2 \cdot \pi^2}{64 \cdot \omega_0^2} \cdot (-2 \cdot \omega_0-(-6 \cdot \omega_0)) + \frac{A^2 \cdot \pi^2}{16 \cdot \omega_0^2} \cdot (2 \cdot \omega_0 -(-2 \cdot \omega_0)) + \frac{A^2 \cdot \pi^2}{64 \cdot \omega_0^2} \cdot (6 \cdot \omega_0 - 2 \cdot \omega_0)\right)=\\
&=\frac{1}{2\pi} \cdot \left(\frac{A^2 \cdot \pi^2}{64 \cdot \omega_0^2} \cdot 4 \cdot \omega_0 + \frac{A^2 \cdot \pi^2}{16 \cdot \omega_0^2} \cdot 4 \cdot \omega_0  + \frac{A^2 \cdot \pi^2}{64 \cdot \omega_0^2} \cdot 4 \cdot \omega_0\right)=\\
&=\frac{1}{2\pi} \cdot \left(\frac{A^2 \cdot \pi^2}{16 \cdot \omega_0} + \frac{A^2 \cdot \pi^2}{4 \cdot \omega_0} + \frac{A^2 \cdot \pi^2}{16 \cdot \omega_0}\right)=\\
&=\frac{1}{2\pi} \cdot \frac{A^2 \cdot \pi^2}{4 \cdot \omega_0} \cdot \left(\frac{1}{4} + 1 + \frac{1}{4}\right)=\\
&=\frac{A^2 \cdot \pi}{8 \cdot \omega_0} \cdot \left(\frac{2}{4} + 1\right)=\\
&=\frac{A^2 \cdot \pi}{8 \cdot \omega_0} \cdot \left(\frac{3}{2}\right)=\\
&=\frac{3 \cdot A^2 \cdot \pi}{16 \cdot \omega_0}
\end{align*}

Energia sygnału $f(t)=A \cdot Sa\left(2 \cdot \omega_0 \cdot t\right) \cdot cos^2\left(2 \cdot \omega_0 \cdot t\right)$ równa się $E=\frac{3 \cdot A^2 \cdot \pi}{16 \cdot \omega_0}$.

Energię sygnału dla pewnego zakresu pulsacji, także można wyznaczyć z twierdzenia Parsevala, ale zmieniając granice w całce zgodnie z oczekiwanym zakresem pulsacji, czyli dla pulsacji $\left| \omega \right| < 2 \cdot \omega_0$ otrzymamy wzór:
\begin{equation}
E_{\left| \omega \right| < 2 \cdot \omega_0} = \frac{1}{2\pi} \cdot \int_{-2 \cdot \omega_0}^{2 \cdot \omega_0} \left|F(\jmath \omega)\right|^2 \cdot d\omega
\end{equation}

Podstawiając dane dla naszego sygnału otrzymamy:
\begin{align*}
E_{\left| \omega \right| < 2 \cdot \omega_0} &= \frac{1}{2\pi} \cdot \int_{-2 \cdot \omega_0}^{2 \cdot \omega_0} \left|F(\jmath \omega)\right|^2 \cdot d\omega=\\
&= \frac{1}{2\pi} \cdot \int_{-2 \cdot \omega_0}^{2 \cdot \omega_0} \left|\frac{A \cdot \pi}{4 \cdot \omega_0}\right|^2 \cdot d\omega=\\
&= \frac{1}{2\pi} \cdot \int_{-2 \cdot \omega_0}^{2 \cdot \omega_0} \left(\frac{A \cdot \pi}{4 \cdot \omega_0}\right)^2 \cdot d\omega=\\
&= \frac{1}{2\pi} \cdot \left(\frac{A \cdot \pi}{4 \cdot \omega_0}\right)^2 \cdot \int_{-2 \cdot \omega_0}^{2 \cdot \omega_0} d\omega=\\
&= \frac{1}{2\pi} \cdot \left(\frac{A^2 \cdot \pi^2}{16 \cdot \omega_0^2}\right) \cdot \left.\omega\right|_{-2 \cdot \omega_0}^{2 \cdot \omega_0}=\\
&= \frac{A^2 \cdot \pi}{32 \cdot \omega_0^2} \cdot (2 \cdot \omega_0 - (-2 \cdot \omega_0))=\\
&= \frac{A^2 \cdot \pi}{32 \cdot \omega_0^2} \cdot (4 \cdot \omega_0)=\\
&= \frac{A^2 \cdot \pi}{8 \cdot \omega_0}
\end{align*}

Podsumowując $ E_{\left| \omega \right| < \omega_0} = \frac{A^2 \cdot \pi}{8 \cdot \omega_0}$.

Teraz możemy obliczyć:
\begin{equation}
\frac{E_{\left| \omega \right| < 2 \cdot \omega_0}}{E} = ?
\end{equation}

Podstawiając nasze wczesniejsze wyniki otrzymujemy:
\begin{align*}
\frac{E_{\left| \omega \right| < 2 \cdot \omega_0}}{E} = \frac{\frac{A^2 \cdot \pi}{8 \cdot \omega_0}}{\frac{3 \cdot A^2 \cdot \pi}{16 \cdot \omega_0}} =\frac{A^2 \cdot \pi}{8 \cdot \omega_0} \cdot \frac{16 \cdot \omega_0}{3 \cdot A^2 \cdot \pi} =\frac{2}{3} \approx 66\%
\end{align*}

Na pulsacje z zakresu $\left| \omega \right| < 2 \cdot \omega_0$ przypada około $66\%$ energii sygnału.

\end{task}

