\begin{task}
Oblicz wartość energii sygnału $f(t)=A\cdot sin^2\left(\omega_0 \cdot t\right)$ okresowego przedstawionego na rysunku 

\begin{figure}[H]
\centering
\begin{tikzpicture}
  %\draw (0,0) circle (1in);
  \draw[->] (-3.0,+0.0) -- (+5.0,+0.0) node[right] {$t$};
  \draw[->] (+0.0,-2.0) -- (+0.0,+2.0) node[above] {$f(t)$};

  \draw[-] (-0.1,+1.5-0.1)--(+0.1,+1.5+0.1) node[midway, left] {$A$};
  \draw[-] (-0.1,-1.5-0.1)--(+0.1,-1.5+0.1) node[midway, left] {$-A$};
  
  \draw[-] (-0.1+2.5,-0.1)--(+0.1+2.5,+0.1) node[midway, below] {$\frac{\pi}{\omega_0}$};
  \draw[-] (-0.1+1.25,-0.1)--(+0.1+1.25,+0.1) node[midway, below] {$\frac{\pi}{2\cdot\omega_0}$};
  \draw[-] (-0.1-1.25,-0.1)--(+0.1-1.25,+0.1) node[midway, below] {$-\frac{\pi}{2\cdot\omega_0}$};
  
  \draw[scale=1.0,domain=-4.0:4.0,samples=100,smooth,variable=\x,red,thick] plot ({\x},{1.5*sin(\x*180.0/3.141592*2*3.141592/5.0)^2});
\end{tikzpicture}
\end{figure}

Energię sygnału okresowego wyznaczamy ze wzoru
\begin{equation}
E=\int_{T}\left|f(t)\right|^2 \cdot dt
\end{equation}

Podstawiamy do wzoru na enargie wzór naszej funkcji dla pierwszego okresu $k=0$
%\begin{equation}
%\begin{aligned}
\begin{align*}
E&=\int_{T}\left|f(t)\right|^2 \cdot dt \\
&=\int_{0}^{T}\left| A\cdot sin^2\left(\omega_0 \cdot t\right) \right|^2 \cdot dt \\
&=\int_{0}^{T} A^2 \cdot sin^4\left(\omega_0 \cdot t\right) \cdot dt \\
&=\left\{ sin(x)=\frac{e^{\jmath \cdot x} - e^{-\jmath \cdot x}}{2 \cdot \jmath}\right\}\\
&=\int_{0}^{T} A^2 \cdot \left( \frac{e^{\jmath \cdot \omega_0 \cdot t} - e^{-\jmath \cdot \omega_0 \cdot t}}{2 \cdot \jmath} \right)^4 \cdot dt \\
&=\int_{0}^{T} A^2 \cdot \frac{\left(e^{\jmath \cdot \omega_0 \cdot t} - e^{-\jmath \cdot \omega_0 \cdot t}\right)^4}{\left(2 \cdot \jmath\right)^4} \cdot dt \\
&=\left\{\begin{matrix}
n=0: &   &   &   &   & 1 &   &   &   &   \\
n=1: &   &   &   & 1 &   & 1 &   &   &   \\
n=2: &   &   & 1 &   & 2 &   & 1 &   &   \\
n=3: &   & 1 &   & 3 &   & 3 &   & 1 &   \\
n=4: & 1 &   & 4 &   & 6 &   & 4 &   & 1 
\end{matrix}\right\}\\
&=\int_{0}^{T} A^2 \cdot \frac{
  1 \cdot \left(e^{\jmath \cdot \omega_0 \cdot t}\right)^4 \cdot \left(-e^{-\jmath \cdot \omega_0 \cdot t}\right)^0 
  + 4 \cdot \left(e^{\jmath \cdot \omega_0 \cdot t}\right)^3 \cdot \left(-e^{-\jmath \cdot \omega_0 \cdot t}\right)^1 
  + 6 \cdot \left(e^{\jmath \cdot \omega_0 \cdot t}\right)^2 \cdot \left(-e^{-\jmath \cdot \omega_0 \cdot t}\right)^2 
  + 4 \cdot \left(e^{\jmath \cdot \omega_0 \cdot t}\right)^1 \cdot \left(-e^{-\jmath \cdot \omega_0 \cdot t}\right)^3 
  + 1 \cdot \left(e^{\jmath \cdot \omega_0 \cdot t}\right)^0 \cdot \left(-e^{-\jmath \cdot \omega_0 \cdot t}\right)^4 
}{\left(2 \cdot \jmath\right)^4} \cdot dt \\
&=\int_{0}^{T} A^2 \cdot \frac{
  e^{4 \cdot \jmath \cdot \omega_0 \cdot t} \cdot e^{-0 \cdot \jmath \cdot \omega_0 \cdot t}
  - 4 \cdot e^{3 \cdot \jmath \cdot \omega_0 \cdot t} \cdot e^{-\jmath \cdot \omega_0 \cdot t} 
  + 6 \cdot e^{2 \cdot \jmath \cdot \omega_0 \cdot t} \cdot e^{-2 \cdot \jmath \cdot \omega_0 \cdot t} 
  - 4 \cdot e^{\jmath \cdot \omega_0 \cdot t} \cdot e^{-3 \cdot \jmath \cdot \omega_0 \cdot t} 
  + e^{0 \cdot \jmath \cdot \omega_0 \cdot t} \cdot e^{-4 \cdot \jmath \cdot \omega_0 \cdot t} 
}{2 ^ 4\cdot \jmath^4} \cdot dt \\
&=\int_{0}^{T} A^2 \cdot \frac{
  e^{4 \cdot \jmath \cdot \omega_0 \cdot t - 0 \cdot \jmath \cdot \omega_0 \cdot t}
  - 4 \cdot e^{3 \cdot \jmath \cdot \omega_0 \cdot t -\jmath \cdot \omega_0 \cdot t} 
  + 6 \cdot e^{2 \cdot \jmath \cdot \omega_0 \cdot t -2 \cdot \jmath \cdot \omega_0 \cdot t} 
  - 4 \cdot e^{\jmath \cdot \omega_0 \cdot t -3 \cdot \jmath \cdot \omega_0 \cdot t} 
  + e^{0 \cdot \jmath \cdot \omega_0 \cdot t -4 \cdot \jmath \cdot \omega_0 \cdot t} 
}{16\cdot 1} \cdot dt \\
&=\int_{0}^{T} A^2 \cdot \frac{
  e^{4 \cdot \jmath \cdot \omega_0 \cdot t}
  - 4 \cdot e^{2 \cdot \jmath \cdot \omega_0 \cdot t} 
  + 6 \cdot e^{0 \cdot \jmath \cdot \omega_0 \cdot t} 
  - 4 \cdot e^{-2 \cdot \jmath \cdot \omega_0 \cdot t} 
  + e^{-4 \cdot \jmath \cdot \omega_0 \cdot t} 
}{16} \cdot dt \\
&=\int_{0}^{T} A^2 \cdot \frac{
  e^{4 \cdot \jmath \cdot \omega_0 \cdot t}
  + e^{-4 \cdot \jmath \cdot \omega_0 \cdot t}
  - 4 \cdot e^{2 \cdot \jmath \cdot \omega_0 \cdot t} 
  - 4 \cdot e^{-2 \cdot \jmath \cdot \omega_0 \cdot t} 
  + 6 \cdot e^{0} 
}{16} \cdot dt \\
&=\int_{0}^{T} A^2 \cdot \frac{
  e^{4 \cdot \jmath \cdot \omega_0 \cdot t}
  + e^{-4 \cdot \jmath \cdot \omega_0 \cdot t} 
  - 4 \cdot e^{2 \cdot \jmath \cdot \omega_0 \cdot t} 
  - 4 \cdot e^{-2 \cdot \jmath \cdot \omega_0 \cdot t} 
  + 6
}{16} \cdot dt \\
&=\frac{A^2}{16} \cdot \int_{0}^{T} \left(
  e^{4 \cdot \jmath \cdot \omega_0 \cdot t}
  + e^{-4 \cdot \jmath \cdot \omega_0 \cdot t} 
  - 4 \cdot e^{2 \cdot \jmath \cdot \omega_0 \cdot t} 
  - 4 \cdot e^{-2 \cdot \jmath \cdot \omega_0 \cdot t} 
  + 6
\right) dt \\
&=\frac{A^2}{16} \cdot \left( \int_{0}^{T} 
e^{4 \cdot \jmath \cdot \omega_0 \cdot t} \cdot dt + \int_{0}^{T} 
 e^{-4 \cdot \jmath \cdot \omega_0 \cdot t} \cdot dt
- 4 \cdot \int_{0}^{T} e^{2 \cdot \jmath \cdot \omega_0 \cdot t} \cdot dt
- 4 \cdot \int_{0}^{T} e^{-2 \cdot \jmath \cdot \omega_0 \cdot t} \cdot dt
+ 6 \cdot \int_{0}^{T} dt \right)\\
&=\begin{Bmatrix}
z_1 = 4 \cdot \jmath \cdot \omega_0 \cdot t & z_2 = -4 \cdot \jmath \cdot \omega_0 \cdot t & z_3 = 2 \cdot \jmath \cdot \omega_0 \cdot t & z_4 = -2 \cdot \jmath \cdot \omega_0 \cdot t \\
dz_1 = 4 \cdot \jmath \cdot \omega_0 \cdot dt & dz_2 = -4 \cdot \jmath \cdot \omega_0 \cdot dt & dz_3 = 2 \cdot \jmath \cdot \omega_0 \cdot dt & dz_4 = -2 \cdot \jmath \cdot \omega_0 \cdot dt \\
dt = \frac{1}{4 \cdot \jmath \cdot \omega_0}\cdot dz_1 & dt = \frac{1}{-4 \cdot \jmath \cdot \omega_0}\cdot dz_2 & dt = \frac{1}{2 \cdot \jmath \cdot \omega_0}\cdot dz_3 & dt = \frac{1}{-2 \cdot \jmath \cdot \omega_0}\cdot dz_4
\end{Bmatrix}\\
&=\frac{A^2}{16} \cdot \left( \int_{0}^{T} 
e^{z_1} \cdot \frac{1}{4 \cdot \jmath \cdot \omega_0}\cdot dz_1 
+ \int_{0}^{T} e^{z_2} \cdot \frac{1}{-4 \cdot \jmath \cdot \omega_0}\cdot dz_2
- 4 \cdot \int_{0}^{T} e^{z_3} \cdot \frac{1}{2 \cdot \jmath \cdot \omega_0}\cdot dz_3
- 4 \cdot \int_{0}^{T} e^{z_4} \cdot \frac{1}{-2 \cdot \jmath \cdot \omega_0}\cdot dz_4
+ 6 \cdot \int_{0}^{T} dt \right)\\
&=\frac{A^2}{16} \cdot \left( \frac{1}{4 \cdot \jmath \cdot \omega_0}\cdot \int_{0}^{T} 
e^{z_1} \cdot  dz_1 
+ \frac{1}{-4 \cdot \jmath \cdot \omega_0}\cdot \int_{0}^{T} e^{z_2} \cdot dz_2
- 4 \cdot \frac{1}{2 \cdot \jmath \cdot \omega_0}\cdot \int_{0}^{T} e^{z_3} \cdot  dz_3
- 4 \cdot \frac{1}{-2 \cdot \jmath \cdot \omega_0}\cdot \int_{0}^{T} e^{z_4} \cdot dz_4
+ 6 \cdot \int_{0}^{T} dt \right)\\
&=\frac{A^2}{16} \cdot \left( \frac{1}{4 \cdot \jmath \cdot \omega_0}\cdot \left. 
e^{z_1} \right|_{0}^{T} 
- \frac{1}{4 \cdot \jmath \cdot \omega_0}\cdot \left. e^{z_2} \right|_{0}^{T}
- \frac{4}{2 \cdot \jmath \cdot \omega_0}\cdot \left. e^{z_3} \right|_{0}^{T}
+ \frac{4}{2 \cdot \jmath \cdot \omega_0}\cdot \left. e^{z_4} \right|_{0}^{T}
+ 6 \cdot \left. t \right|_{0}^{T}\right)\\
&=\frac{A^2}{16} \cdot \left( \frac{1}{4 \cdot \jmath \cdot \omega_0}\cdot \left. 
e^{4 \cdot \jmath \cdot \omega_0 \cdot t} \right|_{0}^{T} 
- \frac{1}{4 \cdot \jmath \cdot \omega_0}\cdot \left. e^{-4 \cdot \jmath \cdot \omega_0 \cdot t} \right|_{0}^{T}
- \frac{4}{2 \cdot \jmath \cdot \omega_0}\cdot \left. e^{2 \cdot \jmath \cdot \omega_0 \cdot t} \right|_{0}^{T}
+ \frac{4}{2 \cdot \jmath \cdot \omega_0}\cdot \left. e^{-2 \cdot \jmath \cdot \omega_0 \cdot t} \right|_{0}^{T}
+ 6 \cdot \left. t \right|_{0}^{T}\right)\\
&=\frac{A^2}{16} \cdot \left( \frac{1}{4 \cdot \jmath \cdot \omega_0}\cdot \left( 
e^{4 \cdot \jmath \cdot \omega_0 \cdot T} -  
e^{4 \cdot \jmath \cdot \omega_0 \cdot 0} \right)
- \frac{1}{4 \cdot \jmath \cdot \omega_0}\cdot \left( e^{-4 \cdot \jmath \cdot \omega_0 \cdot T} - e^{-4 \cdot \jmath \cdot \omega_0 \cdot 0} \right)
- \frac{4}{2 \cdot \jmath \cdot \omega_0}\cdot \left( e^{2 \cdot \jmath \cdot \omega_0 \cdot T} -e^{2 \cdot \jmath \cdot \omega_0 \cdot 0}\right)
+ \frac{4}{2 \cdot \jmath \cdot \omega_0}\cdot \left( e^{-2 \cdot \jmath \cdot \omega_0 \cdot T} -e^{-2 \cdot \jmath \cdot \omega_0 \cdot 0}\right)
+ 6 \cdot \left(T - 0 \right)\right)\\
&=\frac{A^2}{16} \cdot \left( \frac{1}{4 \cdot \jmath \cdot \omega_0}\cdot \left( 
e^{4 \cdot \jmath \cdot \omega_0 \cdot T} -  
e^{0} \right)
- \frac{1}{4 \cdot \jmath \cdot \omega_0}\cdot \left( e^{-4 \cdot \jmath \cdot \omega_0 \cdot T} - e^{0} \right)
- \frac{4}{2 \cdot \jmath \cdot \omega_0}\cdot \left( e^{2 \cdot \jmath \cdot \omega_0 \cdot T} -e^{0}\right)
+ \frac{4}{2 \cdot \jmath \cdot \omega_0}\cdot \left( e^{-2 \cdot \jmath \cdot \omega_0 \cdot T} -e^{0}\right)
+ 6 \cdot T \right)\\
&=\frac{A^2}{16} \cdot \left( \frac{1}{4 \cdot \jmath \cdot \omega_0}\cdot \left( 
e^{4 \cdot \jmath \cdot \omega_0 \cdot T} - 1 \right)
- \frac{1}{4 \cdot \jmath \cdot \omega_0}\cdot \left( e^{-4 \cdot \jmath \cdot \omega_0 \cdot T} - 1 \right)
- \frac{4}{2 \cdot \jmath \cdot \omega_0}\cdot \left( e^{2 \cdot \jmath \cdot \omega_0 \cdot T} -1\right)
+ \frac{4}{2 \cdot \jmath \cdot \omega_0}\cdot \left( e^{-2 \cdot \jmath \cdot \omega_0 \cdot T} -1\right)
+ 6 \cdot T \right)\\
&=\frac{A^2}{16} \cdot \left( 
\frac{e^{4 \cdot \jmath \cdot \omega_0 \cdot T}}{4 \cdot \jmath \cdot \omega_0} - \frac{1}{4 \cdot \jmath \cdot \omega_0} 
- \frac{e^{-4 \cdot \jmath \cdot \omega_0 \cdot T}}{4 \cdot \jmath \cdot \omega_0}  + \frac{1}{4 \cdot \jmath \cdot \omega_0} 
- \frac{4 \cdot e^{2 \cdot \jmath \cdot \omega_0 \cdot T}}{2 \cdot \jmath \cdot \omega_0} + \frac{4}{2 \cdot \jmath \cdot \omega_0}
+ \frac{4 \cdot e^{-2 \cdot \jmath \cdot \omega_0 \cdot T}}{2 \cdot \jmath \cdot \omega_0} -\frac{4}{2 \cdot \jmath \cdot \omega_0}
+ 6 \cdot T \right)\\
&=\frac{A^2}{16} \cdot \left( 
\frac{e^{4 \cdot \jmath \cdot \omega_0 \cdot T}}{4 \cdot \jmath \cdot \omega_0} 
- \frac{e^{-4 \cdot \jmath \cdot \omega_0 \cdot T}}{4 \cdot \jmath \cdot \omega_0}  
- \frac{4 \cdot e^{2 \cdot \jmath \cdot \omega_0 \cdot T}}{2 \cdot \jmath \cdot \omega_0}
+ \frac{4 \cdot e^{-2 \cdot \jmath \cdot \omega_0 \cdot T}}{2 \cdot \jmath \cdot \omega_0}
+ 6 \cdot T \right)\\
&=\frac{A^2}{16} \cdot \left( 
\frac{e^{4 \cdot \jmath \cdot \omega_0 \cdot T} - e^{-4 \cdot \jmath \cdot \omega_0 \cdot T}}{4 \cdot \jmath \cdot \omega_0} 
- \frac{4 \cdot e^{2 \cdot \jmath \cdot \omega_0 \cdot T} - 4 \cdot e^{-2 \cdot \jmath \cdot \omega_0 \cdot T}}{2 \cdot \jmath \cdot \omega_0}
+ 6 \cdot T \right)\\
&=\frac{A^2}{16} \cdot \left( \frac{1}{2 \cdot \omega_0} \cdot
\frac{e^{4 \cdot \jmath \cdot \omega_0 \cdot T} - e^{-4 \cdot \jmath \cdot \omega_0 \cdot T}}{2 \cdot \jmath} 
- \frac{4}{\omega_0} \cdot \frac{e^{2 \cdot \jmath \cdot \omega_0 \cdot T} - e^{-2 \cdot \jmath \cdot \omega_0 \cdot T}}{2 \cdot \jmath}
+ 6 \cdot T \right)\\
&=\begin{Bmatrix}
sin(x)=\frac{e^{\jmath \cdot x} - e^{-\jmath \ cdot x}}{2\cdot \jmath}
\end{Bmatrix}\\
&=\frac{A^2}{16} \cdot \left( \frac{1}{2 \cdot \omega_0} \cdot
sin\left(4 \cdot \omega_0 \cdot T\right)
- \frac{4}{\omega_0} \cdot sin\left(2 \cdot \omega_0 \cdot T\right)
+ 6 \cdot T \right)\\
&=\begin{Bmatrix}
T=\frac{2\pi}{\omega_0}
\end{Bmatrix}\\
&=\frac{A^2}{16} \cdot \left( \frac{1}{2 \cdot \omega_0} \cdot
sin\left(4 \cdot \omega_0 \cdot \frac{2\pi}{\omega_0}\right)
- \frac{4}{\omega_0} \cdot sin\left(2 \cdot \omega_0 \cdot \frac{2\pi}{\omega_0}\right)
+ 6 \cdot \frac{2\pi}{\omega_0} \right)\\
&=\frac{A^2}{16} \cdot \left( \frac{1}{2 \cdot \omega_0} \cdot
sin\left(8 \pi\right)
- \frac{4}{\omega_0} \cdot sin\left(4\pi\right)
+ 6 \cdot \frac{2\pi}{\omega_0} \right)\\
&=\frac{A^2}{16} \cdot \left( \frac{1}{2 \cdot \omega_0} \cdot 0
- \frac{4}{\omega_0} \cdot 0 + 6 \cdot \frac{2\pi}{\omega_0} \right)\\
&=\frac{A^2}{16} \cdot \frac{12\pi}{\omega_0}\\
&=\frac{A^2}{4} \cdot \frac{3\pi}{\omega_0}\\
\end{align*}
%\end{aligned}
%\end{equation}

Energia sygnału wynosi $\frac{A^2}{4} \cdot \frac{3\pi}{\omega_0}$
\end{task}