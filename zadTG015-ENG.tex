\begin{task}
\TT{Oblicz moc sygnału okresowego $f(t)$ przedstawionego na rysunku:}{Compute the average power for the following periodic signal $f(t)$:}

\begin{figure}[H]
\centering
\begin{tikzpicture}
  %\draw (0,0) circle (1in);
  \draw[->] (-4.0,+0.0) -- (+4.0,+0.0) node[right] {$t$};
  \draw[->] (+0.0,-1.5) -- (+0.0,+1.5) node[above] {$f(t)$};

  \draw[scale=1.0,domain=-3.0:-1.0,samples=200,smooth,variable=\x,red,thick] plot ({\x},{1-((\x+2.0)*(\x+2.0))});
  \draw[scale=1.0,domain=-1.0:1.0,samples=200,smooth,variable=\x,red,thick] plot ({\x},{1-(\x*\x)});
  \draw[scale=1.0,domain=1.0:3.0,samples=200,smooth,variable=\x,red,thick] plot ({\x},{1-((\x-2.0)*(\x-2.0))});
  
  \draw[-] (-3.0-0.1,-0.1)--(-3.0+0.1,0.1) node[midway, below, outer sep=5pt,align=center] {$-3$};
  \draw[-] (-1.0-0.1,-0.1)--(-1.0+0.1,0.1) node[midway, below, outer sep=5pt,align=center] {$-1$};
  \draw[-] (+1.0-0.1,-0.1)--(+1.0+0.1,0.1) node[midway, below, outer sep=5pt] {$1$};
  \draw[-] (+3.0-0.1,-0.1)--(+3.0+0.1,0.1) node[midway, below, outer sep=5pt] {$3$};
  \draw[-] (-0.1,+1.0-0.1)--(+0.1,+1.0+0.1) node[midway, above right] {$1$};
\end{tikzpicture}
\end{figure}

\TT{Dla zakresu $t \in \left( -1; 1 \right )$ sygnał opisany jest wzorem:}{Signal in the range $t \in \left( -1; 1 \right )$ is described as:}
\begin{equation}
f(t)=1-t^2
\end{equation}

\TT{Moc sygnału okresowego wyznaczamy ze wzoru:}{The average power for periodic signals is defined by:}

\begin{equation}
	P=\frac{1}{T} \cdot \int_{T}^{}\left|f(t)\right|^2 \cdot dt
\end{equation}

\TT{Dla tego sygnału okres $T$ równa się $2$.}{In this case period $T$ is equal to $2$.}
 
\begin{align*}
	P&=\frac{1}{T} \cdot \int_{T}\left|f(t)\right|^2 \cdot dt =\\
	&=\frac{1}{2} \cdot  \int_{-1}^{1}\left| 1-t^2 \right|^2 \cdot dt =\\
    &=\frac{1}{2} \cdot  \int_{-1}^{1}\left(1-t^2\right)^2 \cdot dt =\\
    &=\frac{1}{2} \cdot  \int_{-1}^{1}\left(1-2 \cdot t^2 + t^4\right) \cdot dt =\\
    &=\frac{1}{2} \cdot \left[ \int_{-1}^{1} 1 \cdot dt +\int_{-1}^{1} (-2) \cdot t^2 \cdot dt +\int_{-1}^{1} t^4 \cdot dt\right]=\\
    &=\frac{1}{2} \cdot \left[ \left. t \right|_{-1}^{1} -2 \cdot \left. \frac{t^3}{3}\right|_{-1}^{1} +\left. \frac{t^5}{5}\right|_{-1}^{1}\right]=\\
    &=\frac{1}{2} \cdot \left[ (1-(-1)) -\frac{2}{3} \cdot (1-(-1)) +\frac{1}{5} \cdot (1-(-1))\right]=\\
    &=\frac{1}{2} \cdot \left[ 2 -\frac{4}{3} +\frac{2}{5}\right]=\\
    &=\frac{1}{2} \cdot \left[ \frac{30}{15} -\frac{20}{15} +\frac{6}{15}\right]=\\
    &=\frac{1}{2} \cdot \frac{16}{15}=\\
    &=\frac{8}{15}\\
\end{align*}

\TT{Moc sygnału równa się $\frac{8}{15}$.}{The average power equals to $\frac{8}{15}$.}
\end{task}
