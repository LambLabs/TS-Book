\begin{task}

\TT{Oblicz transformatę Fouriera sygnału $f(t)$ przedstawionego na rysunku wykorzystując twierdzenia opisujące własciwości transformacji Fouriera. Wykorzystaj informację o tym, że $\mathcal F\{\Pi(t)\}=Sa\left(\frac{\omega}{2}\right)$.}{Compute the Fourier transform of the $f(t)$ signal given below using theorems describing the properties of the Fourier transformation. Exploit the following transform $\mathcal F\{\Pi(t)\}=Sa\left(\frac{\omega}{2}\right)$.}

\begin{figure}[H]
\centering
\begin{tikzpicture}
  %\draw (0,0) circle (1in);
  \draw[->] (-4.0,+0.0) -- (+4.0,+0.0) node[right] {$t$};
  \draw[->] (+0.0,-1.5) -- (+0.0,+1.5) node[above] {$f(t)$};
  \draw[-,red, thick] (-3.5,+0.0) -- (-1.0,+0.0);
  \draw[scale=1.0,domain=-1.0:1.0,samples=200,smooth,variable=\x,red,thick] plot ({\x},{1-(\x*\x)});
  \draw[-,red, thick] (+1.0,+0.0) -- (+3.0,0.0);
  
  \draw[-] (-1.0-0.1,-0.1)--(-1.0+0.1,0.1) node[midway, below, outer sep=5pt,align=center] {$-1$};
  \draw[-] (+1.0-0.1,-0.1)--(+1.0+0.1,0.1) node[midway, below, outer sep=5pt] {$1$};
  \draw[-] (-0.1,+1.0-0.1)--(+0.1,+1.0+0.1) node[midway, above right] {$1$};
\end{tikzpicture}
\end{figure}

\TT{Sygnał $f(t)$ możemy opisać jako:}{The $f(t)$ signal, as a piecewise function, may be described as:}

\begin{equation}
f(t)=\begin{cases}
0 & t \in \left( -\infty; -1 \right ) \\
1-t^2\ & t \in \left( -1; 1 \right ) \\
0 & t \in \left( 1; \infty \right )
\end{cases} 
\end{equation}

\TT{W pierwszej kolejności wyznaczamy pochodną sygnału $f(t)$}{First of all, let's calculate the $f(t)$ signal derivative:}

\begin{equation}
g(t)=f'(t)=\begin{cases}
0 & t \in \left( -\infty; -1 \right ) \\
-2 \cdot t\ & t \in \left( -1; 1 \right ) \\
0 & t \in \left( 1; \infty \right )
\end{cases} 
\end{equation}

\begin{figure}[H]
	\centering
	\begin{tikzpicture}
	%\draw (0,0) circle (1in);
	\draw[->] (-4.0,+0.0) -- (+4.0,+0.0) node[right] {$t$};
	\draw[->] (+0.0,-2.5) -- (+0.0,+2.5) node[above] {$g(t)$};
	\draw[-,red, thick] (-3.5,+0.0) -- (-1.0,+0.0);
	\draw[-,red, thick] (+1.0,+0.0) -- (+3.5,+0.0);
	\draw[-,red, thick] (-1.0,+0.0) -- (-1.0,+2.0);
	\draw[-,red, thick] (-1.0,+2.0) -- (+1.0,-2.0);
	\draw[-,red, thick] (+1.0,+0.0) -- (+1.0,-2.0);
	
	\draw[-] (-1.0-0.1,-0.1)--(-1.0+0.1,0.1) node[midway, below, outer sep=5pt,align=center] {$-1$};
	\draw[-] (+1.0-0.1,-0.1)--(+1.0+0.1,0.1) node[midway, above, outer sep=5pt] {$1$};
	\draw[-] (-0.1,+2.0-0.1)--(+0.1,+2.0+0.1) node[midway, left] {$2$};
	\draw[-] (-0.1,-2.0-0.1)--(+0.1,-2.0+0.1) node[midway, left] {$-2$};
	\end{tikzpicture}
\end{figure}

\TT{Można sprawdzić, że całkując sygnał $g(t)$ otrzymamy sygnał $f(t)$, czyli:}{You can check that by integrating the $g(t)$ signal we'll get the $f(t)$ signal:}

\begin{equation}
f(t) = \int_{-\infty}^{t} g(x) \cdot dx
\end{equation}

\TT{Skoro tak jest, to transformatę sygnału $f(t)$ mozna wyznaczyć z twierdzenia o całkowaniu sygnału, w tym przypadku całkować będziemy sygnał $g(t)$:}{Therefore, the Fourier transform of the $f(t)$ signal can be determined from the integration theorem. In this case we will integrate the $g(t)$ signal:}

\begin{equation}
F(\jmath \omega) = \frac{1}{\jmath \cdot \omega} \cdot G(\jmath \omega) + \pi \cdot \delta(\omega) \cdot G(0)
\end{equation}

\TT{Pytanie, czy można dalej uprościc sygnał $g(t)$ dokonując jego rózniczkowania. Wyznaczmy pochodną sygnału $g(t)$, czyli drugą pochodną sygnału $f(t)$:}{Is it possible to calculate derivative again to simplify the analysed signal? Let's calculate the derivative of the $g(t)$ signal, so the second derivative of the $f(t)$ signal:}

\begin{equation}
h(t)= \frac{\partial}{\partial t}g(t) = \frac{\partial^2}{\partial t^2}f(t) 
\end{equation}

\begin{equation}
h(t)=g'(t)=\begin{cases}
0 & t \in \left( -\infty; -1 \right ) \\
-2& t \in \left( -1; 1 \right ) \\
0 & t \in \left( 1; \infty \right )
\end{cases} + 2 \cdot \delta(t+1) + 2 \cdot \delta(t-1)
\end{equation}

\begin{figure}[H]
  \centering
  \begin{tikzpicture}
  %\draw (0,0) circle (1in);
  \draw[->] (-4.0,+0.0) -- (+4.0,+0.0) node[right] {$t$};
  \draw[->] (+0.0,-2.5) -- (+0.0,+2.5) node[above] {$h(t)$};
  \draw[-,red, thick] (-3.5,+0.0) -- (-1.0,+0.0);
  \draw[-,red, thick] (-1.0,+0.0) -- (-1.0,-2.0);
  \draw[-,red, thick] (-1.0,-2.0) -- (+1.0,-2.0);
  \draw[-,red, thick] (+1.0,-2.0) -- (+1.0,+0.0);
  \draw[-,red, thick] (+1.0,+0.0) -- (+3.5,+0.0);
  \draw[->,red, thick] (+1.0,+0.0) -- (+1.0,+2.0);
  \draw[->,red, thick] (-1.0,+0.0) -- (-1.0,+2.0);
  
  \draw[-] (-1.0-0.1,-0.1)--(-1.0+0.1,0.1) node[midway, below left, outer sep=5pt,align=center] {$-1$};
  \draw[-] (+1.0-0.1,-0.1)--(+1.0+0.1,0.1) node[midway, below right, outer sep=5pt] {$1$};
  \draw[-] (-0.1,-2.0-0.1)--(+0.1,-2.0+0.1) node[midway, below left] {$-2$};
  \draw[] (-1.0,+2.0)--(-1.0,+2.0) node[midway, above] {($2$)};
  \draw[] (1.0,+2.0)--(1.0,+2.0) node[midway, above] {($2$)};

  \end{tikzpicture}
\end{figure}

\TT{Funkcja $h(t)$ składa się z dwóch sygnałów $h_1(t)$ i $h_2(t)$:}{The $h(t)$ signal is a linear combination of signals $h_1(t)$ i $h_2(t)$:}

\begin{equation}
h(t)=h_1(t)+h_2(t)
\end{equation}

\TT{gdzie:}{where:}

\begin{equation}
h_1(t)=-2 \cdot \Pi(\frac{t}{2})
\end{equation}

\begin{equation}
h_2(t)= 2 \cdot \delta(t+1) + 2 \cdot \delta(t-1)
\end{equation}

\begin{figure}[H]
  \centering
  \begin{tikzpicture}
  %\draw (0,0) circle (1in);
  \draw[->] (-4.0,+0.0) -- (+4.0,+0.0) node[right] {$t$};
  \draw[->] (+0.0,-2.5) -- (+0.0,+2.0) node[above] {$h_1(t)$};
  \draw[-,red, thick] (-3.5,+0.0) -- (-1.0,+0.0);
  \draw[-,red, thick] (-1.0,+0.0) -- (-1.0,-2.0);
  \draw[-,red, thick] (-1.0,-2.0) -- (+1.0,-2.0);
  \draw[-,red, thick] (+1.0,-2.0) -- (+1.0,+0.0);
  \draw[-,red, thick] (+1.0,+0.0) -- (+3.5,+0.0);
  
  \draw[-] (-1.0-0.1,-0.1)--(-1.0+0.1,0.1) node[midway, above, outer sep=5pt,align=center] {$-1$};
  \draw[-] (+1.0-0.1,-0.1)--(+1.0+0.1,0.1) node[midway, above, outer sep=5pt] {$1$};
  \draw[-] (-0.1,-2.0-0.1)--(+0.1,-2.0+0.1) node[midway, below left] {$-2$};
  \end{tikzpicture}
\end{figure}

\begin{figure}[H]
	\centering
	\begin{tikzpicture}
	%\draw (0,0) circle (1in);
	\draw[->] (-4.0,+0.0) -- (+4.0,+0.0) node[right] {$t$};
	\draw[->] (+0.0,-1.0) -- (+0.0,+2.5) node[above] {$h_2(t)$};
	\draw[-,red, thick] (-3.5,+0.0) -- (3.5,+0.0);
	\draw[->,red, thick] (+1.0,+0.0) -- (+1.0,+2.0);
	\draw[->,red, thick] (-1.0,+0.0) -- (-1.0,+2.0);
	
	\draw[-] (-1.0-0.1,-0.1)--(-1.0+0.1,0.1) node[midway, below left, outer sep=5pt,align=center] {$-1$};
	\draw[-] (+1.0-0.1,-0.1)--(+1.0+0.1,0.1) node[midway, below right, outer sep=5pt] {$1$};
	\draw[] (-1.0,+2.0)--(-1.0,+2.0) node[midway, above] {($2$)};
	\draw[] (1.0,+2.0)--(1.0,+2.0) node[midway, above] {($2$)};
	
	\end{tikzpicture}
\end{figure}

\TT{Wyznaczmy transformtę sygnału $h_{1}(t)$, czyli $H_{1}(\jmath \omega)$.}{Let's calculate the Fourier transform $H_{1}(\jmath \omega)$ for the first signal $h_{1}(t)$.}

\TT{Z treści zadania wiemy, że:}{We know that:}
$\mathcal F \{\Pi(t)\} = Sa\left(\frac{\omega}{2}\right)$.

\TT{Wykorzystując twierdzenie o zmianie skali mamy:}{Based on the scaling theorem:}

\begin{align*}
\TimeScalingTeorem{x}{X}{h_1}{H_1}
\end{align*}

\TT{otrzymujemy:}{we get:}

\begin{align*}
\Pi(t) \xrightarrow{\mathcal F} & Sa\left(\frac{\omega}{2}\right)\\
\Pi(\frac{1}{2} \cdot t) \xrightarrow{\mathcal F} & \frac{1}{\left|\frac{1}{2}\right|} \cdot Sa\left(\frac{ \frac{\omega}{ \frac{1}{2} }}{2}\right)\\
\Pi(\frac{t}{2}) \xrightarrow{\mathcal F} & 2 \cdot Sa\left(\frac{\omega \cdot 2}{2}\right)\\
\Pi(\frac{t}{2}) \xrightarrow{\mathcal F} & 2 \cdot Sa\left(\omega\right)\\
-2 \cdot \Pi(\frac{t}{2}) \xrightarrow{\mathcal F} & -2 \cdot 2 \cdot Sa\left(\omega\right)\\
-2 \cdot \Pi(\frac{t}{2}) \xrightarrow{\mathcal F} & -4 \cdot Sa\left(\omega\right)\\
\end{align*}

\TT{Transformata sygnału $h_{1}(t)$ to:}{The Fourier transform of the $h_{1}(t)$ signal is equal to:}

\begin{equation}
H_{1}(\jmath \omega) = \mathcal F\{H_{1}(t)\} = -4 \cdot Sa\left(\omega\right)
\end{equation}

\TT{Wyznaczenie transformaty sygnału $h_2(t)$ złożonego z delt Diracka jest znacznie prostsze.}{Let's calculate the Fourier transform of the $h_2(t)$ signal using the sampling property of the Dirac impulse.}

\begin{equation}
H_2(\jmath \omega )=\int_{-\infty }^{\infty}h_2(t) \cdot e^{-\jmath \cdot \omega \cdot t}\cdot dt
\end{equation}

\begin{align*}
H_2(\jmath \omega )&=\int_{-\infty }^{\infty}h_2(t) \cdot e^{-\jmath \cdot \omega \cdot t}\cdot dt=\\
&=\int_{-\infty }^{\infty}\left( 2 \cdot \delta(t+1) + 2 \cdot \delta(t-1)\right) \cdot e^{-\jmath \cdot \omega \cdot t}\cdot dt=\\
&=2 \cdot \int_{-\infty }^{\infty} \left( \delta(t+1) + \delta(t-1) \right) \cdot e^{-\jmath \cdot \omega \cdot t}\cdot dt=\\
&=2 \cdot \left( \int_{-\infty }^{\infty} \delta(t+1) \cdot e^{-\jmath \cdot \omega \cdot t}\cdot dt + \int_{-\infty }^{\infty} \delta(t-1) \cdot e^{-\jmath \cdot \omega \cdot t}\cdot dt \right)=\\
&=\begin{Bmatrix}
\SamplingPropertyOfDelta
\end{Bmatrix}=\\
&= 2 \cdot \left( e^{-\jmath \cdot \omega \cdot (-1)} + e^{-\jmath \cdot \omega \cdot 1} \right)=\\
&= 2 \cdot \left( e^{\jmath \cdot \omega} + e^{-\jmath \cdot \omega} \right)=\\
&= 2 \cdot \left( e^{\jmath \cdot \omega} + e^{-\jmath \cdot \omega} \right) \cdot \frac{2}{2}=\\
&= 4 \cdot \frac{ e^{\jmath \cdot \omega} + e^{-\jmath \cdot \omega} }{2} =\\
&=\begin{Bmatrix}
\EulerCos
\end{Bmatrix}=\\
&= 4 \cdot cos\left( \omega\right) 
\end{align*}

\TT{Transformata sygnału $h_2(t)$ to $H_2(\jmath \omega)=4 \cdot cos\left( \omega\right)$.}{The Fourier transform of the $h_2(t)$ signal is equal to $H_2(\jmath \omega)=4 \cdot cos\left( \omega\right)$.}

\TT{Korzystając z liniowości transformacji Fouriera:}{Based on the linearity theorem:}
\begin{align*}
\HomogeneousTeorem{h}{H}
\end{align*}

\TT{można wyznaczyć transformatę Fouriera $H(\jmath \omega)$ funkcji $h(t)$}{The Fourier transform $H(\jmath \omega)$ of the $h(t)$ signal can be derived as:}

\begin{align*}
H(\jmath \omega) &= H_1(\jmath \omega)+H_2(\jmath \omega)=\\
&= -4\cdot Sa\left( \omega\right) + 4 \cdot cos\left( \omega \right)=\\
&= 4\cdot \left(cos\left( \omega\right) - Sa\left( \omega\right)\right)
\end{align*}

\TT{Mamy wyznaczoną transformatę $H(\jmath \omega)$. Teraz, z twierdzenia o całkowaniu sygnału, możemy wyznaczyc transformatę $G(\jmath \omega)$:}{We derived the $H(\jmath \omega)$ transform. Now, based on the integration theorem, we can calculate the $G(\jmath \omega)$ Fourier transform.}

\begin{align*}
\IntegralTeorem{h}{H}{g}{G}
\end{align*}
 
 \begin{align*}
G(\jmath \omega) &= \frac{1}{\jmath \cdot \omega} \cdot H(\jmath \omega) + \pi \cdot \delta(\omega) \cdot H(0)=\\
&=\begin{Bmatrix*}[l]
H(0)=4\cdot \left(cos\left( 0\right) - Sa\left( 0\right)\right)\\
H(0)=4\cdot \left(1 -1 \right)\\
H(0)=4\cdot 0\\
H(0)=0\\
\end{Bmatrix*}=\\
&=\frac{1}{\jmath \cdot \omega} \cdot \left( 4\cdot \left(cos\left( \omega\right) - Sa\left( \omega\right)\right) \right) + 0=\\
&=\frac{4}{\jmath \cdot \omega} \cdot \left(cos\left( \omega\right) - Sa\left( \omega\right)\right)
\end{align*}

\TT{Ostatecznie transformata sygnału $g(t)$ jest równa $G(\jmath \omega)=\frac{4}{\jmath \cdot \omega} \cdot \left(cos\left( \omega\right) - Sa\left( \omega\right)\right)$.}{The Fourier transform of the $g(t)$ signal is equal to $G(\jmath \omega)=\frac{4}{\jmath \cdot \omega} \cdot \left(cos\left( \omega\right) - Sa\left( \omega\right)\right)$.}

\TT{Mamy wyznaczoną transformatę $G(\jmath \omega)$. Teraz, kolejny raz z twierdzenia o całkowaniu sygnału, możemy wyznaczyć transformatę $F(\jmath \omega)$:}{We derived the $G(\jmath \omega)$ transform. Using the integration theorem once again, we can calculate the $F(\jmath \omega)$ Fourier transform.}

\begin{align*}
\IntegralTeorem{g}{G}{f}{F}
\end{align*}

\begin{align*}
F(\jmath \omega) &= \frac{1}{\jmath \cdot \omega} \cdot G(\jmath \omega) + \pi \cdot \delta(\omega) \cdot G(0)=\\
&=\begin{Bmatrix*}[l]
G(0)=\frac{4}{\jmath \cdot 0} \cdot \left(cos\left( 0\right) - Sa\left(0\right)\right)\\
G(0)=\frac{0}{0}!!!\\
G(0)=\int_{-\infty}^{\infty} g(t) \cdot dt=\int_{-1}^{1} (-2) \cdot t \cdot dt = \left. (-2)\cdot \frac{t^2}{2} \right|_{-1}^{1}\\
G(0)=(-2) \cdot \left(\frac{1}{2} - \frac{1}{2}\right)= (-2) \cdot 0\\
G(0)=0\\
\end{Bmatrix*}=\\
&=\frac{1}{\jmath \cdot \omega} \cdot \frac{4}{\jmath \cdot \omega} \cdot \left(cos\left( \omega\right) - Sa\left( \omega\right)\right) + 0=\\
&=\frac{4}{\jmath^2 \cdot \omega^2} \cdot \left(cos\left( \omega\right) - Sa\left( \omega\right)\right)=\\
&=\frac{4}{(-1) \cdot \omega^2} \cdot \left(cos\left( \omega\right) - Sa\left( \omega\right)\right)=\\
&=\frac{4}{\omega^2} \cdot \left(Sa\left( \omega\right) - cos\left( \omega\right)\right)
\end{align*}

\TT{Ostatecznie transformata sygnału $f(t)$ jest równa $F(\jmath \omega)=\frac{4}{\omega^2} \cdot \left(Sa\left( \omega\right) - cos\left( \omega\right)\right)$.}{The Fourier transform of the $f(t)$ signal is equal to $F(\jmath \omega)=\frac{4}{\omega^2} \cdot \left(Sa\left( \omega\right) - cos\left( \omega\right)\right)$.}

\end{task}

