\begin{task}
Oblicz splot sygnałów $f(t)=\Pi\left(\frac{t}{T}\right)$ i $g(t)=\Lambda\left(\frac{t}{T}\right)$


\begin{figure}[H]
\centering
\begin{tikzpicture}
  %\draw (0,0) circle (1in);
  \draw[->] (-2.0-5.0,+0.0) -- (+2.0-5.0,+0.0) node[right] {$t$};
  \draw[->] (+0.0-5.0,-1.0) -- (+0.0-5.0,+2.5) node[above] {$f(t)$};
  
  \draw[-,red, thick] (-1.0-5.0,+2.0) -- (+1.0-5.0,+2.0);
  \draw[-,red, thick] (-1.0-5.0,+0.0) -- (-1.0-5.0,+2.0);
  \draw[-,red, thick] (+1.0-5.0,+0.0) -- (+1.0-5.0,+2.0);
  
  \draw[-] (-0.1-5.0,+2.0-0.1)--(+0.1-5.0,+2.0+0.1) node[midway, above left] {$1$};
  \draw[-] (-1.0-0.1-5.0,-0.1)--(-1.0+0.1-5.0,0.1) node[midway, below, outer sep=5pt,align=center] {$-\frac{T}{2}$};
  \draw[-] (+1.0-0.1-5.0,-0.1)--(+1.0+0.1-5.0,0.1) node[midway, below, outer sep=5pt,align=center] {$\frac{T}{2}$};
  
  
  \draw[->] (-2.0,+0.0) -- (+2.0,+0.0) node[right] {$t$};
  \draw[->] (+0.0,-1.0) -- (+0.0,+2.5) node[above] {$g(t)$};
  
  %\draw[-,red, thick] (-1.0,+2.0) -- (+1.0,+2.0);
  \draw[-,green, thick] (-1.0,+0.0) -- (+0.0,+2.0);
  \draw[-,green, thick] (+1.0,+0.0) -- (+0.0,+2.0);
  
  \draw[-] (-0.1,+2.0-0.1)--(+0.1,+2.0+0.1) node[midway, above left] {$1$};
  \draw[-] (-1.0-0.1,-0.1)--(-1.0+0.1,0.1) node[midway, below, outer sep=5pt,align=center] {$-T$};
  \draw[-] (+1.0-0.1,-0.1)--(+1.0+0.1,0.1) node[midway, below, outer sep=5pt,align=center] {$T$};
  
\end{tikzpicture}
\end{figure}

Wzór na slot sygnałów
\begin{equation}
\Convolution{h}{f}{g}
\end{equation}

Wzory sygnałów pod całką
\begin{align*}
f(\tau)&=\Pi\left(\frac{\tau}{T}\right)\\
g(t-\tau)&=\Lambda\left(\frac{t - \tau}{T}\right)
\end{align*}

Wykresy obu funkcji dla różnych wartości $t$

\begin{figure}[H]
  \centering
  \begin{animateinline}[controls,autoplay,loop,poster = 25,palindrome]{10}
    \multiframe{101}{n=-2.5+0.05}{%Number of Frames, variable = initial + increment
      \begin{tikzpicture}
      %\draw (0,0) circle (1in);
      \draw[->] (-4.5,+0.0) -- (+4.5,+0.0) node[right] {$\tau$};
      \draw[->] (+0.0,-1.0) -- (+0.0,+2.5) node[above] {$f(\tau),g(t-\tau)$};
      
      \draw[-,red, thick] (-0.5,+2.0) -- (+0.5,+2.0);
      \draw[-,red, thick] (-0.5,+0.0) -- (-0.5,+2.0);
      \draw[-,red, thick] (+0.5,+0.0) -- (+0.5,+2.0);
      
      \draw[-,green, thick] (-1.0+\n,+0.0) -- (+0.0+\n,+2.0);
      \draw[-,green, thick] (+1.0+\n,+0.0) -- (-0.0+\n,+2.0);
      \draw[-,green, dashed] (+0.0+\n,+0.0) -- (-0.0+\n,+2.0);
      
      \draw[-] (-0.1,+2.0-0.1)--(+0.1,+2.0+0.1) node[midway, above left] {$1$};
      \draw[-] (-1.0-0.1+\n,-0.1)--(-1.0+0.1+\n,0.1) node[midway, below, outer sep=25pt,align=center] {$t-T$};
      \draw[-] (+1.0-0.1+\n,-0.1)--(+1.0+0.1+\n,0.1) node[midway, below, outer sep=25pt,align=center] {$t+T$};
      \draw[-] (+0.0-0.1+\n,-0.1)--(+0.0+0.1+\n,0.1) node[midway, below, outer sep=25pt,align=center] {$t$};
      
      \draw[-] (-0.5-0.1,-0.1)--(-0.5+0.1,0.1) node[midway, below, outer sep=5pt,align=center] {$-\frac{T}{2}$};
      \draw[-] (+0.5-0.1,-0.1)--(+0.5+0.1,0.1) node[midway, below, outer sep=5pt,align=center] {$\frac{T}{2}$};
      
      \end{tikzpicture}  
    }
  \end{animateinline}
\end{figure}

Po wymnożeniu obu funkcji dla przykładowych wartości $t$ otrzymujemy

\begin{figure}[H]
  \centering
  \begin{animateinline}[controls,autoplay,loop,poster = 25,palindrome]{10}
    \multiframe{101}{n=-2.5+0.05}{%Number of Frames, variable = initial + increment
      \begin{tikzpicture}
      \tikzmath{
        function fun1(\x,\n,\T) {
          if (\x < -\T/2) || (\x < \n- \T) || (\x > \T/2) || (\x > \n + \T) then {
            return 0.0;
          } else {
            if \x < \n then {
              return \x/\T -(\n-\T)/\T;
            } else {
              return -\x/\T - (-\n-\T)/\T;
            };
          };
        };
      }
      
      %\draw (0,0) circle (1in);
      \draw[->] (-4.5,+0.0) -- (+4.5,+0.0) node[right] {$\tau$};
      \draw[->] (+0.0,-1.0) -- (+0.0,+2.5) node[above] {$f(\tau) \cdot g(t-\tau)$};
      
      \draw[-,red, dotted] (-0.5,+2.0) -- (+0.5,+2.0);
      \draw[-,red, dotted] (-0.5,+0.0) -- (-0.5,+2.0);
      \draw[-,red, dotted] (+0.5,+0.0) -- (+0.5,+2.0);
      
      \draw[-,green, dotted] (-1.0+\n,+0.0) -- (+0.0+\n,+2.0);
      \draw[-,green, dotted] (+1.0+\n,+0.0) -- (-0.0+\n,+2.0);
      \draw[-,green, dashed] (+0.0+\n,+0.0) -- (-0.0+\n,+2.0);
      
      \draw[-] (-0.1,+2.0-0.1)--(+0.1,+2.0+0.1) node[midway, above left] {$1$};
      \draw[-] (-1.0-0.1+\n,-0.1)--(-1.0+0.1+\n,0.1) node[midway, below, outer sep=25pt,align=center] {$t-T$};
      \draw[-] (+1.0-0.1+\n,-0.1)--(+1.0+0.1+\n,0.1) node[midway, below, outer sep=25pt,align=center] {$t+T$};
      \draw[-] (+0.0-0.1+\n,-0.1)--(+0.0+0.1+\n,0.1) node[midway, below, outer sep=25pt,align=center] {$t$};
      
      \draw[-] (-0.5-0.1,-0.1)--(-0.5+0.1,0.1) node[midway, below, outer sep=5pt,align=center] {$-\frac{T}{2}$};
      \draw[-] (+0.5-0.1,-0.1)--(+0.5+0.1,0.1) node[midway, below, outer sep=5pt,align=center] {$\frac{T}{2}$};
      
      %\draw[-,pink, thick] (-1.0+\n,+0.0) -- (+0.0+\n,+2.0);
      \draw[scale=1.0,domain=-3.5:3.5,smooth,samples=200,variable=\x,red,thick] plot ({\x},{2*fun1(\x,\n,1)});
      
      \end{tikzpicture}  
    }
  \end{animateinline}
\end{figure}

Jak widać dla różnych wartości $t$ otrzymujemy różny kształt funkcji podcałkowej $f(\tau)\cdot g(t-\tau)$.

Przedział 1.

\begin{figure}[H]
  \centering
  \begin{animateinline}[controls,autoplay,loop,poster = 21,palindrome]{10}
    \multiframe{22}{n=-2.5+0.05}{%Number of Frames, variable = initial + increment
      \begin{tikzpicture}
      \tikzmath{
        function fun1(\x,\n,\T) {
          if (\x < -\T/2) || (\x < \n- \T) || (\x > \T/2) || (\x > \n + \T) then {
            return 0.0;
          } else {
            if \x < \n then {
              return \x/\T -(\n-\T)/\T;
            } else {
              return -\x/\T - (-\n-\T)/\T;
            };
          };
        };
      }
      
      %\draw (0,0) circle (1in);
      \draw[->] (-4.5,+0.0) -- (+4.5,+0.0) node[right] {$\tau$};
      \draw[->] (+0.0,-1.0) -- (+0.0,+2.5) node[above] {$f(\tau) \cdot g(t-\tau)$};
      
      \draw[-,red, dotted] (-0.5,+2.0) -- (+0.5,+2.0);
      \draw[-,red, dotted] (-0.5,+0.0) -- (-0.5,+2.0);
      \draw[-,red, dotted] (+0.5,+0.0) -- (+0.5,+2.0);
      
      \draw[-,green, dotted] (-1.0+\n,+0.0) -- (+0.0+\n,+2.0);
      \draw[-,green, dotted] (+1.0+\n,+0.0) -- (-0.0+\n,+2.0);
      \draw[-,green, dashed] (+0.0+\n,+0.0) -- (-0.0+\n,+2.0);
      
      \draw[-] (-0.1,+2.0-0.1)--(+0.1,+2.0+0.1) node[midway, above left] {$1$};
      \draw[-] (-1.0-0.1+\n,-0.1)--(-1.0+0.1+\n,0.1) node[midway, below, outer sep=25pt,align=center] {$t-T$};
      \draw[-] (+1.0-0.1+\n,-0.1)--(+1.0+0.1+\n,0.1) node[midway, below, outer sep=25pt,align=center] {$t+T$};
      \draw[-] (+0.0-0.1+\n,-0.1)--(+0.0+0.1+\n,0.1) node[midway, below, outer sep=25pt,align=center] {$t$};
      
      \draw[-] (-0.5-0.1,-0.1)--(-0.5+0.1,0.1) node[midway, below, outer sep=5pt,align=center] {$-\frac{T}{2}$};
      \draw[-] (+0.5-0.1,-0.1)--(+0.5+0.1,0.1) node[midway, below, outer sep=5pt,align=center] {$\frac{T}{2}$};
      
      %\draw[-,pink, thick] (-1.0+\n,+0.0) -- (+0.0+\n,+2.0);
      \draw[scale=1.0,domain=-3.5:3.5,smooth,samples=200,variable=\x,red,thick] plot ({\x},{2*fun1(\x,\n,1)});
      
      \end{tikzpicture}  
    }
  \end{animateinline}
\end{figure}

Dla wartości $t$ spełniających warunek $t+T<\frac{T}{2}$ 

\end{task}

