\begin{task}
Oblicz splot sygnałów $f(t)=\Pi\left(\frac{t}{T}\right)$ i $g(t)=\Lambda\left(\frac{t}{T}\right)$


\begin{figure}[H]
\centering
\begin{tikzpicture}
  %\draw (0,0) circle (1in);
  \draw[->] (-2.0-5.0,+0.0) -- (+2.0-5.0,+0.0) node[right] {$t$};
  \draw[->] (+0.0-5.0,-1.0) -- (+0.0-5.0,+2.5) node[above] {$f(t)$};
  
  \draw[-,red, thick] (-1.0-5.0,+2.0) -- (+1.0-5.0,+2.0);
  \draw[-,red, thick] (-1.0-5.0,+0.0) -- (-1.0-5.0,+2.0);
  \draw[-,red, thick] (+1.0-5.0,+0.0) -- (+1.0-5.0,+2.0);
  
  \draw[-] (-0.1-5.0,+2.0-0.1)--(+0.1-5.0,+2.0+0.1) node[midway, above left] {$1$};
  \draw[-] (-1.0-0.1-5.0,-0.1)--(-1.0+0.1-5.0,0.1) node[midway, below, outer sep=5pt,align=center] {$-\frac{T}{2}$};
  \draw[-] (+1.0-0.1-5.0,-0.1)--(+1.0+0.1-5.0,0.1) node[midway, below, outer sep=5pt,align=center] {$\frac{T}{2}$};
  
  
  \draw[->] (-2.0,+0.0) -- (+2.0,+0.0) node[right] {$t$};
  \draw[->] (+0.0,-1.0) -- (+0.0,+2.5) node[above] {$g(t)$};
  
  %\draw[-,red, thick] (-1.0,+2.0) -- (+1.0,+2.0);
  \draw[-,green, thick] (-1.0,+0.0) -- (+0.0,+2.0);
  \draw[-,green, thick] (+1.0,+0.0) -- (+0.0,+2.0);
  
  \draw[-] (-0.1,+2.0-0.1)--(+0.1,+2.0+0.1) node[midway, above left] {$1$};
  \draw[-] (-1.0-0.1,-0.1)--(-1.0+0.1,0.1) node[midway, below, outer sep=5pt,align=center] {$-T$};
  \draw[-] (+1.0-0.1,-0.1)--(+1.0+0.1,0.1) node[midway, below, outer sep=5pt,align=center] {$T$};
  
\end{tikzpicture}
\end{figure}

Wzór na slot sygnałów
\begin{equation}
\Convolution{h}{f}{g}
\end{equation}

Wzory sygnałów pod całką
\begin{align*}
f(\tau)&=\Pi\left(\frac{\tau}{T}\right)\\
g(t-\tau)&=\Lambda\left(\frac{t - \tau}{T}\right)
\end{align*}

\begin{align*}
f(\tau) &= \begin{cases}
0 & \tau \in \left( -\infty; -\frac{T}{2} \right) \\
A & \tau \in \left(-\frac{T}{2}; \frac{T}{2} \right) \\
0 & \tau \in \left(\frac{T}{2}; \infty \right) \\
\end{cases}\\
g(t-\tau) &= \begin{cases}
0 & \tau \in \left(-\infty; t-T\right);\\
\frac{1}{T}\cdot \tau - \frac{t-T}{T} & \tau \in \left(t-T; t\right)\\
-\frac{1}{T}\cdot \tau - \frac{-t-T}{T} & \tau \in \left(t; t+T\right)\\
0 & \tau \in \left(t+T; \infty\right);
\end{cases}
\end{align*}

Wykresy obu funkcji dla różnych wartości $t$

\begin{figure}[H]
  \centering
  \begin{animateinline}[controls,autoplay,loop,poster = 25,palindrome]{10}
    \multiframe{101}{n=-2.5+0.05}{%Number of Frames, variable = initial + increment
      \begin{tikzpicture}
      %\draw (0,0) circle (1in);
      \draw[->] (-4.5,+0.0) -- (+4.5,+0.0) node[right] {$\tau$};
      \draw[->] (+0.0,-1.0) -- (+0.0,+2.5) node[above] {$f(\tau),g(t-\tau)$};
      
      \draw[-,red, thick] (-0.5,+2.0) -- (+0.5,+2.0);
      \draw[-,red, thick] (-0.5,+0.0) -- (-0.5,+2.0);
      \draw[-,red, thick] (+0.5,+0.0) -- (+0.5,+2.0);
      
      \draw[-,green, thick] (-1.0+\n,+0.0) -- (+0.0+\n,+2.0);
      \draw[-,green, thick] (+1.0+\n,+0.0) -- (-0.0+\n,+2.0);
      \draw[-,green, dashed] (+0.0+\n,+0.0) -- (-0.0+\n,+2.0);
      
      \draw[-] (-0.1,+2.0-0.1)--(+0.1,+2.0+0.1) node[midway, above left] {$1$};
      \draw[-] (-1.0-0.1+\n,-0.1)--(-1.0+0.1+\n,0.1) node[midway, below, outer sep=25pt,align=center] {$t-T$};
      \draw[-] (+1.0-0.1+\n,-0.1)--(+1.0+0.1+\n,0.1) node[midway, below, outer sep=25pt,align=center] {$t+T$};
      \draw[-] (+0.0-0.1+\n,-0.1)--(+0.0+0.1+\n,0.1) node[midway, below, outer sep=25pt,align=center] {$t$};
      
      \draw[-] (-0.5-0.1,-0.1)--(-0.5+0.1,0.1) node[midway, below, outer sep=5pt,align=center] {$-\frac{T}{2}$};
      \draw[-] (+0.5-0.1,-0.1)--(+0.5+0.1,0.1) node[midway, below, outer sep=5pt,align=center] {$\frac{T}{2}$};
      
      \end{tikzpicture}  
    }
  \end{animateinline}
\end{figure}

Po wymnożeniu obu funkcji dla przykładowych wartości $t$ otrzymujemy

\begin{figure}[H]
  \centering
  \begin{animateinline}[controls,autoplay,loop,poster = 25,palindrome]{10}
    \multiframe{101}{n=-2.5+0.05}{%Number of Frames, variable = initial + increment
      \begin{tikzpicture}
      \tikzmath{
        function fun1(\x,\n,\T) {
          if (\x < -\T/2) || (\x < \n- \T) || (\x > \T/2) || (\x > \n + \T) then {
            return 0.0;
          } else {
            if \x < \n then {
              return \x/\T -(\n-\T)/\T;
            } else {
              return -\x/\T - (-\n-\T)/\T;
            };
          };
        };
      }
      
      %\draw (0,0) circle (1in);
      \draw[->] (-4.5,+0.0) -- (+4.5,+0.0) node[right] {$\tau$};
      \draw[->] (+0.0,-1.0) -- (+0.0,+2.5) node[above] {$f(\tau) \cdot g(t-\tau)$};
      
      \draw[-,red, dotted] (-0.5,+2.0) -- (+0.5,+2.0);
      \draw[-,red, dotted] (-0.5,+0.0) -- (-0.5,+2.0);
      \draw[-,red, dotted] (+0.5,+0.0) -- (+0.5,+2.0);
      
      \draw[-,green, dotted] (-1.0+\n,+0.0) -- (+0.0+\n,+2.0);
      \draw[-,green, dotted] (+1.0+\n,+0.0) -- (-0.0+\n,+2.0);
      \draw[-,green, dashed] (+0.0+\n,+0.0) -- (-0.0+\n,+2.0);
      
      \draw[-] (-0.1,+2.0-0.1)--(+0.1,+2.0+0.1) node[midway, above left] {$1$};
      \draw[-] (-1.0-0.1+\n,-0.1)--(-1.0+0.1+\n,0.1) node[midway, below, outer sep=25pt,align=center] {$t-T$};
      \draw[-] (+1.0-0.1+\n,-0.1)--(+1.0+0.1+\n,0.1) node[midway, below, outer sep=25pt,align=center] {$t+T$};
      \draw[-] (+0.0-0.1+\n,-0.1)--(+0.0+0.1+\n,0.1) node[midway, below, outer sep=25pt,align=center] {$t$};
      
      \draw[-] (-0.5-0.1,-0.1)--(-0.5+0.1,0.1) node[midway, below, outer sep=5pt,align=center] {$-\frac{T}{2}$};
      \draw[-] (+0.5-0.1,-0.1)--(+0.5+0.1,0.1) node[midway, below, outer sep=5pt,align=center] {$\frac{T}{2}$};
      
      %\draw[-,pink, thick] (-1.0+\n,+0.0) -- (+0.0+\n,+2.0);
      \draw[scale=1.0,domain=-3.5:3.5,smooth,samples=200,variable=\x,red,thick] plot ({\x},{2*fun1(\x,\n,1)});
      
      \end{tikzpicture}  
    }
  \end{animateinline}
\end{figure}

Jak widać dla różnych wartości $t$ otrzymujemy różny kształt funkcji podcałkowej $f(\tau)\cdot g(t-\tau)$.

\paragraph{Przedział 1}.

\begin{figure}[H]
  \centering
  \begin{animateinline}[controls,autoplay,loop,poster = 20,palindrome]{10}
    \multiframe{21}{n=-2.5+0.05}{%Number of Frames, variable = initial + increment
      \begin{tikzpicture}
      \tikzmath{
        function fun1(\x,\n,\T) {
          if (\x < -\T/2) || (\x < \n- \T) || (\x > \T/2) || (\x > \n + \T) then {
            return 0.0;
          } else {
            if \x < \n then {
              return \x/\T -(\n-\T)/\T;
            } else {
              return -\x/\T - (-\n-\T)/\T;
            };
          };
        };
      }
      
      %\draw (0,0) circle (1in);
      \draw[->] (-4.5,+0.0) -- (+4.5,+0.0) node[right] {$\tau$};
      \draw[->] (+0.0,-1.0) -- (+0.0,+2.5) node[above] {$f(\tau) \cdot g(t-\tau)$};
      
      \draw[-,red, dotted] (-0.5,+2.0) -- (+0.5,+2.0);
      \draw[-,red, dotted] (-0.5,+0.0) -- (-0.5,+2.0);
      \draw[-,red, dotted] (+0.5,+0.0) -- (+0.5,+2.0);
      
      \draw[-,green, dotted] (-1.0+\n,+0.0) -- (+0.0+\n,+2.0);
      \draw[-,green, dotted] (+1.0+\n,+0.0) -- (-0.0+\n,+2.0);
      \draw[-,green, dashed] (+0.0+\n,+0.0) -- (-0.0+\n,+2.0);
      
      \draw[-] (-0.1,+2.0-0.1)--(+0.1,+2.0+0.1) node[midway, above left] {$1$};
      \draw[-] (-1.0-0.1+\n,-0.1)--(-1.0+0.1+\n,0.1) node[midway, below, outer sep=25pt,align=center] {$t-T$};
      \draw[-] (+1.0-0.1+\n,-0.1)--(+1.0+0.1+\n,0.1) node[midway, below, outer sep=25pt,align=center] {$t+T$};
      \draw[-] (+0.0-0.1+\n,-0.1)--(+0.0+0.1+\n,0.1) node[midway, below, outer sep=25pt,align=center] {$t$};
      
      \draw[-] (-0.5-0.1,-0.1)--(-0.5+0.1,0.1) node[midway, below, outer sep=5pt,align=center] {$-\frac{T}{2}$};
      \draw[-] (+0.5-0.1,-0.1)--(+0.5+0.1,0.1) node[midway, below, outer sep=5pt,align=center] {$\frac{T}{2}$};
      
      %\draw[-,pink, thick] (-1.0+\n,+0.0) -- (+0.0+\n,+2.0);
      \draw[scale=1.0,domain=-3.5:3.5,smooth,samples=200,variable=\x,red,thick] plot ({\x},{2*fun1(\x,\n,1)});
      
      \end{tikzpicture}  
    }
  \end{animateinline}
\end{figure}

Dla wartości $t$ spełniających warunek $t+T<-\frac{T}{2}$ 

\begin{align*}
t+T&<-\frac{T}{2}\\
t&<-\frac{T}{2}-T\\
t&<-\frac{3}{2}\cdot T\\
\end{align*}

w wyniku mnożenia otrzymyjemy $0$ a więc wartość splotu jest także równa $0$

\begin{align*}
h(t)&=\int_{-\infty}^{\infty} 0 \cdot d\tau\\
&=0
\end{align*}

\paragraph{Przedział 2}.

\begin{figure}[H]
  \centering
  \begin{animateinline}[controls,autoplay,loop,poster = 20,palindrome]{10}
    \multiframe{21}{n=-1.5+0.05}{%Number of Frames, variable = initial + increment
      \begin{tikzpicture}
      \tikzmath{
        function fun1(\x,\n,\T) {
          if (\x < -\T/2) || (\x < \n- \T) || (\x > \T/2) || (\x > \n + \T) then {
            return 0.0;
          } else {
            if \x < \n then {
              return \x/\T -(\n-\T)/\T;
            } else {
              return -\x/\T - (-\n-\T)/\T;
            };
          };
        };
      }
      
      %\draw (0,0) circle (1in);
      \draw[->] (-4.5,+0.0) -- (+4.5,+0.0) node[right] {$\tau$};
      \draw[->] (+0.0,-1.0) -- (+0.0,+2.5) node[above] {$f(\tau) \cdot g(t-\tau)$};
      
      \draw[-,red, dotted] (-0.5,+2.0) -- (+0.5,+2.0);
      \draw[-,red, dotted] (-0.5,+0.0) -- (-0.5,+2.0);
      \draw[-,red, dotted] (+0.5,+0.0) -- (+0.5,+2.0);
      
      \draw[-,green, dotted] (-1.0+\n,+0.0) -- (+0.0+\n,+2.0);
      \draw[-,green, dotted] (+1.0+\n,+0.0) -- (-0.0+\n,+2.0);
      \draw[-,green, dashed] (+0.0+\n,+0.0) -- (-0.0+\n,+2.0);
      
      \draw[-] (-0.1,+2.0-0.1)--(+0.1,+2.0+0.1) node[midway, above left] {$1$};
      \draw[-] (-1.0-0.1+\n,-0.1)--(-1.0+0.1+\n,0.1) node[midway, below, outer sep=25pt,align=center] {$t-T$};
      \draw[-] (+1.0-0.1+\n,-0.1)--(+1.0+0.1+\n,0.1) node[midway, below, outer sep=25pt,align=center] {$t+T$};
      \draw[-] (+0.0-0.1+\n,-0.1)--(+0.0+0.1+\n,0.1) node[midway, below, outer sep=25pt,align=center] {$t$};
      
      \draw[-] (-0.5-0.1,-0.1)--(-0.5+0.1,0.1) node[midway, below, outer sep=5pt,align=center] {$-\frac{T}{2}$};
      \draw[-] (+0.5-0.1,-0.1)--(+0.5+0.1,0.1) node[midway, below, outer sep=5pt,align=center] {$\frac{T}{2}$};
      
      %\draw[-,pink, thick] (-1.0+\n,+0.0) -- (+0.0+\n,+2.0);
      \draw[scale=1.0,domain=-3.5:3.5,smooth,samples=200,variable=\x,red,thick] plot ({\x},{2*fun1(\x,\n,1)});
      
      \end{tikzpicture}  
    }
  \end{animateinline}
\end{figure}

Dla wartości $t$ spełniających warunki $t+T \geq -\frac{T}{2}$ i $t<-\frac{T}{2}$

\begin{align*}
t+T& \geq -\frac{T}{2} &\wedge&&  t&<-\frac{T}{2}\\
t&\geq-\frac{T}{2}-T  &\wedge&&  t&<-\frac{T}{2}\\
t&\geq-\frac{3}{2}\cdot T &\wedge&&  t&<-\frac{T}{2}\\
\end{align*}

a więc $t\in \left<-\frac{3}{2}\cdot T, -\frac{T}{2} \right)$

w wyniku mnożenia otrzymujemy prostą zdefiniowaną na odcinku $t \in \left(-\frac{T}{2}, t+T\right)$.

\begin{align*}
f(\tau) \cdot g(t-\tau)&=\begin{cases}
0 & \tau \in \left(-\infty; -\frac{T}{2}\right)\\
-\frac{1}{T}\cdot \tau - \frac{-t-T}{T} & \tau \in \left(-\frac{T}{2}; t+T\right)\\
0 & \tau \in \left( t+T; \infty \right)\\
\end{cases}
\end{align*}

wartość splotu wyznaczamy z ze wzoru

\begin{align*}
h(t)&=\int_{-\infty}^{\infty} f(\tau) \cdot g(t-\tau) \cdot d\tau\\
&=\int_{-\infty}^{\frac{T}{2}} 0 \cdot d\tau + \int_{-\frac{T}{2}}^{t+T}\left( -\frac{1}{T}\cdot \tau - \frac{-t-T}{T} \right)\cdot d\tau +\int_{t+T}^{\infty} 0 \cdot d\tau\\
&=0 - \int_{-\frac{T}{2}}^{t+T} \frac{1}{T}\cdot \tau d\tau + \int_{-\frac{T}{2}}^{t+T} \frac{t+T}{T} \cdot d\tau +0\\
&=- \frac{1}{T}\cdot \int_{-\frac{T}{2}}^{t+T} \tau \cdot d\tau + \frac{t+T}{T} \cdot \int_{-\frac{T}{2}}^{t+T} d\tau\\
&=- \frac{1}{T}\cdot \left( \frac{1}{2} \cdot \tau^2 \right)_{-\frac{T}{2}}^{t+T} + \frac{t+T}{T} \cdot \left(\tau \right)_{-\frac{T}{2}}^{t+T}\\
&=- \frac{1}{T}\cdot \frac{1}{2} \cdot \left(  \left(t+T\right)^2 -\left(-\frac{T}{2}\right)^2 \right) + \frac{t+T}{T} \cdot \left(t+T - \left(-\frac{T}{2}\right) \right)\\
&=- \frac{1}{2\cdot T} \cdot \left(  t^2+2\cdot t\cdot T + T^2 -\frac{T^2}{4} \right) + \frac{t+T}{T} \cdot \left(t+T + \frac{T}{2} \right)\\
&=- \frac{1}{2\cdot T} \cdot \left(  t^2+2\cdot t\cdot T + \frac{3}{4}\cdot T^2 \right) + \frac{t+T}{T} \cdot \left(t+\frac{3}{2}\cdot T\right)\\
&=- \frac{1}{2\cdot T} \cdot \left(  t^2+2\cdot t\cdot T + \frac{3}{4}\cdot T^2 \right) + \frac{1}{T} \cdot \left(t^2 + \frac{3}{2} \cdot t \cdot T + t\cdot T +\frac{3}{2}\cdot T^2\right)\\
&=- \frac{1}{2\cdot T} \cdot \left(  t^2+2\cdot t\cdot T + \frac{3}{4}\cdot T^2 \right) + \frac{2}{2\cdot T} \cdot \left(t^2 + \frac{5}{2} \cdot t \cdot T+\frac{3}{2}\cdot T^2\right)\\
&=\frac{1}{2\cdot T} \cdot \left(  -t^2-2\cdot t\cdot T - \frac{3}{4}\cdot T^2 \right) + \frac{1}{2\cdot T} \cdot \left(2\cdot t^2 + 5 \cdot t \cdot T+3\cdot T^2\right)\\
&=\frac{1}{2\cdot T} \cdot \left(  -t^2-2\cdot t\cdot T - \frac{3}{4}\cdot T^2 + 2\cdot t^2 + 5 \cdot t \cdot T+3\cdot T^2\right)\\
&=\frac{1}{2\cdot T} \cdot \left( t^2 + 3 \cdot t \cdot T+2 \frac{1}{4} \cdot  T^2\right)\\
&=\frac{1}{2\cdot T} \cdot t^2 + \frac{1}{2 \cdot T} \cdot 3 \cdot t \cdot T+\frac{1}{2\cdot T} \cdot \frac{9}{4} \cdot  T^2\\
&=\frac{1}{2\cdot T} \cdot t^2 + \frac{3}{2} \cdot t+\frac{9}{8} \cdot  T\\
\end{align*}

\paragraph{Przedział 3}.

\begin{figure}[H]
  \centering
  \begin{animateinline}[controls,autoplay,loop,poster = 20,palindrome]{10}
    \multiframe{21}{n=-0.5+0.05}{%Number of Frames, variable = initial + increment
      \begin{tikzpicture}
      \tikzmath{
        function fun1(\x,\n,\T) {
          if (\x < -\T/2) || (\x < \n- \T) || (\x > \T/2) || (\x > \n + \T) then {
            return 0.0;
          } else {
            if \x < \n then {
              return \x/\T -(\n-\T)/\T;
            } else {
              return -\x/\T - (-\n-\T)/\T;
            };
          };
        };
      }
      
      %\draw (0,0) circle (1in);
      \draw[->] (-4.5,+0.0) -- (+4.5,+0.0) node[right] {$\tau$};
      \draw[->] (+0.0,-1.0) -- (+0.0,+2.5) node[above] {$f(\tau) \cdot g(t-\tau)$};
      
      \draw[-,red, dotted] (-0.5,+2.0) -- (+0.5,+2.0);
      \draw[-,red, dotted] (-0.5,+0.0) -- (-0.5,+2.0);
      \draw[-,red, dotted] (+0.5,+0.0) -- (+0.5,+2.0);
      
      \draw[-,green, dotted] (-1.0+\n,+0.0) -- (+0.0+\n,+2.0);
      \draw[-,green, dotted] (+1.0+\n,+0.0) -- (-0.0+\n,+2.0);
      \draw[-,green, dashed] (+0.0+\n,+0.0) -- (-0.0+\n,+2.0);
      
      \draw[-] (-0.1,+2.0-0.1)--(+0.1,+2.0+0.1) node[midway, above left] {$1$};
      \draw[-] (-1.0-0.1+\n,-0.1)--(-1.0+0.1+\n,0.1) node[midway, below, outer sep=25pt,align=center] {$t-T$};
      \draw[-] (+1.0-0.1+\n,-0.1)--(+1.0+0.1+\n,0.1) node[midway, below, outer sep=25pt,align=center] {$t+T$};
      \draw[-] (+0.0-0.1+\n,-0.1)--(+0.0+0.1+\n,0.1) node[midway, below, outer sep=25pt,align=center] {$t$};
      
      \draw[-] (-0.5-0.1,-0.1)--(-0.5+0.1,0.1) node[midway, below, outer sep=5pt,align=center] {$-\frac{T}{2}$};
      \draw[-] (+0.5-0.1,-0.1)--(+0.5+0.1,0.1) node[midway, below, outer sep=5pt,align=center] {$\frac{T}{2}$};
      
      %\draw[-,pink, thick] (-1.0+\n,+0.0) -- (+0.0+\n,+2.0);
      \draw[scale=1.0,domain=-3.5:3.5,smooth,samples=200,variable=\x,red,thick] plot ({\x},{2*fun1(\x,\n,1)});
      
      \end{tikzpicture}  
    }
  \end{animateinline}
\end{figure}

Dla wartości $t$ spełniających warunki $t \geq -\frac{T}{2}$ i $t<\frac{T}{2}$

\begin{align*}
t& \geq -\frac{T}{2} &\wedge&&  t&<\frac{T}{2}\\
\end{align*}

a więc $t\in \left<-\frac{1}{2}\cdot T, \frac{1}{2}\cdot T \right)$

w wyniku mnożenia otrzymujemy dwie proste zdefiniowaną na odcinkach $t \in \left(-\frac{T}{2}, t\right)$ oraz $t \in \left(t, \frac{T}{2}\right)$.

\begin{align*}
f(\tau) \cdot g(t-\tau)&=\begin{cases}
0 & \tau \in \left(-\infty; -\frac{T}{2}\right)\\
\frac{1}{T}\cdot \tau - \frac{t-T}{T} & \tau \in \left(-\frac{T}{2}; t\right)\\
-\frac{1}{T}\cdot \tau - \frac{-t-T}{T} & \tau \in \left(t; \frac{T}{2}\right)\\
0 & \tau \in \left( \frac{T}{2}; \infty \right)\\
\end{cases}
\end{align*}

wartość splotu wyznaczamy z ze wzoru

\begin{align*}
h(t)&=\int_{-\infty}^{\infty} f(\tau) \cdot g(t-\tau) \cdot d\tau\\
&=\int_{-\infty}^{-\frac{T}{2}} 0 \cdot d\tau 
+ \int_{-\frac{T}{2}}^{t}\left( \frac{1}{T}\cdot \tau - \frac{t-T}{T} \right)\cdot d\tau 
+ \int_{t}^{\frac{T}{2}}\left( -\frac{1}{T}\cdot \tau - \frac{-t-T}{T} \right)\cdot d\tau 
+\int_{\frac{T}{2}}^{\infty} 0 \cdot d\tau\\
&=0 
+ \int_{-\frac{T}{2}}^{t} \frac{1}{T}\cdot \tau \cdot d\tau -  \int_{-\frac{T}{2}}^{t} \frac{t-T}{T} \cdot d\tau 
+ \int_{t}^{\frac{T}{2}}\left( -\frac{1}{T}\cdot \tau\right) \cdot d\tau - \int_{t}^{\frac{T}{2}} \frac{-t-T}{T} \cdot d\tau 
+0\\
&=\frac{1}{T}\cdot \int_{-\frac{T}{2}}^{t} \tau \cdot d\tau -  \frac{t-T}{T} \cdot \int_{-\frac{T}{2}}^{t} d\tau 
-\frac{1}{T}\cdot \int_{t}^{\frac{T}{2}} \tau \cdot d\tau +  \frac{t+T}{T} \cdot \int_{t}^{\frac{T}{2}} d\tau\\
&=\frac{1}{T}\cdot \left. \frac{1}{2} \cdot \tau^2\right|_{-\frac{T}{2}}^{t} -  \frac{t-T}{T} \cdot \left. \tau 
\right|_{-\frac{T}{2}}^{t} - \frac{1}{T}\cdot \left. \frac{1}{2} \cdot \tau^2 \right|_{t}^{\frac{T}{2}} +  \frac{t+T}{T} \cdot \left. \tau \right|_{t}^{\frac{T}{2}}\\
&=\frac{1}{2 \cdot T}\cdot \left( t^2 - \left(-\frac{T}{2}\right)^2\right) -  \frac{t-T}{T} \cdot \left( t - \left(-\frac{T}{2}\right) \right) - \frac{1}{2\cdot T}\cdot \left( \left(\frac{T}{2}\right)^2 - t^2\right) +  \frac{t+T}{T} \cdot \left( \frac{T}{2} - t \right)\\
&=\frac{1}{2 \cdot T}\cdot \left( t^2 + \frac{1}{4} \cdot T^2\right) -  \frac{t-T}{T} \cdot \left( t + \frac{T}{2} \right) - \frac{1}{2\cdot T}\cdot \left( \frac{1}{4}\cdot T^2 - t^2\right) +  \frac{t+T}{T} \cdot \left( \frac{T}{2} - t \right)\\
&=\frac{1}{2 \cdot T}\cdot \left( t^2 + \frac{1}{4} \cdot T^2\right) -  \frac{2}{2\cdot T} \cdot \left(t-T\right) \cdot \left( t + \frac{T}{2} \right) - \frac{1}{2\cdot T}\cdot \left( \frac{1}{4}\cdot T^2 - t^2\right) +  \frac{2}{2 \cdot T} \cdot \left(t+T\right) \cdot \left( \frac{T}{2} - t \right)\\
&=\frac{1}{2 \cdot T}\cdot \left( t^2 + \frac{1}{4} \cdot T^2\right) -  \frac{2}{2\cdot T} \cdot \left(t^2 + \frac{1}{2} \cdot t \cdot T - t \cdot T - \frac{1}{2} \cdot T^2 \right) - \frac{1}{2\cdot T}\cdot \left( \frac{1}{4}\cdot T^2 - t^2\right) +  \frac{2}{2 \cdot T} \cdot \left(\frac{1}{2}\cdot t \cdot T - t^2 + \frac{1}{2} \cdot T^2 - t \cdot T \right)\\
&=\frac{1}{2 \cdot T}\cdot \left( t^2 + \frac{1}{4} \cdot T^2\right) +  \frac{1}{2\cdot T} \cdot \left(-2 \cdot t^2 - t \cdot T + 2\cdot t \cdot T + T^2 \right) + \frac{1}{2\cdot T}\cdot \left( -\frac{1}{4}\cdot T^2 + t^2\right) +  \frac{1}{2 \cdot T} \cdot \left( t \cdot T - 2 \cdot t^2 + T^2 - 2 \cdot t \cdot T \right)\\
&=\frac{1}{2 \cdot T}\cdot \left( t^2 + \frac{1}{4} \cdot T^2 -2 \cdot t^2 - t \cdot T + 2\cdot t \cdot T + T^2 -\frac{1}{4}\cdot T^2 + t^2 + t \cdot T - 2 \cdot t^2 + T^2 - 2 \cdot t \cdot T \right)\\
&=\frac{1}{2 \cdot T}\cdot \left( - 2 \cdot t^2 + 2 \cdot T^2  \right)\\
&=\frac{1}{T}\cdot \left( - t^2 + T^2  \right)\\
&=- \frac{1}{T}\cdot t^2 + T \\
\end{align*}


\paragraph{Przedział 4}.

\begin{figure}[H]
  \centering
  \begin{animateinline}[controls,autoplay,loop,poster = 20,palindrome]{10}
    \multiframe{21}{n=0.5+0.05}{%Number of Frames, variable = initial + increment
      \begin{tikzpicture}
      \tikzmath{
        function fun1(\x,\n,\T) {
          if (\x < -\T/2) || (\x < \n- \T) || (\x > \T/2) || (\x > \n + \T) then {
            return 0.0;
          } else {
            if \x < \n then {
              return \x/\T -(\n-\T)/\T;
            } else {
              return -\x/\T - (-\n-\T)/\T;
            };
          };
        };
      }
      
      %\draw (0,0) circle (1in);
      \draw[->] (-4.5,+0.0) -- (+4.5,+0.0) node[right] {$\tau$};
      \draw[->] (+0.0,-1.0) -- (+0.0,+2.5) node[above] {$f(\tau) \cdot g(t-\tau)$};
      
      \draw[-,red, dotted] (-0.5,+2.0) -- (+0.5,+2.0);
      \draw[-,red, dotted] (-0.5,+0.0) -- (-0.5,+2.0);
      \draw[-,red, dotted] (+0.5,+0.0) -- (+0.5,+2.0);
      
      \draw[-,green, dotted] (-1.0+\n,+0.0) -- (+0.0+\n,+2.0);
      \draw[-,green, dotted] (+1.0+\n,+0.0) -- (-0.0+\n,+2.0);
      \draw[-,green, dashed] (+0.0+\n,+0.0) -- (-0.0+\n,+2.0);
      
      \draw[-] (-0.1,+2.0-0.1)--(+0.1,+2.0+0.1) node[midway, above left] {$1$};
      \draw[-] (-1.0-0.1+\n,-0.1)--(-1.0+0.1+\n,0.1) node[midway, below, outer sep=25pt,align=center] {$t-T$};
      \draw[-] (+1.0-0.1+\n,-0.1)--(+1.0+0.1+\n,0.1) node[midway, below, outer sep=25pt,align=center] {$t+T$};
      \draw[-] (+0.0-0.1+\n,-0.1)--(+0.0+0.1+\n,0.1) node[midway, below, outer sep=25pt,align=center] {$t$};
      
      \draw[-] (-0.5-0.1,-0.1)--(-0.5+0.1,0.1) node[midway, below, outer sep=5pt,align=center] {$-\frac{T}{2}$};
      \draw[-] (+0.5-0.1,-0.1)--(+0.5+0.1,0.1) node[midway, below, outer sep=5pt,align=center] {$\frac{T}{2}$};
      
      %\draw[-,pink, thick] (-1.0+\n,+0.0) -- (+0.0+\n,+2.0);
      \draw[scale=1.0,domain=-3.5:3.5,smooth,samples=200,variable=\x,red,thick] plot ({\x},{2*fun1(\x,\n,1)});
      
      \end{tikzpicture}  
    }
  \end{animateinline}
\end{figure}

Dla wartości $t$ spełniających warunki $t-T \geq -\frac{T}{2}$ i $t-T<\frac{T}{2}$

\begin{align*}
t-T& \geq -\frac{T}{2} &\wedge&&  t-T&<\frac{T}{2}\\
t& \geq -\frac{T}{2}+T &\wedge&&  t&<\frac{T}{2}+T\\
t& \geq \frac{1}{2}\cdot T &\wedge&&  t&<\frac{3}{2}\cdot T\\
\end{align*}

a więc $t\in \left<\frac{1}{2}\cdot T, \frac{3}{2}\cdot T \right)$

w wyniku mnożenia otrzymujemy prostą zdefiniowaną na odcinku $t \in \left(t-T, \frac{T}{2}\right)$.

\begin{align*}
f(\tau) \cdot g(t-\tau)&=\begin{cases}
0 & \tau \in \left(-\infty; t-T\right)\\
\frac{1}{T}\cdot \tau - \frac{t-T}{T} & \tau \in \left(t-T; \frac{T}{2}\right)\\
0 & \tau \in \left( \frac{T}{2}; \infty \right)\\
\end{cases}
\end{align*}

wartość splotu wyznaczamy z ze wzoru

\begin{align*}
h(t)&=\int_{-\infty}^{\infty} f(\tau) \cdot g(t-\tau) \cdot d\tau\\
&=\int_{-\infty}^{t-T} 0 \cdot d\tau 
+ \int_{t-T}^{\frac{T}{2}}\left( \frac{1}{T}\cdot \tau - \frac{t-T}{T} \right)\cdot d\tau 
+\int_{\frac{T}{2}}^{\infty} 0 \cdot d\tau\\
&=0
+ \int_{t-T}^{\frac{T}{2}} frac{1}{T}\cdot \tau \cdot d\tau - \int_{t-T}^{\frac{T}{2}} \frac{t-T}{T} \cdot d\tau 
+0\\
&=\frac{1}{T}\cdot \int_{t-T}^{\frac{T}{2}}  \tau \cdot d\tau - \frac{t-T}{T} \cdot \int_{t-T}^{\frac{T}{2}}  d\tau\\
&=\frac{1}{T}\cdot \left. \frac{1}{2} \cdot \tau^2 \right|_{t-T}^{\frac{T}{2}} - \frac{t-T}{T} \cdot \left.  \tau \right|_{t-T}^{\frac{T}{2}}\\
&=\frac{1}{T}\cdot \frac{1}{2} \cdot \left(  \left(\frac{T}{2}\right)^2 - \left(t-T\right)^2\right) - \frac{t-T}{T} \cdot \left(  \frac{T}{2} - \left(t-T\right)\right)\\
&=\frac{1}{2 \cdot T} \cdot \left(  \frac{1}{4}\cdot T^2 - \left(t^2 -2\cdot t \cdot T + T^2 \right)\right) - \frac{t-T}{T} \cdot \left(  \frac{T}{2} - t+T\right)\\
&=\frac{1}{2 \cdot T} \cdot \left(  \frac{1}{4}\cdot T^2 -t^2 +2\cdot t \cdot T - T^2 \right) - \frac{1}{T} \cdot \left(t-T\right)\cdot \left(  \frac{3}{2}\cdot T - t\right)\\
&=\frac{1}{2 \cdot T} \cdot \left(  -\frac{3}{4}\cdot T^2 -t^2 +2\cdot t \cdot T \right) - \frac{2}{2\cdot T} \cdot \left(\frac{3}{2} \cdot t \cdot T - t^2 - \frac{3}{2}\cdot T^2 + t \cdot T\right)\\
&=\frac{1}{2 \cdot T} \cdot \left(  -\frac{3}{4}\cdot T^2 -t^2 +2\cdot t \cdot T \right) - \frac{1}{2 \cdot T} \cdot \left(\frac{6}{2} \cdot t \cdot T - 2\cdot t^2 - \frac{6}{2}\cdot T^2 + 2\cdot t \cdot T\right)\\
&=\frac{1}{2 \cdot T} \cdot \left(  -\frac{3}{4}\cdot T^2 -t^2 +2\cdot t \cdot T  - \frac{6}{2} \cdot t \cdot T + 2\cdot t^2 + \frac{6}{2}\cdot T^2 - 2\cdot t \cdot T\right)\\
&=\frac{1}{2 \cdot T} \cdot \left(  \frac{9}{4}\cdot T^2 -3\cdot t \cdot T  +  t^2 \right)\\
&=\frac{1}{2 \cdot T} \cdot  \frac{9}{4}\cdot T^2 - \frac{1}{2 \cdot T} \cdot 3\cdot t \cdot T  +  \frac{1}{2 \cdot T} \cdot t^2\\
&=\frac{9}{8}\cdot T - \frac{3}{2} \cdot t  +  \frac{1}{2 \cdot T} \cdot t^2\\
\end{align*}

\paragraph{Przedział 5}.

\begin{figure}[H]
  \centering
  \begin{animateinline}[controls,autoplay,loop,poster = 20,palindrome]{10}
    \multiframe{21}{n=1.5+0.05}{%Number of Frames, variable = initial + increment
      \begin{tikzpicture}
      \tikzmath{
        function fun1(\x,\n,\T) {
          if (\x < -\T/2) || (\x < \n- \T) || (\x > \T/2) || (\x > \n + \T) then {
            return 0.0;
          } else {
            if \x < \n then {
              return \x/\T -(\n-\T)/\T;
            } else {
              return -\x/\T - (-\n-\T)/\T;
            };
          };
        };
      }
      
      %\draw (0,0) circle (1in);
      \draw[->] (-4.5,+0.0) -- (+4.5,+0.0) node[right] {$\tau$};
      \draw[->] (+0.0,-1.0) -- (+0.0,+2.5) node[above] {$f(\tau) \cdot g(t-\tau)$};
      
      \draw[-,red, dotted] (-0.5,+2.0) -- (+0.5,+2.0);
      \draw[-,red, dotted] (-0.5,+0.0) -- (-0.5,+2.0);
      \draw[-,red, dotted] (+0.5,+0.0) -- (+0.5,+2.0);
      
      \draw[-,green, dotted] (-1.0+\n,+0.0) -- (+0.0+\n,+2.0);
      \draw[-,green, dotted] (+1.0+\n,+0.0) -- (-0.0+\n,+2.0);
      \draw[-,green, dashed] (+0.0+\n,+0.0) -- (-0.0+\n,+2.0);
      
      \draw[-] (-0.1,+2.0-0.1)--(+0.1,+2.0+0.1) node[midway, above left] {$1$};
      \draw[-] (-1.0-0.1+\n,-0.1)--(-1.0+0.1+\n,0.1) node[midway, below, outer sep=25pt,align=center] {$t-T$};
      \draw[-] (+1.0-0.1+\n,-0.1)--(+1.0+0.1+\n,0.1) node[midway, below, outer sep=25pt,align=center] {$t+T$};
      \draw[-] (+0.0-0.1+\n,-0.1)--(+0.0+0.1+\n,0.1) node[midway, below, outer sep=25pt,align=center] {$t$};
      
      \draw[-] (-0.5-0.1,-0.1)--(-0.5+0.1,0.1) node[midway, below, outer sep=5pt,align=center] {$-\frac{T}{2}$};
      \draw[-] (+0.5-0.1,-0.1)--(+0.5+0.1,0.1) node[midway, below, outer sep=5pt,align=center] {$\frac{T}{2}$};
      
      %\draw[-,pink, thick] (-1.0+\n,+0.0) -- (+0.0+\n,+2.0);
      \draw[scale=1.0,domain=-3.5:3.5,smooth,samples=200,variable=\x,red,thick] plot ({\x},{2*fun1(\x,\n,1)});
      
      \end{tikzpicture}  
    }
  \end{animateinline}
\end{figure}

Dla wartości $t$ spełniających warunek $t-T \geq \frac{T}{2}$.

\begin{align*}
t-T& \geq \frac{T}{2}\\
t& \geq \frac{T}{2}+T\\
t& \geq \frac{3}{2}\cdot T\\
\end{align*}

a więc $t\in \left<\frac{3}{2}\cdot T, \infty \right)$

w wyniku mnożenia otrzymujemy sygnał zerowy

\begin{align*}
f(\tau) \cdot g(t-\tau)&=0
\end{align*}

a więc wartość splotu wyznaczona ze wzoru

\begin{align*}
h(t)&=\int_{-\infty}^{\infty} f(\tau) \cdot g(t-\tau) \cdot d\tau\\
&=\int_{-\infty}^{\infty} 0 \cdot d\tau \\
&=0
\end{align*}

\paragraph{Podsumowanie}

Zbierająć wyniki, wynik splotu wyrażony jest jako funkcja o pięciu przedziałach

\begin{align*}
h(t)&=\int_{-\infty}^{\infty} f(\tau) \cdot g(t-\tau) \cdot d\tau\\
&=\begin{cases}
0 & \tau \in \left(-\infty; -\frac{3}{2}\cdot T\right);\\
\frac{1}{2\cdot T} \cdot t^2 + \frac{3}{2} \cdot t+\frac{9}{8} \cdot  T & \tau \in \left( -\frac{3}{2}\cdot T; -\frac{1}{2} \cdot T\right);\\
- \frac{1}{T}\cdot t^2 + \frac{3}{4} \cdot T & \tau \in \left( -\frac{1}{2}\cdot T; \frac{1}{2} \cdot T\right);\\
\frac{9}{8}\cdot T - \frac{3}{2} \cdot t  +  \frac{1}{2 \cdot T} \cdot t^2 & \tau \in \left( \frac{1}{2}\cdot T; \frac{3}{2} \cdot T\right);\\
0 & \tau \in \left(\frac{3}{2}\cdot T ; \infty\right);\\
\end{cases}
\end{align*}

\begin{figure}[H]
  \centering
      \begin{tikzpicture}
      \tikzmath{
        %function fun1(\x,\n,\T) {
        %  if (\x < -\T/2) || (\x < \n- \T) || (\x > \T/2) || (\x > \n + \T) then {
        %    return 0.0;
        %  } else {
        %    if \x < \n then {
        %      return \x/\T -(\n-\T)/\T;
        %    } else {
        %      return -\x/\T - (-\n-\T)/\T;
        %    };
        %  };
        %};
        function fun2(\x,\n,\T) {
          if (\x < (-3.0 * \T/2.0)) then {
            return 0;
          } else {
              if (\x < -\T/2.0) then {
                return \x*\x/(2*\T) + 3*\x/2 + 9*\T/8;
              } else {
                  if (\x < \T/2.0) then {
                    return -\x*\x/\T +3*\T / 4;
                  } else {
                      if (\x < 3*\T/2) then {
                        return \x*\x/(2*\T) - 3*\x/2 + 9*\T/8;
                      } else {
                        return 0;
                      };
                  };
              };
          };
          %return 0.0;
        };
      }
      
      %\draw (0,0) circle (1in);
      \draw[->] (-4.5,+0.0) -- (+4.5,+0.0) node[right] {$t$};
      \draw[->] (+0.0,-1.0) -- (+0.0,+2.5) node[above] {$h(t)$};
      
      \draw[-] (-0.1,+2.0-0.1)--(+0.1,+2.0+0.1) node[midway, above left] {$1$};
      \draw[-] (-0.1,+2.0*3/4-0.1)--(+0.1,+2.0*3/4+0.1) node[midway, above left] {$\frac{3}{4} \cdot T$};


      \draw[-] (-1.5-0.1,-0.1)--(-1.5+0.1,0.1) node[midway, below, outer sep=5pt,align=center] {$-\frac{3}{2}\cdot T$};
      \draw[-] (+1.5-0.1,-0.1)--(+1.5+0.1,0.1) node[midway, below, outer sep=5pt,align=center] {$\frac{3}{2}\cdot T$};
      \draw[-] (-0.5-0.1,-0.1)--(-0.5+0.1,0.1) node[midway, below, outer sep=5pt,align=center] {$-\frac{1}{2}\cdot T$};
      \draw[-] (+0.5-0.1,-0.1)--(+0.5+0.1,0.1) node[midway, below, outer sep=5pt,align=center] {$\frac{1}{2}\cdot T$};
      
      %\draw[-,pink, thick] (-1.0+\n,+0.0) -- (+0.0+\n,+2.0);
      \draw[scale=1.0,domain=-3.5:3.5,smooth,samples=200,variable=\x,red,thick] plot ({\x},{2*fun2(\x,1,1)});
      
      \end{tikzpicture}  
\end{figure}

\end{task}