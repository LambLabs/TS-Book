\documentclass[a4paper,11pt]{LambBook} %6 inch 800x600

\usepackage[T1]{fontenc}
\usepackage[polish]{babel}
\usepackage[utf8]{inputenc}
\usepackage{lmodern}
\usepackage{mathtools}
\usepackage{amsmath}
\usepackage{cancel}
\usepackage{graphicx} 
\usepackage{array}
\usepackage{float} %Wymuszenie wstawienia obrazka w miejscu wstawienia
%\usepackage[europeancurrents, europeanvoltages, europeanresistors, americaninductors, europeanports]{circuitikz} %Do rysowania obwodów elektrycznych
\usepackage[ISBN=978-83-939620-1-3,SC2]{ean13isbn} %Drukowanie numeru ISBN
\usepackage{tikz}
\usepackage{ifthen}
\usepackage{ bbold }
\usepackage{animate}


\selectlanguage{polish}

\allowdisplaybreaks
%https://www.codecogs.com/latex/eqneditor.php
%http://www.sciweavers.org/free-online-latex-equation-editor

\def\booktitle{Teoria Sygnałów w zadaniach}
\def\bookauthors{Tomasz Grajek, Krzysztof Wegner}

\usepackage[pdftex,
        unicode=true, % Aby działały polskie literki
        colorlinks=true,
        urlcolor=rltblue,       % \href{...}{...} external (URL)
        filecolor=rltgreen,     % \href{...} local file
        linkcolor=rltred,       % \ref{...} and \pageref{...}
        citecolor=blue,
        pdfstartview={FitV},
        pdftitle={\booktitle},
        pdfauthor={\bookauthors},
        pdfsubject={Teoria Sygnałów},
        pdfkeywords={Teoria Sygnałów, Zadania},
        pdfproducer={pdfLaTeX},
        %pdfadjustspacing=1,
        pagebackref=false, %activate back references inside bibliography. Must be specified as part
        bookmarksopen=true]{hyperref}

\setcounter{secnumdepth}{2}

%sinc
\usetikzlibrary{math}
\tikzmath{
	function sinc(\x) {
		if  abs(\x) < .001 then { % (|x| < .001) ~ (x = 0)
			return 1;
		} else {
			return sin(\x r)/\x;
		};
	};
}

% Własność próbkująca delty diraca
\newcommand{\SamplingPropertyOfDelta}{
  \int_{-\infty }^{\infty} \delta(t-t_0)\cdot f(t) \cdot dt = f(t_0)
}
% Wzory Eulera
\newcommand{\EulerSin}{
  sin\left(x\right) = \frac{e^{\jmath \cdot x} - e^{-\jmath \cdot x}}{2 \cdot \jmath}
}
\newcommand{\EulerCos}{
  cos\left(x\right) = \frac{e^{\jmath \cdot x} + e^{-\jmath \cdot x}}{2}
}
% Definicja Sa
\newcommand{\SaDef}{
  Sa\left(x\right) = \frac{sin\left(x\right)}{x}
}
% Twierdzenie o całkowaniu % \IntegralTeorem{g}{G}{f}{F}
\newcommand{\IntegralTeorem}[4]{
  #1(t) &\xrightarrow{\mathcal F} #2(\jmath \omega)\\
  #3(t) = \int_{-\infty}^{t} #1(\tau)\cdot d\tau &\xrightarrow{\mathcal F}
  #4(\jmath \omega) = \frac{1}{\jmath \cdot \omega} \cdot #2(\jmath \omega) + \pi \cdot \delta(\omega) \cdot #2(0)
}
% Twierdzenie o przesunięciu w czasu częstotliwości % \TimeShiftTeorem{g}{G}{f}{F}
\newcommand{\TimeShiftTeorem}[4]{
  #1(t) &\xrightarrow{\mathcal F} #2(\jmath \omega)\\
  #3(t) = #1(t-t_0) &\xrightarrow{\mathcal F} 
  #4(\jmath \omega) = #2(\jmath \omega) \cdot e^{-\jmath \cdot \omega \cdot t_0}
}
% Twierdzenie o przesunięciu w dziedzinie częstotliwości % \FrequencyShiftTeorem{g}{G}{f}{F}
\newcommand{\FrequencyShiftTeorem}[4]{
  #1(t) &\xrightarrow{\mathcal F} #2(\jmath \omega)\\
  #3(t) = #1(t) \cdot e^{\jmath \cdot \omega_0 \cdot t} &\xrightarrow{\mathcal F} 
  #4(\jmath \omega) = #2(\jmath \left(\omega - \omega_0\right))
}
% Twierdzenie o symetrii % \SymetryTeorem{g}{G}{f}{F}
\newcommand{\SymetryTeorem}[4]{
  #1(t) &\xrightarrow{\mathcal F} #2(\jmath \omega)\\
  #3(t) = #2(t) &\xrightarrow{\mathcal F}
  #4(\jmath \omega) = 2\pi \cdot #1(-\omega)
}
% Twierdzenie o zmianie skali % \TimeScalingTeorem{g}{G}{f}{F}
\newcommand{\TimeScalingTeorem}[4]{
  #1(t) &\xrightarrow{\mathcal F} #2(\jmath \omega)\\
  #3(t) = #1(\alpha \cdot t) &\xrightarrow{\mathcal F}
  #4(\jmath \omega) = \frac{1}{\left|\alpha \right|} \cdot #2(\jmath \frac{\omega}{\alpha})
}

% Twierdzenie o liniowości % \HomogeneousTeorem{f}{F}
\newcommand{\HomogeneousTeorem}[2]{
  #1_1(t) &\xrightarrow{\mathcal F} #2_1(\jmath \omega) \\
  #1_2(t) &\xrightarrow{\mathcal F} #2_2(\jmath \omega) \\
  #1(t) = \alpha \cdot #1_1(t) + \beta \cdot #1_2(t) &\xrightarrow{\mathcal F} #2(\jmath \omega)=\alpha \cdot #2_1(\jmath \omega) + \beta \cdot #2_2(\jmath \omega)
}

%Splot %\Convolution{h}{f}{g}
\newcommand{\Convolution}[3]{
#1(t) = \int_{-\infty}^{\infty} #2(\tau) \cdot #3(t-\tau) \cdot d\tau
}

%Splot %\Parseval{F}
\newcommand{\Parseval}[1]{
 E = \frac{1}{2\pi} \cdot \int_{-\infty}^{\infty} \left|#1(\jmath \omega)\right|^2 \cdot d\omega
}

\setdebug
%\setrelease
%\setPL
\setENG
%====================================================
\begin{document}
%====================================================
\definecolor{rltred}{rgb}{0.75,0,0} %Definicja kolorów
\definecolor{rltgreen}{rgb}{0,0.5,0}
\definecolor{rltblue}{rgb}{0,0,0.75}
%====================================================
\title{\booktitle}
\author{\bookauthors}
%====================================================
%Strona Tytułowa
\label{page:titlepage}
%====================================================
% Title Page definition
\thispagestyle{empty}
\begin{titlepage}
  \begin{center}
    
    %{\fontsize{100}{120}\selectfont \@title}
    %\@title
    \baselineskip=100pt
    {\fontsize{100}{120}\selectfont  Teoria Sygnałów} \\ {\fontsize{80}{100}\selectfont w zadaniach}
    \\[10em]
    \bookauthors
    
    \vfill
    
    % Bottom of the page
    %{\large \date }
    {\large \today }
    
  \end{center}
\end{titlepage}
\cleardoublepage
%====================================================
%====================================================
%Druga Strona z ISBN'em
\label{page:firstpage}
%====================================================
% First Page definition
\thispagestyle{empty}
\begin{flushleft}
  \textsc{\TT{Politechnika Poznańska}{Poznan University of Technology}}\\%
  \TT{Wydział Informatyki i Telekomunikacji}{Faculty of Computing and Telecommunications}\\%
  \TT{Instytut Telekomunikacji Multimedialnej}{Departament of Multimedia Telecomunications}\\[1em]
  
  pl. M. Skłodowskiej-Curie 5\\
  60-965 Poznań\\[1em]
  
  www.et.put.poznan.pl\\
  www.multimedia.edu.pl
\end{flushleft}

\vfill

\begin{flushleft}
  Copyright © Krzysztof Wegner, 2019\\
  \TT{Wszelkie prawa zastrzeżone}{All right reserved}\\
  \ISBN\\
  \TT{Wydrukowano w Polsce}{Printed in Poland}\\[1em]
  
  %Książka współfinansowana ze środków Unii Europejskiej w ramach Europejskiego Funduszu Społecznego.\\
\end{flushleft}
\clearpage
%====================================================
%====================================================
%\input{01_opor_zastepczy/_opor_zastepczy.tex}
%\input{02_zrodla/_zrodla.tex}
%\input{03_prawa_kirchhoffa/_prawa_kirchhoffa.tex}
%\input{04_metoda_wezlowa/_metoda_wezlowa.tex}
%\input{05_metoda_oczkowa/_metoda_oczkowa.tex}
%\input{06_thevenin_norton/_thevenin_norton.tex}
%\input{07_moc/_moc.tex}
%\input{08_superpozycja/_metoda_superpozycji.tex}
%\input{08_uklady_nieliniowe/_uklady_nieliniowe.tex}
%\input{09_impedancja_zastepcza/_impedancja_zastepcza.tex}
%\input{10_metoda_oczkowa_wezlowa_AC/_metoda_oczkowa_wezlowa_AC.tex}
%\input{11_moc_AC/_moc_AC.tex}

\chapter{Podstawowe własności sygnałów}
\inputTask{zadTG017-ENG.tex}
%Fundamental concepts and measures
%	Signals and their models
%	Signal classes and examples
%		Continuous, discrete, analogue, quantized and digital signals
%		Periodic signals
%		Sinusoidal signals: real and complex
%		Non-periodic signals
\section{Podstawowe własności sygnałów}
%	Basic signal metrics
%		Amplitude
%		Mean value
\subsection{Wartość średnia}
\inputTask{zadKW001-ENG.tex}
\inputTask{zadKW002-ENG.tex} % ToDo - dodać wersję ze wzorami Eulera - łatwiejsza niż cykliczna
\inputTask[Debug]{zadKW007-ENG.tex}
%		Energy of a signal
\subsection{Energia sygnału}
\inputTask{zadKW006-ENG.tex}
%		Power of s signal
\subsection{Moc sygnału}
\inputTask{zadKW016-ENG.tex}
\inputTask{zadKW004.tex}
\inputTask{zadKW003-ENG.tex}
\inputTask{zadTG016-ENG.tex}
\inputTask{zadKW005.tex}
\inputTask[Debug]{zadKW017-ENG.tex} %Wynik ok, ale do sprawdzenia granice calkowania i okres, bo uśredniamy za dwa okresy sin^2, czyli zła ideologia
\inputTask{zadKW041.tex}
\inputTask{zadTG015-ENG.tex}
%		Effective value of a signal (RMS)
%	Energy signals vs power signals
%	Orthogonality. Orthogonal signals and vectors
%	Signal components
%		DC and AC signal components
%		Odd and even signal components
%Analysis of periodic signals using orthogonal series
\chapter{Analiza sygnałów okresowych za pomocą szeregów ortogonalnych}
%	Hilbert space
%	Orthogonal bases
%	Orthogonal series of functions
%	Trigonometric Fourier series
\section{Trygonometryczny szerego Fouriera}
\inputTask{zadKW008-ENG.tex}
\inputTask{zadKW009-ENG.tex}
\inputTask{zadKW010-ENG.tex}
\inputTask{zadKW011.tex}
%	The influence of signal symmetries on the coefficients of the trigonometric Fourier series
%	Complex exponential Fourier series
\section{Zespolony szerego Fouriera}
\inputTask{zadKW012.tex}
\inputTask{zadKW013.tex}
\inputTask{zadKW014.tex}
\inputTask{zadKW015.tex}
\inputTask{zadKW018.tex}
%	The harmonic spectrum of a real signal
%	The relationship of the complex exponential and the trigonometric Fourier series
%	Linearity of Fourier series
%	The influence of signal symmetries on the coefficients of complex exponential Fourier series
%	The effect of signal shift in time on the complex exponential Fourier series
%	Spectrum of a product of two signals
%	Computing the power of a signal – the Parseval theorem
\section{Obliczenia mocy sygnałów - twierdzenie Parsevala}
\inputTask{zadTG014.tex}
%Analysis of non-periodic signals. Fourier Transformation and Transform
\chapter{Analiza sygnałów nieokresowych. Transformata Fouriera}
%	An intuitive introduction
%	Definition
\section{Wyznaczanie transformaty Fouriera z definicji}
\inputTask{zadTG001.tex}
\inputTask{zadTG002.tex}
\inputTask{zadTG003.tex}
\inputTask{zadTG004.tex}
\inputTask{zadKW019.tex}
\inputTask{zadKW020.tex}
\inputTask{zadKW029.tex}
%	Fourier Transform vs Laplace Transform
%	The Magnitude Spectrum and Phase Spectrum
%	Symmetries of the Fourier Transform for real-valued signals
%	Special case of Fourier Transform for symmetrical signals
%	Theorems describing the properties of Fourier Transformation
%		Linearity
%		Shift theorem – shifting in time domain
%		Shifting in frequency domain (also known as modulation theorem)
%		Scaling theorem (also called the similarity theorem)
%		Time-frequency duality (also known as the symmetry theorem)
%		Derivative theorem (differentiation in time domain)
%		Integration theorem
\section{Wykorzystanie twierdzeń do obliczeń transformaty Fouriera}
\inputTask{zadTG005.tex}
\inputTask{zadTG006.tex}
\inputTask{zadKW021.tex}
\inputTask{zadKW022.tex}
\inputTask{zadKW023.tex}
\inputTask{zadTG007.tex}
\inputTask{zadKW024.tex}
\inputTask{zadKW025.tex}
\inputTask{zadKW026.tex}
\inputTask{zadKW027.tex}
\inputTask{zadKW030.tex}
\inputTask{zadKW031.tex}

%	Calculating energy of the signal from its Fourier transform. The Parseval's theorem
\section{Obliczenia energii sygnału za pomocą transformaty Fouriera. Twierdzenie Parsevala}
\inputTask{zadTG009.tex}
\inputTask{zadTG013.tex}
\inputTask{zadTG010.tex}
%	Generalization of the Fourier transformation for infinite energy signals
%	Fourier transform of a periodic signal
%	Calculating the power of a signal from its Fourier transform. The Parseval's theorem for power signals
%Processing of signals by linear and time invariant (LTI) systems
\chapter{Przetwarzanie sygnałów za pomocą układów LTI}
%	Introduction to LTI systems. Fundamental properties
%	Impulse response of an LTI system
%	Impulse response of a causal system
%	The response of an LTI system to arbitrary input
%	Properties of linear convolution
\section{Obliczanie splotu ze wzoru}
\inputTask{zadTG008.tex}
\inputTask{zadKW028.tex}
%	Frequency response
%	Determining the frequency response of an electronic circuit
%	Filters
\section{Filtry}
\inputTask{zadTG011.tex}
\inputTask{zadTG012.tex}
%Sampling. Discrete-time signals
%	Introduction to discrete signals
%	Spectrum of a sampled signal
%	Spectral efect of sampling a continuous signal
%	Reconstruction of the continuous signal from its samples
%	Non-periodic and periodic discrete-time signals
%	Fourier transforms of discrete-time signals
%	Processing of discrete-time signals
%	Frequency response of discrete-time LTI systems

%====================================================
% Ostatna stona z ISBN'em
%====================================================
\label{page:lastpage}
%====================================================
% Last Page definition
\newpage
\thispagestyle{empty}
\null
\vfill
\begin{center}
    © \the\year\\
    Wszelkie prawa zastrzeżone.\\[1em]
    \begin{minipage}{0.5\textwidth}
    \EANisbn    
    \end{minipage}
\end{center}
%====================================================
%====================================================
\end{document}

%Tom pierwszy zadania
%Tom drugi rozwiązania

%TODO
