\documentclass[a4paper,11pt]{LambBook} %6 inch 800x600

\usepackage[T1]{fontenc}
\usepackage[polish]{babel}
\usepackage[utf8]{inputenc}
\usepackage{lmodern}
\usepackage{mathtools}
\usepackage{amsmath}
\usepackage{cancel}
\usepackage{graphicx} 
\usepackage{array}
\usepackage{float} %Wymuszenie wstawienia obrazka w miejscu wstawienia
%\usepackage[europeancurrents, europeanvoltages, europeanresistors, americaninductors, europeanports]{circuitikz} %Do rysowania obwodów elektrycznych
\usepackage{tikz}
\usepackage{ifthen}

\selectlanguage{polish}

\allowdisplaybreaks
%https://www.codecogs.com/latex/eqneditor.php
%http://www.sciweavers.org/free-online-latex-equation-editor

\def\booktitle{Teoria Sygnałów w zadaniach}
\def\bookauthors{Tomasz Grajek, Krzysztof Wegner}

\usepackage[pdftex,
        unicode=true, % Aby działały polskie literki
        colorlinks=true,
        urlcolor=rltblue,       % \href{...}{...} external (URL)
        filecolor=rltgreen,     % \href{...} local file
        linkcolor=rltred,       % \ref{...} and \pageref{...}
        citecolor=blue,
        pdfstartview={FitV},
        pdftitle={\booktitle},
        pdfauthor={\bookauthors},
        pdfsubject={Teoria Sygnałów},
        pdfkeywords={Teoria Sygnałów, Zadania},
        pdfproducer={pdfLaTeX},
        %pdfadjustspacing=1,
        pagebackref=false, %activate back references inside bibliography. Must be specified as part
        bookmarksopen=true]{hyperref}

\setcounter{secnumdepth}{2}

%sinc
\usetikzlibrary{math}
\tikzmath{
	function sinc(\x) {
		if  abs(\x) < .001 then { % (|x| < .001) ~ (x = 0)
			return 1;
		} else {
			return sin(\x r)/\x;
		};
	};
}




%\setdebug
\setrelease
%====================================================
\begin{document}
%====================================================
\definecolor{rltred}{rgb}{0.75,0,0} %Definicja kolorów
\definecolor{rltgreen}{rgb}{0,0.5,0}
\definecolor{rltblue}{rgb}{0,0,0.75}
%====================================================
\title{\booktitle}
\author{\bookauthors}
%====================================================
%Strona Tytułowa
\label{page:titlepage}
\makebooktitle
%====================================================
%Druga Strona z ISBN'em
\thispagestyle{empty}
\begin{flushleft}
\textsc{Politechnika Poznańska}\\%
Wydział Elektroniki i Telekomunikacji\\%
Katedra Telekomunikacji Multimedialnej i~Mikroelektroniki\\[1em]

pl. M. Skłodowskiej-Curie 5\\
60-965 Poznań\\[1em]

www.et.put.poznan.pl\\
www.multimedia.edu.pl
\end{flushleft}

\vfill

\begin{flushleft}
Copyright © Krzysztof Wegner, 2019\\
Wszelkie prawa zastrzeżone\\
%\ISBN\\
Wydrukowano w Polsce\\[1em]

Książka współfinansowana ze środków Unii Europejskiej w ramach Europejskiego Funduszu Społecznego.\\
\end{flushleft}
\clearpage
%====================================================
%\input{01_opor_zastepczy/_opor_zastepczy.tex}
%\input{02_zrodla/_zrodla.tex}
%\input{03_prawa_kirchhoffa/_prawa_kirchhoffa.tex}
%\input{04_metoda_wezlowa/_metoda_wezlowa.tex}
%\input{05_metoda_oczkowa/_metoda_oczkowa.tex}
%\input{06_thevenin_norton/_thevenin_norton.tex}
%\input{07_moc/_moc.tex}
%\input{08_superpozycja/_metoda_superpozycji.tex}
%\input{08_uklady_nieliniowe/_uklady_nieliniowe.tex}
%\input{09_impedancja_zastepcza/_impedancja_zastepcza.tex}
%\input{10_metoda_oczkowa_wezlowa_AC/_metoda_oczkowa_wezlowa_AC.tex}
%\input{11_moc_AC/_moc_AC.tex}


%Fundamental concepts and measures
%	Signals and their models
%	Signal classes and examples
%		Continuous, discrete, analogue, quantized and digital signals
%		Periodic signals
%		Sinusoidal signals: real and complex
%		Non-periodic signals
%	Basic signal metrics
%		Amplitude
%		Mean value
\inputTask{zadKW001.tex}
\inputTask{zadKW002.tex}
\inputTask{zadKW007.tex}
%		Energy of a signal
\inputTask{zadKW003.tex}
\inputTask{zadKW004.tex}
\inputTask{zadKW005.tex}
\inputTask{zadKW006.tex}
\inputTask{zadKW016.tex}
\inputTask{zadKW017.tex}
%		Power of s signal
%		Effective value of a signal (RMS)
%	Energy signals vs power signals
%	Orthogonality. Orthogonal signals and vectors
%	Signal components
%		DC and AC signal components
%		Odd and even signal components
%Analysis of periodic signals using orthogonal series
%	Hilbert space
%	Orthogonal bases
%	Orthogonal series of functions
%	Trigonometric Fourier series
\inputTask{zadKW008.tex}
\inputTask{zadKW009.tex}
\inputTask{zadKW010.tex}
\inputTask{zadKW011.tex}
%	The influence of signal symmetries on the coefficients of the trigonometric Fourier series
%	Complex exponential Fourier series
\inputTask{zadKW012.tex}
\inputTask{zadKW013.tex}
\inputTask{zadKW014.tex}
\inputTask{zadKW015.tex}
\inputTask{zadKW018.tex}
%	The harmonic spectrum of a real signal
%	The relationship of the complex exponential and the trigonometric Fourier series
%	Linearity of Fourier series
%	The influence of signal symmetries on the coefficients of complex exponential Fourier series
%	The effect of signal shift in time on the complex exponential Fourier series
%	Spectrum of a product of two signals
%	Computing the power of a signal – the Parseval theorem
%Analysis of non-periodic signals. Fourier Transformation and Transform
%	An intuitive introduction
%	Definition
\inputTask{zadTG001.tex}
\inputTask{zadTG002.tex}
\inputTask{zadTG003.tex}
\inputTask{zadTG004.tex}
\inputTask{zadKW019.tex}
\inputTask{zadKW020.tex}
%	Fourier Transform vs Laplace Transform
%	The Magnitude Spectrum and Phase Spectrum
%	Symmetries of the Fourier Transform for real-valued signals
%	Special case of Fourier Transform for symmetrical signals
%	Theorems describing the properties of Fourier Transformation
%		Linearity
%		Shift theorem – shifting in time domain
%		Shifting in frequency domain (also known as modulation theorem)
%		Scaling theorem (also called the similarity theorem)
%		Time-frequency duality (also known as the symmetry theorem)
%		Derivative theorem (differentiation in time domain)
%		Integration theorem
%	Calculating energy of the signal from its Fourier transform. The Parseval's theorem
%	Generalization of the Fourier transformation for infinite energy signals
%	Fourier transform of a periodic signal
%	Calculating the power of a signal from its Fourier transform. The Parseval's theorem for power signals
%Processing of signals by linear and time invariant (LTI) systems
%	Introduction to LTI systems. Fundamental properties
%	Impulse response of an LTI system
%	Impulse response of a causal system
%	The response of an LTI system to arbitrary input
%	Properties of linear convolution
%	Frequency response
%	Determining the frequency response of an electronic circuit
%Filters
%	Sampling. Discrete-time signals
%	Introduction to discrete signals
%	Spectrum of a sampled signal
%	Spectral efect of sampling a continuous signal
%	Reconstruction of the continuous signal from its samples
%	Non-periodic and periodic discrete-time signals
%	Fourier transforms of discrete-time signals
%	Processing of discrete-time signals
%	Frequency response of discrete-time LTI systems

%====================================================
\end{document}

%Tom pierwszy zadania
%Tom drugi rozwiązania

%TODO
