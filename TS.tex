\documentclass[a4paper,11pt]{LambBook} %6 inch 800x600

\usepackage[T1]{fontenc}
\usepackage[polish]{babel}
\usepackage[utf8]{inputenc}
\usepackage{lmodern}
\usepackage{mathtools}
\usepackage{amsmath}
\usepackage{cancel}
\usepackage{graphicx} 
\usepackage{array}
\usepackage{float} %Wymuszenie wstawienia obrazka w miejscu wstawienia
%\usepackage[europeancurrents, europeanvoltages, europeanresistors, americaninductors, europeanports]{circuitikz} %Do rysowania obwodów elektrycznych
\usepackage{tikz}
\usepackage{ifthen}

\selectlanguage{polish}

\allowdisplaybreaks
%https://www.codecogs.com/latex/eqneditor.php

\def\booktitle{Teoria Sygnałów w zadaniach}
\def\bookauthors{Tomasz Grajek, Krzysztof Wegner}

\usepackage[pdftex,
        unicode=true, % Aby działały polskie literki
        colorlinks=true,
        urlcolor=rltblue,       % \href{...}{...} external (URL)
        filecolor=rltgreen,     % \href{...} local file
        linkcolor=rltred,       % \ref{...} and \pageref{...}
        citecolor=blue,
        pdfstartview={FitV},
        pdftitle={\booktitle},
        pdfauthor={\bookauthors},
        pdfsubject={Teoria Sygnałów},
        pdfkeywords={Teoria Sygnałów, Zadania},
        pdfproducer={pdfLaTeX},
        %pdfadjustspacing=1,
        pagebackref=false, %activate back references inside bibliography. Must be specified as part
        bookmarksopen=true]{hyperref}

\def\debugbook{0}
\newcommand{\setdebug}{\def\debugbook{1}}

\newcommand{\inputTask}[1]
{
  \ifthenelse{\equal{debugbook}{0}}{\input{#1}}{}
}
\setcounter{secnumdepth}{2}

\setdebug
%====================================================
\begin{document}
%====================================================
\definecolor{rltred}{rgb}{0.75,0,0} %Definicja kolorów
\definecolor{rltgreen}{rgb}{0,0.5,0}
\definecolor{rltblue}{rgb}{0,0,0.75}
%====================================================
\title{\booktitle}
\author{\bookauthors}
%====================================================
%Strona Tytułowa
\label{page:titlepage}
\makebooktitle
%====================================================
%Druga Strona z ISBN'em
\thispagestyle{empty}
\begin{flushleft}
\textsc{Politechnika Poznańska}\\%
Wydział Elektroniki i Telekomunikacji\\%
Katedra Telekomunikacji Multimedialnej i~Mikroelektroniki\\[1em]

pl. M. Skłodowskiej-Curie 5\\
60-965 Poznań\\[1em]

www.et.put.poznan.pl\\
www.multimedia.edu.pl
\end{flushleft}

\vfill

\begin{flushleft}
Copyright © Krzysztof Wegner, 2019\\
Wszelkie prawa zastrzeżone\\
%\ISBN\\
Wydrukowano w Polsce\\[1em]

Książka współfinansowana ze środków Unii Europejskiej w ramach Europejskiego Funduszu Społecznego.\\
\end{flushleft}
\clearpage
%====================================================
%\input{01_opor_zastepczy/_opor_zastepczy.tex}
%\input{02_zrodla/_zrodla.tex}
%\input{03_prawa_kirchhoffa/_prawa_kirchhoffa.tex}
%\input{04_metoda_wezlowa/_metoda_wezlowa.tex}
%\input{05_metoda_oczkowa/_metoda_oczkowa.tex}
%\input{06_thevenin_norton/_thevenin_norton.tex}
%\input{07_moc/_moc.tex}
%\input{08_superpozycja/_metoda_superpozycji.tex}
%\input{08_uklady_nieliniowe/_uklady_nieliniowe.tex}
%\input{09_impedancja_zastepcza/_impedancja_zastepcza.tex}
%\input{10_metoda_oczkowa_wezlowa_AC/_metoda_oczkowa_wezlowa_AC.tex}
%\input{11_moc_AC/_moc_AC.tex}


%Fundamental concepts and measures
%	Signals and their models
%	Signal classes and examples
%		Continuous, discrete, analogue, quantized and digital signals
%		Periodic signals
%		Sinusoidal signals: real and complex
%		Non-periodic signals
%	Basic signal metrics
%		Amplitude
%		Mean value
\inputTask{zadKW001.tex}
\inputTask{zadKW002.tex}
\inputTask{zadKW007.tex}
%		Energy of a signal
\inputTask{zadKW003.tex}
\inputTask{zadKW004.tex}
\inputTask{zadKW005.tex}
\inputTask{zadKW006.tex}
\inputTask{zadKW016.tex}
%		Power of s signal
%		Effective value of a signal (RMS)
%	Energy signals vs power signals
%	Orthogonality. Orthogonal signals and vectors
%	Signal components
%		DC and AC signal components
%		Odd and even signal components
%Analysis of periodic signals using orthogonal series
%	Hilbert space
%	Orthogonal bases
%	Orthogonal series of functions
%	Trigonometric Fourier series
\inputTask{zadKW008.tex}
\inputTask{zadKW009.tex}
\inputTask{zadKW010.tex}
\inputTask{zadKW011.tex}
%	The influence of signal symmetries on the coefficients of the trigonometric Fourier series
%	Complex exponential Fourier series
\inputTask{zadKW012.tex}
\inputTask{zadKW013.tex}
\inputTask{zadKW014.tex}
\inputTask{zadKW015.tex}
%	The harmonic spectrum of a real signal
%	The relationship of the complex exponential and the trigonometric Fourier series
%	Linearity of Fourier series
%	The influence of signal symmetries on the coefficients of complex exponential Fourier series
%	The effect of signal shift in time on the complex exponential Fourier series
%	Spectrum of a product of two signals
%	Computing the power of a signal – the Parseval theorem
%Analysis of non-periodic signals. Fourier Transformation and Transform
%	An intuitive introduction
%	Definition
\begin{task}
Oblicz transformatę Fouriera sygnału $f(t)$ przedstawionego na rysunku oraz narysuj jego widmo amplitudowe i fazowe

\begin{figure}[H]
\centering
\begin{tikzpicture}
  %\draw (0,0) circle (1in);
  \draw[->] (-3.0,+0.0) -- (+5.0,+0.0) node[right] {$t$};
  \draw[->] (+0.0,-1.5) -- (+0.0,+1.5) node[above] {$f(t)$};
  \draw[-,red, thick] (-3.5,+0.0) -- (-1.0,+0.0) -- (-1.0,+1.0) -- (1.0,+1.0) -- (1.0,+0.0) -- (3.0,0.0);
  \draw[-] (-1.0-0.1,-0.1)--(-1.0+0.1,0.1) node[midway, below, outer sep=5pt,align=center] {$-\frac{\tau}{2}$};
  \draw[-] (+1.0-0.1,-0.1)--(+1.0+0.1,0.1) node[midway, below, outer sep=5pt] {$\frac{\tau}{2}$};
  \draw[-] (-0.1,1.0-0.1)--(+0.1,1.0+0.1) node[midway, above left] {$A$};
\end{tikzpicture}
\end{figure}

W pierwszej kolejności opiszmy sygnał za pomocą sygnałów elementarnych:

\begin{equation}
f(t)=A \cdot \Pi(\frac{t}{\tau}) 
\end{equation}

Transformatę Fouriera obliczamy ze wzoru:

\begin{equation}
F(\jmath \omega )=\int_{-\infty }^{\infty}f(t) \cdot e^{-\jmath \cdot \omega \cdot t}\cdot dt
\end{equation}

Podstawiamy do wzoru na transformatę wzór naszej funkcji

\begin{align*}
F(\jmath \omega )&=\int_{-\infty }^{\infty}f(t) \cdot e^{-\jmath \cdot \omega \cdot t}\cdot dt\\
&=\int_{-\infty }^{\infty} A\cdot\Pi(\frac{t}{\tau})  \cdot e^{-\jmath \cdot \omega \cdot t}\cdot dt\\
&=\int_{-\infty }^{-\frac{\tau}{2}} 0 \cdot e^{-\jmath \cdot \omega \cdot t}\cdot dt + \int_{-\frac{\tau}{2} }^{\frac{\tau}{2}} A\cdot e^{-\jmath \cdot \omega \cdot t}\cdot dt + \int_{\frac{\tau}{2} }^{\infty} 0\cdot e^{-\jmath \cdot \omega \cdot t}\cdot dt\\
&=\int_{-\infty }^{-\frac{\tau}{2}} 0 \cdot dt + \int_{-\frac{\tau}{2} }^{\frac{\tau}{2}} A\cdot e^{-\jmath \cdot \omega \cdot t}\cdot dt + \int_{\frac{\tau}{2} }^{\infty} 0 \cdot dt\\
&=0 + \int_{-\frac{\tau}{2} }^{\frac{\tau}{2}} A\cdot e^{-\jmath \cdot \omega \cdot t}\cdot dt + 0\\
&=A\cdot \int_{-\frac{\tau}{2} }^{\frac{\tau}{2}} \cdot e^{-\jmath \cdot \omega \cdot t}\cdot dt\\
&=\begin{Bmatrix}
z&=-\jmath \cdot \omega \cdot t\\
dz&=-\jmath \cdot \omega \cdot dt\\
dt&=\frac{1}{-\jmath \cdot \omega} \cdot dz\\
\end{Bmatrix}\\
&=A\cdot \int_{-\frac{\tau}{2} }^{\frac{\tau}{2}} \cdot e^{z}\cdot \frac{1}{-\jmath \cdot \omega} \cdot dz\\
&=A\cdot \frac{1}{-\jmath \cdot \omega} \cdot \int_{-\frac{\tau}{2} }^{\frac{\tau}{2}} \cdot e^{z}\cdot dz\\
&=A\cdot \frac{1}{-\jmath \cdot \omega} \cdot \left. e^{z}\right|_{-\frac{\tau}{2} }^{\frac{\tau}{2}}\\
&=A\cdot \frac{1}{-\jmath \cdot \omega} \cdot \left. e^{-\jmath \cdot \omega \cdot t}\right|_{-\frac{\tau}{2} }^{\frac{\tau}{2}}\\
%&=A\cdot \left(\left. \frac{e^{-\jmath \cdot \omega \cdot t}}{-\jmath \cdot \omega}\right |_{-\frac{\tau}{2}}^{\frac{\tau}{2}} \right )\\
&=\frac{A}{-\jmath \cdot \omega} \cdot \left(e^{-\jmath \cdot \omega \cdot \frac{\tau}{2}} - e^{-\jmath \cdot \omega \cdot (-\frac{\tau}{2})}\right)\\
&=\frac{A}{\jmath \cdot \omega} \cdot \left(e^{\jmath \cdot \omega \cdot \frac{\tau}{2}} - e^{-\jmath \cdot \omega \cdot \frac{\tau}{2}}\right)\\
&=\begin{Bmatrix*}[l]
sin(x)=\frac{e^{\jmath \cdot x} - e^{-\jmath \cdot x}}{2 \cdot \jmath}
\end{Bmatrix*}\\
&=\frac{2 \cdot A}{\omega} \cdot sin\left(\omega \cdot \frac{\tau}{2}\right)\\
&=\begin{Bmatrix*}[l]
\frac{sin(x)}{x}=Sa(x)
\end{Bmatrix*}\\
&=A\cdot \tau \cdot Sa\left(\omega \cdot \frac{\tau}{2}\right)
\end{align*}

Transformata sygnału $f(t)=A\cdot\Pi(\frac{t}{\tau})$ to $F(\jmath \omega)=A \cdot \tau \cdot Sa\left(\omega \cdot \frac{\tau}{2}\right)$


Narysujmy widmo sygnału $f(t)=A\cdot\Pi(\frac{t}{\tau})$ czyli:
\begin{equation}
F(\jmath \omega)=A \cdot \tau \cdot Sa\left(\omega \cdot \frac{\tau}{2}\right)
\end{equation}


\begin{figure}[H]
	\centering
	\begin{tikzpicture}
	\draw[->] (-5.0,+0.0) -- (+5.0,+0.0) node[right] {$\omega$};
	\draw[->] (+0.0,-1.5) -- (+0.0,+4.0) node[above] {$F(\jmath \cdot \omega)$};

	\draw[scale=1.0,domain=-4.0:4.0,samples=2000,smooth,variable=\x,red,thick] plot ({\x},{3*sinc(3.141592*\x)});

	\draw[-] (-1.0-0.1,-0.1)--(-1.0+0.1,0.1) node[midway, below, outer sep=5pt] {-$\frac{2 \pi}{\tau}$};
	\draw[-] (-2.0-0.1,-0.1)--(-2.0+0.1,0.1) node[midway, below, outer sep=5pt] {-$\frac{4 \pi}{\tau}$};
	\draw[-] (-3.0-0.1,-0.1)--(-3.0+0.1,0.1) node[midway, below, outer sep=5pt] {-$\frac{6 \pi}{\tau}$};
	\draw[-] (-4.0-0.1,-0.1)--(-4.0+0.1,0.1) node[midway, below, outer sep=5pt] {-$\frac{8 \pi}{\tau}$};
	\draw[-] (+1.0-0.1,-0.1)--(+1.0+0.1,0.1) node[midway, below, outer sep=5pt] {$\frac{2 \pi}{\tau}$};
	\draw[-] (+2.0-0.1,-0.1)--(+2.0+0.1,0.1) node[midway, below, outer sep=5pt] {$\frac{4 \pi}{\tau}$};
	\draw[-] (+3.0-0.1,-0.1)--(+3.0+0.1,0.1) node[midway, below, outer sep=5pt] {$\frac{6 \pi}{\tau}$};
	\draw[-] (+4.0-0.1,-0.1)--(+4.0+0.1,0.1) node[midway, below, outer sep=5pt] {$\frac{8 \pi}{\tau}$};
	\draw[-] (-0.1,+3.0-0.1)--(+0.1,+3.0+0.1) node[midway, left] {$A \cdot \tau$};

	\end{tikzpicture}
\end{figure}



Widmo amplitudowe obliczamy ze wzoru:
\begin{equation}
M(\omega)=\left | F(j \cdot \omega) \right |
\end{equation}

\begin{figure}[H]
	\centering
	\begin{tikzpicture}
	\draw[->] (-5.0,+0.0) -- (+5.0,+0.0) node[right] {$\omega$};
	\draw[->] (+0.0,-1.5) -- (+0.0,+4.0) node[above] {$M(\omega)$};
	
	\draw[scale=1.0,domain=-4.0:4.0,samples=2000,smooth,variable=\x,red,thick] plot ({\x},{abs(3*sinc(3.141592*\x))});
	
	\draw[-] (-1.0-0.1,-0.1)--(-1.0+0.1,0.1) node[midway, below, outer sep=5pt] {-$\frac{2 \pi}{\tau}$};
	\draw[-] (-2.0-0.1,-0.1)--(-2.0+0.1,0.1) node[midway, below, outer sep=5pt] {-$\frac{4 \pi}{\tau}$};
	\draw[-] (-3.0-0.1,-0.1)--(-3.0+0.1,0.1) node[midway, below, outer sep=5pt] {-$\frac{6 \pi}{\tau}$};
	\draw[-] (-4.0-0.1,-0.1)--(-4.0+0.1,0.1) node[midway, below, outer sep=5pt] {-$\frac{8 \pi}{\tau}$};
	\draw[-] (+1.0-0.1,-0.1)--(+1.0+0.1,0.1) node[midway, below, outer sep=5pt] {$\frac{2 \pi}{\tau}$};
	\draw[-] (+2.0-0.1,-0.1)--(+2.0+0.1,0.1) node[midway, below, outer sep=5pt] {$\frac{4 \pi}{\tau}$};
	\draw[-] (+3.0-0.1,-0.1)--(+3.0+0.1,0.1) node[midway, below, outer sep=5pt] {$\frac{6 \pi}{\tau}$};
	\draw[-] (+4.0-0.1,-0.1)--(+4.0+0.1,0.1) node[midway, below, outer sep=5pt] {$\frac{8 \pi}{\tau}$};
	\draw[-] (-0.1,+3.0-0.1)--(+0.1,+3.0+0.1) node[midway, left] {$A \cdot \tau$};
		
	\end{tikzpicture}
\end{figure}

Widmo fazowe obliczamy ze wzoru:
\begin{equation}
\Phi ( \omega )=arctg(\frac{Im\{F(\jmath \cdot \omega )\}}{Re\{F(\jmath \cdot \omega )\}})
\end{equation}

\begin{figure}[H]
	\centering
	\begin{tikzpicture}
	\draw[->] (-5.0,+0.0) -- (+5.0,+0.0) node[right] {$\omega$};
	\draw[->] (+0.0,-1.5) -- (+0.0,+2.0) node[above] {$\Phi(\omega)$};
	
	\draw[-,red] (-4.0,-1.0) -- (-3.0,-1.0);
	\draw[-,red] (-3.0,0.0) -- (-2.0,0.0);
	\draw[-,red] (-2.0,1.0) -- (-1.0,1.0);
	\draw[-,red] (-1.0,0.0) -- (1.0,0.0);
	\draw[-,red] (1.0,-1.0) -- (2.0,-1.0);
	\draw[-,red] (2.0,0.0) -- (3.0,0.0);
	\draw[-,red] (3.0,1.0) -- (4.0,1.0);
	
	  
	\draw[-] (-1.0-0.1,-0.1)--(-1.0+0.1,0.1) node[midway, below, outer sep=5pt] {-$\frac{2 \pi}{\tau}$};
	\draw[-] (-2.0-0.1,-0.1)--(-2.0+0.1,0.1) node[midway, below, outer sep=5pt] {-$\frac{4 \pi}{\tau}$};
	\draw[-] (-3.0-0.1,-0.1)--(-3.0+0.1,0.1) node[midway, below, outer sep=5pt] {-$\frac{6 \pi}{\tau}$};
	\draw[-] (-4.0-0.1,-0.1)--(-4.0+0.1,0.1) node[midway, below, outer sep=5pt] {-$\frac{8 \pi}{\tau}$};
	\draw[-] (+1.0-0.1,-0.1)--(+1.0+0.1,0.1) node[midway, below, outer sep=5pt] {$\frac{2 \pi}{\tau}$};
	\draw[-] (+2.0-0.1,-0.1)--(+2.0+0.1,0.1) node[midway, below, outer sep=5pt] {$\frac{4 \pi}{\tau}$};
	\draw[-] (+3.0-0.1,-0.1)--(+3.0+0.1,0.1) node[midway, below, outer sep=5pt] {$\frac{6 \pi}{\tau}$};
	\draw[-] (+4.0-0.1,-0.1)--(+4.0+0.1,0.1) node[midway, below, outer sep=5pt] {$\frac{8 \pi}{\tau}$};
	\draw[-] (-0.1,+1.0-0.1)--(+0.1,+1.0+0.1) node[midway, above left] {$\pi$};
	\draw[-] (-0.1,-1.0-0.1)--(+0.1,-1.0+0.1) node[midway, above left] {-$\pi$};
	
	\end{tikzpicture}
\end{figure}

\end{task}


%	Fourier Transform vs Laplace Transform
%	The Magnitude Spectrum and Phase Spectrum
%	Symmetries of the Fourier Transform for real-valued signals
%	Special case of Fourier Transform for symmetrical signals
%	Theorems describing the properties of Fourier Transformation
%		Linearity
%		Shift theorem – shifting in time domain
%		Shifting in frequency domain (also known as modulation theorem)
%		Scaling theorem (also called the similarity theorem)
%		Time-frequency duality (also known as the symmetry theorem)
%		Derivative theorem (differentiation in time domain)
%		Integration theorem
%	Calculating energy of the signal from its Fourier transform. The Parseval's theorem
%	Generalization of the Fourier transformation for infinite energy signals
%	Fourier transform of a periodic signal
%	Calculating the power of a signal from its Fourier transform. The Parseval's theorem for power signals
%Processing of signals by linear and time invariant (LTI) systems
%	Introduction to LTI systems. Fundamental properties
%	Impulse response of an LTI system
%	Impulse response of a causal system
%	The response of an LTI system to arbitrary input
%	Properties of linear convolution
%	Frequency response
%	Determining the frequency response of an electronic circuit
%Filters
%	Sampling. Discrete-time signals
%	Introduction to discrete signals
%	Spectrum of a sampled signal
%	Spectral efect of sampling a continuous signal
%	Reconstruction of the continuous signal from its samples
%	Non-periodic and periodic discrete-time signals
%	Fourier transforms of discrete-time signals
%	Processing of discrete-time signals
%	Frequency response of discrete-time LTI systems

%====================================================
\end{document}

%Tom pierwszy zadania
%Tom drugi rozwiązania

%TODO
