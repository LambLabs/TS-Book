\documentclass[a4paper,11pt]{LambBook} %6 inch 800x600

\usepackage[T1]{fontenc}
\usepackage[polish]{babel}
\usepackage[utf8]{inputenc}
\usepackage{lmodern}
\usepackage{mathtools}
\usepackage{amsmath}
\usepackage{cancel}
\usepackage{graphicx} 
\usepackage{array}
\usepackage{float} %Wymuszenie wstawienia obrazka w miejscu wstawienia
%\usepackage[europeancurrents, europeanvoltages, europeanresistors, americaninductors, europeanports]{circuitikz} %Do rysowania obwodów elektrycznych
\usepackage{tikz}
\selectlanguage{polish}

\allowdisplaybreaks
%https://www.codecogs.com/latex/eqneditor.php

\def\booktitle{Teoria Sygnałów w zadaniach}
\def\bookauthors{Tomasz Grajek, Jakub Stankowski, Krzysztof Wegner}

\usepackage[pdftex,
        unicode=true, % Aby działały polskie literki
        colorlinks=true,
        urlcolor=rltblue,       % \href{...}{...} external (URL)
        filecolor=rltgreen,     % \href{...} local file
        linkcolor=rltred,       % \ref{...} and \pageref{...}
        citecolor=blue,
        pdfstartview={FitV},
        pdftitle={\booktitle},
        pdfauthor={\bookauthors},
        pdfsubject={Teoria Sygnałów},
        pdfkeywords={Teoria Sygnałów, Zadania},
        pdfproducer={pdfLaTeX},
        %pdfadjustspacing=1,
        pagebackref=false, %activate back references inside bibliography. Must be specified as part
        bookmarksopen=true]{hyperref}

\setcounter{secnumdepth}{2}
%====================================================
\begin{document}
%====================================================
\definecolor{rltred}{rgb}{0.75,0,0} %Definicja kolorów
\definecolor{rltgreen}{rgb}{0,0.5,0}
\definecolor{rltblue}{rgb}{0,0,0.75}
%====================================================
\title{\booktitle}
\author{\bookauthors}
%====================================================
%Strona Tytułowa
\label{page:titlepage}
\makebooktitle
%====================================================
%Druga Strona z ISBN'em
\thispagestyle{empty}
\begin{flushleft}
\textsc{Politechnika Poznańska}\\%
Wydział Elektroniki i Telekomunikacji\\%
Katedra Telekomunikacji Multimedialnej i~Mikroelektroniki\\[1em]

pl. M. Skłodowskiej-Curie 5\\
60-965 Poznań\\[1em]

www.et.put.poznan.pl\\
www.multimedia.edu.pl
\end{flushleft}

\vfill

\begin{flushleft}
Copyright © Krzysztof Wegner, 2019\\
Wszelkie prawa zastrzeżone\\
%\ISBN\\
Wydrukowano w Polsce\\[1em]

Książka współfinansowana ze środków Unii Europejskiej w ramach Europejskiego Funduszu Społecznego.\\
\end{flushleft}
\clearpage
%====================================================
%\input{01_opor_zastepczy/_opor_zastepczy.tex}
%\input{02_zrodla/_zrodla.tex}
%\input{03_prawa_kirchhoffa/_prawa_kirchhoffa.tex}
%\input{04_metoda_wezlowa/_metoda_wezlowa.tex}
%\input{05_metoda_oczkowa/_metoda_oczkowa.tex}
%\input{06_thevenin_norton/_thevenin_norton.tex}
%\input{07_moc/_moc.tex}
%\input{08_superpozycja/_metoda_superpozycji.tex}
%\input{08_uklady_nieliniowe/_uklady_nieliniowe.tex}
%\input{09_impedancja_zastepcza/_impedancja_zastepcza.tex}
%\input{10_metoda_oczkowa_wezlowa_AC/_metoda_oczkowa_wezlowa_AC.tex}
%\input{11_moc_AC/_moc_AC.tex}


%Fundamental concepts and measures
%	Signals and their models
%	Signal classes and examples
%		Continuous, discrete, analogue, quantized and digital signals
%		Periodic signals
%		Sinusoidal signals: real and complex
%		Non-periodic signals
%	Basic signal metrics
%		Amplitude
%		Mean value
Oblicz wartość średnią okresowego sygnału $f(t)$ przedstawionego na rysunku 

\begin{figure}[H]
\centering
\begin{tikzpicture}
  %\draw (0,0) circle (1in);
  \draw[->] (-3.0,+0.0) -- (+5.0,+0.0) node[right] {$t$};
  \draw[->] (+0.0,-1.5) -- (+0.0,+1.5) node[above] {$f(t)$};
  \draw[-,red, thick] (-2.5,+0.0) -- (-2.0,+0.0) -- (-2.0,+1.0) -- (-1.0,+1.0) -- (-1.0,+0.0)--(+0.0,+0.0) -- (+0.0,+1.0) -- (+1.0,+1.0) -- (+1.0,+0.0) -- (+2.0,+0.0) -- (+2.0,+1.0) -- (3.0,1.0) -- (3.0,0.0) -- (3.5,0.0);
  %\draw[-] (-1.0-0.1,-0.1)--(-1.0+0.1,0.1) node[midway, below, outer sep=10pt,align=center] {$-\frac{T}{2}$};
  \draw[-] (-1.0-0.1,-0.1)--(-1.0+0.1,0.1) node[midway, below, outer sep=5pt,align=center] {$-\frac{T}{2}$};
  \draw[-] (+1.0-0.1,-0.1)--(+1.0+0.1,0.1) node[midway, below, outer sep=5pt] {$\frac{T}{2}$};
  \draw[-] (+2.0-0.1,-0.1)--(+2.0+0.1,0.1) node[midway, below, outer sep=5pt] {$T$};
  \draw[-] (-0.1,1.0-0.1)--(+0.1,1.0+0.1) node[midway, left] {$A$};
\end{tikzpicture}
\end{figure}

W pierwszej kolejności należy opisać sygnał za pomocą wzoru.

\begin{equation}
   f(x)=\begin{cases}A & t \in \left (  0+k \cdot T; \frac{T}{2}+k \cdot T \right ) \\0 & t \in \left ( \frac{T}{2}+k \cdot T; T +k \cdot T\right )\end{cases} \wedge k \in C %\mathbb{C}
\end{equation}

Wartość średnią sygnału wyznaczamy z wzoru

\begin{equation}
\bar{f}=\frac{1}{T}\int_{T}f(t) \cdot dt
\end{equation}

Podstawiamy do wzoru wzór naszej funkcji w pierwszym okresie $k=0$

\begin{equation}
\begin{aligned}
\bar{f} &=\frac{1}{T}\int_{T}f(t) \cdot dt =\\
&=\frac{1}{T} \left( \int_{0}^{\frac{T}{2}} A \cdot dt + \int_{\frac{T}{2}}^{T} 0 \cdot dt \right ) = \\
&=\frac{1}{T} \left( A \cdot \int_{0}^{\frac{T}{2}} dt + 0 \right ) = \\
&=\frac{1}{T} \left( A \cdot \left. t\right |_{0}^{\frac{T}{2}} \right ) = \\
&=\frac{A}{T} \cdot \left. t\right |_{0}^{\frac{T}{2}} = \\
&=\frac{A}{T} \cdot \left( \frac{T}{2} - 0 \right ) = \\
&=\frac{A}{T} \cdot \left( \frac{T}{2} \right ) = \\
&=\frac{A}{\cancel{T}} \cdot \left( \frac{\cancel{T}}{2} \right ) = \\
&=\frac{A}{2} 
\end{aligned}
\end{equation}

Średnia wartość sygnału wynosi $\frac{A}{2}$
\newpage

Oblicz wartość średnią sygnału $f(t)=\mathbf{1}(t)\cdot e^{-a\cdot t}\cdot sin\left(\frac{2\pi}{T}\cdot t \right )$ przedstawionego na rysunku 

\begin{figure}[H]
\centering
\begin{tikzpicture}
  %\draw (0,0) circle (1in);
  \draw[->] (-3.0,+0.0) -- (+5.0,+0.0) node[right] {$t$};
  \draw[->] (+0.0,-1.5) -- (+0.0,+1.5) node[above] {$f(t)$};
  \draw[-,red, thick] (-2.5,+0.0) -- (+0.0,+0.0);
  %\draw[-] (-1.0-0.1,-0.1)--(-1.0+0.1,0.1) node[midway, below, outer sep=10pt,align=center] {$-\frac{T}{2}$};
  \draw[-] (+0.5-0.1,-0.1)--(+0.5+0.1,0.1) node[midway, below, outer sep=5pt] {$\frac{T}{2}$};
  \draw[-] (+1.0-0.1,-0.1)--(+1.0+0.1,0.1) node[midway, below, outer sep=5pt] {$T$};
  \draw[-] (-0.1,1.0-0.1)--(+0.1,1.0+0.1) node[midway, left] {$A$};
  
  \draw[scale=1.0,domain=0:4.0,smooth,variable=\x,red,thick] plot ({\x},{exp(-0.5*\x)*sin(\x*180.0/3.141592*2*3.141592/1.0)});
  
  \draw[scale=1.0,domain=0:4.0,smooth,variable=\x,red,thick,dashed] plot ({\x},{exp(-0.5*\x)});
\end{tikzpicture}
\end{figure}

Wartość średnią sygnału wyznaczamy z wzoru

\begin{equation}
\bar{f}=\lim_{\tau \rightarrow \infty }\frac{1}{\tau}\int_{-\frac{\tau}{2}}^{\frac{\tau}{2}}f(t) \cdot dt
\end{equation}

Podstawiamy do wzoru wzór naszej funkcji

%\begin{equation}
%\begin{aligned}
\begin{align*}
\bar{f} &=\lim_{\tau \rightarrow \infty }\frac{1}{\tau}\int_{-\frac{\tau}{2}}^{\frac{\tau}{2}}f(t) \cdot dt\\
 &=\lim_{\tau \rightarrow \infty }\frac{1}{\tau}\int_{-\frac{\tau}{2}}^{\frac{\tau}{2}} \mathbf{1}(t)\cdot e^{-a\cdot t}\cdot sin\left(\frac{2\pi}{T}\cdot t \right) \cdot dt\\
 &=\lim_{\tau \rightarrow \infty }\frac{1}{\tau}\left(
 \int_{-\frac{\tau}{2}}^{0} 0 \cdot e^{-a\cdot t}\cdot sin\left(\frac{2\pi}{T}\cdot t \right) \cdot dt +
 \int_{0}^{\frac{\tau}{2}} 1 \cdot e^{-a\cdot t}\cdot sin\left(\frac{2\pi}{T}\cdot t \right) \cdot dt \right)\\
 &=\lim_{\tau \rightarrow \infty }\frac{1}{\tau}\left(
 \int_{-\frac{\tau}{2}}^{0} 0 \cdot dt +
 \int_{0}^{\frac{\tau}{2}} e^{-a\cdot t}\cdot sin\left(\frac{2\pi}{T}\cdot t \right) \cdot dt \right)\\
 &=\lim_{\tau \rightarrow \infty }\frac{1}{\tau}\left(
 0 +
 \int_{0}^{\frac{\tau}{2}} e^{-a\cdot t}\cdot sin\left(\frac{2\pi}{T}\cdot t \right) \cdot dt \right)\\
 &=\lim_{\tau \rightarrow \infty }\frac{1}{\tau}\left(
 \int_{0}^{\frac{\tau}{2}} e^{-a\cdot t}\cdot sin\left(\frac{2\pi}{T}\cdot t \right) \cdot dt \right)\\
 &=\begin{Bmatrix*}[l]
 u=sin(\frac{2\pi}{T}\cdot t) & dv = e^{-a \cdot t}\cdot dt\\ 
 du=\frac{T}{2\pi} \cdot cos(\frac{2\pi}{T}\cdot t)\cdot dt & v=-\frac{1}{a}\cdot e^{-a\cdot t}
 \end{Bmatrix*}\\
 &=\lim_{\tau \rightarrow \infty }\frac{1}{\tau}\left(
 \left. -\frac{1}{a}\cdot e^{-a\cdot t} \cdot sin \left(\frac{2\pi}{T}\cdot t\right) \right|_{0}^{\frac{\tau}{2}}
 -\int_{0}^{\frac{\tau}{2}} -\frac{1}{a}\cdot e^{-a\cdot t} \cdot \frac{T}{2\pi} \cdot cos\left(\frac{2\pi}{T}\cdot t\right)\cdot dt
 \right)\\
 &=\lim_{\tau \rightarrow \infty }\frac{1}{\tau}\left(
 \left( -\frac{1}{a}\cdot e^{-a\cdot \frac{\tau}{2}} \cdot sin \left(\frac{2\pi}{T}\cdot \frac{\tau}{2}\right) + \frac{1}{a}\cdot e^{-a\cdot 0} \cdot sin \left(\frac{2\pi}{T}\cdot 0\right) \right)\right.\\
 &\left.+\frac{1}{a} \cdot \frac{T}{2\pi} \cdot \int_{0}^{\frac{\tau}{2}} \cdot e^{-a\cdot t} \cdot cos\left(\frac{2\pi}{T}\cdot t\right)\cdot dt \right)\\
 &=\begin{Bmatrix*}[l]
 u=cos(\frac{2\pi}{T}\cdot t) & dv = e^{-a \cdot t}\cdot dt\\ 
 du=-\frac{T}{2\pi} \cdot sin(\frac{2\pi}{T}\cdot t)\cdot dt & v=-\frac{1}{a}\cdot e^{-a\cdot t}
 \end{Bmatrix*}\\
 &=\lim_{\tau \rightarrow \infty }\frac{1}{\tau}\left(
 \left( -\frac{1}{a}\cdot e^{-a\cdot \frac{\tau}{2}} \cdot sin \left(\frac{2\pi}{T}\cdot \frac{\tau}{2}\right) + \frac{1}{a}\cdot e^{-a\cdot 0} \cdot sin \left(\frac{2\pi}{T}\cdot 0\right) \right)\right.\\
 &\left.+\frac{1}{a} \cdot \frac{T}{2\pi} \cdot 
 \left(
 \left. -\frac{1}{a}\cdot e^{-a\cdot t} \cdot cos \left(\frac{2\pi}{T}\cdot t\right) \right|_{0}^{\frac{\tau}{2}}
 -\int_{0}^{\frac{\tau}{2}} -\frac{1}{a}\cdot e^{-a\cdot t} \cdot \frac{T}{2\pi} \cdot sin\left(\frac{2\pi}{T}\cdot t\right)\cdot dt
 \right)
 \right)\\
 &=\lim_{\tau \rightarrow \infty }\frac{1}{\tau}\left(
 \left( -\frac{1}{a}\cdot e^{-a\cdot \frac{\tau}{2}} \cdot sin \left(\frac{2\pi}{T}\cdot \frac{\tau}{2}\right) + \frac{1}{a}\cdot 1 \cdot 0 \right)\right.\\
 &\left.+\frac{1}{a} \cdot \frac{T}{2\pi} \cdot 
 \left(
 \left( -\frac{1}{a}\cdot e^{-a\cdot \frac{\tau}{2}} \cdot cos \left(\frac{2\pi}{T}\cdot \frac{\tau}{2}\right) + \frac{1}{a}\cdot e^{-a\cdot 0} \cdot cos \left(\frac{2\pi}{T}\cdot 0\right) \right)
 \right.\right. \\
 &\left.\left.+\frac{1}{a}\cdot \frac{T}{2\pi} \cdot \int_{0}^{\frac{\tau}{2}} e^{-a\cdot t} \cdot  sin\left(\frac{2\pi}{T}\cdot t\right)\cdot dt
 \right)
 \right)\\
 &=\lim_{\tau \rightarrow \infty }\frac{1}{\tau}\left(
 \left( -\frac{1}{a}\cdot e^{-a\cdot \frac{\tau}{2}} \cdot sin \left(\frac{2\pi}{T}\cdot \frac{\tau}{2}\right) + 0 \right)\right.\\
 &\left.+\frac{1}{a} \cdot \frac{T}{2\pi} \cdot 
 \left(
 \left( -\frac{1}{a}\cdot e^{-a\cdot \frac{\tau}{2}} \cdot cos \left(\frac{2\pi}{T}\cdot \frac{\tau}{2}\right) + \frac{1}{a}\cdot 1 \cdot 1 \right)
 \right.\right. \\
 &\left.\left.+\frac{1}{a}\cdot \frac{T}{2\pi} \cdot \int_{0}^{\frac{\tau}{2}} e^{-a\cdot t} \cdot  sin\left(\frac{2\pi}{T}\cdot t\right)\cdot dt
 \right)
 \right)\\
 &=\lim_{\tau \rightarrow \infty }\frac{1}{\tau}\left(
 -\frac{1}{a}\cdot e^{-a\cdot \frac{\tau}{2}} \cdot sin \left(\frac{2\pi}{T}\cdot \frac{\tau}{2}\right)\right.\\
 &\left.-\frac{1}{a^2} \cdot \frac{T}{2\pi} \cdot 
 e^{-a\cdot \frac{\tau}{2}} \cdot cos \left(\frac{2\pi}{T}\cdot \frac{\tau}{2}\right) + \frac{1}{a^2}\cdot \frac{T}{2\pi}
 \right. \\
 &\left.+\frac{1}{a^2}\cdot \frac{T^2}{4\pi^2} \cdot \int_{0}^{\frac{\tau}{2}} e^{-a\cdot t} \cdot  sin\left(\frac{2\pi}{T}\cdot t\right)\cdot dt
 \right)\\
 &=\begin{Bmatrix*}[l]
 -\frac{1}{a}\cdot e^{-a\cdot \frac{\tau}{2}} \cdot sin \left(\frac{2\pi}{T}\cdot \frac{\tau}{2}\right)\\
 -\frac{1}{a^2} \cdot \frac{T}{2\pi} \cdot 
 e^{-a\cdot \frac{\tau}{2}} \cdot cos \left(\frac{2\pi}{T}\cdot \frac{\tau}{2}\right) + \frac{1}{a^2}\cdot \frac{T}{2\pi}
  \\
 +\frac{1}{a^2}\cdot \frac{T^2}{4\pi^2} \cdot \int_{0}^{\frac{\tau}{2}} e^{-a\cdot t} \cdot  sin\left(\frac{2\pi}{T}\cdot t\right)\cdot dt = \int_{0}^{\frac{\tau}{2}} e^{-a\cdot t} \cdot  sin\left(\frac{2\pi}{T}\cdot t\right)\cdot dt
 \end{Bmatrix*}\\
 &=\begin{Bmatrix*}[l]
 -\frac{1}{a}\cdot e^{-a\cdot \frac{\tau}{2}} \cdot sin \left(\frac{2\pi}{T}\cdot \frac{\tau}{2}\right)
 -\frac{1}{a^2} \cdot \frac{T}{2\pi} \cdot 
 e^{-a\cdot \frac{\tau}{2}} \cdot cos \left(\frac{2\pi}{T}\cdot \frac{\tau}{2}\right) + \frac{1}{a^2}\cdot \frac{T}{2\pi}
  \\
 = \int_{0}^{\frac{\tau}{2}} e^{-a\cdot t} \cdot  sin\left(\frac{2\pi}{T}\cdot t\right)\cdot dt - \frac{1}{a^2}\cdot \frac{T^2}{4\pi^2} \cdot \int_{0}^{\frac{\tau}{2}} e^{-a\cdot t} \cdot  sin\left(\frac{2\pi}{T}\cdot t\right)\cdot dt
 \end{Bmatrix*}\\
 &=\begin{Bmatrix*}[l]
 -\frac{1}{a}\cdot e^{-a\cdot \frac{\tau}{2}} \cdot sin \left(\frac{2\pi}{T}\cdot \frac{\tau}{2}\right)
 -\frac{1}{a^2} \cdot \frac{T}{2\pi} \cdot 
 e^{-a\cdot \frac{\tau}{2}} \cdot cos \left(\frac{2\pi}{T}\cdot \frac{\tau}{2}\right) + \frac{1}{a^2}\cdot \frac{T}{2\pi}
 \\
 = \left(1 - \frac{1}{a^2}\cdot \frac{T^2}{4\pi^2}\right) \cdot \int_{0}^{\frac{\tau}{2}} e^{-a\cdot t} \cdot  sin\left(\frac{2\pi}{T}\cdot t\right)\cdot dt
 \end{Bmatrix*}\\
 &=\begin{Bmatrix*}[l]
 \frac{-\frac{1}{a}\cdot e^{-a\cdot \frac{\tau}{2}} \cdot sin \left(\frac{2\pi}{T}\cdot \frac{\tau}{2}\right)
 -\frac{1}{a^2} \cdot \frac{T}{2\pi} \cdot 
 e^{-a\cdot \frac{\tau}{2}} \cdot cos \left(\frac{2\pi}{T}\cdot \frac{\tau}{2}\right) + \frac{1}{a^2}\cdot \frac{T}{2\pi}}
 {\left(1 - \frac{1}{a^2}\cdot \frac{T^2}{4\pi^2}\right)}
 \\
 = \int_{0}^{\frac{\tau}{2}} e^{-a\cdot t} \cdot  sin\left(\frac{2\pi}{T}\cdot t\right)\cdot dt
 \end{Bmatrix*}\\
 &=\lim_{\tau \rightarrow \infty }\frac{1}{\tau}\left(\frac{-\frac{1}{a}\cdot e^{-a\cdot \frac{\tau}{2}} \cdot sin \left(\frac{2\pi}{T}\cdot \frac{\tau}{2}\right)
   -\frac{1}{a^2} \cdot \frac{T}{2\pi} \cdot 
   e^{-a\cdot \frac{\tau}{2}} \cdot cos \left(\frac{2\pi}{T}\cdot \frac{\tau}{2}\right) + \frac{1}{a^2}\cdot \frac{T}{2\pi}}
 {\left(1 - \frac{1}{a^2}\cdot \frac{T^2}{4\pi^2}\right)} \right)\\
 &=0
\end{align*}
%\end{aligned}
%\end{equation}

Średnia wartość sygnału wynosi $0$
\newpage

%		Energy of a signal
Oblicz energię sygnału okresowego $f(t)=A + B \cdot sin\left(\frac{2\pi}{T}\cdot t \right )$ przedstawionego na rysunku 

\begin{figure}[H]
\centering
\begin{tikzpicture}
  %\draw (0,0) circle (1in);
  \draw[->] (-3.0,+0.0) -- (+5.0,+0.0) node[right] {$t$};
  \draw[->] (+0.0,-1.5) -- (+0.0,+2.0) node[above] {$f(t)$};
  \draw[-,red, dashed] (-4.0,+1.4) -- (+4.0,+1.4);
  \draw[-,red, dashed] (-4.0,-0.6) -- (+4.0,-0.6);
  \draw[-,red, dashed] (-4.0,+0.4) -- (+4.0,+0.4);
  \draw[-,red, dashed] (+1.0,+0.0) -- (+1.0,+0.4);
  \draw[-,red, dashed] (+2.0,+0.0) -- (+2.0,+0.4);
  %\draw[-] (-1.0-0.1,-0.1)--(-1.0+0.1,0.1) node[midway, below, outer sep=10pt,align=center] {$-\frac{T}{2}$};
  \draw[-] (+1.0-0.1,-0.1)--(+1.0+0.1,0.1) node[midway, below, outer sep=5pt] {$\frac{T}{2}$};
  \draw[-] (+2.0-0.1,-0.1)--(+2.0+0.1,0.1) node[midway, below, outer sep=5pt] {$T$};
  \draw[-] (-0.1,+1.4-0.1)--(+0.1,+1.4+0.1) node[midway, left] {$A+B$};
  \draw[-] (-0.1,+0.4-0.1)--(+0.1,+0.4+0.1) node[midway, left] {$A$};
  \draw[-] (-0.1,-0.6-0.1)--(+0.1,-0.6+0.1) node[midway, left] {$A-B$};
  
  \draw[scale=1.0,domain=-4.0:4.0,smooth,variable=\x,red,thick] plot ({\x},{0.4+ sin(\x*180.0/3.141592*2*3.141592/2.0)});

\end{tikzpicture}
\end{figure}

Energię sygnału okresowego wyznaczamy z wzoru

\begin{equation}
E=\int_{T}^{}\left|f(t)\right|^2 \cdot dt
\end{equation}

Podstawiamy do wzoru wzór naszej funkcji

%\begin{equation}
%\begin{aligned}
\begin{align*}
E&=\int_{T}^{}\left|f(t)\right|^2 \cdot dt\\
 &=\int_{0}^{T}\left|A + B \cdot sin\left(\frac{2\pi}{T}\cdot t \right) \right|^2 \cdot dt\\ 
 &=\int_{0}^{T}\left(A + B \cdot sin\left(\frac{2\pi}{T}\cdot t \right) \right)^2 \cdot dt\\ 
 &=\int_{0}^{T}\left(A^2 + 2\cdot A \cdot B \cdot sin\left(\frac{2\pi}{T}\cdot t \right) + B^2 \cdot sin^2\left(\frac{2\pi}{T}\cdot t \right) \right) \cdot dt\\
 &=\int_{0}^{T}A^2 \cdot dt + \int_{0}^{T} 2\cdot A \cdot B \cdot sin\left(\frac{2\pi}{T}\cdot t \right) \cdot dt + \int_{0}^{T} B^2 \cdot sin^2\left(\frac{2\pi}{T}\cdot t \right)  \cdot dt\\
 &=A^2 \cdot \int_{0}^{T} dt + 2\cdot A \cdot B \cdot \int_{0}^{T} sin\left(\frac{2\pi}{T}\cdot t \right) \cdot dt + B^2 \cdot \int_{0}^{T}  sin^2\left(\frac{2\pi}{T}\cdot t \right)  \cdot dt\\
 &=\begin{Bmatrix*}[l]
 z=\frac{2\pi}{T} \cdot t\\
 dz = \frac{2\pi}{T} \cdot dt & dt = \frac{dz}{\frac{2\pi}{T}} =\frac{T}{2\pi} \cdot dz 
 \end{Bmatrix*}\\
 &=A^2 \cdot \left. t \right|_{0}^{T} + 2\cdot A \cdot B \cdot \int_{0}^{T} sin\left(z \right) \cdot \frac{T}{2\pi} \cdot dz + B^2 \cdot \int_{0}^{T} \frac{1}{2} \cdot \left(1 - cos\left(2 \cdot \frac{2\pi}{T}\cdot t \right) \right)  \cdot dt\\
 &=A^2 \cdot \left( T - 0 \right) + 2\cdot A \cdot B \cdot \frac{T}{2\pi} \cdot \int_{0}^{T} sin\left(z \right) \cdot dz + B^2 \cdot \frac{1}{2} \cdot \int_{0}^{T} \left(1 - cos\left(2 \cdot \frac{2\pi}{T}\cdot t \right) \right)  \cdot dt\\
 &=A^2 \cdot T + 2\cdot A \cdot B \cdot \frac{T}{2\pi} \cdot \left( \left.-cos\left(z \right)\right|_{0}^{T}  \right) + B^2 \cdot \frac{1}{2} \cdot \left( \int_{0}^{T} 1 \cdot dt - \int_{0}^{T} cos\left(2 \cdot \frac{2\pi}{T}\cdot t \right)  \cdot dt \right)\\
 &=\begin{Bmatrix*}[l]
 w=2\cdot \frac{2\pi}{T} \cdot t\\
 dw = 2\cdot \frac{2\pi}{T} \cdot dt & dt = \frac{dw}{\frac{4\pi}{T}} =\frac{T}{4\pi} \cdot dw 
 \end{Bmatrix*}\\
 &=A^2 \cdot T + 2\cdot A \cdot B \cdot \frac{T}{2\pi} \cdot \left( \left.-cos\left(\frac{2\pi}{T} \cdot t \right)\right|_{0}^{T}  \right) + B^2 \cdot \frac{1}{2}\cdot \left( \left. t \right|_{0}^{T} - \int_{0}^{T} cos\left(w \right)  \cdot \frac{T}{4\pi} \cdot dw \right)\\
 &=A^2 \cdot T + 2\cdot A \cdot B \cdot \frac{T}{2\pi} \cdot  \left(-cos\left(\frac{2\pi}{T} \cdot T \right)+cos\left(\frac{2\pi}{T} \cdot 0 \right)\right) + B^2 \cdot \frac{1}{2} \cdot \left( \left(T - 0 \right) - \frac{T}{4\pi} \cdot \int_{0}^{T} cos\left(w \right)  \cdot dw \right)\\
 &=A^2 \cdot T + 2\cdot A \cdot B \cdot \frac{T}{2\pi} \cdot  \left(-cos\left(2\pi \right)+cos\left(0 \right)\right) + B^2 \cdot \frac{1}{2} \cdot \left( T - \frac{T}{4\pi} \cdot \left. -sin\left(w \right)  \right|_{0}^{T} \right)\\
 &=A^2 \cdot T + 2\cdot A \cdot B \cdot \frac{T}{2\pi} \cdot  \left(-1+1\right) + B^2 \cdot \frac{1}{2} \cdot \left( T + \frac{T}{4\pi} \cdot \left. sin\left(2\cdot \frac{2\pi}{T}\cdot t \right)  \right|_{0}^{T} \right)\\
 &=A^2 \cdot T + 2\cdot A \cdot B \cdot \frac{T}{2\pi} \cdot 0 + B^2 \cdot \frac{1}{2} \cdot \left( T + \frac{T}{4\pi} \cdot \left( sin\left(2\cdot \frac{2\pi}{T}\cdot T \right) - sin\left(2\cdot \frac{2\pi}{T}\cdot 0 \right) \right) \right)\\
 &=A^2 \cdot T + B^2 \cdot \frac{1}{2} \cdot \left( T + \frac{T}{4\pi} \cdot \left( sin\left(4\pi \right) - sin\left( 0 \right) \right) \right)\\
 &=A^2 \cdot T + B^2 \cdot \frac{1}{2} \cdot \left( T + \frac{T}{4\pi} \cdot \left( 0 - 0 \right) \right)\\
 &=A^2 \cdot T + B^2 \cdot \frac{1}{2} \cdot \left( T \right)\\
 &=A^2 \cdot T + \frac{B^2}{2} \cdot T\\
\end{align*}
%\end{aligned}
%\end{equation}

Energia sygnału wynosi $A^2 \cdot T + \frac{B^2}{2} \cdot T$
\newpage

Oblicz energię sygnału okresowego $f(t)$ przedstawionego na rysunku 

\begin{figure}[H]
\centering
\begin{tikzpicture}
  %\draw (0,0) circle (1in);
  \draw[->] (-3.0,+0.0) -- (+5.0,+0.0) node[right] {$t$};
  \draw[->] (+0.0,-1.5) -- (+0.0,+2.0) node[above] {$f(t)$};
  \draw[-,red, thick] (-1.5,0.5) -- (-1.0,1.0) -- (-1.0,0.0) -- (0.0,1.0) -- (0.0,+0.0) -- (+1.0,+1.0) -- (1.0,+1.0) -- (+1.0,+0.0) -- (1.0,+0.0) -- (+2.0,+1.0) -- (+2.0,+0.0) -- (2.5,0.5);
  \draw[-,red, dashed] (-1.5,1.0) -- (2.5,1.0);
  %\draw[-] (-1.0-0.1,-0.1)--(-1.0+0.1,0.1) node[midway, below, outer sep=10pt,align=center] {$-\frac{T}{2}$};
  \draw[-] (-1.0-0.1,-0.1)--(-1.0+0.1,0.1) node[midway, below, outer sep=5pt] {$-T$};
  \draw[-] (+1.0-0.1,-0.1)--(+1.0+0.1,0.1) node[midway, below, outer sep=5pt] {$T$};
  \draw[-] (+2.0-0.1,-0.1)--(+2.0+0.1,0.1) node[midway, below, outer sep=5pt] {$2 \cdot T$};
  \draw[-] (-0.1,+1.0-0.1)--(+0.1,+1.0+0.1) node[midway, left] {$A$};

\end{tikzpicture}
\end{figure}

W pierwszej kolejności należy ustalić wzór funkcji przedstawionej na rysunku.
Jest to funkcja odcinkowa. W pierwszym okresie możemy ja opisać ogólnym równaniem prostej:

\begin{equation}
f(t) = a \cdot t + b
\end{equation}

 W pierwszym okresie wykres funkcji jest prostą przechodzącą przez dwa punkty: $(0,0)$ oraz $(T,A)$. Możemy wiec napisać układ równań rozwiązać go i znaleźć nie znane parametry $a$ i $b$.  

\begin{align*}
&\left\{\begin{matrix}
0 = a\cdot 0 +b\\ 
A = a\cdot T +b
\end{matrix}\right. \\
&\left\{\begin{matrix}
0 = b\\ 
A = a\cdot T +b
\end{matrix}\right. \\
&\left\{\begin{matrix}
0 = b\\ 
A = a\cdot T +0
\end{matrix}\right. \\
&\left\{\begin{matrix}
0 = b\\ 
\frac{A}{T} = a
\end{matrix}\right.
\end{align*}
A więc funkcję przedstawioną na rysunku, w pierwszy okresie można opisać wzorem
\begin{align*}
f(t) = \frac{A}{T}\cdot t
\end{align*}
I ogólniej całą funkcję można wyrazić następującym wzorem
\begin{align*}
f(t) = \frac{A}{T}\cdot \left(t-k\cdot T\right) \wedge k \in C
\end{align*}
Energię sygnału okresowego wyznaczamy z wzoru


\begin{equation}
E=\int_{T}^{}\left|f(t)\right|^2 \cdot dt
\end{equation}

Podstawiamy do wzoru wzór naszej funkcji

%\begin{equation}
%\begin{aligned}
\begin{align*}
E&=\int_{T}^{}\left|f(t)\right|^2 \cdot dt\\
 &=\int_{0}^{T}\left|\frac{A}{T}\cdot t \right|^2 \cdot dt\\ 
 &=\int_{0}^{T}\left(\frac{A}{T}\cdot t \right)^2 \cdot dt\\ 
 &=\int_{0}^{T}\frac{A^2}{T^2}\cdot t^2 \cdot dt\\ 
 &=\frac{A^2}{T^2}\cdot \int_{0}^{T} t^2 \cdot dt\\ 
 &=\frac{A^2}{T^2}\cdot \left(\left. \frac{1}{3}\cdot t^3 \right|_{0}^{T}\right)\\ 
 &=\frac{A^2}{T^2}\cdot \left(\frac{1}{3}\cdot T^3  - \frac{1}{3}\cdot 0^3 \right)\\
 &=\frac{A^2}{T^2}\cdot \left(\frac{1}{3}\cdot T^3  - 0 \right)\\
 &=\frac{A^2}{T^2}\cdot \frac{1}{3}\cdot T^3\\
 &=\frac{A^2}{3}\cdot T\\
\end{align*}
%\end{aligned}
%\end{equation}

Energia sygnału wynosi $\frac{A^2}{3}\cdot T$
\newpage

Oblicz energię sygnału okresowego $f(t)$ przedstawionego na rysunku 

\begin{figure}[H]
\centering
\begin{tikzpicture}
  %\draw (0,0) circle (1in);
  \draw[->] (-3.0,+0.0) -- (+5.0,+0.0) node[right] {$t$};
  \draw[->] (+0.0,-1.5) -- (+0.0,+2.0) node[above] {$f(t)$};
  \draw[-,red, thick] (-1.5,0.75) -- (-1.0,1.5)--(-1.0,0.0) -- (0.0,-1.0) -- (0.0,+0.0) -- (+1.0,+1.5) -- (+1.0,+0.0) -- (+2.0,-1.0) -- (+2.0,+0.0) -- (2.5,0.75);
  \draw[-,red, dashed] (-1.5,1.5) -- (2.5,1.5);
  \draw[-,red, dashed] (-1.5,-1.0) -- (2.5,-1.0);
  %\draw[-] (-1.0-0.1,-0.1)--(-1.0+0.1,0.1) node[midway, below, outer sep=10pt,align=center] {$-\frac{T}{2}$};
  \draw[-] (-1.0-0.1,-0.1)--(-1.0+0.1,0.1) node[midway, below, outer sep=5pt] {$-(1-a)\cdot T$};
  \draw[-] (+1.0-0.1,-0.1)--(+1.0+0.1,0.1) node[midway, below, outer sep=5pt] {$a\cdot T$};
  \draw[-] (+2.0-0.1,-0.1)--(+2.0+0.1,0.1) node[midway, below, outer sep=5pt] {$T$};
  \draw[-] (-0.1,+1.5-0.1)--(+0.1,+1.5+0.1) node[midway, left] {$A$};
  \draw[-] (-0.1,-1.0-0.1)--(+0.1,-1.0+0.1) node[midway, left] {$B$};

\end{tikzpicture}
\end{figure}

W pierwszej kolejności należy ustalić wzór funkcji przedstawionej na rysunku.
Jest to funkcja odcinkowa. W pierwszym okresie możemy ja opisać za pomocą dwuch prostych. Ogólne równanie prostej:

\begin{equation}
f(t) = m \cdot t + b
\end{equation}

 W pierwszym okresie w pierwszej części wykres funkcji jest prostą przechodzącą przez dwa punkty: $(0,0)$ oraz $(a\cdot T,A)$. Możemy wiec napisać układ równań rozwiązać go i znaleźć nie znane parametry $m$ i $b$.  

\begin{align*}
&\left\{\begin{matrix*}[l]
0 = m\cdot 0 +b\\ 
A = m\cdot a \cdot T +b
\end{matrix*}\right. \\
&\left\{\begin{matrix*}[l]
0 = b\\ 
A = m \cdot a\cdot T +b
\end{matrix*}\right. \\
&\left\{\begin{matrix*}[l]
0 = b\\ 
A = m \cdot a\cdot T +0
\end{matrix*}\right. \\
&\left\{\begin{matrix*}[l]
0 = b\\ 
\frac{A}{a\cdot T} = m
\end{matrix*}\right.
\end{align*}
A więc pierwszy odcinek funkcji przedstawionej na rysunku, w pierwszy okresie można opisać wzorem
\begin{align*}
f(t) = \frac{A}{a \cdot T}\cdot t
\end{align*}

Drugi odcinek funkcji jest prostą przechodzącą przez następujące dwa punkty: $(a \cdot T,0)$ oraz $(T,-B)$. Możemy wiec napisać układ równań rozwiązać go i znaleźć nie znane parametry $m$ i $b$.

\begin{align*}
&\left\{\begin{matrix*}[l]
0 = m\cdot a \cdot T +b\\ 
-B = m\cdot T +b
\end{matrix*}\right. \\
&\left\{\begin{matrix*}[l]
-m \cdot a \cdot T = b\\ 
-B = m \cdot T -m \cdot a \cdot T
\end{matrix*}\right. \\
&\left\{\begin{matrix*}[l]
-m \cdot a \cdot T = b\\ 
-B = m \cdot \left( T - a \cdot T\right)
\end{matrix*}\right. \\
&\left\{\begin{matrix*}[l]
-m \cdot a \cdot T = b\\ 
-\frac{B}{T - a \cdot T} = m
\end{matrix*}\right. \\
&\left\{\begin{matrix*}[l]
\frac{B}{T - a \cdot T} \cdot a \cdot T = b\\ 
-\frac{B}{T - a \cdot T} = m
\end{matrix*}\right. \\
&\left\{\begin{matrix*}[l]
\frac{B}{1 - a} \cdot a = b\\ 
-\frac{B}{T - a \cdot T} = m
\end{matrix*}\right.
\end{align*}
A więc drugi odcinek funkcji przedstawionej na rysunku, w pierwszy okresie można opisać wzorem
\begin{align*}
f(t) = -\frac{B}{T - a \cdot T}\cdot t + \frac{B}{1 - a} \cdot a
\end{align*}
W związku z tym całą funkcję w pierwszym okresie można zapisać jako funkcje przedziałową
\begin{align*}
f(t) = \left\{\begin{matrix*}[l]
\frac{A}{a \cdot T}\cdot t & dla &t \in (0; a \cdot T)\\ 
-\frac{B}{T - a \cdot T}\cdot t + \frac{B}{1 - a} \cdot a & dla & t \in (a \cdot T; T)
\end{matrix*}\right.
\end{align*}
I ogólniej całą funkcję można wyrazić następującym wzorem
\begin{align*}
f(t) = \left\{\begin{matrix*}[l]
\frac{A}{a \cdot T}\cdot \left( t - k\cdot T \right) & dla &t \in (0 + k\cdot T; a \cdot T + k\cdot T)\\ 
-\frac{B}{T - a \cdot T}\cdot \left( t - k\cdot T \right) + \frac{B}{1 - a} \cdot a & dla & t \in (a \cdot T+ k\cdot T; T+ k\cdot T)
\end{matrix*}\right. \wedge k \in C
\end{align*}
Energię sygnału okresowego wyznaczamy z wzoru
\begin{equation}
E=\int_{T}^{}\left|f(t)\right|^2 \cdot dt
\end{equation}
Podstawiamy do wzoru wzór naszej funkcji
%\begin{equation}
%\begin{aligned}
\begin{align*}
E&=\int_{T}^{}\left|f(t)\right|^2 \cdot dt\\
 &=\int_{0}^{a \cdot T}\left|\frac{A}{a \cdot T}\cdot t \right|^2 \cdot dt
  +\int_{a \cdot T}^{T}\left|\frac{B}{T - a \cdot T}\cdot t - \frac{B}{1 - a} \cdot a\right|^2 \cdot dt\\ 
 &=\int_{0}^{a \cdot T}\left(\frac{A}{a \cdot T}\cdot t \right)^2 \cdot dt
  +\int_{a \cdot T}^{T}\left(\frac{B}{T - a \cdot T}\cdot t - \frac{B}{1 - a} \cdot a\right)^2 \cdot dt\\ 
 &=\int_{0}^{a \cdot T}\frac{A^2}{a^2 \cdot T^2}\cdot t^2 \cdot dt
 +\int_{a \cdot T}^{T}\left(\left(\frac{B}{T - a \cdot T}\cdot t\right)^2 - 2\cdot \frac{B}{T - a \cdot T}\cdot t \cdot  \frac{B}{1 - a} \cdot a + \left( \frac{B}{1 - a} \cdot a\right)^2 \right) \cdot dt\\ 
 &=\frac{A^2}{a^2 \cdot T^2}\cdot \int_{0}^{a \cdot T} t^2 \cdot dt
 +\int_{a \cdot T}^{T}\left(\frac{B^2}{T^2 \cdot \left(1 - a \right)^2}\cdot t^2 - 2\cdot \frac{B^2}{T\cdot \left(1 - a\right)^2}\cdot t \cdot a + \frac{B^2}{\left(1 - a \right)^2} \cdot a^2 \right) \cdot dt\\
 &=\frac{A^2}{a^2 \cdot T^2}\cdot \left(\left. \frac{1}{3}\cdot t^3 \cdot dt \right|_{0}^{a \cdot T} \right)
 +\int_{a \cdot T}^{T} \frac{B^2}{T^2 \cdot \left(1 - a \right)^2}\cdot t^2 \cdot dt - \int_{a \cdot T}^{T} 2\cdot \frac{B^2}{T\cdot \left(1 - a\right)^2}\cdot t \cdot a \cdot dt + \int_{a \cdot T}^{T} \frac{B^2}{\left(1 - a \right)^2} \cdot a^2 \cdot dt\\ 
 &=\frac{A^2}{a^2 \cdot T^2}\cdot \left(\left. \frac{1}{3}\cdot t^3 \right|_{0}^{a \cdot T} \right)
 +\frac{B^2}{T^2 \cdot \left(1 - a \right)^2}\cdot \int_{a \cdot T}^{T}  t^2 \cdot dt 
 - \frac{2\cdot B^2}{T\cdot \left(1 - a\right)^2}\cdot a \cdot \int_{a \cdot T}^{T} t \cdot dt 
 + \frac{B^2}{\left(1 - a \right)^2} \cdot a^2 \cdot \int_{a \cdot T}^{T} dt\\
 &=\frac{A^2}{a^2 \cdot T^2}\cdot \left(\frac{1}{3}\cdot \left(a \cdot T\right)^3 - \frac{1}{3}\cdot 0^3 \right)
 +\frac{B^2}{T^2 \cdot \left(1 - a \right)^2}\cdot \left( \left. \frac{1}{3} \cdot t^3 \right|_{a \cdot T}^{T} \right) 
 - \frac{2\cdot B^2}{T\cdot \left(1 - a\right)^2}\cdot a \cdot \left( \left. \frac{1}{2} \cdot t^2 \right|_{a \cdot T}^{T} \right) \\
 &+ \frac{B^2}{\left(1 - a \right)^2} \cdot a^2 \cdot \left( \left. t \right|_{a \cdot T}^{T} \right)\\
 &=\frac{A^2}{a^2 \cdot T^2}\cdot \left(\frac{1}{3}\cdot a^3 \cdot T^3 - 0 \right)
 +\frac{B^2}{T^2 \cdot \left(1 - a \right)^2}\cdot \left( \frac{1}{3} \cdot T^3 -\frac{1}{3} \cdot \left( a\cdot T \right)^3 \right) \\
 &- \frac{2\cdot B^2}{T\cdot \left(1 - a\right)^2}\cdot a \cdot \left( \frac{1}{2} \cdot T^2 -  \frac{1}{2} \cdot \left(a \cdot T\right)^2 \right)
 + \frac{B^2}{\left(1 - a \right)^2} \cdot a^2 \cdot \left( T - a \cdot T \right)\\
 &=\frac{A^2}{a^2 \cdot T^2}\cdot \frac{1}{3}\cdot a^3 \cdot T^3
 +\frac{B^2}{T^2 \cdot \left(1 - a \right)^2}\cdot \left( \frac{1}{3} \cdot T^3 -\frac{1}{3} \cdot a^3 \cdot T ^3 \right) \\
 &- \frac{2\cdot B^2}{T\cdot \left(1 - a\right)^2}\cdot a \cdot \left( \frac{1}{2} \cdot T^2 -  \frac{1}{2} \cdot a^2 \cdot T^2 \right)
 + \frac{B^2}{\left(1 - a \right)^2} \cdot a^2 \cdot \left( 1 - a\right) \cdot T\\
 &=\frac{A^2}{3}\cdot a \cdot T
 +\frac{B^2}{T^2 \cdot \left(1 - a \right)^2}\cdot \left(1 -a^3 \right)\cdot \frac{1}{3} \cdot T^3 \\
 &- \frac{2\cdot B^2}{T\cdot \left(1 - a\right)^2}\cdot a \cdot \left( 1 -   a^2 \right) \cdot \frac{1}{2} \cdot T^2
 + \frac{B^2}{\left(1 - a \right)^2} \cdot a^2 \cdot \left( 1 - a\right) \cdot T\\ 
 &=\frac{A^2}{3}\cdot a \cdot T
 +\frac{B^2}{\left(1 - a\right)^2}\cdot \left(1-a\right)\cdot \left(1 +a + a^2 \right)\cdot \frac{1}{3} \cdot T \\
 &- \frac{2\cdot B^2}{\left(1 - a\right)^2}\cdot a \cdot \left( 1 - a \right) \cdot \left( 1 + a \right) \cdot \frac{1}{2} \cdot T
 + \frac{B^2}{1 - a} \cdot a^2 \cdot T\\
 &=\frac{A^2}{3}\cdot a \cdot T
 +\frac{B^2}{1 - a}\cdot \left(1 +a + a^2 \right)\cdot \frac{1}{3} \cdot T
 - \frac{2\cdot B^2}{1 - a}\cdot a \cdot \left( 1 + a \right) \cdot \frac{1}{2} \cdot T
 + \frac{B^2}{1 - a} \cdot a^2 \cdot T\\
 &=\frac{A^2}{3}\cdot a \cdot T
 +\frac{B^2}{1 - a}\cdot T \cdot \left( \left(1 +a + a^2 \right)\cdot \frac{1}{3}
 - 2 \cdot a \cdot \left( 1 + a \right) \cdot \frac{1}{2}
 + a^2 \right)\\
 &=\frac{A^2}{3}\cdot a \cdot T
 +\frac{B^2}{1 - a}\cdot T \cdot \left( \left(1 +a + a^2 \right)\cdot \frac{2}{6}
 - 2 \cdot a \cdot \left( 1 + a \right) \cdot \frac{3}{6}
 + a^2 \cdot \frac{6}{6} \right)\\ 
 &=\frac{A^2}{3}\cdot a \cdot T
 +\frac{B^2}{1 - a}\cdot T \cdot \frac{1}{6} \cdot \left( \left(1 +a + a^2 \right)\cdot 2
 - 2\cdot a \cdot \left( 1 + a \right) \cdot 3
 + a^2 \cdot 6 \right)\\  
 &=\frac{A^2}{3}\cdot a \cdot T
 +\frac{B^2}{1 - a}\cdot T \cdot \frac{1}{6} \cdot \left( 2 +2\cdot a + 2\cdot a^2 - 6\cdot a - 6\cdot a^2 + 6 \cdot a^2 \right)\\
 &=\frac{A^2}{3}\cdot a \cdot T
 +\frac{B^2}{1 - a}\cdot T \cdot \frac{1}{6} \cdot \left( 2 - 4 \cdot a + 2\cdot a^2 \right)\\
 &=\frac{A^2}{3}\cdot a \cdot T
 +\frac{B^2}{1 - a}\cdot T \cdot \frac{1}{3} \cdot \left( 1 - 2 \cdot a + a^2 \right)\\
 &=\frac{A^2}{3}\cdot a \cdot T
 +\frac{B^2}{1 - a}\cdot T \cdot \frac{1}{3} \cdot \left( 1 - a \right)^2\\
 &=\frac{A^2}{3}\cdot a \cdot T
 + \frac{B^2}{3} \cdot \left( 1 - a \right)\cdot T\\
\end{align*}
%\end{aligned}
%\end{equation}

Energia sygnału wynosi $\frac{A^2}{3}\cdot a \cdot T + \frac{B^2}{3} \cdot \left( 1 - a \right)\cdot T$
\newpage

Oblicz energię sygnału $f(t)$ przedstawionego na rysunku 

\begin{figure}[H]
\centering
\begin{tikzpicture}
  %\draw (0,0) circle (1in);
  \draw[->] (-3.0,+0.0) -- (+5.0,+0.0) node[right] {$t$};
  \draw[->] (+0.0,-2.0) -- (+0.0,+2.0) node[above] {$f(t)$};

  \draw[-] (-0.1,+1.5-0.1)--(+0.1,+1.5+0.1) node[midway, left] {$A$};
  \draw[-] (-0.1,-1.5-0.1)--(+0.1,-1.5+0.1) node[midway, left] {$-A$};
  
  \draw[scale=1.0,domain=0.0:4.0,smooth,variable=\x,red,thick] plot ({\x},{1.5*exp(-\x)});
  \draw[scale=1.0,domain=-2.5:0.0,smooth,variable=\x,red,thick] plot ({\x},{-1.5*exp(\x)});
\end{tikzpicture}
\end{figure}

\begin{equation}
f(t) = \left\{\begin{matrix*}[l]
-A \cdot e^{a\cdot t} & dla & t \in \left(-\infty; 0\right)\\
A \cdot e^{-a\cdot t} & dla & t \in \left(0; \infty\right)
\end{matrix*}\right.
\end{equation}
Energię sygnału nieokresowego wyznaczamy z wzoru
\begin{equation}
E=\lim_{\tau \rightarrow \infty}\int_{-\frac{\tau}{2}}^{\frac{\tau}{2}}\left|f(t)\right|^2 \cdot dt
\end{equation}
Podstawiamy do wzoru na enargie wzór naszej funkcji
%\begin{equation}
%\begin{aligned}
\begin{align*}
E&=\lim_{\tau \rightarrow \infty}\int_{-\frac{\tau}{2}}^{\frac{\tau}{2}}\left|f(t)\right|^2 \cdot dt\\
 &=\lim_{\tau \rightarrow \infty} \left( 
   \int_{-\frac{\tau}{2}}^{0}\left| -A \cdot e^{a\cdot t} \right|^2 \cdot dt 
 + \int_{0}^{\frac{\tau}{2}}\left| A \cdot e^{-a\cdot t} \right|^2 \cdot dt \right)\\
 &=\lim_{\tau \rightarrow \infty} \left( 
   \int_{-\frac{\tau}{2}}^{0}\left( -A \cdot e^{a\cdot t} \right)^2 \cdot dt 
 + \int_{0}^{\frac{\tau}{2}}\left( A \cdot e^{-a\cdot t} \right)^2 \cdot dt \right)\\
 &=\lim_{\tau \rightarrow \infty} \left( 
   \int_{-\frac{\tau}{2}}^{0}\left( -A\right)^2 \cdot \left(e^{a\cdot t} \right)^2 \cdot dt 
 + \int_{0}^{\frac{\tau}{2}}\left( A \right)^2 \cdot \left(e^{-a\cdot t} \right)^2 \cdot dt \right)\\
 &=\lim_{\tau \rightarrow \infty} \left( 
   \int_{-\frac{\tau}{2}}^{0} A^2 \cdot e^{2\cdot a\cdot t} \cdot dt 
 + \int_{0}^{\frac{\tau}{2}} A^2 \cdot e^{-2 \cdot a\cdot t} \cdot dt \right)\\
 &=\lim_{\tau \rightarrow \infty} \left( 
    A^2 \cdot \int_{-\frac{\tau}{2}}^{0}  e^{2\cdot a\cdot t} \cdot dt 
 +  A^2 \cdot \int_{0}^{\frac{\tau}{2}} e^{-2 \cdot a\cdot t} \cdot dt \right)\\
 &=\lim_{\tau \rightarrow \infty} A^2 \cdot \left( 
   \int_{-\frac{\tau}{2}}^{0}  e^{2\cdot a\cdot t} \cdot dt 
 + \int_{0}^{\frac{\tau}{2}} e^{-2 \cdot a\cdot t} \cdot dt \right)\\
 &=\begin{Bmatrix*}[l]
   z=2\cdot a \cdot t & w=-2 \cdot a \cdot t\\
   dz = 2 \cdot a \cdot dt & dw=-2 \cdot a \cdot dt\\
   dt = \frac{dz}{2\cdot a} & dt = \frac{dw}{-2 \cdot a}
   \end{Bmatrix*}\\
 &=\lim_{\tau \rightarrow \infty} A^2 \cdot \left( 
    \int_{-\frac{\tau}{2}}^{0}  e^{z} \cdot \frac{dz}{2\cdot a} 
 +  \int_{0}^{\frac{\tau}{2}} e^{w} \cdot \frac{dw}{-2\cdot a} \right)\\
 &=\lim_{\tau \rightarrow \infty} \frac{A^2}{2\cdot a} \cdot \left( 
    \int_{-\frac{\tau}{2}}^{0}  e^{z} \cdot dz 
 -  \int_{0}^{\frac{\tau}{2}} e^{w} \cdot dw \right)\\
 &=\lim_{\tau \rightarrow \infty} \frac{A^2}{2\cdot a} \cdot \left( 
    \left. e^{z} \right|_{-\frac{\tau}{2}}^{0} 
 -  \left. e^{w} \right|_{0}^{\frac{\tau}{2}} \right)\\ 
 &=\lim_{\tau \rightarrow \infty} \frac{A^2}{2\cdot a} \cdot \left( 
    \left. e^{2\cdot a \cdot t} \right|_{-\frac{\tau}{2}}^{0} 
 -  \left. e^{-2 \cdot a \cdot dt} \right|_{0}^{\frac{\tau}{2}} \right)\\ 
 &=\lim_{\tau \rightarrow \infty} \frac{A^2}{2\cdot a} \cdot \left( 
    \left(e^{2\cdot a \cdot 0} -e^{-2\cdot a \cdot \frac{\tau}{2}} \right)
 -  \left(e^{-2 \cdot a \cdot \frac{\tau}{2}}-e^{-2 \cdot a \cdot 0} \right) \right)\\ 
 &=\lim_{\tau \rightarrow \infty} \frac{A^2}{2\cdot a} \cdot \left( 
    \left(e^{0} -e^{-a \cdot \tau} \right)
 -  \left(e^{-a \cdot \tau}-e^{0} \right)\right)\\
 &=\lim_{\tau \rightarrow \infty} \frac{A^2}{2\cdot a} \cdot \left( 
    1 -e^{-a \cdot \tau} -  e^{-a \cdot \tau}+1 \right)\\
 &=\lim_{\tau \rightarrow \infty} \frac{A^2}{2\cdot a} \cdot \left( 
    2 - 2\cdot e^{-a \cdot \tau} \right)\\
 &=\lim_{\tau \rightarrow \infty} \frac{A^2}{2\cdot a} \cdot 2 \cdot \left( 
    1 - e^{-a \cdot \tau} \right)\\
 &=\lim_{\tau \rightarrow \infty} \frac{A^2}{a} \cdot \left( 
 1 - e^{-a \cdot \tau} \right)\\
 &= \frac{A^2}{a}\\
\end{align*}
%\end{aligned}
%\end{equation}

Energia sygnału wynosi $\frac{A^2}{a}$
\newpage

%		Power of s signal
%		Effective value of a signal (RMS)
%	Energy signals vs power signals
%	Orthogonality. Orthogonal signals and vectors
%	Signal components
%		DC and AC signal components
%		Odd and even signal components
%Analysis of periodic signals using orthogonal series
%	Hilbert space
%	Orthogonal bases
%	Orthogonal series of functions
%	Trigonometric Fourier series
%	The influence of signal symmetries on the coefficients of the trigonometric Fourier series
%	Complex exponential Fourier series
%	The harmonic spectrum of a real signal
%	The relationship of the complex exponential and the trigonometric Fourier series
%	Linearity of Fourier series
%	The influence of signal symmetries on the coefficients of complex exponential Fourier series
%	The effect of signal shift in time on the complex exponential Fourier series
%	Spectrum of a product of two signals
%	Computing the power of a signal – the Parseval theorem
%Analysis of non-periodic signals. Fourier Transformation and Transform
%	An intuitive introduction
%	Definition
%	Fourier Transform vs Laplace Transform
%	The Magnitude Spectrum and Phase Spectrum
%	Symmetries of the Fourier Transform for real-valued signals
%	Special case of Fourier Transform for symmetrical signals
%	Theorems describing the properties of Fourier Transformation
%		Linearity
%		Shift theorem – shifting in time domain
%		Shifting in frequency domain (also known as modulation theorem)
%		Scaling theorem (also called the similarity theorem)
%		Time-frequency duality (also known as the symmetry theorem)
%		Derivative theorem (differentiation in time domain)
%		Integration theorem
%	Calculating energy of the signal from its Fourier transform. The Parseval's theorem
%	Generalization of the Fourier transformation for infinite energy signals
%	Fourier transform of a periodic signal
%	Calculating the power of a signal from its Fourier transform. The Parseval's theorem for power signals
%Processing of signals by linear and time invariant (LTI) systems
%	Introduction to LTI systems. Fundamental properties
%	Impulse response of an LTI system
%	Impulse response of a causal system
%	The response of an LTI system to arbitrary input
%	Properties of linear convolution
%	Frequency response
%	Determining the frequency response of an electronic circuit
%Filters
%	Sampling. Discrete-time signals
%	Introduction to discrete signals
%	Spectrum of a sampled signal
%	Spectral efect of sampling a continuous signal
%	Reconstruction of the continuous signal from its samples
%	Non-periodic and periodic discrete-time signals
%	Fourier transforms of discrete-time signals
%	Processing of discrete-time signals
%	Frequency response of discrete-time LTI systems

%====================================================
\end{document}

%Tom pierwszy zadania
%Tom drugi rozwiązania

%TODO
