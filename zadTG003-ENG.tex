\begin{task}
\TT{Oblicz transformatę Fouriera sygnału $f(t)$ przedstawionego na rysunku.}{Compute the Fourier transform of a triangle impulse shown below.}

\begin{figure}[H]
\centering
\begin{tikzpicture}
	\draw[->] (-3.0,+0.0) -- (+5.0,+0.0) node[right] {$t$};
	\draw[->] (+0.0,-1.5) -- (+0.0,+1.5) node[above] {$f(t)$};
	\draw[-,red, thick] (-3.5,+0.0) -- (-1.0,+0.0) -- (0.0,+1.0) -- (1.0,+0.0) -- (3.0,0.0);
	\draw[-] (-1.0-0.1,-0.1)--(-1.0+0.1,0.1) node[midway, below, outer sep=5pt,align=center] {$-t_{0}$};
	\draw[-] (+1.0-0.1,-0.1)--(+1.0+0.1,0.1) node[midway, below, outer sep=5pt] {$t_{0}$};
	\draw[-] (-0.1,1.0-0.1)--(+0.1,1.0+0.1) node[midway, above left] {$A$};
\end{tikzpicture}
\end{figure}

\TT{W pierwszej kolejności opiszmy sygnał za pomocą sygnałów elementarnych:}{First of all, describe the $f(t)$ signal using elementary signals:}

\begin{equation}
f(t) = A \cdot \Lambda(\frac{t}{t_{0}})
\end{equation}

\TT{Transformatę Fouriera obliczamy ze wzoru:}{The Fourier transform is defined as:}

\begin{equation}
F(\jmath \omega )=\int_{-\infty }^{\infty}f(t) \cdot e^{-\jmath \cdot \omega \cdot t}\cdot dt
\end{equation}

\TT{Do obliczenia całki potrzebujemy jawnej postaci równań opisujących proste na odcinkach $(-t_{0}, 0)$ oraz $(0, t_{0})$.}{In order to integrate the $f(t)$ signal, we need to describe it as a piecewise signal.}

\TT{Ogólne równanie prostej to:}{The simplest form of a linear function is:}

\begin{equation}
f(t) = m \cdot t + b
\end{equation}

\TT{Dla pierwszego zakresu wartości $t$ wykres funkcji jest prostą przechodzącą przez dwa punkty: $(-t_{0},0)$ oraz $(0,A)$. Możemy więc napisać układ równań, rozwiązać go i wyznaczyć parametry prostej $m$ i $b$.}{In the first interval (e.g. $t \in (-t_0; 0)$), linear function crosses two points: $(-t_0,0)$ and $(0,A)$. So, in order to derive $m$ and $b$, the following system of the equations has to be solved.}

\begin{align*}
&\left\{\begin{matrix*}[l]
0 = m\cdot (-t_{0}) +b\\ 
A = m\cdot 0 +b
\end{matrix*}\right. \\
&\left\{\begin{matrix*}[l]
-b = m\cdot (-t_{0})\\ 
A = b
\end{matrix*}\right. \\
&\left\{\begin{matrix*}[l]
\frac{b}{t_{0}} = m\\ 
A = b
\end{matrix*}\right. \\
&\left\{\begin{matrix*}[l]
A = b\\ 
\frac{A}{t_{0}} = m
\end{matrix*}\right.
\end{align*}

\TT{Równianie prostej dla $t$ z zakresu  $(-t_{0},0)$ to:}{As a result we get:}

\begin{align*}
f(t) = \frac{A}{t_{0}}\cdot t + A
\end{align*}

\TT{Dla drugiego zakresu wartości $t$ wykres funkcji jest prostą przechodzącą przez dwa punkty: $(0,A)$ oraz $(t_{0},0)$. Możemy więc napisać układ równań, rozwiązać go i wyznaczyć parametry prostej $m$ i $b$.}{In the second interval (e.g. $t \in (0;t_0)$), linear function crosses two points: $(0;A)$ and $(t_0,0)$. So, in order to derive $m$ and $b$, the following system of the equations has to be solved.}   

\begin{align*}
&\left\{\begin{matrix*}[l]
0 = m\cdot t_{0} +b\\ 
A = m\cdot 0 +b
\end{matrix*}\right. \\
&\left\{\begin{matrix*}[l]
-b = m\cdot t_{0}\\ 
A = b
\end{matrix*}\right. \\
&\left\{\begin{matrix*}[l]
-\frac{b}{t_{0}} = m\\ 
A = b
\end{matrix*}\right. \\
&\left\{\begin{matrix*}[l]
A = b\\ 
-\frac{A}{t_{0}} = m
\end{matrix*}\right.
\end{align*}

\TT{Równianie prostej dla $t$ z zakresu  $(0,t_{0})$ to:}{As a result we get:}

\begin{align*}
f(t) = -\frac{A}{t_{0}}\cdot t + A
\end{align*}

\TT{Podsumowując, sygnał $f(t)$ możemy opisać jako funkcję przedziałową:}{As a result the piecewise linear function is given by:}

\begin{equation}
f(t) = A \cdot \Lambda(\frac{t}{t_{0}}) = \left\{\begin{matrix*}[l]
0 & \TT{dla}{for} & t \in \left(-\infty; -t_{0}\right)\\
\frac{A}{t_{0}} \cdot t + A & \TT{dla}{for} & t \in \left(-t_{0}; 0\right)\\
-\frac{A}{t_{0}} \cdot t + A & \TT{dla}{for} & t \in \left(0; t_{0}\right)\\
0 & \TT{dla}{for} & t \in \left(t_{0};\infty\right)
\end{matrix*}\right.
\end{equation}

\TT{Podstawiamy do wzoru na transformatę wzór naszej funkcji:}{For the given $f(t)$ signal we get:}

\begin{align*}
F(\jmath \omega )&=\int_{-\infty }^{\infty}f(t) \cdot e^{-\jmath \cdot \omega \cdot t}\cdot dt=\\ %1
&=\int_{-\infty}^{-t_{0}} 0 \cdot e^{-\jmath \cdot \omega \cdot t}\cdot dt %2
+\int_{-t_{0}}^{0} \left(\frac{A}{t_{0}} \cdot t + A\right) \cdot e^{-\jmath \cdot \omega \cdot t}\cdot dt=\\
&+\int_{0}^{t_{0}} \left(-\frac{A}{t_{0}} \cdot t + A\right) \cdot e^{-\jmath \cdot \omega \cdot t}\cdot dt
+\int_{t_{0}}^{\infty} 0 \cdot e^{-\jmath \cdot \omega \cdot t}\cdot dt=\\
&=\int_{-\infty}^{-t_{0}} 0 \cdot dt +\int_{-t_{0}}^{0}\frac{A}{t_{0}} \cdot t \cdot e^{-\jmath \cdot \omega \cdot t}\cdot dt %3
+\int_{-t_{0}}^{0}A \cdot e^{-\jmath \cdot \omega \cdot t}\cdot dt=\\
&+\int_{0}^{t_{0}} -\frac{A}{t_{0}} \cdot t \cdot e^{-\jmath \cdot \omega \cdot t}\cdot dt 
+\int_{0}^{t_{0}} A \cdot e^{-\jmath \cdot \omega \cdot t}\cdot dt + \int_{t_{0}}^{\infty} 0 \cdot dt=\\
&= 0  + \frac{A}{t_{0}} \cdot \int_{-t_{0}}^{0} t \cdot e^{-\jmath \cdot \omega \cdot t}\cdot dt %4
+A \cdot \int_{-t_{0}}^{0} e^{-\jmath \cdot \omega \cdot t}\cdot dt=\\
&-\frac{A}{t_{0}} \cdot \int_{0}^{t_{0}} t \cdot e^{-\jmath \cdot \omega \cdot t}\cdot dt 
+A \cdot \int_{0}^{t_{0}} e^{-\jmath \cdot \omega \cdot t}\cdot dt + 0=\\
&=\begin{Bmatrix*}[l] %5
u&=t & dv&=e^{ -\jmath \cdot \omega \cdot t} \cdot dt \\
du&=dt & v&=\frac{1}{-\jmath \cdot \omega}\cdot e^{ -\jmath \cdot \omega \cdot t}
\end{Bmatrix*}=\\
&=\frac{A}{t_{0}}\cdot \left( \left. t \cdot \frac{1}{-\jmath \cdot \omega}\cdot e^{ -\jmath \cdot \omega \cdot t} \right|_{-t_{0}}^{0} %6
- \int_{-t_{0}}^{0} \frac{1}{-\jmath \cdot \omega}\cdot e^{ -\jmath \cdot \omega \cdot t} \cdot dt \right)=\\
&+ A\cdot \left( \left. \frac{1}{-\jmath \cdot \omega}\cdot e^{ -\jmath \cdot \omega \cdot t} \right|_{-t_{0}}^{0}\right)=\\
&-\frac{A}{t_{0}}\cdot \left( \left. t \cdot \frac{1}{-\jmath \cdot \omega}\cdot e^{ -\jmath \cdot \omega \cdot t} \right|_{0}^{t_{0}}
- \int_{0}^{t_{0}} \frac{1}{-\jmath \cdot \omega}\cdot e^{ -\jmath \cdot \omega \cdot t} \cdot dt \right)=\\
&+ A\cdot \left( \left. \frac{1}{-\jmath \cdot \omega}\cdot e^{ -\jmath \cdot \omega \cdot t} \right|_{0}^{t_{0}}\right)=\\
&=\frac{A}{t_{0}}\cdot \left( 0 \cdot e^{-\jmath \cdot \omega \cdot 0} - (-t_{0}) \cdot \frac{1}{-\jmath \cdot \omega}\cdot e^{ -\jmath \cdot \omega \cdot (-t_{0})} %7
+ \frac{1}{\jmath \cdot \omega} \left( \left. \frac{1}{-\jmath \cdot \omega}\cdot e^{ -\jmath \cdot \omega \cdot t} \right|_{-t_{0}}^{0}\right)\right)=\\
&+ \frac{A}{-\jmath \cdot \omega} \cdot \left( e^{ -\jmath \cdot \omega \cdot 0} - e^{ -\jmath \cdot \omega \cdot (-t_{0})} \right)=\\
&-\frac{A}{t_{0}}\cdot \left(t_{0} \cdot \frac{1}{-\jmath \cdot \omega}\cdot e^{ -\jmath \cdot \omega \cdot t_{0}} - 0 \cdot e^{-\jmath \cdot \omega \cdot 0} 
+ \frac{1}{\jmath \cdot \omega} \left( \left. \frac{1}{-\jmath \cdot \omega}\cdot e^{ -\jmath \cdot \omega \cdot t} \right|_{0}^{t_{0}}\right)\right)=\\
&+ \frac{A}{-\jmath \cdot \omega} \cdot \left( e^{ -\jmath \cdot \omega \cdot t_{0}} - e^{ -\jmath \cdot \omega \cdot 0} \right)=\\
&=\frac{A}{t_{0}}\cdot \left( 0 - t_{0} \cdot \frac{1}{\jmath \cdot \omega}\cdot e^{ \jmath \cdot \omega \cdot t_{0}} %8
- \frac{1}{\jmath^{2} \cdot \omega^{2}} \left( e^{ -\jmath \cdot \omega \cdot 0} - e^{ -\jmath \cdot \omega \cdot (-t_{0})} \right)\right)=\\
&- \frac{A}{\jmath \cdot \omega} \cdot \left( 1 - e^{ \jmath \cdot \omega \cdot t_{0}} \right)=\\
&-\frac{A}{t_{0}}\cdot \left(t_{0} \cdot \frac{1}{-\jmath \cdot \omega}\cdot e^{ -\jmath \cdot \omega \cdot t_{0}} - 0 
- \frac{1}{\jmath^{2} \cdot \omega^{2}} \left(e^{ -\jmath \cdot \omega \cdot t_{0}}-e^{ -\jmath \cdot \omega \cdot 0}\right)\right)=\\
&- \frac{A}{\jmath \cdot \omega} \cdot \left( e^{ -\jmath \cdot \omega \cdot t_{0}} - 1\right)=\\
&=-\frac{A}{\jmath \cdot \omega} \cdot e^{ \jmath \cdot \omega \cdot t_{0}} - \frac{A}{t_{0} \cdot \jmath^{2} \cdot \omega^{2}} + \frac{A}{t_{0} \cdot \jmath^{2} \cdot \omega^{2}} \cdot e^{ \jmath \cdot \omega \cdot t_{0}} %9
- \frac{A}{\jmath \cdot \omega} + \frac{A}{\jmath \cdot \omega} \cdot e^{ \jmath \cdot \omega \cdot t_{0}}=\\
&+\frac{A}{\jmath \cdot \omega} \cdot e^{ -\jmath \cdot \omega \cdot t_{0}}  + \frac{A}{t_{0} \cdot \jmath^{2} \cdot \omega^{2}} \cdot e^{- \jmath \cdot \omega \cdot t_{0}} - \frac{A}{t_{0} \cdot \jmath^{2} \cdot \omega^{2}}
- \frac{A}{\jmath \cdot \omega} \cdot e^{ -\jmath \cdot \omega \cdot t_{0}} + \frac{A}{\jmath \cdot \omega}=\\
&=- \frac{2 \cdot A}{t_{0} \cdot \jmath^{2} \cdot \omega^{2}} + \frac{A}{t_{0} \cdot \jmath^{2} \cdot \omega^{2}}  \cdot \left(e^{ \jmath \cdot \omega \cdot t_{0}}+e^{ -\jmath \cdot \omega \cdot t_{0}} \right)=\\ %10
&= \frac{2 \cdot A}{t_{0} \cdot \omega^{2}} - \frac{A}{t_{0} \cdot \omega^{2}} \cdot \left(e^{ \jmath \cdot \omega \cdot t_{0}}+e^{ -\jmath \cdot \omega \cdot t_{0}} \right)=\\%11
&=\begin{Bmatrix*}[l]%12
cos(x)=\frac{e^{\jmath \cdot x} + e^{-\jmath \cdot x}}{2}\\
\end{Bmatrix*}=\\
&= \frac{2 \cdot A}{t_{0} \cdot \omega^{2}} - \frac{2 \cdot A}{t_{0} \cdot \omega^{2}} \cdot cos(\omega \cdot t_{0})=\\%13
&= \frac{2 \cdot A}{t_{0} \cdot \omega^{2}} \cdot (1 - cos(\omega \cdot t_{0}))=\\%14
&=\begin{Bmatrix*}[l]%15
sin^{2}(x)=\frac{1}{2}-\frac{1}{2}\cdot cos(2 \cdot x)\\
cos(2 \cdot x)= 1-2\cdot sin^{2}(x)\\
\end{Bmatrix*}=\\
&= \frac{2 \cdot A}{t_{0} \cdot \omega^{2}} \cdot (1 - 1 + 2\cdot sin^{2}(\frac{\omega \cdot t_{0}}{2}))=\\%16
&= \frac{4 \cdot A}{t_{0} \cdot \omega^{2}} \cdot sin^{2}(\frac{\omega \cdot t_{0}}{2})=\\%16
&= \frac{A \cdot t_{0}}{\frac{t_{0}^{2} \cdot \omega^{2}}{4}} \cdot sin^{2}(\frac{\omega \cdot t_{0}}{2})=\\%17
&=\begin{Bmatrix*}[l]%18
\frac{sin(x)}{x}=Sa(x)
\end{Bmatrix*}=\\
&= A \cdot t_{0} \cdot Sa^{2}(\frac{\omega \cdot t_{0}}{2})%17
\end{align*}

\TT{Transformata sygnału $f(t) = A \cdot \Lambda(\frac{t}{t_{0}})$ to $F(\jmath \omega)=A \cdot t_{0} \cdot Sa^{2}(\frac{\omega \cdot t_{0}}{2})$.}{The Fourier transform of the $f(t) = A \cdot \Lambda(\frac{t}{t_{0}})$ is equal to $F(\jmath \omega)=A \cdot t_{0} \cdot Sa^{2}(\frac{\omega \cdot t_{0}}{2})$.}

\end{task}
