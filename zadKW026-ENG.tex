\begin{task}

\TT{Oblicz transformatę Fouriera sygnału $f(t)=sgn(t)$ wykorzystując twierdzenia opisujące własciwości transformacji Fouriera.}{Compute the Fourier transform of the $f(t)=sgn(t)$ signal using theorems describing the properties of Fourier transformation.}

\begin{figure}[H]
\centering
\begin{tikzpicture}
  %\draw (0,0) circle (1in);
  \draw[->] (-4.0,+0.0) -- (+4.0,+0.0) node[right] {$t$};
  \draw[->] (+0.0,-2.5) -- (+0.0,+2.5) node[above] {$f(t)$};
  
  \draw[-,red, thick] (-3.5,-2.0) -- (+0.0,-2.0);
  \draw[-,red, thick] (+0.0,-2.0) -- (+0.0,+2.0);
  \draw[-,red, thick] (+0.0,+2.0) -- (+3.5,+2.0);
  
  \draw[-] (-0.1,+2.0-0.1)--(+0.1,+2.0+0.1) node[midway, left] {$1$};
  \draw[-] (-0.1,-2.0-0.1)--(+0.1,-2.0+0.1) node[midway, right] {$-1$};
\end{tikzpicture}
\end{figure}

\TT{Opiszmy sygnał za pomocą skoków jednostkowych. }{Let's describe the $f(t)$ signal using the elementary signals - step signals:}

\begin{align*}
f(t) & = sgn(t) =\\
& = \mathbb{1}(t) - \mathbb{1}(-t)=\\
& = f_1(t) - f_2(t)
\end{align*}

\TT{gdzie:}{where:}

\begin{align*}
f_1(t)&= \mathbb{1}(t)\\
f_2(t)&= \mathbb{1}(-t)\\
\end{align*}

\begin{figure}[H]
  \centering
  \begin{tikzpicture}
  %\draw (0,0) circle (1in);
  \draw[->] (-4.0,+0.0) -- (+4.0,+0.0) node[right] {$t$};
  \draw[->] (+0.0,-1.0) -- (+0.0,+2.5) node[above] {$f_1(t)$};
  
  \draw[-,red, thick] (-3.5,+0.0) -- (+0.0,+0.0);
  \draw[-,red, thick] (+0.0,+0.0) -- (+0.0,+2.0);
  \draw[-,red, thick] (+0.0,+2.0) -- (+3.5,+2.0);
  
  \draw[-] (-0.1,+2.0-0.1)--(+0.1,+2.0+0.1) node[midway, left] {$1$};
  \end{tikzpicture}
\end{figure}

\TT{Transformaty sygnału $f_1(t)=\mathbb{1}(t)$ nie można wyznaczyć wprost ze wzoru. Ale łatwo można wyznaczyć pochodnią $f_1'(t)$:}{The transform of the $f_1(t)=\mathbb{1}(t)$ signal cannot be derived by classical Fourier transformation. However, the derivative $f_1'(t)$ can be easily derived:}

\begin{align*}
g(t) = f_1'(t) = \delta(t)
\end{align*}

\begin{figure}[H]
  \centering
  \begin{tikzpicture}
  %\draw (0,0) circle (1in);
  \draw[->] (-4.0,+0.0) -- (+4.0,+0.0) node[right] {$t$};
  \draw[->] (+0.0,-1.0) -- (+0.0,+2.5) node[above] {$g(t)$};
  
  \draw[-,red, thick] (-3.5,+0.0) -- (+3.5,+0.0);
  \draw[->,red, thick] (+0.0,+0.0) -- (+0.0,+2.0);
  
  \draw[-] (-0.1,+2.0-0.1)--(+0.1,+2.0+0.1) node[midway, left] {$1$};
  \end{tikzpicture}
\end{figure}

\TT{Dla sygnału $g(t) = \delta(t)$ w bardzo łatwy sposób można wyznaczyć transformatę Fouriera.}{The Fourier transform for the $g(t) = \delta(t)$ signal can be calculated as:}

\begin{align*}
G(\jmath \omega) &= \int_{-\infty}^{\infty} g(t) \cdot e^{-\jmath \cdot \omega \cdot t}=\\
&=\int_{-\infty}^{\infty} \delta(t) \cdot e^{-\jmath \cdot \omega \cdot t}=\\
&=\begin{Bmatrix}
\SamplingPropertyOfDelta
\end{Bmatrix}=\\
&=e^{-\jmath \cdot \omega \cdot 0}=\\
&=e^{0}=\\
&=1
\end{align*}

\TT{Transformatą Fouriera sygnału $g(t)=\delta(t)$ jest $G(\jmath \omega)=1$.}{The Fourier transform of the $g(t)=\delta(t)$ signal is equal to $G(\jmath \omega)=1$.}

\TT{Korzystając z twierdzenia o całkowaniu można wyznaczyć transformatę funkcji $f_1(t)$:}{Based on the integration theorem we can calculate the Fourier transform of the $f_1(t)$ signal:}

\begin{align*}
\IntegralTeorem{g}{G}{f_1}{F_1}
\end{align*}

\begin{align*}
F_1(\jmath \omega)&=\frac{1}{\jmath \cdot \omega } \cdot G(\jmath \omega) + \pi \cdot \delta(\omega) \cdot G(0)=\\
&=\frac{1}{\jmath \cdot \omega } \cdot 1 + \pi \cdot \delta(\omega) \cdot 1=\\
&=\frac{1}{\jmath \cdot \omega } + \pi \cdot \delta(\omega)
\end{align*}

\TT{A więc transformatą skoku jednostkowego jest $F_1(\jmath \omega)=\frac{1}{\jmath \cdot \omega } + \pi \cdot \delta(\omega)$.}{The Fourier transform of the step signal $f_1(t)=\mathbb{1}(t)$ is equal to $F_1(\jmath \omega)=\frac{1}{\jmath \cdot \omega } + \pi \cdot \delta(\omega)$.}

\TT{Funkcję $f_2(t)$ można zapisać jako:}{The $f_2(t)$ signal can be described as:}

\begin{align*}
f_2(t)&=\mathbb{1}(-t)=\\
&=\mathbb{1}(-1 \cdot t)=\\
&=f_1(-1 \cdot t)
\end{align*}

\TT{A wiec transformatę funkcji $f_2(t)$ można wyznaczyć z twierdzenia o zmianie skali}{Therefore, the Fourier transform of the $f_2(t)$ signal can be derived using the scaling theorem:}

\begin{align*}
\TimeScalingTeorem{f_1}{F_1}{f_2}{F_2}
\end{align*}

\begin{align*}
F_2(\jmath \omega)&=\frac{1}{\left| a \right|} \cdot F_1(\jmath \frac{\omega}{a})=\\
&=\begin{Bmatrix}
a=-1
\end{Bmatrix}=\\
&=\frac{1}{\left| -1 \right|} \cdot \frac{1}{\jmath \cdot \frac{\omega}{-1} } + \pi \cdot \delta(\frac{\omega}{-1})=\\
&=\frac{1}{1} \cdot \frac{1}{-\jmath \cdot \omega } + \pi \cdot \delta(-\omega)=\\
&=- \frac{1}{\jmath \cdot \omega } + \pi \cdot \delta(\omega)
\end{align*}

\TT{Transformata funkcji $f_2(t)$ jest równa $F_2(\jmath \omega)- \frac{1}{\jmath \cdot \omega } + \pi \cdot \delta(\omega)$.}{The Fourier transform of the $f_2(t)$ signal is equal to $F_2(\jmath \omega)- \frac{1}{\jmath \cdot \omega } + \pi \cdot \delta(\omega)$.}

\TT{Transformatę funkcji $f(t)$ możemy wyznaczyć z twierdzenia o jednorodności}{Finally, the Fourier transform of the $f(t)$ signal can be derived using the linearity theorem:}

\begin{align*}
\HomogeneousTeorem{f}{F}
\end{align*}

\begin{align*}
F(\jmath \omega)&=F_1(\jmath \omega) - F_2(\jmath \omega)=\\
&=\frac{1}{\jmath \cdot \omega } + \pi \cdot \delta(\omega) - \left(-\frac{1}{\jmath \cdot \omega } + \pi \cdot \delta(\omega)\right)=\\
&=\frac{1}{\jmath \cdot \omega } + \pi \cdot \delta(\omega) +\frac{1}{\jmath \cdot \omega } - \pi \cdot \delta(\omega)=\\
&=\frac{2}{\jmath \cdot \omega }
\end{align*}

\TT{Transformata funkcji $f(t)=sgn(t)$ jest równa $F(\jmath \omega)=\frac{2}{\jmath \cdot \omega }$.}{The Fourier transform of the $f(t)=sgn(t)$ is equal to $F(\jmath \omega)=\frac{2}{\jmath \cdot \omega}$.}

\end{task}

