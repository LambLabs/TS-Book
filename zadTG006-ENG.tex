\begin{task}
\TT{Oblicz transformatę Fouriera sygnału $f(t)$ wykorzystując twierdzenia opisujące własciwości transformacji Fouriera. Wykorzystaj wiedzę o właściwościach próbkujących delty Diraca.}{Compute the Fourier transform of the $f(t)$ signal given below using theorems describing the properties of Fourier transformation. Exploit the sampling property of Dirac delta.}

\begin{figure}[H]
\centering
\begin{tikzpicture}
	\draw[->] (-3.0,+0.0) -- (+5.0,+0.0) node[right] {$t$};
	\draw[->] (+0.0,-1.5) -- (+0.0,+1.5) node[above] {$f(t)$};
	\draw[-,red, thick] (-3.5,+0.0) -- (-1.0,+0.0) -- (0.0,+1.0) -- (1.0,+0.0) -- (3.0,0.0);
	\draw[-] (-1.0-0.1,-0.1)--(-1.0+0.1,0.1) node[midway, below, outer sep=5pt,align=center] {$-t_{0}$};
	\draw[-] (+1.0-0.1,-0.1)--(+1.0+0.1,0.1) node[midway, below, outer sep=5pt] {$t_{0}$};
	\draw[-] (-0.1,1.0-0.1)--(+0.1,1.0+0.1) node[midway, above left] {$A$};
\end{tikzpicture}
\end{figure}

\TT{W pierwszej kolejności opiszmy sygnał za pomocą sygnałów elementarnych:}{First of all, describe the $f(t)$ signal using the elementary signals:}

\begin{equation}
f(t) = A \cdot \Lambda(\frac{t}{t_{0}})
\end{equation}

\TT{Wyznaczmy pochodną sygnału $f(t)$, czyli sygnał $g(t)= \frac{\partial}{\partial t}f(t)$.}{Let's derive derivative of the $f(t)$ signal as $g(t)= \frac{\partial}{\partial t}f(t)$:}

\begin{figure}[H]
	\centering
	\begin{tikzpicture}
	\draw[->] (-3.0,+0.0) -- (+5.0,+0.0) node[right] {$t$};
	\draw[->] (+0.0,-1.5) -- (+0.0,+2.0) node[above] {$g(t)$};
	\draw[-,red, thick] (-3.5,+0.0) -- (-1.0,+0.0) -- (-1.0,+1.0) -- (0.0,+1.0) -- (0.0,-1.0) -- (1.0,-1.0) -- (1.0,0.0) -- (3.0,0.0);
	\draw[-] (-1.0-0.1,-0.1)--(-1.0+0.1,0.1) node[midway, below, outer sep=5pt,align=center] {$-t_{0}$};
	\draw[-] (+1.0-0.1,-0.1)--(+1.0+0.1,0.1) node[midway, below, outer sep=5pt] {$t_{0}$};
	\draw[-] (-0.1,1.0-0.1)--(+0.1,1.0+0.1) node[midway, above left] {$\frac{A}{t_{0}}$};
	\draw[-] (-0.1,-1.0-0.1)--(+0.1,-1.0+0.1) node[midway, below left] {$-\frac{A}{t_{0}}$};
	\end{tikzpicture}
\end{figure}

\TT{Można sprawdzić, że całkując sygnał $g(t)$ otrzymamy sygnał $f(t)$, czyli:}{You can check that by integrating the $g(t)$ signal we'll get the $f(t)$ signal:}

\begin{equation}
f(t) = \int_{-\infty}^{t} g(x) \cdot dx
\end{equation}

\TT{Skoro tak jest, to transformatę sygnału $f(t)$ mozna wyznaczyć z twierdzenia o całkowaniu sygnału, w tym przypadku całkować będziemy sygnał $g(t)$:}{Therefore, the Fourier transform of the $f(t)$ signal can be determined from the integration theorem. In this case we will integrate the $g(t)$ signal:}

\begin{equation}
F(\jmath \omega) = \frac{1}{\jmath \cdot \omega} \cdot G(\jmath \omega) + \pi \cdot \delta(\omega) \cdot G(0)
\end{equation}

\TT{Pytanie, czy można dalej uprościc sygnał $g(t)$ dokonując jego rózniczkowania. Wyznaczmy pochodną sygnału $g(t)$, czyli drugą pochodną sygnału $f(t)$:}{Is it possible to calculate derivative again to simply the analysed signal? Let's calculate the derivative of the $g(t)$ signal, so the second derivative of the $f(t)$ signal:}

\begin{equation}
h(t)= \frac{\partial}{\partial t}g(t) = \frac{\partial^2}{\partial t^2}f(t) 
\end{equation}

\begin{figure}[H]
	\centering
	\begin{tikzpicture}
	\draw[->] (-3.0,+0.0) -- (+5.0,+0.0) node[right] {$t$};
	\draw[->] (+0.0,-2.5) -- (+0.0,+1.5) node[above] {$h(t)$};
	\draw[-,red, thick] (-3.5,+0.0) -- (3.5,0.0);
	\draw[->,red, thick] (-1.0,+0.0) -- (-1.0,1.0);
	\draw[->,red, thick] (0.0,+0.0) -- (0.0,-2.0);
	\draw[->,red, thick] (+1.0,+0.0) -- (+1.0,1.0);
	
	\draw[-] (-1.0-0.1,-0.1)--(-1.0+0.1,0.1) node[midway, below, outer sep=5pt,align=center] {$-t_0$};
	\draw[-] (+1.0-0.1,-0.1)--(+1.0+0.1,0.1) node[midway, below, outer sep=5pt] {$t_0$};
	\draw[] (-1.0,+1.0)--(-1.0,+1.0) node[midway, above] {($\frac{A}{t_{0}}$)};
	\draw[] (0.0,-2.0)--(0.0,-2.0) node[midway, below left] {($-\frac{2 \cdot A}{t_{0}}$)};
	\draw[] (1.0,1.0)--(1.0,1.0) node[midway, above] {($\frac{A}{t_{0}}$)};
	\end{tikzpicture}
\end{figure}

\TT{Sygnał $h(t)$ można opisać, wykorzystując sygnały elementarne:}{Using the elementary signals we can write:}

\begin{equation}
h(t) = \frac{A}{t_{0}} \cdot \delta(t-(-t_{0})) -\frac{2 \cdot A}{t_{0}} \cdot \delta(t) + \frac{A}{t_{0}} \cdot \delta(t-(t_{0}))
\end{equation}

\TT{Można sprawdzić, że całkując sygnał $h(t)$ otrzymamy sygnał $g(t)$, czyli:}{You can check that by integrating the $h(t)$ signal we'll get the $g(t)$ signal:}

\begin{equation}
g(t) = \int_{-\infty}^{t} h(x) \cdot dx
\end{equation}


\TT{Skoro tak jest, to transformatę sygnału $g(t)$ mozna wyznaczyć z twierdzenia o całkowaniu sygnału, w tym przypadku całkować będziemy sygnał $h(t)$:}{Therefore, the Fourier transform of the $g(t)$ signal can be determined from the integration theorem. In this case we will integrate the $h(t)$ signal:}

\begin{equation}
G(\jmath \omega) = \frac{1}{\jmath \cdot \omega} \cdot H(\jmath \omega) + \pi \cdot \delta(\omega) \cdot H(0)
\end{equation}

\TT{Z powyższego równania widać, że musimy znać $H(\jmath \omega)$, czyli transformatę sygnału $h(t)$:}{In order to derive $G(\jmath \omega)$ we have to calculate the $H(\jmath \omega)$ transform of the $h(t)$ signal:}

\begin{equation}
h(t) = \frac{A}{t_{0}} \cdot \delta(t-(-t_{0})) -\frac{2 \cdot A}{t_{0}} \cdot \delta(t) + \frac{A}{t_{0}} \cdot \delta(t-(t_{0}))
\end{equation}

\TT{Ponieważ transformacja Fouriera jest przekształceniem liniowym, dlatego można wyznaczyć osobno transformaty poszczególnych delt Diraca, czyli:}{Based on the linearity theorem, we can derive transforms separately for each Dirac delta signal:}

\begin{align*}
H(\jmath \omega)&={\mathcal F}\{h(t)\}=\\
&={\mathcal F}\left\{\frac{A}{t_{0}} \cdot \delta(t-(-t_{0})) -\frac{2 \cdot A}{t_{0}} \cdot \delta(t) + \frac{A}{t_{0}} \cdot \delta(t-(t_{0}))\right\}=\\
&={\mathcal F}\left\{\frac{A}{t_{0}} \cdot \delta(t-(-t_{0}))\right\} - {\mathcal F}\left\{\frac{2 \cdot A}{t_{0}} \cdot \delta(t)\right\} + {\mathcal F}\left\{\frac{A}{t_{0}} \cdot \delta(t-(t_{0}))\right\}=\\
&=\frac{A}{t_{0}} \cdot {\mathcal F}\left\{\delta(t-(-t_{0}))\right\} - \frac{2 \cdot A}{t_{0}} \cdot {\mathcal F}\left\{\delta(t)\right\} + \frac{A}{t_{0}} \cdot {\mathcal F}\left\{\delta(t-(t_{0}))\right\}=\\
&=\begin{Bmatrix*}[l]
\delta(t)\xrightarrow{\mathcal F} 1\\
\delta(t-(-t_{0}))\xrightarrow{\mathcal F} 1 \cdot e^{-\jmath \cdot \omega \cdot (-t_{0})}\\
\delta(t-(t_{0}))\xrightarrow{\mathcal F} 1 \cdot e^{-\jmath \cdot \omega \cdot t_{0}}
\end{Bmatrix*}=\\
&=\frac{A}{t_{0}} \cdot e^{-\jmath \cdot \omega \cdot (-t_{0})} - \frac{2 \cdot A}{t_{0}} \cdot 1 + \frac{A}{t_{0}} \cdot e^{-\jmath \cdot \omega \cdot t_{0}}=\\
&=\frac{A}{t_{0}} \cdot \left(e^{\jmath \cdot \omega \cdot t_{0}} -2 + e^{-\jmath \cdot \omega \cdot t_{0}}\right)=\\
&=\begin{Bmatrix*}[l]
cos(x)=\frac{e^{\jmath \cdot x} + e^{-\jmath \cdot x}}{2}
\end{Bmatrix*}=\\
&=\frac{A}{t_{0}} \cdot \left(2 \cdot cos(\omega \cdot t_{0}) -2\right)=\\
&=\frac{2 \cdot A}{t_{0}} \cdot \left(cos(\omega \cdot t_{0}) -1\right)
\end{align*}

\TT{Czyli transformata sygnału $h(t)$ to:}{The Fourier transform of the $h(t)$ signal is equal to:}

\begin{equation}
H(\jmath \omega) = \frac{2 \cdot A}{t_{0}} \cdot \left(cos(\omega \cdot t_{0}) -1\right)
\end{equation}

\TT{Mamy wyznaczoną transformatę $H(\jmath \omega)$. Teraz, z twierdzenia o całkowaniu sygnału, możemy wyznaczyc transformatę $G(\jmath \omega)$:}{We derived the $H(\jmath \omega)$ transform. Now, based on the integration theorem, we can calculate the $G(\jmath \omega)$ Fourier transform.}

\begin{align*}
G(\jmath \omega)&= \frac{1}{\jmath \cdot \omega} \cdot H(\jmath \omega) + \pi \cdot \delta(\omega) \cdot H(0)=\\
&=\frac{1}{\jmath \cdot \omega} \cdot \frac{2 \cdot A}{t_{0}} \cdot \left(cos(\omega \cdot t_{0}) -1\right)+ \pi \cdot \delta(\omega) \cdot H(0)=\\
&=\begin{Bmatrix*}[l]
H(0)=\frac{2 \cdot A}{t_{0}} \cdot \left(cos(0 \cdot t_{0}) -1\right)\\
H(0)=\frac{2 \cdot A}{t_{0}} \cdot \left(cos(0) -1\right)\\
H(0)=\frac{2 \cdot A}{t_{0}} \cdot \left(1 -1\right)\\
H(0)=0\\
\end{Bmatrix*}=\\
&=\frac{2 \cdot A}{\jmath \cdot \omega \cdot t_{0}} \cdot \left(cos(\omega \cdot t_{0}) -1\right)
\end{align*}

\TT{Mamy wyznaczoną transformatę $G(\jmath \omega)$. Teraz, kolejny raz z twierdzenia o całkowaniu sygnału, możemy wyznaczyć transformatę $F(\jmath \omega)$:}{We derived the $G(\jmath \omega)$ transform. Using the integration theorem once again, we can calculate the $F(\jmath \omega)$ Fourier transform.}

\begin{align*}
F(\jmath \omega)&= \frac{1}{\jmath \cdot \omega} \cdot G(\jmath \omega) + \pi \cdot \delta(\omega) \cdot G(0)=\\
&=\frac{1}{\jmath \cdot \omega} \cdot \frac{2 \cdot A}{\jmath \cdot \omega \cdot t_{0}} \cdot \left(cos(\omega \cdot t_{0}) -1\right) + \pi \cdot \delta(\omega) \cdot G(0)=\\
&=\begin{Bmatrix*}[l]
G(0)=\frac{2 \cdot A}{\jmath \cdot 0 \cdot t_{0}} \cdot \left(cos(0 \cdot t_{0}) -1\right)\\
G(0)=\frac{0}{0}!!!\\
G(0)=\int_{-\infty}^{\infty} g(t) \cdot dt=\int_{-t_{0}}^{0} \frac{A}{t_{0}} \cdot dt+\int_{0}^{t_{0}} (-\frac{A}{t_{0}}) \cdot dt\\
G(0)=\frac{A}{t_{0}} \cdot (0 -(-t_{0}))-\frac{A}{t_{0}} \cdot (t_{0} - 0)= A- A\\
G(0)=0\\
\end{Bmatrix*}=\\
&=\frac{1}{\jmath \cdot \omega} \cdot \frac{2 \cdot A}{\jmath \cdot \omega \cdot t_{0}} \cdot \left(cos(\omega \cdot t_{0}) -1\right)=\\
&=\frac{2 \cdot A}{\jmath^{2} \cdot \omega^{2} \cdot t_{0}} \cdot \left(cos(\omega \cdot t_{0}) -1\right)=\\
&=\frac{2 \cdot A}{\omega^{2} \cdot t_{0}} \cdot \left(1 - cos(\omega \cdot t_{0})\right)=\\
&=\begin{Bmatrix*}[l]
sin^{2}(x)= \frac{1}{2}-\frac{1}{2} \cdot cos(2 \cdot x)\\
cos(2 \cdot x)= 1 - 2 \cdot sin^{2}(x)\\
\end{Bmatrix*}=\\
&=\frac{2 \cdot A}{\omega^{2} \cdot t_{0}} \cdot \left(1 - 1 + 2 \cdot sin^{2}\left(\frac{\omega \cdot t_{0}}{2}\right)\right)=\\
&=\frac{4 \cdot A}{\omega^{2} \cdot t_{0}} \cdot sin^{2}\left(\frac{\omega \cdot t_{0}}{2}\right)=\\
&=\begin{Bmatrix*}[l]
\frac{sin(x)}{x}=Sa(x)
\end{Bmatrix*}=\\
&=A \cdot t_{0} \cdot Sa^{2}(\frac{\omega \cdot t_{0}}{2})
\end{align*}

\TT{Transformata sygnału $f(t) = A \cdot \Lambda(\frac{t}{t_{0}})$ to $F(\jmath \omega)=A \cdot t_{0} \cdot Sa^{2}(\frac{\omega \cdot t_{0}}{2})$}{The Fourier transform of $f(t) = A \cdot \Lambda(\frac{t}{t_{0}})$ is equal to $F(\jmath \omega)=A \cdot t_{0} \cdot Sa^{2}(\frac{\omega \cdot t_{0}}{2})$.}

\end{task}
