\begin{task}

Wyznacz odpowied\'{z} implusową $h(t)$ układu LTI, wiedząc, że sygnały $u(t)$ oraz $y(t)$ wygladają jak na poniższych wykresach. Wykorzystaj informacje o transformatach sygnałów: $\Pi(t) \xrightarrow{\mathcal F} Sa\left(\frac{\omega}{2}\right)$ oraz $\Lambda(t) \xrightarrow{\mathcal F} Sa^2\left(\frac{\omega}{2}\right)$.

\begin{figure}[H]
	\centering
	\begin{tikzpicture}
	%\draw (0,0) circle (1in);
	\draw[->] (-2.0-5.0,+0.0) -- (+2.0-5.0,+0.0) node[right] {$t$};
	\draw[->] (+1.0-5.0,-1.0) -- (+1.0-5.0,+2.5) node[above] {$u(t)$};
	
	\draw[-,red, thick] (-1.0-5.0,+1.5) -- (+1.0-5.0,+1.5);
	\draw[-,red, thick] (-1.0-5.0,+0.0) -- (-1.0-5.0,+1.5);
	\draw[-,red, thick] (+1.0-5.0,+0.0) -- (+1.0-5.0,+1.5);
	
	\draw[-] (-0.1-4.0,+1.5-0.1)--(+0.1-4.0,+1.5+0.1) node[midway, above left] {$A$};
	\draw[-] (-1.0-0.1-5.0,-0.1)--(-1.0+0.1-5.0,0.1) node[midway, below, outer sep=5pt,align=center] {$-t_0$};
	
	
	\draw[->] (-2.0,+0.0) -- (+2.0,+0.0) node[right] {$t$};
	\draw[->] (+0.0,-1.0) -- (+0.0,+2.5) node[above] {$y(t)$};
	
	%\draw[-,red, thick] (-1.0,+2.0) -- (+1.0,+2.0);
	\draw[-,green, thick] (-1.0,+0.0) -- (+0.0,+2.0);
	\draw[-,green, thick] (+1.0,+0.0) -- (+0.0,+2.0);
	
	\draw[-] (-0.1,+2.0-0.1)--(+0.1,+2.0+0.1) node[midway, above left] {$A \cdot t_0$};
	\draw[-] (-1.0-0.1,-0.1)--(-1.0+0.1,0.1) node[midway, below, outer sep=5pt,align=center] {$-t_0$};
	\draw[-] (+1.0-0.1,-0.1)--(+1.0+0.1,0.1) node[midway, below, outer sep=5pt,align=center] {$t_0$};
	
	\end{tikzpicture}
\end{figure}

Wiemy, że transformatę odpowiedzi układu można wyznaczyć ze wzoru $Y(\jmath \omega)=U(\jmath \omega) \cdot H(\jmath \omega)$ oraz że $h(t) \xrightarrow{\mathcal F} H(\jmath \omega)$. W związku z tym $H(\jmath \omega)=\frac{Y(\jmath \omega)}{U(\jmath \omega)}$ oraz $h(t) \xrightarrow{\mathcal F^{-1}} H(\jmath \omega)$.

W pierwszym kroku wyznaczmy transformaty sygnałów $u(t)$ oraz $y(t)$:
\begin{align*}
u(t)&= A \cdot \Pi\left(\frac{t+\frac{t_0}{2}}{t_0}\right) & y(t)&= A \cdot t_0 \cdot \Lambda\left(\frac{t}{t_0}\right)\\
U(\jmath \omega)&=\mathcal F \{u(t)\} & Y(\jmath \omega)&=\mathcal F \{y(t)\} \\
\Pi(t) &\xrightarrow{\mathcal F} Sa\left(\frac{\omega}{2}\right) & \Lambda(t) &\xrightarrow{\mathcal F} Sa^2\left(\frac{\omega}{2}\right)\\
\Pi\left(\frac{1}{t_0} \cdot t \right) &\xrightarrow{\mathcal F} t_0 \cdot Sa\left(\frac{\omega \cdot t_0}{2}\right) & \Lambda\left(\frac{1}{t_0} \cdot t \right) &\xrightarrow{\mathcal F} t_0 \cdot Sa^2\left(\frac{\omega \cdot t_0}{2}\right)\\
\Pi\left(\frac{t+\frac{t_0}{2}}{t_0}\right) &\xrightarrow{\mathcal F} t_0 \cdot Sa\left(\frac{\omega \cdot t_0}{2}\right) \cdot e^{\jmath \cdot \omega \cdot \frac{t_0}{2}} & A \cdot t_0 \cdot \Lambda\left(\frac{t}{t_0}\right) &\xrightarrow{\mathcal F} A \cdot t_0^2 \cdot Sa^2\left(\frac{\omega \cdot t_0}{2}\right)\\
A \cdot \Pi\left(\frac{t+\frac{t_0}{2}}{t_0}\right) &\xrightarrow{\mathcal F} A \cdot t_0 \cdot Sa\left(\frac{\omega \cdot t_0}{2}\right) \cdot e^{\jmath \cdot \omega \cdot \frac{t_0}{2}} &  &
\end{align*}

Skoro zanmy transformaty sygnałów wejściowego i wyjściowego, to możemy wyznaczyc transmitancję układu, czyli $H(\jmath \omega)$. 
\begin{align*}
H(\jmath \omega)&=\frac{Y(\jmath \omega)}{U(\jmath \omega)}\\
&=\frac{A \cdot t_0^2 \cdot Sa^2\left(\frac{\omega \cdot t_0}{2}\right)}{A \cdot t_0 \cdot Sa\left(\frac{\omega \cdot t_0}{2}\right) \cdot e^{\jmath \cdot \omega \cdot \frac{t_0}{2}}}\\
&=t_0 \cdot Sa\left(\frac{\omega \cdot t_0}{2}\right) \cdot e^{-\jmath \cdot \omega \cdot \frac{t_0}{2}}
\end{align*}

Teraz możemy wyznaczć odpowied\'{z} implusową układu $h(t)$:
\begin{align*}
h(t) &\xrightarrow{\mathcal F} H(\jmath \omega)\\
? &\xrightarrow{\mathcal F} t_0 \cdot Sa\left(\frac{\omega \cdot t_0}{2}\right) \cdot e^{-\jmath \cdot \omega \cdot \frac{t_0}{2}}\\
\Pi(t) &\xrightarrow{\mathcal F} Sa\left(\frac{\omega}{2}\right)\\
\Pi\left(\frac{1}{t_0} \cdot t \right) &\xrightarrow{\mathcal F} t_0 \cdot Sa\left(\frac{\omega \cdot t_0}{2}\right)\\
\Pi\left(\frac{t-\frac{t_0}{2}}{t_0}\right) &\xrightarrow{\mathcal F} t_0 \cdot Sa\left(\frac{\omega \cdot t_0}{2}\right) \cdot e^{-\jmath \cdot \omega \cdot \frac{t_0}{2}}\\
\end{align*}


Odpowied\'{z} implusowa układu to $h(t)= \Pi\left(\frac{t-\frac{t_0}{2}}{t_0}\right)$.
\end{task}

