\begin{task}
Wyznacz udział mocy podstawowej (pierwszej) harmonicznej w całkowitej mocy okresowego sygnału $f(t)$ przedstawionego na rysunku: 

\begin{figure}[H]
\centering
\begin{tikzpicture}
  \draw[->] (-3.0,+0.0) -- (+5.0,+0.0) node[right] {$t$};
  \draw[->] (+0.0,-1.5) -- (+0.0,+1.5) node[above] {$f(t)$};
  \draw[-,red, thick] (-2.0,+0.0) -- (-2.0,+1.0) -- (-1.0,+1.0) -- (-1.0,-1.0)--(+0.0,-1.0) -- (+0.0,+1.0) -- (+1.0,+1.0) -- (+1.0,-1.0) -- (+2.0,-1.0) -- (+2.0,+1.0) -- (3.0,1.0) -- (3.0,0.0);
  \draw[-] (-1.0-0.1,-0.1)--(-1.0+0.1,0.1) node[midway, below right, outer sep=5pt,align=center] {$-\frac{T}{2}$};
  \draw[-] (+1.0-0.1,-0.1)--(+1.0+0.1,0.1) node[midway, below right, outer sep=5pt,align=center] {$\frac{T}{2}$};
  \draw[-] (+2.0-0.1,-0.1)--(+2.0+0.1,0.1) node[midway, below right, outer sep=5pt,align=center] {$T$};
  \draw[-] (-0.1,1.0-0.1)--(+0.1,1.0+0.1) node[midway, left] {$A$};
  \draw[-] (-0.1,-1.0-0.1)--(+0.1,-1.0+0.1) node[midway, right] {$-A$};
\end{tikzpicture}
\end{figure}

\begin{equation}
\frac{P_1}{P} = ?
\end{equation}

W pierwszej kolejności należy opisać sygnał za pomocą wzoru:

\begin{equation}
   f(x)=\begin{cases}A & t \in \left (  0+k \cdot T; \frac{T}{2}+k \cdot T \right ) \\
   -A & t \in \left ( \frac{T}{2}+k \cdot T; T +k \cdot T\right )\end{cases} \wedge k \in C %\mathbb{C}
\end{equation}

Moc sygnału możemy obliczyć ze wzoru:
\begin{equation}
P=\frac{1}{T} \cdot \int_{T}^{}\left|f(t)\right|^2 \cdot dt
\end{equation}

Podstawiając wartości sygnału $f(t)$ do wzoru na moc otrzymujemy:
\begin{align*}
P &=\frac{1}{T} \cdot \int_{T}\left|f(t)\right|^2 \cdot dt=\\
&=\frac{1}{T} \cdot \left( \int_{0}^{\frac{T}{2}} \left|A\right|^2 \cdot dt + \int_{\frac{T}{2}}^{T} \left|-A\right|^2 \cdot dt \right )=\\
&=\frac{1}{T} \cdot \left( A^2 \cdot \int_{0}^{\frac{T}{2}} dt + A^2 \cdot \int_{\frac{T}{2}}^{T} dt  \right )=\\
&=\frac{A^2}{T} \cdot \left( \left. t\right |_{0}^{\frac{T}{2}} + \left. t\right |_{\frac{T}{2}}^{T} \right )=\\
&=\frac{A^2}{T} \cdot \left( \frac{T}{2} - 0 + T - \frac{T}{2}\right )=\\
&=\frac{A^2}{T} \cdot \left( T \right )=\\
&=A^2 
\end{align*}

Moc sygnału $f(t)$ równa się $A^2$.

Moc podstawowej (pierwszej) harmonicznej to (na podstawie twierdzenia Parsevala):
\begin{equation}
P_1=\left|F_{1}\right|^2+\left|F_{-1}\right|^2
\end{equation}

Ponieważ sygnał $f(t)$ jest sygnałem rzeczywistym, to $\left|F_{1}\right|=\left|F_{-1}\right|$, czyli moc podstawowej harmonicznej: 
\begin{equation}
P_1=2 \cdot \left|F_{1}\right|^2
\end{equation}

W związku z tym, należy obliczyć wartość współczynnika $F_1$. Można to zrobić bezpośrednio ze wzoru na $F_k$:
\begin{equation}
F_k=\frac{1}{T} \cdot \int_{T}f(t) \cdot e^{-\jmath \cdot k \cdot \frac{2\pi}{T} \cdot t} \cdot dt\\
\end{equation}
podstawiając $k=1$:
\begin{equation}
F_1=\frac{1}{T} \cdot \int_{T}f(t) \cdot e^{-\jmath \cdot \frac{2\pi}{T} \cdot t} \cdot dt
\end{equation}

Podstawiając wartości sygnału $f(t)$ do wzoru na $F_1$ otrzymujemy:
\begin{align*}
F_1&=\frac{1}{T} \cdot \int_{T}f(t) \cdot e^{-\jmath \cdot \frac{2\pi}{T} \cdot t} \cdot dt=\\
&=\frac{1}{T} \cdot \left(\int_{0}^{\frac{T}{2}}A \cdot e^{-\jmath \cdot \frac{2\pi}{T} \cdot t} \cdot dt + \int_{\frac{T}{2}}^{T}-A \cdot e^{-\jmath \cdot \frac{2\pi}{T} \cdot t} \cdot dt\right)=\\
&=\frac{A}{T} \cdot \left(\int_{0}^{\frac{T}{2}}e^{-\jmath \cdot \frac{2\pi}{T} \cdot t} \cdot dt - \int_{\frac{T}{2}}^{T}e^{-\jmath \cdot \frac{2\pi}{T} \cdot t} \cdot dt\right)=\\
&=\begin{Bmatrix*}[l]
z&=-\jmath \cdot \frac{2\pi}{T} \cdot t\\
dz&=-\jmath \cdot \frac{2\pi}{T} \cdot dt\\
dt&=\frac{dz}{-\jmath \cdot \frac{2\pi}{T}}\\
\end{Bmatrix*}=\\
&=\frac{A}{T} \cdot \left(\int_{0}^{\frac{T}{2}} e^{z} \cdot \frac{dz}{-\jmath \cdot \frac{2\pi}{T}}-\int_{\frac{T}{2}}^{T} e^{z} \cdot \frac{dz}{-\jmath \cdot \frac{2\pi}{T}}\right)=\\
&=-\frac{A}{T \cdot \jmath \cdot \frac{2\pi}{T}} \cdot \left(\int_{0}^{\frac{T}{2}} e^{z} \cdot dz - \int_{\frac{T}{2}}^{T} e^{z} \cdot dz\right)=\\
&=-\frac{A}{\jmath \cdot 2 \pi} \cdot \left(\left. e^{z} \right|_{0}^{\frac{T}{2}} - \left. e^{z} \right|_{\frac{T}{2}}^{T}\right)=\\
&=-\frac{A}{\jmath \cdot 2 \pi} \cdot \left(\left. e^{-\jmath \cdot \frac{2\pi}{T} \cdot t} \right|_{0}^{\frac{T}{2}} - \left. e^{-\jmath \cdot \frac{2\pi}{T} \cdot t} \right|_{\frac{T}{2}}^{T}\right)=\\
&=-\frac{A}{\jmath \cdot 2 \pi} \cdot \left( e^{-\jmath \cdot \frac{2\pi}{T} \cdot \frac{T}{2}} - e^{-\jmath \cdot \frac{2\pi}{T} \cdot 0} -e^{-\jmath \cdot \frac{2\pi}{T} \cdot T} + e^{-\jmath \cdot \frac{2\pi}{T} \cdot \frac{T}{2}}\right)=\\
&=-\frac{A}{\jmath \cdot 2 \pi} \cdot \left( e^{ -\jmath \cdot \pi } - e^{ 0} - e^{ -\jmath \cdot 2 \cdot \pi}+e^{ -\jmath \cdot \pi }\right)=\\
&=\begin{Bmatrix}
e^{ -\jmath \cdot 2\cdot \pi }&=cos(2\pi)-\jmath \cdot sin(2\pi)=1\\
e^{ -\jmath \cdot \pi }&=cos(\pi)-\jmath \cdot sin(\pi)=-1\\
\end{Bmatrix}=\\
&=-\frac{A}{\jmath \cdot 2 \pi} \cdot \left( -1 - 1 - 1 -1\right)=\\
&=-\frac{A}{\jmath \cdot 2 \pi} \cdot \left( -4\right)=\\
&=\frac{2 \cdot A}{\jmath \cdot \pi}=\\
&=-\jmath \cdot \frac{2 \cdot A}{\pi}
\end{align*}

Wartość współczynnika $F_1$ to $-\jmath \cdot \frac{2 \cdot A}{\pi}$

Podstawiając wartość współczynnika $F_1$ do wzoru na moc podstawowej harmonicznej otrzymujemy:
\begin{align*}
P_1&=2 \cdot \left|F_{1}\right|^2=\\
&=2 \cdot \left|-\jmath \cdot \frac{2 \cdot A}{\pi}\right|^2=\\
&=2 \cdot \left(\frac{2 \cdot A}{\pi}\right)^2=\\
&=2 \cdot \frac{4 \cdot A^2}{\pi^2}=\\
&=\frac{8 \cdot A^2}{\pi^2}
\end{align*}

Moc podstawowej harmonicznej równa się $P_1=\frac{8 \cdot A^2}{\pi^2}$.

Teraz można wyznaczyć udział mocy podstawowej (pierwszej) harmonicznej w całkowitej mocy okresowego sygnału $f(t)$:
\begin{equation}
\frac{P_1}{P} = \frac{\frac{8 \cdot A^2}{\pi^2}}{A^2} = \frac{8}{\pi^2} \approx 81\%
\end{equation}


\end{task}