\begin{task}
Oblicz transformatę Fouriera sygnału $f(t)$ przedstawionego na rysunku za pomocą twierdzeń.

\begin{figure}[H]
\centering
\begin{tikzpicture}
  %\draw (0,0) circle (1in);
  \draw[->] (-4.0,+0.0) -- (+4.0,+0.0) node[right] {$t$};
  \draw[->] (+0.0,-1.5) -- (+0.0,+1.5) node[above] {$f(t)$};
  \draw[-,red, thick] (-3.5,+0.0) -- (-1.0,+0.0);
  \draw[-,red, thick] (+1.0,+0.0) -- (+3.5,+0.0);
  \draw[-,red, thick] (-1.0,+0.0) -- (-1.0,+1.0);
  \draw[-,red, thick] (+1.0,+0.0) -- (+1.0,+1.0);
  \draw[-,red, thick] (-1.0,+1.0) -- (+1.0,+1.0);
  
  \draw[-] (-1.0-0.1,-0.1)--(-1.0+0.1,0.1) node[midway, below, outer sep=5pt,align=center] {$-t_0$};
  \draw[-] (+1.0-0.1,-0.1)--(+1.0+0.1,0.1) node[midway, below, outer sep=5pt] {$t_0$};
  \draw[-] (-0.1,+1.0-0.1)--(+0.1,+1.0+0.1) node[midway, above left] {$A$};
\end{tikzpicture}
\end{figure}

\begin{equation}
f(t)=\begin{cases}
0 & t \in \left( -\infty; -t_0 \right ) \\
A & t \in \left( -t_0; t_0 \right ) \\
0 & t \in \left( t_0; \infty \right )
\end{cases} 
\end{equation}

W pierwszej kolejności wyznaczamy pochodna sygnału $f(t)$

\begin{equation}
g(t)=f'(t)=\begin{cases}
0 & t \in \left( -\infty; -t_0 \right ) \\
0 & t \in \left( -t_0; t_0 \right ) \\
0 & t \in \left( t_0; \infty \right )
\end{cases} + A \cdot \delta(t+t_0) - A \cdot \delta(t-t_0)
\end{equation}

czyli po prostu

\begin{equation}
g(t)=f'(t)= A \cdot \delta(t+t_0) - A \cdot \delta(t-t_0)
\end{equation}

\begin{figure}[H]
  \centering
  \begin{tikzpicture}
  %\draw (0,0) circle (1in);
  \draw[->] (-4.0,+0.0) -- (+4.0,+0.0) node[right] {$t$};
  \draw[->] (+0.0,-1.5) -- (+0.0,+1.5) node[above] {$g(t)$};
  \draw[-,red, thick] (-3.5,+0.0) -- (+3.5,+0.0);
  \draw[->,red, thick] (-1.0,+0.0) -- (-1.0,+1.0);
  \draw[->,red, thick] (+1.0,+0.0) -- (+1.0,-1.0);
  
  \draw[-] (-1.0-0.1,-0.1)--(-1.0+0.1,0.1) node[midway, below, outer sep=5pt,align=center] {$-t_0$};
  \draw[-] (+1.0-0.1,-0.1)--(+1.0+0.1,0.1) node[midway, above, outer sep=5pt] {$t_0$};
  \draw[-] (-0.1,+1.0-0.1)--(+0.1,+1.0+0.1) node[midway, left] {$A$};
  \draw[-] (-0.1,-1.0-0.1)--(+0.1,-1.0+0.1) node[midway, left] {$-A$};
  \end{tikzpicture}
\end{figure}

Wyznaczanie transformaty sygnału $g(t)$ złożonego z delt diracka jest znacznie prostsze. 

\begin{equation}
G(\jmath \omega )=\int_{-\infty }^{\infty}g(t) \cdot e^{-\jmath \cdot \omega \cdot t}\cdot dt
\end{equation}

Podstawiamy do wzoru na transformatę wzór naszej funkcji

\begin{align*}
G(\jmath \omega )&=\int_{-\infty }^{\infty}g(t) \cdot e^{-\jmath \cdot \omega \cdot t}\cdot dt\\
&=\int_{-\infty }^{\infty}\left( A \cdot \delta(t+t_0) - A \cdot \delta(t-t_0) \right) \cdot e^{-\jmath \cdot \omega \cdot t}\cdot dt\\
&=\int_{-\infty }^{\infty}\left( A \cdot \delta(t+t_0)\cdot e^{-\jmath \cdot \omega \cdot t} - A \cdot \delta(t-t_0)\cdot e^{-\jmath \cdot \omega \cdot t} \right) \cdot dt\\
&=\int_{-\infty }^{\infty} A \cdot \delta(t+t_0)\cdot e^{-\jmath \cdot \omega \cdot t}\cdot dt - \int_{-\infty }^{\infty} A \cdot \delta(t-t_0)\cdot e^{-\jmath \cdot \omega \cdot t} \cdot dt\\
&= A \cdot \int_{-\infty }^{\infty} \delta(t+t_0)\cdot e^{-\jmath \cdot \omega \cdot t}\cdot dt - A \cdot \int_{-\infty }^{\infty}  \delta(t-t_0)\cdot e^{-\jmath \cdot \omega \cdot t} \cdot dt\\
&=\begin{Bmatrix}
\SamplingPropertyOfDelta
\end{Bmatrix}\\
&= A \cdot e^{-\jmath \cdot \omega \cdot (-t_0)} - A \cdot e^{-\jmath \cdot \omega \cdot t_0}\\
&= A \cdot e^{\jmath \cdot \omega \cdot t_0} - A \cdot e^{-\jmath \cdot \omega \cdot t_0}\\
&= A \cdot \left(e^{\jmath \cdot \omega \cdot t_0} - e^{-\jmath \cdot \omega \cdot t_0}\right)\\
&= A \cdot \left(e^{\jmath \cdot \omega \cdot t_0} - e^{-\jmath \cdot \omega \cdot t_0}\right) \cdot \frac{2\cdot \jmath}{2\cdot \jmath}\\
&= A \cdot 2\cdot \jmath \cdot \frac{e^{\jmath \cdot \omega \cdot t_0} - e^{-\jmath \cdot \omega \cdot t_0}}{2\cdot \jmath}\\
&=\begin{Bmatrix}
\EulerSin
\end{Bmatrix}\\
&= A \cdot 2\cdot \jmath \cdot sin\left( \omega \cdot t_0\right)\\
&= \jmath \cdot 2 \cdot A \cdot sin\left( \omega \cdot t_0\right)\\
\end{align*}

Transformata sygnału $g(t)$ to $G(\jmath \omega)=\jmath \cdot 2 \cdot A \cdot sin\left( \omega \cdot t_0\right)$
\\

Następnie możemy wykorzystać twierdzenie o całkowaniu aby wyznaczyć transformatę sygnału $f(t)$ na podstawie transformaty sygnału $g(t)=f'(t)$
%\begin{align*}
%g(t) &\overset{F}{\rightarrow} G(\jmath \omega)\\
%f(t) &= \int_{-\infty}^{t} g(\tau)\cdot d\tau\\
%F(\jmath \omega) &= \frac{1}{\jmath \cdot \omega} G(\jmath \omega) + \pi \cdot \delta(0) \cdot G(0)
%\end{align*}
\begin{align*}
\IntegralTeorem{g}{G}{f}{F}
\end{align*}

Podstawiając obliczona wcześniej transformatę $G(\jmath \omega)$ sygnału $g(t)$ otrzymujemy transformatę $F(\jmath \omega)$ sygnału $f(t)$

\begin{align*}
F(\jmath \omega) &= \frac{1}{\jmath \cdot \omega} \cdot G(\jmath \omega) + \pi \cdot \delta(0) \cdot G(0)\\
&=\frac{1}{\jmath \cdot \omega} \cdot \jmath \cdot 2 \cdot A \cdot sin\left( \omega \cdot t_0\right) + \pi \cdot \delta(0) \cdot \jmath \cdot 2 \cdot A \cdot sin\left( 0 \cdot t_0\right)\\
&=\frac{1}{\omega} \cdot 2 \cdot A \cdot sin\left( \omega \cdot t_0\right) + \pi \cdot \delta(0) \cdot \jmath \cdot 2 \cdot A \cdot sin\left( 0\right)\\
&=\frac{1}{\omega} \cdot 2 \cdot A \cdot sin\left( \omega \cdot t_0\right) \cdot \frac{t_0}{t_0} + \pi \cdot \delta(0) \cdot \jmath \cdot 2 \cdot A \cdot 0\\
&= 2 \cdot A \cdot t_0 \cdot \frac{sin\left( \omega \cdot t_0\right)}{\omega \cdot t_0} + 0\\
&=\begin{Bmatrix}
\SaDef
\end{Bmatrix}\\
&= 2 \cdot A \cdot t_0 \cdot Sa\left( \omega \cdot t_0\right)\\
\end{align*}

Ostatecznie transformata sygnału $f(t)$ jest równa $F(\jmath \omega)=2 \cdot A \cdot t_0 \cdot Sa\left( \omega \cdot t_0\right)$.
\end{task}

