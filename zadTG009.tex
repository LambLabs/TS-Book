\begin{task}
Oblicz energię sygnału $f(t)=Sa\left(\omega_0 \cdot t\right)$, wiedząc że transformata sygnału $\Pi(t)$ jest równa $Sa\left(\frac{\omega}{2}\right)$.

\begin{equation}
f(t) = Sa\left(\omega_0 \cdot t\right)
\end{equation}

\begin{equation}
\Pi(t) \xrightarrow{\mathcal F} Sa\left(\frac{\omega}{2}\right)
\end{equation}

Energię sygnału można wyznaczyc ze wzoru:

\begin{equation}
E= \int_{-\infty }^{\infty}\left|f(t)\right| ^2 \cdot dt
\end{equation}

Podstawiając dany sygnał $f(t)$ do wzoru na energie otrzymujemy:
\begin{align*}
E &= \int_{-\infty }^{\infty}\left|f(t)\right| ^2 \cdot dt=\\
&= \int_{-\infty }^{\infty}\left|Sa\left(\omega_0 \cdot t\right)\right| ^2 \cdot dt=\\
&=\begin{Bmatrix}
&\SaDef
\end{Bmatrix}=\\
&= \int_{-\infty }^{\infty}\left|\frac{sin\left(\omega_0 \cdot t\right)}{\left(\omega_0 \cdot t\right)}\right| ^2 \cdot dt=\\
&= \int_{-\infty }^{\infty}\frac{sin^2\left(\omega_0 \cdot t\right)}{\left(\omega_0 \cdot t\right)^2} \cdot dt=\\
&=....
\end{align*}

Próbując obliczyc energię tym sposobem musimy obliczyć całkę cykliczną. A może jest łatwiejszy sposób?

Spróbumy wykorzystać twierdzenie Parsevala:

\begin{equation}
\Parseval{F}
\end{equation}

W tym podejściu musimy obliczyć transformatę Fouriera sygnału $f(t)$, czyli $F(\jmath \omega)$.
Skoro wiemy, że:

\begin{equation}
g(t) = \Pi(t) \xrightarrow{\mathcal F} Sa\left(\frac{\omega}{2}\right)
\end{equation}

to, na podstawie twierdzenia o symetrii przekształcenia Fouriera:
\begin{align*}
\SymetryTeorem{g}{G}{f}{F}
\end{align*}
otrzymujemy:
\begin{align*}
Sa\left(\frac{t}{2}\right) &\xrightarrow{\mathcal F} 2\pi \cdot \Pi(-\omega)\\
Sa\left(\frac{t}{2}\right) &\xrightarrow{\mathcal F} 2\pi \cdot \Pi(\omega)
\end{align*}

Teraz musimy przeskalować $Sa\left(\frac{t}{2}\right)$ tak, aby otrzymać $Sa\left(\omega_0 \cdot t\right)$. W tym celu skorzystamy z twierdzenia o zmianie skali podstawiając $\alpha=2 \cdot \omega_0$:
\begin{align*}
\TimeScalingTeorem{f}{F}{g}{G}
\end{align*}

\begin{align*}
Sa\left(\frac{t}{2}\right) &\xrightarrow{\mathcal F} 2\pi \cdot \Pi(\omega)\\
Sa\left(2 \cdot \omega_0 \cdot \frac{t}{2}\right) &\xrightarrow{\mathcal F} \frac{1}{2 \cdot \omega_0} \cdot 2\pi \cdot \Pi(\frac{\omega}{2 \cdot \omega_0})\\
Sa\left(\omega_0 \cdot t\right) &\xrightarrow{\mathcal F} \frac{\pi}{\omega_0} \cdot \Pi(\frac{\omega}{2 \cdot \omega_0})
\end{align*}

Narysujmy widmo amplitudowe sygnału $f(t)$, czyli $\left|F(\jmath \omega)\right|$.

\begin{figure}[H]
	\centering
	\begin{tikzpicture}
		\draw[->] (-3.0,+0.0) -- (+3.0,+0.0) node[right] {$\omega$};
		\draw[->] (+0.0,-1.0) -- (+0.0,+2.5) node[above] {$\left|F(\jmath \omega)\right|$};
		
		\draw[-,red, thick] (-1.5,+1.5) -- (1.5,+1.5);
		\draw[-,red, thick] (-1.5,+0.0) -- (-1.5,+1.5);
		\draw[-,red, thick] (1.5,+0.0) -- (1.5,+1.5);
		
		
		\draw[-] (-0.1,+1.5-0.1)--(+0.1,+1.5+0.1) node[midway, above left] {$\frac{\pi}{\omega_0}$};
		
		\draw[-] (-1.5-0.1,-0.1)--(-1.5+0.1,0.1) node[midway, below, outer sep=5pt,align=center] {$-\omega_0$};
		\draw[-] (1.5-0.1,-0.1)--(1.5+0.1,0.1) node[midway, below, outer sep=5pt,align=center] {$\omega_0$};
	
	\end{tikzpicture}  
\end{figure}

\begin{align*}
\left|F(\jmath \omega)\right| &= \begin{cases}
0 & \omega \in \left( -\infty; -\omega_0 \right) \\
\frac{\pi}{\omega_0} & \omega \in \left(-\omega_0; \omega_0 \right) \\
0 & \omega \in \left(\omega_0; \infty \right) \\
\end{cases}\\
\end{align*}

Ponieważ energię wyznaczamy ze wzoru:
\begin{equation}
\Parseval{F}
\end{equation}

to wyznaczmy $\left|F(\jmath \omega)\right|^2$:
\begin{align*}
\left|F(\jmath \omega)\right|^2 &= \begin{cases}
0 & \omega \in \left( -\infty; -\omega_0 \right) \\
\left(\frac{\pi}{\omega_0}\right)^2 & \omega \in \left(-\omega_0; \omega_0 \right) \\
0 & \omega \in \left(\omega_0; \infty \right) \\
\end{cases}\\
\end{align*}

\begin{align*}
E &= \frac{1}{2\pi} \cdot \int_{-\infty}^{\infty} \left|F(\jmath \omega)\right|^2 \cdot d\omega=\\
&=\frac{1}{2\pi} \cdot \left(\int_{-\infty}^{-\omega_0} 0 \cdot d\omega + \int_{-\omega_0}^{\omega_0} \left(\frac{\pi}{\omega_0}\right)^2 \cdot d\omega + \int_{\omega_0}^{\infty} 0 \cdot d\omega\right)=\\
&=\frac{1}{2\pi} \cdot \left(0 + \left(\frac{\pi}{\omega_0}\right)^2 \cdot \int_{-\omega_0}^{\omega_0} d\omega + 0\right)=\\
&=\frac{1}{2\pi} \cdot \left(\frac{\pi}{\omega_0}\right)^2 \cdot \left( \left.\omega \right|_{-\omega_0}^{\omega_0}\right)=\\
&=\frac{\pi}{2 \cdot \omega_0^2} \cdot \left(\omega_0 -(-\omega_0)\right)=\\
&=\frac{\pi}{2 \cdot \omega_0^2} \cdot \left(2 \cdot \omega_0\right)=\\
&=\frac{\pi}{\omega_0}
\end{align*}

Energia sygnału $f(t)=Sa\left(\omega_0 \cdot t\right)$ równa się $E=\frac{\pi}{\omega_0}$.

\end{task}

