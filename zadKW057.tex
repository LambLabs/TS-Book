\begin{task}
\TT{Oblicz transformatę Fouriera sygnału $f(t)$ przedstawionego na rysunku oraz narysuj jego widmo amplitudowe i fazowe.}{Compute the Fourier transform of a signal shown below. Compute and draw magnitude and phase spectra.}

\begin{figure}[H]
\centering
\begin{tikzpicture}
  \draw[->] (-3.0,+0.0) -- (+7.0,+0.0) node[right] {$t$};
  \draw[->] (+0.0,-2.0) -- (+0.0,+2.0) node[above] {$f(t)$};

  \draw[scale=1.0,domain=-2.5:0.0,smooth,variable=\x,red,thick] plot ({\x},{1.5*exp(2.2*\x)});
  \draw[scale=1.0,domain=0.0:6.0,smooth,variable=\x,red,thick] plot ({\x},{1.5*exp(-0.5*\x)});%96%3.141592
  
  \draw[-] (+0.0-0.1,1.5-0.1)--(+0.0+0.1,1.5+0.1) node[midway, left, outer sep=5pt] {$A$}; 
\end{tikzpicture}
\end{figure}

\TT{Sygnał $f(t)$ możemy opisać jako funkcję przedziałową:}{The $f(t)$ signal, as a piecewise function is given by:}

\begin{equation}
f(t) = \begin{cases}
A \cdot e^{\alpha\cdot t} & \TT{\text{ dla }}{\text{ for }} t \in \left(-\infty; 0\right)\\
A \cdot e^{-\beta\cdot t} & \TT{\text{ dla }}{\text{ for }} t \in \left(0; \infty\right)
\end{cases}
\end{equation}

\TT{Transformatę Fouriera obliczamy ze wzoru:}{The Fourier transform is defined as:}

\begin{equation}
F(\jmath \omega )=\int_{-\infty }^{\infty}f(t) \cdot e^{-\jmath \cdot \omega \cdot t}\cdot dt
\end{equation}

\TT{Podstawiamy do wzoru na transformatę wzór naszej funkcji:}{For the given $f(t)$ signal we get:}

\begin{align*}
F(\jmath \omega )&=\int_{-\infty }^{\infty}f(t) \cdot e^{-\jmath \cdot \omega \cdot t}\cdot dt=\\
&=\int_{-\infty}^{0} A \cdot e^{\alpha \cdot t} \cdot e^{-\jmath \cdot \omega \cdot t}\cdot dt
+\int_{0}^{\infty} A \cdot e^{-\beta \cdot t} \cdot e^{-\jmath \cdot \omega \cdot t}\cdot dt=\\
&=\int_{-\infty}^{0} A \cdot e^{\alpha \cdot t-\jmath \cdot \omega \cdot t}\cdot dt
+\int_{0}^{\infty} A \cdot e^{-\beta \cdot t -\jmath \cdot \omega \cdot t}\cdot dt=\\
&=A \cdot \int_{-\infty}^{0} e^{\left(\alpha -\jmath \cdot \omega \right)\cdot t}\cdot dt
+A \cdot \int_{0}^{\infty} e^{\left(-\beta -\jmath \cdot \omega \right) \cdot t}\cdot dt=\\
&=\begin{Bmatrix*}[l]%12
z_1 &= \left(\alpha -\jmath \cdot \omega \right)\cdot t & z_2&=\left(-\beta -\jmath \cdot \omega \right) \cdot t\\
dz_1&=\left(\alpha -\jmath \cdot \omega \right)\cdot dt & dz_2&=\left(-\beta -\jmath \cdot \omega \right) \cdot dt\\
dt&=\frac{dz_1}{\alpha -\jmath \cdot \omega } & dt&=\frac{dz_2}{-\beta -\jmath \cdot \omega }\\
\end{Bmatrix*}=\\
&=A \cdot \int_{-\infty}^{0} e^{z_1}\cdot \frac{dz_1}{\alpha -\jmath \cdot \omega }
+A \cdot \int_{0}^{\infty} e^{z_2}\cdot \frac{dz_2}{-\beta -\jmath \cdot \omega }=\\
&=A \cdot \frac{1}{\alpha -\jmath \cdot \omega } \cdot \int_{-\infty}^{0} e^{z_1}\cdot dz_1
+A \cdot \frac{1}{-\beta -\jmath \cdot \omega } \cdot \int_{0}^{\infty} e^{z_2}\cdot dz_2=\\
&=A \cdot \frac{1}{\alpha -\jmath \cdot \omega } \cdot \lim_{\tau \rightarrow \infty } \int_{-\tau}^{0} e^{z_1}\cdot dz_1
+A \cdot \frac{1}{-\beta -\jmath \cdot \omega } \cdot \lim_{\tau \rightarrow \infty } \int_{0}^{\tau} e^{z_2}\cdot dz_2=\\
&=A \cdot \frac{1}{\alpha -\jmath \cdot \omega } \cdot \lim_{\tau \rightarrow \infty } \left. e^{z_1}\right|_{-\tau}^{0}
+A \cdot \frac{1}{-\beta -\jmath \cdot \omega } \cdot \lim_{\tau \rightarrow \infty } \left. e^{z_2}\right|_{0}^{\tau}=\\
&=A \cdot \frac{1}{\alpha -\jmath \cdot \omega } \cdot \lim_{\tau \rightarrow \infty } \left. e^{\left(\alpha -\jmath \cdot \omega \right)\cdot t}\right|_{-\tau}^{0}
+A \cdot \frac{1}{-\beta -\jmath \cdot \omega } \cdot \lim_{\tau \rightarrow \infty } \left. e^{\left(-\beta -\jmath \cdot \omega \right) \cdot t}\right|_{0}^{\tau}=\\
&=A \cdot \frac{1}{\alpha -\jmath \cdot \omega } \cdot \lim_{\tau \rightarrow \infty } \left( e^{\left(\alpha -\jmath \cdot \omega \right)\cdot 0} -  e^{\left(\alpha -\jmath \cdot \omega \right)\cdot \left(-\tau\right)}\right)
+A \cdot \frac{1}{-\beta -\jmath \cdot \omega } \cdot \lim_{\tau \rightarrow \infty } \left( e^{\left(-\beta -\jmath \cdot \omega \right) \cdot \tau}-e^{\left(-\beta -\jmath \cdot \omega \right) \cdot 0}\right)=\\
&=A \cdot \frac{1}{\alpha -\jmath \cdot \omega } \cdot \lim_{\tau \rightarrow \infty } \left( e^{0} -  e^{\left(\alpha -\jmath \cdot \omega \right)\cdot \left(-\tau\right)}\right)
+A \cdot \frac{1}{-\beta -\jmath \cdot \omega } \cdot \lim_{\tau \rightarrow \infty } \left( e^{\left(-\beta -\jmath \cdot \omega \right) \cdot \tau}-e^{0}\right)=\\
&=A \cdot \frac{1}{\alpha -\jmath \cdot \omega } \cdot \lim_{\tau \rightarrow \infty } \left( 1 -  e^{\left(\alpha -\jmath \cdot \omega \right)\cdot \left(-\tau\right)}\right)
+A \cdot \frac{1}{-\beta -\jmath \cdot \omega } \cdot \lim_{\tau \rightarrow \infty } \left( e^{\left(-\beta -\jmath \cdot \omega \right) \cdot \tau}-1\right)=\\
&=A \cdot \frac{1}{\alpha -\jmath \cdot \omega } \cdot \left( 1 -  \lim_{\tau \rightarrow \infty } e^{\left(\alpha -\jmath \cdot \omega \right)\cdot \left(-\tau\right)}\right)
+A \cdot \frac{1}{-\beta -\jmath \cdot \omega } \cdot \left(\lim_{\tau \rightarrow \infty } e^{\left(-\beta -\jmath \cdot \omega \right) \cdot \tau}-1\right)=\\
&=A \cdot \frac{1}{\alpha -\jmath \cdot \omega } \cdot \left( 1 - 0\right)
+A \cdot \frac{1}{-\beta -\jmath \cdot \omega } \cdot \left(0-1\right)=\\
&=A \cdot \frac{1}{\alpha -\jmath \cdot \omega } \cdot \left( 1\right)
+A \cdot \frac{1}{-\beta -\jmath \cdot \omega } \cdot \left(-1\right)=\\
&=A \cdot \frac{1}{\alpha -\jmath \cdot \omega }
-A \cdot \frac{1}{-\beta -\jmath \cdot \omega }=\\
&=A \cdot \frac{1}{\alpha -\jmath \cdot \omega }
+A \cdot \frac{1}{\beta +\jmath \cdot \omega }=\\
\end{align*}

\TT{Transformata sygnału $f(t)$ to $F(\jmath \omega)=A \cdot \frac{1}{\alpha -\jmath \cdot \omega }
  +A \cdot \frac{1}{\beta +\jmath \cdot \omega }$.}{The Fourier transform of the $f(t)$ is equal to $F(\jmath \omega)=A \cdot \frac{1}{\alpha -\jmath \cdot \omega }
  +A \cdot \frac{1}{\beta +\jmath \cdot \omega }$.}

\TT{Widmo amplitudowe obliczamy ze wzoru:}{The magnitude spectrum is defined as:}

\begin{align*}
M(\omega)&=\left| F(j \omega) \right|=\\
&=\left| A \cdot \frac{1}{\alpha -\jmath \cdot \omega }
+A \cdot \frac{1}{\beta +\jmath \cdot \omega } \right|=\\
&=\left| A \cdot \left( \frac{1}{\alpha -\jmath \cdot \omega }
+ \frac{1}{\beta +\jmath \cdot \omega }\right) \right|=\\
&=\left| A \cdot \left( \frac{\beta +\jmath \cdot \omega}{\left(\alpha -\jmath \cdot \omega \right) \cdot \left(\beta +\jmath \cdot \omega \right)}
+ \frac{\alpha -\jmath \cdot \omega}{\left(\alpha -\jmath \cdot \omega \right) \cdot \left(\beta +\jmath \cdot \omega \right) }\right) \right|=\\
&=\left| A \cdot \frac{\beta +\jmath \cdot \omega + \alpha -\jmath \cdot \omega}{\left(\alpha -\jmath \cdot \omega \right) \cdot \left(\beta +\jmath \cdot \omega \right)} \right|=\\
&=\left| A \cdot \frac{\beta + \alpha}{ \alpha \cdot \beta + \jmath \cdot \alpha \cdot \omega - \jmath \cdot \beta \cdot \omega - \jmath \cdot \omega \cdot \jmath \cdot \omega} \right|=\\
&=\left| A \cdot \frac{\beta + \alpha}{ \alpha \cdot \beta + \jmath \cdot\left( \alpha - \beta \right)\cdot \omega + \omega^2} \right|=\\
&=\left| A \right| \cdot \left| \frac{\beta + \alpha}{ \alpha \cdot \beta + \jmath \cdot\left( \alpha - \beta \right)\cdot \omega + \omega^2} \right|=\\
&=\left| A \right| \cdot \frac{ \left|\beta + \alpha \right|}{ \left| \alpha \cdot \beta + \jmath \cdot\left( \alpha - \beta \right)\cdot \omega + \omega^2 \right|}=\\
&=A \cdot \frac{ \beta + \alpha }{ \left| \alpha \cdot \beta + \omega^2 + \jmath \cdot\left( \alpha - \beta \right)\cdot \omega\right|}=\\
&=A \cdot \frac{ \beta + \alpha }{ \sqrt{\left( \alpha \cdot \beta + \omega^2 \right)^2+ \left(\left( \alpha - \beta \right)\cdot \omega\right)^2}}=\\
\end{align*}

\begin{figure}[H]
  \centering
  \begin{tikzpicture}
  \draw[->] (-5.0,+0.0) -- (+5.0,+0.0) node[right] {$\omega$};
  \draw[->] (+0.0,-1.5) -- (+0.0,+6.0) node[above] {$M(\omega)$};
  
  \draw[scale=1.0,domain=-4.0:4.0,samples=2000,smooth,variable=\x,red,thick] plot ({\x},{(0.2+2.5)/sqrt((0.2*2.5+\x^2)^2+((0.2-2.5)*\x)^2)});
  
 % \draw[-] (-2.0-0.1,-0.1)--(-2.0+0.1,0.1) node[midway, below, outer sep=5pt] {$-\omega_{0}$};
 % \draw[-] (+2 .0-0.1,-0.1)--(+2.0+0.1,0.1) node[midway, below, outer sep=5pt] {$\omega_{0}$};
  \pgfmathsetlengthmacro{\radius}{(0.2+2.5)/sqrt((0.2*2.5)^2+(0)^2)*1cm}
  \draw[-] (-0.1,\radius-0.1)--(+0.1,\radius+0.1) node[midway, left] {$A \cdot \frac{ \beta + \alpha }{\alpha \cdot \beta}$};

  %\draw[-,dashed] (-2.0,\radius)--(+2.0,\radius);
  
  %\draw[-,dashed] (+2.0,+0.0)--(+2.0,\radius);
  %\draw[-,dashed] (-2.0,+0.0)--(-2.0,\radius);
  \end{tikzpicture}
\end{figure}

\TT{Widmo aplitudowe sygnału rzeczywistego jest zawsze parzyste.}{The magnitude spectrum of a \underline{real signal} is an even-symmetric function of $w$.}

\TT{Widmo fazowe obliczamy ze wzoru:}{The phase spectrum is defined as:}

\begin{align*}
\Phi ( \omega )&=arg\left(A \cdot \frac{1}{\alpha -\jmath \cdot \omega }
+A \cdot \frac{1}{\beta +\jmath \cdot \omega } \right)=\\
&=arg\left(A \cdot \left( \frac{1}{\alpha -\jmath \cdot \omega }
+ \frac{1}{\beta +\jmath \cdot \omega }\right)\right)=\\
&=arg\left( A \cdot \left( \frac{\beta +\jmath \cdot \omega}{\left(\alpha -\jmath \cdot \omega \right) \cdot \left(\beta +\jmath \cdot \omega \right)}
+ \frac{\alpha -\jmath \cdot \omega}{\left(\alpha -\jmath \cdot \omega \right) \cdot \left(\beta +\jmath \cdot \omega \right) }\right)\right)=\\
&=arg\left(A \cdot \frac{\beta +\jmath \cdot \omega + \alpha -\jmath \cdot \omega}{\left(\alpha -\jmath \cdot \omega \right) \cdot \left(\beta +\jmath \cdot \omega \right)}\right)=\\
&=arg\left( A \cdot \frac{\beta + \alpha}{ \alpha \cdot \beta + \jmath \cdot \alpha \cdot \omega - \jmath \cdot \beta \cdot \omega - \jmath \cdot \omega \cdot \jmath \cdot \omega} \right)=\\
&=arg\left( A \cdot \frac{\beta + \alpha}{ \alpha \cdot \beta + \jmath \cdot\left( \alpha - \beta \right)\cdot \omega + \omega^2} \right)=\\
&=\begin{Bmatrix}
arg\left(\frac{z_1}{z_2}\right) = arg\left(z_1\right) -arg\left(z_2\right)
\end{Bmatrix}=\\
&=arg\left(  A \cdot \left(\beta + \alpha\right)\right) - arg\left(\alpha \cdot \beta + \jmath \cdot\left( \alpha - \beta \right)\cdot \omega + \omega^2 \right)=\\
&=\begin{Bmatrix}
arg\left(a + \jmath \cdot b\right) = \TT{arctg}{arctan}(\frac{b}{a})\\
\end{Bmatrix}=\\
&=\TT{arctg}{arctan}\left(\frac{0}{ A \cdot \left(\beta + \alpha\right)}\right) - \TT{arctg}{arctan}\left(\frac{\left( \alpha - \beta \right)\cdot \omega}{\alpha \cdot \beta+ \omega^2}\right)=\\
&=0 - \TT{arctg}{arctan}\left(\frac{\left( \alpha - \beta \right)\cdot \omega}{\alpha \cdot \beta+ \omega^2}\right)=\\
&= - \TT{arctg}{arctan}\left(\frac{\left( \alpha - \beta \right)\cdot \omega}{\alpha \cdot \beta+ \omega^2}\right)
\end{align*}

\begin{figure}[H]
  \centering
  \begin{tikzpicture}
  \draw[->] (-5.0,+0.0) -- (+5.0,+0.0) node[right] {$\omega$};
  \draw[->] (+0.0,-1.5) -- (+0.0,+2.0) node[above] {$\Phi(\omega)$};
  
  \draw[scale=1.0,domain=-4.0:4.0,samples=2000,smooth,variable=\x,red,thick] plot ({\x},{-atan((0.2-0.5)*\x/(0.2*0.5+\x*\x)*3.1415/180});
  
%  \draw[-] (-2.0-0.1,-0.1)--(-2.0+0.1,0.1) node[midway, below, outer sep=5pt] {$-\omega_0$};
%  \draw[-] (+2.0-0.1,-0.1)--(+2.0+0.1,0.1) node[midway, below, outer sep=5pt] {$\omega_0$};
  \draw[-] (-0.1,+1.0-0.1)--(+0.1,+1.0+0.1) node[midway, above left] {$\frac{\pi}{2}$};
  \draw[-] (-0.1,-1.0-0.1)--(+0.1,-1.0+0.1) node[midway, above left] {$-\frac{\pi}{2}$};
  
  \end{tikzpicture}
\end{figure}

\TT{Widmo fazowe sygnału rzeczywistego jest zawsze nieparzyste.}{The phase spectrum of a \underline{real signal} is an odd-symmetric function of $w$.}

\end{task}
