\begin{task}

\TT{Wyznacz odpowied\'{z} implusową $h(t)$ układu LTI, wiedząc, że sygnały $u(t)$ oraz $y(t)$ wygladają jak na poniższych wykresach. Wykorzystaj informacje o transformatach sygnałów: $\Pi(t) \xrightarrow{\mathcal F} Sa\left(\frac{\omega}{2}\right)$ oraz $\Lambda(t) \xrightarrow{\mathcal F} Sa^2\left(\frac{\omega}{2}\right)$.}{Calculate the impulse response $h(t)$ of an LTI system for input $u(t)$ and output $y(t)$ signals given below. Exploit transforms: $\Pi(t) \xrightarrow{\mathcal F} Sa\left(\frac{\omega}{2}\right)$ and $\Lambda(t) \xrightarrow{\mathcal F} Sa^2\left(\frac{\omega}{2}\right)$.}

\begin{figure}[H]
	\centering
	\begin{tikzpicture}
	%\draw (0,0) circle (1in);
	\draw[->] (-2.0-5.0,+0.0) -- (+2.0-5.0,+0.0) node[right] {$t$};
	\draw[->] (+1.0-5.0,-1.0) -- (+1.0-5.0,+2.5) node[above] {$u(t)$};
	
	\draw[-,red, thick] (-1.0-5.0,+1.5) -- (+1.0-5.0,+1.5);
	\draw[-,red, thick] (-1.0-5.0,+0.0) -- (-1.0-5.0,+1.5);
	\draw[-,red, thick] (+1.0-5.0,+0.0) -- (+1.0-5.0,+1.5);
	
	\draw[-] (-0.1-4.0,+1.5-0.1)--(+0.1-4.0,+1.5+0.1) node[midway, above left] {$A$};
	\draw[-] (-1.0-0.1-5.0,-0.1)--(-1.0+0.1-5.0,0.1) node[midway, below, outer sep=5pt,align=center] {$-t_0$};
	
	
	\draw[->] (-2.0,+0.0) -- (+2.0,+0.0) node[right] {$t$};
	\draw[->] (+0.0,-1.0) -- (+0.0,+2.5) node[above] {$y(t)$};
	
	%\draw[-,red, thick] (-1.0,+2.0) -- (+1.0,+2.0);
	\draw[-,green, thick] (-1.0,+0.0) -- (+0.0,+2.0);
	\draw[-,green, thick] (+1.0,+0.0) -- (+0.0,+2.0);
	
	\draw[-] (-0.1,+2.0-0.1)--(+0.1,+2.0+0.1) node[midway, above left] {$A \cdot t_0$};
	\draw[-] (-1.0-0.1,-0.1)--(-1.0+0.1,0.1) node[midway, below, outer sep=5pt,align=center] {$-t_0$};
	\draw[-] (+1.0-0.1,-0.1)--(+1.0+0.1,0.1) node[midway, below, outer sep=5pt,align=center] {$t_0$};
	
	\end{tikzpicture}
\end{figure}

\TT{Wiemy, że transformatę odpowiedzi układu można wyznaczyć ze wzoru $Y(\jmath \omega)=U(\jmath \omega) \cdot H(\jmath \omega)$ oraz że $h(t) \xrightarrow{\mathcal F} H(\jmath \omega)$. W związku z tym $H(\jmath \omega)=\frac{Y(\jmath \omega)}{U(\jmath \omega)}$ oraz $h(t) \xrightarrow{\mathcal F^{-1}} H(\jmath \omega)$.}{We know that for the LTI systems: $Y(\jmath \omega)=U(\jmath \omega) \cdot H(\jmath \omega)$ and $h(t) \xrightarrow{\mathcal F} H(\jmath \omega)$. So, $h(t)=\mathcal F^{-1}\{H(\jmath \omega)\}$ and $H(\jmath \omega)=\frac{Y(\jmath \omega)}{U(\jmath \omega)}$.}

\TT{Aby wyznaczyć transmitancję $H(\jmath \omega)$ trzeba obliczyć sygnałów $u(t)$ oraz $y(t)$:}{In order to derive frequency response $H(\jmath \omega)$, the Fourier transforms of the $u(t)$ and $y(t)$ signals have to be computed:}

\begin{align*}
u(t)&= A \cdot \Pi\left(\frac{t+\frac{t_0}{2}}{t_0}\right) & y(t)&= A \cdot t_0 \cdot \Lambda\left(\frac{t}{t_0}\right)\\
u(t) &\xrightarrow{\mathcal F}U(\jmath \omega) & y(t) &\xrightarrow{\mathcal F}Y(\jmath \omega) \\
\Pi(t) &\xrightarrow{\mathcal F} Sa\left(\frac{\omega}{2}\right) & \Lambda(t) &\xrightarrow{\mathcal F} Sa^2\left(\frac{\omega}{2}\right)\\
\Pi\left(\frac{1}{t_0} \cdot t \right) &\xrightarrow{\mathcal F} t_0 \cdot Sa\left(\frac{\omega \cdot t_0}{2}\right) & \Lambda\left(\frac{1}{t_0} \cdot t \right) &\xrightarrow{\mathcal F} t_0 \cdot Sa^2\left(\frac{\omega \cdot t_0}{2}\right)\\
\Pi\left(\frac{t+\frac{t_0}{2}}{t_0}\right) &\xrightarrow{\mathcal F} t_0 \cdot Sa\left(\frac{\omega \cdot t_0}{2}\right) \cdot e^{\jmath \cdot \omega \cdot \frac{t_0}{2}} & A \cdot t_0 \cdot \Lambda\left(\frac{t}{t_0}\right) &\xrightarrow{\mathcal F} A \cdot t_0^2 \cdot Sa^2\left(\frac{\omega \cdot t_0}{2}\right)\\
A \cdot \Pi\left(\frac{t+\frac{t_0}{2}}{t_0}\right) &\xrightarrow{\mathcal F} A \cdot t_0 \cdot Sa\left(\frac{\omega \cdot t_0}{2}\right) \cdot e^{\jmath \cdot \omega \cdot \frac{t_0}{2}} &  &
\end{align*}

\TT{Skoro zanmy transformaty sygnałów wejściowego i wyjściowego, to możemy wyznaczyc transmitancję układu, czyli $H(\jmath \omega)$.}{Now, frequency response $H(\jmath \omega)$ can be derived:}

\begin{align*}
H(\jmath \omega)&=\frac{Y(\jmath \omega)}{U(\jmath \omega)}=\\
&=\frac{A \cdot t_0^2 \cdot Sa^2\left(\frac{\omega \cdot t_0}{2}\right)}{A \cdot t_0 \cdot Sa\left(\frac{\omega \cdot t_0}{2}\right) \cdot e^{\jmath \cdot \omega \cdot \frac{t_0}{2}}}=\\
&=t_0 \cdot Sa\left(\frac{\omega \cdot t_0}{2}\right) \cdot e^{-\jmath \cdot \omega \cdot \frac{t_0}{2}}
\end{align*}

\TT{Teraz możemy wyznaczć odpowied\'{z} implusową układu $h(t)$:}{Finally, the impulse response $h(t)$ can be calculated:}

\begin{align*}
h(t) &\xrightarrow{\mathcal F} H(\jmath \omega)\\
? &\xrightarrow{\mathcal F} t_0 \cdot Sa\left(\frac{\omega \cdot t_0}{2}\right) \cdot e^{-\jmath \cdot \omega \cdot \frac{t_0}{2}}\\
\Pi(t) &\xrightarrow{\mathcal F} Sa\left(\frac{\omega}{2}\right)\\
\Pi\left(\frac{1}{t_0} \cdot t \right) &\xrightarrow{\mathcal F} t_0 \cdot Sa\left(\frac{\omega \cdot t_0}{2}\right)\\
\Pi\left(\frac{t-\frac{t_0}{2}}{t_0}\right) &\xrightarrow{\mathcal F} t_0 \cdot Sa\left(\frac{\omega \cdot t_0}{2}\right) \cdot e^{-\jmath \cdot \omega \cdot \frac{t_0}{2}}\\
\end{align*}


\TT{Odpowied\'{z} implusowa układu to $h(t)= \Pi\left(\frac{t-\frac{t_0}{2}}{t_0}\right)$.}{The impulse response of the system is equal to $h(t)= \Pi\left(\frac{t-\frac{t_0}{2}}{t_0}\right)$.}


\end{task}

