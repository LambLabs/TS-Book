\begin{task}
\TT{Oblicz moc okresowego sygnału $f(t)$ przedstawionego na rysunku:}{Compute the average power for the following periodic signal $f(t)$:}

\begin{figure}[H]
\centering
\begin{tikzpicture}
  %\draw (0,0) circle (1in);
  \draw[->] (-3.0,+0.0) -- (+5.0,+0.0) node[right] {$t$};
  \draw[->] (+0.0,-1.5) -- (+0.0,+1.5) node[above] {$f(t)$};
  \draw[-,red, thick] (-2.5,+0.0) -- (-2.0,+0.0) -- (-2.0,+1.0) -- (-1.4,+1.0) -- (-1.4,+0.0)--(+0.0,+0.0) -- (+0.0,+1.0) -- (+0.6,+1.0) -- (+0.6,+0.0) -- (+2.0,+0.0) -- (+2.0,+1.0) -- (2.6,1.0) -- (2.6,0.0) -- (3.5,0.0);
  %\draw[-] (-1.0-0.1,-0.1)--(-1.0+0.1,0.1) node[midway, below, outer sep=10pt,align=center] {$-\frac{T}{2}$};
  \draw[-] (-1.4-0.1,-0.1)--(-1.4+0.1,0.1) node[midway, below, outer sep=5pt,align=center] {$-\frac{2\cdot T}{3}$};
  \draw[-] (+0.6-0.1,-0.1)--(+0.6+0.1,0.1) node[midway, below, outer sep=5pt] {$\frac{T}{3}$};
  \draw[-] (+2.0-0.1,-0.1)--(+2.0+0.1,0.1) node[midway, below, outer sep=5pt] {$T$};
  \draw[-] (-0.1,1.0-0.1)--(+0.1,1.0+0.1) node[midway, left] {$A$};
\end{tikzpicture}
\end{figure}

\TT{Zaczynamy od zapisania wzoru funkcji przedstawionej na rysunku:}{Signal $f(t)$ can de described as:}

\begin{equation}
f(x)=\begin{cases}A & t \in \left (  0+k \cdot T; \frac{T}{3}+k \cdot T \right ) \\0 & t \in \left ( \frac{T}{3}+k \cdot T; T +k \cdot T\right )\end{cases} \wedge k \in \TT{C}{Z} 
\end{equation}

\TT{Moc sygnału okresowego wyznaczamy ze wzoru:}{The average power for periodic signals is defined by:}

\begin{equation}
P=\frac{1}{T} \cdot \int_{T}^{}\left|f(t)\right|^2 \cdot dt
\end{equation}

\TT{Podstawiamy do wzoru na moc wzór naszej funkcji dla pierwszego okresu $k=0$}{Compute average power for period $k=0$}

\begin{align*}
P&=\frac{1}{T} \cdot \int_{T}\left|f(t)\right|^2 \cdot dt =\\
&=\frac{1}{T} \cdot \left( \int_{0}^{\frac{T}{3}}\left| A \right|^2 \cdot dt + \int_{\frac{T}{3}}^{T}\left| 0 \right|^2 \cdot dt\right)=\\
&=\frac{1}{T} \cdot \left( \int_{0}^{\frac{T}{3}} A ^2 \cdot dt + \int_{\frac{T}{3}}^{T}0 \cdot dt\right)=\\
&=\frac{1}{T} \cdot \left( A ^2 \cdot \int_{0}^{\frac{T}{3}} dt + 0\right)=\\
&=\frac{A^2}{T} \cdot \left. t \right|_{0}^{\frac{T}{3}}=\\
&=\frac{A^2}{T} \cdot \left( \frac{T}{3} - 0 \right)=\\
&=\frac{A^2}{T} \cdot \frac{T}{3}=\\
&=\frac{A^2}{3}\\
\end{align*}

\TT{Moc sygnału wynosi $\frac{A^2}{3}$.}{Average power equals to $\frac{A^2}{3}.$}
\end{task}