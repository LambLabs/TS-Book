\begin{task}
\TT{Oblicz transformatę Fouriera sygnału $f(t)$ podanego poniżej za pomocą twierdzeń, wiedząc że transformata sygnału $\Lambda(t)$ jest równa $Sa^2\left(\frac{\omega}{2}\right)$.}{Compute the Fourier transform of the $f(t)$ signal given below using theorems describing the properties of Fourier transformation. Exploit the following transform $\mathcal F\{\Lambda(t)\} = Sa^2\left(\frac{\omega}{2}\right)$.}

\begin{equation}
f(t) = \begin{cases}
0 & \TT{\text{ dla }}{\text{ for }} t \in \left(-\infty; -1\right)\\
(t+1)^2 & \TT{\text{ dla }}{\text{ for }} t \in \left(-1; 0\right)\\
-(t-1)^2 & \TT{\text{ dla }}{\text{ for }} t \in \left(0; 1\right)\\
0 & \TT{\text{ dla }}{\text{ for }} t \in \left(1; \infty\right)
\end{cases}
\end{equation}

\begin{equation}
\Lambda(t) \overset{F}{\rightarrow} Sa^2\left(\frac{\omega}{2}\right)
\end{equation}

\begin{figure}[H]
\centering
\begin{tikzpicture}
  %\draw (0,0) circle (1in);
  \draw[->] (-3.0,+0.0) -- (+3.0,+0.0) node[right] {$t$};
  \draw[->] (+0.0,-2.5) -- (+0.0,+2.5) node[above] {$f(t)$};
  \draw[-,red, thick] (-2.5,+0.0) -- (-2.0,0.0);
  \draw[scale=1.0,domain=-2.0:0.0,samples=1000,smooth,variable=\x,red,thick] plot ({\x},{2*(((\x/2)+1)^2)});
  \draw[scale=1.0,domain=0.0:2.0,samples=1000,smooth,variable=\x,red,thick] plot ({\x},{-2*(((\x/2)-1)^2)});
  \draw[-,red, thick] (2.0,+0.0) -- (2.5,0.0);  
  \draw[-] (-2.0-0.1,-0.1)--(-2.0+0.1,0.1) node[midway, below, outer sep=5pt,align=center] {$-1$};
  \draw[-] (+2.0-0.1,-0.1)--(+2.0+0.1,0.1) node[midway, above, outer sep=5pt] {$1$};
  \draw[-] (-0.1,+2.0-0.1)--(+0.1,+2.0+0.1) node[midway, left] {$1$};
  \draw[-] (-0.1,-2.0-0.1)--(+0.1,-2.0+0.1) node[midway, left] {$-1$};
\end{tikzpicture}
\end{figure}

\TT{Wyznaczmy pochodną sygnału $f(t)$, czyli sygnał $g(t)= \frac{\partial}{\partial t}f(t)$.}{Let's derive derivative of the $f(t)$ signal as $g(t)= \frac{\partial}{\partial t}f(t)$:}

\begin{figure}[H]
    \centering
    \begin{tikzpicture}
    \draw[->] (-3.0,+0.0) -- (+3.0,+0.0) node[right] {$t$};
    \draw[->] (+0.0,-2.5) -- (+0.0,+4.5) node[above] {$g(t)$};
    \draw[-,red, thick] (-2.5,+0.0) -- (-2.0,0.0);
    \draw[scale=1.0,domain=-2.0:0.0,samples=1000,smooth,variable=\x,red,thick] plot ({\x},{(2*\x)+4});
    \draw[->,red, thick] (0.0,+0.0) -- (0.0,-2.0);
    \draw[scale=1.0,domain=0.0:2.0,samples=1000,smooth,variable=\x,red,thick] plot ({\x},{(-2*\x)+4});
    \draw[-,red, thick] (2.0,+0.0) -- (2.5,0.0);  
    
    \draw[-] (-2.0-0.1,-0.1)--(-2.0+0.1,0.1) node[midway, below, outer sep=5pt,align=center] {$-1$};
    \draw[-] (+2.0-0.1,-0.1)--(+2.0+0.1,0.1) node[midway, below, outer sep=5pt] {$1$};
    \draw[-] (0.0-0.1,4.0-0.1)--(0.0+0.1,4.0+0.1) node[midway, left] {$2$};
    \draw[] (0.0,-2.0)--(0.0,-2.0) node[midway, below left] {($-2$)};
    \end{tikzpicture}
\end{figure}

\TT{Sygnał $g(t)$ można opisać, wykorzystując sygnały elementarne:}{Using the elementary signals we can write:}

\begin{equation}
g(t) = 2 \cdot \Lambda(t) -2 \cdot \delta(t)
\end{equation}

\TT{Można sprawdzić, że całkując sygnał $g(t)$ otrzymamy sygnał $f(t)$, czyli:}{You can check that by integrating the $g(t)$ signal we'll get the $f(t)$ signal:}

\begin{equation}
f(t) = \int_{-\infty}^{t} g(x) \cdot dx
\end{equation}

\TT{Skoro tak jest, to transformatę sygnału $f(t)$ mozna wyznaczyć z twierdzenia o całkowaniu sygnału, w tym przypadku całkować będziemy sygnał $g(t)$:}{Therefore, the Fourier transform of the $f(t)$ signal can be determined from the integration theorem. In this case we will integrate the $g(t)$ signal:}

\begin{equation}
F(\jmath \omega) = \frac{1}{\jmath \cdot \omega} \cdot G(\jmath \omega) + \pi \cdot \delta(\omega) \cdot G(0)
\end{equation}

\TT{Z powyższego równania widać, że musimy znać $G(\jmath \omega)$, czyli transformatę sygnału $g(t)$.}{In order to derive $F(\jmath \omega)$ we have to calculate the $G(\jmath \omega)$ transform of the $g(t)$ signal.}

\TT{Ponieważ transformacja Fouriera jest przekształceniem liniowym, dlatego można wyznaczyć osobno transformaty poszczególnych sygnałów elementarnych, czyli:}{Based on linearity of the Fourier transformation, we can calculate transforms for elementary signals separately:}

\begin{equation}
g(t) = 2 \cdot \left( g_{1}(t) - g_{2}(t) \right)
\end{equation}
\TT{gdzie:}{where:}
\begin{align*}
g_{1}(t) = \Lambda(t)\\
g_{2}(t) = \delta(t)
\end{align*}

\TT{Wyznaczmy transformtę sygnału $g_{1}(t)$, czyli $G_{1}(\jmath \omega)$.}{Calculate the Fourier transform $G_{1}(\jmath \omega)$ for the first signal $g_{1}(t)$.}

\TT{Z treści zadania wiemy, że:}{We know that:}
$\mathcal F\{\Lambda(t)\} = Sa^2\left(\frac{\omega}{2}\right)$.

\TT{Transformata sygnału $g_{1}(t)$ to:}{The Fourier transform of $g_{1}(t)$ signal is equal to:}
\begin{equation}
G_{1}(\jmath \omega) = \mathcal F\{g_{1}(t)\} = Sa^2\left(\frac{\omega}{2}\right)
\end{equation}

\TT{Wyznaczmy transformtę sygnału $g_{2}(t)$, czyli $G_{2}(\jmath \omega)$.}{Now, let's calculate the Fourier transform $G_{2}(\jmath \omega)$ for the second signal $g_{2}(t)$.}
\TT{Dla sygnału $g_2(t) = \delta(t)$ w bardzo łatwy sposób można wyznaczyć transformatę Fouriera.}{The Fourier transform for the $g_2(t) = \delta(t)$ signal can be calculated as:}

\begin{align*}
G_2(\jmath \omega) &= \int_{-\infty}^{\infty} g_2(t) \cdot e^{-\jmath \cdot \omega \cdot t}=\\
&=\int_{-\infty}^{\infty} \delta(t) \cdot e^{-\jmath \cdot \omega \cdot t}=\\
&=\begin{Bmatrix}
\SamplingPropertyOfDelta
\end{Bmatrix}=\\
&=e^{-\jmath \cdot \omega \cdot 0}=\\
&=e^{0}=\\
&=1
\end{align*}

\TT{Transformatą Fouriera sygnału $g_2(t)=\delta(t)$ jest $G_2(\jmath \omega)=1$.}{The Fourier transform of the $g_2(t)=\delta(t)$ signal is equal to $G_2(\jmath \omega)=1$.}

\TT{Transformatę funkcji $g(t)$ możemy wyznaczyć z twierdzenia o jednorodności}{Finally, the Fourier transform of the $g(t)$ signal can be derived using the linearity theorem:}

\begin{align*}
\HomogeneousTeorem{g}{G}
\end{align*}

\begin{align*}
g(t) &= 2 \cdot \left( g_{1}(t) - g_{2}(t) \right)\\
G(\jmath \omega)&=2 \cdot \left(G_1(\jmath \omega) - G_2(\jmath \omega)\right)=\\
&=2 \cdot \left(Sa^2\left(\frac{\omega}{2}\right) - 1\right)
\end{align*}

\TT{Mamy wyznaczoną transformatę $G(\jmath \omega)$. Teraz, z twierdzenia o całkowaniu sygnału, możemy wyznaczyc transformatę $F(\jmath \omega)$:}{We derived the $G(\jmath \omega)$ transform. Now, based on the integration theorem, we can calculate the $F(\jmath \omega)$ Fourier transform.}

\begin{align*}
F(\jmath \omega)&= \frac{1}{\jmath \cdot \omega} \cdot G(\jmath \omega) + \pi \cdot \delta(\omega) \cdot G(0)=\\
&=\frac{1}{\jmath \cdot \omega} \cdot 2 \cdot \left(Sa^2\left(\frac{\omega}{2}\right) - 1\right)+ \pi \cdot \delta(\omega) \cdot G(0)=\\
&=\begin{Bmatrix*}[l]
G(0)=2 \cdot \left(Sa^2\left(\frac{0}{2}\right) - 1\right)\\
G(0)=2 \cdot \left(Sa^2\left(0\right) - 1\right)\\
G(0)=2 \cdot \left(1 -1\right)\\
G(0)=0\\
\end{Bmatrix*}=\\
&=\frac{2}{\jmath \cdot \omega} \cdot \left(Sa^2\left(\frac{\omega}{2}\right) - 1\right)
\end{align*}

\TT{Transformata funkcji $f(t)$ jest równa $F(\jmath \omega)=\frac{2}{\jmath \cdot \omega} \cdot \left(Sa^2\left(\frac{\omega}{2}\right) - 1\right)$.}{The Fourier transform of the $f(t)$ is equal to $F(\jmath \omega)=\frac{2}{\jmath \cdot \omega} \cdot \left(Sa^2\left(\frac{\omega}{2}\right) - 1\right)$.}

\end{task}

