\begin{task}
\TT{Oblicz transformatę Fouriera sygnału $f(t)=Sa^2\left(\frac{t-t_0}{T}\right) \cdot cos\left(\omega_0 \cdot t\right)$ za pomocą twierdzeń, wiedząc że transformata sygnału $\Lambda(t)$ jest równa $Sa^2\left(\frac{\omega}{2}\right)$.}{Compute the Fourier transform of the $f(t)=Sa^2\left(\frac{t-t_0}{T}\right) \cdot cos\left(\omega_0 \cdot t\right)$ signal using theorems describing the properties of Fourier transformation. Exploit the following transform $\mathcal F\{\Lambda(t)\} = Sa^2\left(\frac{\omega}{2}\right)$.}

\begin{equation}
f(t) = Sa^2\left(\frac{t-t_0}{T}\right) \cdot cos\left(\omega_0 \cdot t\right)
\end{equation}

\begin{equation}
\Lambda(t) \overset{F}{\rightarrow} Sa^2\left(\frac{\omega}{2}\right)
\end{equation}

\begin{figure}[H]
\centering
\begin{tikzpicture}
  %\draw (0,0) circle (1in);
  \draw[->] (-4.0,+0.0) -- (+4.0,+0.0) node[right] {$t$};
  \draw[->] (+0.0,-2.5) -- (+0.0,+2.5) node[above] {$f(t)$};
  \draw[scale=1.0,domain=-3.5:3.5,samples=1000,smooth,variable=\x,red,thick] plot ({\x},{2*sinc((\x-1.8)*3.141592*2)*sinc((\x-1.8)*3.141592*2)*cos(\x*3.141592*2)});%
  
  \draw[-] (-1.0-0.1,-0.1)--(-1.0+0.1,0.1) node[midway, above, outer sep=5pt,align=center] {$-\frac{2\pi}{\omega_0}$};
  \draw[-] (-2.0-0.1,-0.1)--(-2.0+0.1,0.1) node[midway, above, outer sep=5pt,align=center] {$-\frac{4\pi}{\omega_0}$};
  \draw[-] (-3.0-0.1,-0.1)--(-3.0+0.1,0.1) node[midway, above, outer sep=5pt,align=center] {$-\frac{6\pi}{\omega_0}$};
  \draw[-] (+1.0-0.1,-0.1)--(+1.0+0.1,0.1) node[midway, below, outer sep=5pt] {$\frac{2\pi}{\omega_0}$};
  \draw[-] (+1.8-0.1,-0.1)--(+1.8+0.1,0.1) node[midway, below, outer sep=25pt] {$t_0$};
  \draw[-] (+2.0-0.1,-0.1)--(+2.0+0.1,0.1) node[midway, below, outer sep=5pt] {$\frac{4\pi}{\omega_0}$};
  \draw[-] (+3.0-0.1,-0.1)--(+3.0+0.1,0.1) node[midway, below, outer sep=5pt] {$\frac{6\pi}{\omega_0}$};
  \draw[-] (-0.1,+2.0-0.1)--(+0.1,+2.0+0.1) node[midway, left] {$A$};
  \draw[-] (-0.1,-2.0-0.1)--(+0.1,-2.0+0.1) node[midway, left] {$-A$};
\end{tikzpicture}
\end{figure}

\TT{W pierwszej kolejności można funkcję $f(t)$ rozpisać następująco:}{Firstly, note that using Euler's identity we can write:}

\begin{align*}
f(t) &= Sa^2\left(\frac{t-t_0}{T}\right) \cdot cos\left(\omega_0 \cdot t\right)=\\
&=\begin{Bmatrix}
\EulerCos
\end{Bmatrix}=\\
&= Sa^2\left(\frac{t-t_0}{T}\right) \cdot \frac{e^{\jmath \cdot \omega_0 \cdot t} + e^{-\jmath \cdot \omega_0 \cdot t}}{2}=\\
&= \frac{1}{2} \cdot \left( Sa^2\left(\frac{t-t_0}{T}\right) \cdot e^{\jmath \cdot \omega_0 \cdot t} + Sa^2\left(\frac{t-t_0}{T}\right) \cdot e^{-\jmath \cdot \omega_0 \cdot t} \right)=\\
&=\begin{Bmatrix}
f_1(t) &= Sa^2\left(\frac{t-t_0}{T}\right) \cdot e^{\jmath \cdot \omega_0 \cdot t} \\
f_2(t) &= Sa^2\left(\frac{t-t_0}{T}\right) \cdot e^{-\jmath \cdot \omega_0 \cdot t}
\end{Bmatrix}=\\
&=\frac{1}{2} \cdot \left(f_1(t) + f_2(t) \right)
\end{align*}

\TT{Należy zauważyć iż funkcja $f_1(t)$ i $f_2(t)$ jest złożeniem funkcji $Sa^2$ i funkcji wykładniczych.}{Please note, that $f_1(t)$ and $f_2(t)$ signals are the product of $Sa^2$ and complex exponential signals:}

\begin{align*}
f_1(t) &= Sa^2\left(\frac{t-t_0}{T}\right) \cdot e^{\jmath \cdot \omega_0 \cdot t} = g(t)\cdot e^{\jmath \cdot \omega_0 \cdot t}\\
f_2(t) &= Sa^2\left(\frac{t-t_0}{T}\right) \cdot e^{-\jmath \cdot \omega_0 \cdot t} = g(t) \cdot e^{-\jmath \cdot \omega_0 \cdot t}
\end{align*}

\TT{Znając transformatę sygnału $g(t) = Sa^2\left(\frac{t-t_0}{T}\right)$ możemy skorzystać z twierdzenia o przesunięciu w dziedzinie częstotliwości.}{Knowing the Fourier transform of $g(t) = Sa^2\left(\frac{t-t_0}{T}\right)$ we can use the modulation theorem to derive the Fourier transform of $f_x(t)$:}

\begin{align*}
\FrequencyShiftTeorem{g}{G}{f}{F}
\end{align*}

\TT{Aby wyznaczyć transformatę sygnału $g(t)$ możemy skorzystać z twierdzenia o symetrii. Znając transformatę $H(\jmath \omega)$ sygnału $h(t)$ można wyznaczyć transformatę $G(\jmath \omega)$ sygnału $g(t)$:}{We know that $\mathcal F\{\Lambda(t)\}=Sa^2\left(\frac{\omega}{2}\right)$. Based on time-frequency duality theorem we can derive transform for $g(t)$ signal:}

\begin{align*}
\SymetryTeorem{h}{H}{g}{G}
\end{align*}

\TT{Tak wiec zacznijmy od transformaty sygnału trójkątnego $h(t)=\Lambda(t)$ i wyznaczymy transformatę funkcji $Sa^2$.}{Let's start with deriving the Fourier transform for the $Sa^2$ signal using $h(t)=\Lambda(t)$:}

\begin{align*}
h(t)=\Lambda(t) &\overset{F}{\rightarrow} H(\jmath \omega) = Sa^2\left(\frac{\omega}{2}\right)\\
g_1(t) = H(t) = Sa^2\left(\frac{t}{2}\right) &\overset{F}{\rightarrow} 
G_1(\jmath \omega) = 2\pi \cdot h(\omega) = 2\pi \cdot  \Lambda\left(-\omega\right) = 2\pi \cdot  \Lambda\left(\omega\right)
\end{align*}

\TT{Wyznaczyliśmy transformatę funkcji $g_1(t)$. Jednak funkcja $g_1(t)$ nie ma takiej samej postaci jak funkcja $g(t)$.}{We derived Fourier transform for the $g_1(t)$ signal. However, this is not exactly the same signal as $g(t)$.}

\begin{align*}
g(t)&=Sa^2\left(\frac{t-t_0}{T}\right)=\\
&=\begin{Bmatrix}
k(t) = Sa^2\left(\frac{t}{T}\right)
\end{Bmatrix}=\\
&=k\left(t-t_0\right)
\end{align*}

\TT{Znająć transformatę funkcji $k\left(t\right)$ możemy z łatwością wyznaczyć transformatę sygnału $g(t)$ za pomoca twierdzenia o przesunięciu w czasie:}{Knowing the Fourier transform of $k(t)$ we can use the shift in time theorem to derive the Fourier transform of $g(t)$:}

\begin{align*}
\TimeShiftTeorem{k}{K}{g}{G}
\end{align*}

\TT{Wyznaczyliśmy transformatę funkcji $g_1(t)$. Jednak funkcja $g_1(t)$ nie ma takiej samej postaci jak funkcja $k(t)$.}{We derived the Fourier transform for the $g_1(t)$ signal. However, this is not exactly the same signal as $k(t)$.}

\begin{align*}
k(t)&=Sa^2\left(\frac{t}{T}\right)=\\
&=Sa^2\left(\frac{2\cdot t}{2\cdot T}\right)=\\
&=Sa^2\left(\frac{2}{T} \cdot \frac{t}{2}\right)=\\
&=\begin{Bmatrix}
a = \frac{2}{T}
\end{Bmatrix}=\\
&=Sa^2\left(a \cdot \frac{t}{2}\right)=\\
&=g_1(a\cdot t) 
\end{align*}

\TT{Znając transformatę funkcji $g_1(t)$ możemy wyznaczyć transformatę funkcji $k(t)=g_1(a \cdot t)$ za pomocą twierdzenia o zmianie skali.}{Now, we can calculate the Fourier transform for $k(t)=g_1(a \cdot t)$ signal using the scaling theorem:}

\begin{align*}
\TimeScalingTeorem{g_1}{G_1}{k}{K}
\end{align*}

\begin{align*}
K(\jmath \omega) &= \frac{1}{\left|\alpha \right|} \cdot G_1(\jmath \frac{\omega}{\alpha})=\\
&=\begin{Bmatrix}
\alpha = \frac{2}{T}
\end{Bmatrix}=\\
&=\frac{1}{\left| \frac{2}{T} \right|} \cdot G_1( \frac{\omega}{\frac{2}{T}})=\\
&=\begin{Bmatrix}%end
G_1(\jmath \omega) = 2\pi \cdot \Lambda\left(\omega\right)
\end{Bmatrix}=\\
&=\frac{1}{ \frac{2}{T}} \cdot 2\pi \cdot \Lambda\left( \frac{\omega}{\frac{2}{T}}\right)=\\
&=\pi \cdot T \cdot \Lambda\left( \frac{\omega \cdot T}{2}\right)
\end{align*}

\TT{Tak wiec transformata sygnału $k(t)=Sa^2\left(\frac{t}{T}\right)$ jest równa $K(\jmath \omega)=\pi \cdot T \cdot \Lambda\left( \frac{\omega \cdot T}{2}\right)$}{As a result the Fourier transform for the $k(t)=Sa^2\left(\frac{t}{T}\right)$ signal is equal to $K(\jmath \omega)=\pi \cdot T \cdot \Lambda\left( \frac{\omega \cdot T}{2}\right)$.}

\TT{Podstawiając do wzoru na transformatę sygnału $g(t)=Sa^2\left(\frac{t-t_0}{T}\right)=k(t-t_0)$ otrzymyjemy $G(\jmath \omega)=K(\jmath \omega)\cdot e^{-\jmath \cdot \omega \cdot t_0} = \pi \cdot T \cdot \Lambda\left( \frac{\omega \cdot T}{2}\right) \cdot e^{-\jmath \cdot \omega \cdot t_0}$.}{Therefore, for the $g(t)=Sa^2\left(\frac{t-t_0}{T}\right)=k(t-t_0)$ signal, the Fourier transform is equal to $G(\jmath \omega)=K(\jmath \omega)\cdot e^{-\jmath \cdot \omega \cdot t_0} = \pi \cdot T \cdot \Lambda\left( \frac{\omega \cdot T}{2}\right) \cdot e^{-\jmath \cdot \omega \cdot t_0}$.}

\TT{Kolejnym krokiem jest wyznaczenie transformaty dwóch sygnałów:}{Now, we are able to derive transforms for:}

\begin{align*}
f_1(t)&=Sa^2\left(\frac{t-t_0}{T}\right) \cdot e^{\jmath \cdot \omega_0 \cdot t}\\
f_2(t)&=Sa^2\left(\frac{t-t_0}{T}\right) \cdot e^{-\jmath \cdot \omega_0 \cdot t}\\
\end{align*}

\TT{Korzystając z twierdzenie o przesunięciu w dziedzinie częstotliwości:}{Based on the modulation theorem:}

\begin{align*}
\FrequencyShiftTeorem{g}{G}{f_1}{F_1}
\end{align*}

\TT{otrzymujemy:}{we get:}

\begin{align*}
F_1(\jmath \omega)&=G\left(\jmath \left(\omega -\omega_0\right)\right)=\\
&=\pi \cdot T \cdot \Lambda\left( \frac{\left(\omega-\omega_0\right) \cdot T}{2}\right) \cdot e^{-\jmath \cdot \left(\omega-\omega_0\right) \cdot t_0}
\end{align*}

\begin{align*}
F_2(\jmath \omega)&=G\left(\jmath \left(\omega +\omega_0\right)\right)=\\
&=\pi \cdot T \cdot \Lambda\left( \frac{\left(\omega+\omega_0\right) \cdot T}{2}\right) \cdot e^{-\jmath \cdot \left(\omega+\omega_0\right) \cdot t_0}
\end{align*}

\TT{Ostatecznie korzystając z liniowości transformacji Fouriera:}{Finally, based on the linearity theorem:}

\begin{align*}
\HomogeneousTeorem{f}{F}
\end{align*}

\TT{otrzymujemy:}{we get:}

\begin{align*}
F(\jmath \omega)&=\frac{1}{2} \cdot \left(F_1(\jmath \omega)+F_2(\jmath \omega)\right)=\\
&=\frac{1}{2} \cdot \left( \pi \cdot T \cdot \Lambda\left( \frac{\left(\omega-\omega_0\right) \cdot T}{2}\right) \cdot e^{-\jmath \cdot \left(\omega-\omega_0\right) \cdot t_0} + \pi \cdot T \cdot \Lambda\left( \frac{\left(\omega+\omega_0\right) \cdot T}{2}\right) \cdot e^{-\jmath \cdot \left(\omega+\omega_0\right) \cdot t_0} \right)
\end{align*}

\TT{Transformata Fouriera sygnału $f(t)=Sa^2\left(\omega_0 \cdot t\right) \cdot cos\left(\omega_0 \cdot t\right)$ jest równa}{The Fourier transform of the $f(t)=Sa^2\left(\omega_0 \cdot t\right) \cdot cos\left(\omega_0 \cdot t\right)$ signal is equal to:}

\begin{align*}
F(\jmath \omega)=\frac{1}{2} \cdot \left( \pi \cdot T \cdot \Lambda\left( \frac{\left(\omega-\omega_0\right) \cdot T}{2}\right) \cdot e^{-\jmath \cdot \left(\omega-\omega_0\right) \cdot t_0} + \pi \cdot T \cdot \Lambda\left( \frac{\left(\omega+\omega_0\right) \cdot T}{2}\right) \cdot e^{-\jmath \cdot \left(\omega+\omega_0\right) \cdot t_0} \right)
\end{align*}

\end{task}

