\begin{task}
\TT{Oblicz transformatę Fouriera sygnału $f(t)$ przedstawionego na rysunku oraz narysuj jego widmo amplitudowe i fazowe.}{Compute the Fourier transform of a signal shown below. Compute and draw magnitude and phase spectra.}

\begin{figure}[H]
\centering
\begin{tikzpicture}
  \draw[->] (-3.0,+0.0) -- (+7.0,+0.0) node[right] {$t$};
  \draw[->] (+0.0,-2.0) -- (+0.0,+2.0) node[above] {$f(t)$};

  \draw[scale=1.0,domain=-2.5:0.0,smooth,variable=\x,red,thick] plot ({\x},{0});
  \draw[scale=1.0,domain=0.0:6.0,smooth,variable=\x,red,thick] plot ({\x},{1.5*exp(-0.5*\x)*sin(\x*180)});%96%3.141592
  \draw[scale=1.0,domain=0.0:6.0,smooth,variable=\x,red,thick,dashed] plot ({\x},{1.5*exp(-0.5*\x)});
  \draw[scale=1.0,domain=0.0:6.0,smooth,variable=\x,red,thick,dashed] plot ({\x},{-1.5*exp(-0.5*\x)});
  
  \draw[-] (+1.0-0.1,-0.1)--(+1.0+0.1,0.1) node[midway, below, outer sep=5pt] {$\frac{\pi}{\omega_0}$};  
  \draw[-] (+2.0-0.1,-0.1)--(+2.0+0.1,0.1) node[midway, below, outer sep=5pt] {$\frac{2\cdot \pi}{\omega_0}$};  
  \draw[-] (+3.0-0.1,-0.1)--(+3.0+0.1,0.1) node[midway, below, outer sep=5pt] {$\frac{3\cdot \pi}{\omega_0}$};  
  
  \draw[-] (+0.0-0.1,1.5-0.1)--(+0.0+0.1,1.5+0.1) node[midway, left, outer sep=5pt] {$A$}; 
  
  \draw[-] (+0.0-0.1,-1.5-0.1)--(+0.0+0.1,-1.5+0.1) node[midway, left, outer sep=5pt] {$-A$}; 
  
\end{tikzpicture}
\end{figure}

\TT{Sygnał $f(t)$ możemy opisać jako funkcję przedziałową:}{The $f(t)$ signal, as a piecewise function is given by:}

\begin{equation}
f(t) = \begin{cases}
0 & \TT{\text{ dla }}{\text{ if }} t \in \left(-\infty; 0\right)\\
e^{-a\cdot t} \cdot sin(\omega_{0} \cdot t) & \TT{\text{ dla }}{\text{ if }} t \in \left(0; \infty\right)
\end{cases}
\end{equation}

\TT{Transformatę Fouriera obliczamy ze wzoru:}{The Fourier transform is defined as:}

\begin{equation}
F(\jmath \omega )=\int_{-\infty }^{\infty}f(t) \cdot e^{-\jmath \cdot \omega \cdot t}\cdot dt
\end{equation}

\TT{Podstawiamy do wzoru na transformatę wzór naszej funkcji:}{For the given $f(t)$ signal we get:}

\begin{align*}
F(\jmath \omega )&=\int_{-\infty }^{\infty}f(t) \cdot e^{-\jmath \cdot \omega \cdot t}\cdot dt=\\
&=\int_{-\infty}^{0} 0 \cdot e^{-\jmath \cdot \omega \cdot t}\cdot dt
+\int_{0}^{\infty} e^{-a \cdot t} \cdot sin(\omega_{0} \cdot t) \cdot e^{-\jmath \cdot \omega \cdot t}\cdot dt=\\
&=\begin{Bmatrix*}[l]%12
sin(x)=\frac{e^{\jmath \cdot x} - e^{-\jmath \cdot x}}{2 \cdot \jmath}\\
\end{Bmatrix*}=\\
&=\int_{-\infty}^{0} 0 \cdot dt +\int_{0}^{\infty} e^{-a \cdot t} \cdot \left(\frac{e^{\jmath \cdot \omega_{0} \cdot t} - e^{-\jmath \cdot  \omega_{0} \cdot t}}{2 \cdot \jmath} \right) \cdot e^{-\jmath \cdot \omega \cdot t}\cdot dt=\\
&=0 + \lim_{\tau \rightarrow \infty }\frac{1}{2 \cdot \jmath}\left(
\int_{0}^{\tau} e^{-a\cdot t}\cdot e^{\jmath \cdot \omega_{0} \cdot t} \cdot e^{-\jmath \cdot \omega \cdot t} \cdot dt
-\int_{0}^{\tau} e^{-a\cdot t}\cdot e^{-\jmath \cdot \omega_{0} \cdot t} \cdot e^{-\jmath \cdot \omega \cdot t} \cdot dt \right)=\\
&=\lim_{\tau \rightarrow \infty }\frac{1}{2 \cdot \jmath}\left(
\int_{0}^{\tau} e^{(-a + \jmath \cdot \omega_{0} -\jmath \cdot \omega) \cdot t} \cdot dt
-\int_{0}^{\tau} e^{(-a - \jmath \cdot \omega_{0} -\jmath \cdot \omega) \cdot t} \cdot dt \right)=\\
&=\begin{Bmatrix}
z&=&(-a + \jmath \cdot \omega_{0} -\jmath \cdot \omega) \cdot t&w&=&(-a - \jmath \cdot \omega_{0} -\jmath \cdot \omega) \cdot t\\
dz&=&(-a + \jmath \cdot \omega_{0} -\jmath \cdot \omega) \cdot dt&dw&=&(-a - \jmath \cdot \omega_{0} -\jmath \cdot \omega) \cdot dt\\
dt&=&\frac{1}{(-a + \jmath \cdot \omega_{0} -\jmath \cdot \omega)} \cdot dz&dt&=&\frac{1}{(-a - \jmath \cdot \omega_{0} -\jmath \cdot \omega)} \cdot dw\\
\end{Bmatrix}=\\
&=\lim_{\tau \rightarrow \infty }\frac{1}{2 \cdot \jmath}
\int_{0}^{\tau} e^{z} \cdot \frac{dz}{(-a + \jmath \cdot \omega_{0} -\jmath \cdot \omega)}
-\lim_{\tau \rightarrow \infty }\frac{1}{2 \cdot \jmath}
\int_{0}^{\tau} e^{w} \cdot \frac{dw}{(-a - \jmath \cdot \omega_{0} -\jmath \cdot \omega)}=\\
&=\frac{1}{2 \cdot \jmath \cdot (-a + \jmath \cdot \omega_{0} -\jmath \cdot \omega)} \cdot \lim_{\tau \rightarrow \infty }
\int_{0}^{\tau} e^{z} \cdot dz
-\frac{1}{2 \cdot \jmath \cdot (-a - \jmath \cdot \omega_{0} -\jmath \cdot \omega)} \cdot \lim_{\tau \rightarrow \infty }
\int_{0}^{\tau} e^{w} \cdot dw=\\
&=\frac{1}{2 \cdot \jmath \cdot (-a + \jmath \cdot \omega_{0} -\jmath \cdot \omega)} \cdot \lim_{\tau \rightarrow \infty }
\left. e^{z}\right|_{0}^{\tau}
-\frac{1}{2 \cdot \jmath \cdot (-a - \jmath \cdot \omega_{0} -\jmath \cdot \omega)} \cdot \lim_{\tau \rightarrow \infty }
\left. e^{w}\right|_{0}^{\tau}=\\
&=\frac{1}{2 \cdot \jmath \cdot (-a + \jmath \cdot \omega_{0} -\jmath \cdot \omega)} \cdot \lim_{\tau \rightarrow \infty }
\left. e^{(-a + \jmath \cdot \omega_{0} -\jmath \cdot \omega) \cdot t}\right|_{0}^{\tau}+\\
&-\frac{1}{2 \cdot \jmath \cdot (-a - \jmath \cdot \omega_{0} -\jmath \cdot \omega)} \cdot \lim_{\tau \rightarrow \infty }
\left. e^{(-a - \jmath \cdot \omega_{0} -\jmath \cdot \omega) \cdot t}\right|_{0}^{\tau}\\
&=\frac{1}{2 \cdot \jmath \cdot (-a + \jmath \cdot \omega_{0} -\jmath \cdot \omega)} \cdot \lim_{\tau \rightarrow \infty }
\left( e^{(-a + \jmath \cdot \omega_{0} -\jmath \cdot \omega) \cdot \tau}-e^{(-a + \jmath \cdot \omega_{0} -\jmath \cdot \omega) \cdot 0} \right)+\\
&-\frac{1}{2 \cdot \jmath \cdot (-a - \jmath \cdot \omega_{0} -\jmath \cdot \omega)} \cdot \lim_{\tau \rightarrow \infty }
\left( e^{(-a - \jmath \cdot \omega_{0} -\jmath \cdot \omega) \cdot \tau}-e^{(-a - \jmath \cdot \omega_{0} -\jmath \cdot \omega) \cdot 0} \right)=\\
&=\frac{1}{2 \cdot \jmath \cdot (-a + \jmath \cdot \omega_{0} -\jmath \cdot \omega)} \cdot \left(\lim_{\tau \rightarrow \infty }
\left( e^{-a \cdot \tau} \cdot e^{(\jmath \cdot \omega_{0} -\jmath \cdot \omega) \cdot \tau} \right)-\lim_{\tau \rightarrow \infty }1\right)+\\
&-\frac{1}{2 \cdot \jmath \cdot (-a - \jmath \cdot \omega_{0} -\jmath \cdot \omega)} \cdot \left(\lim_{\tau \rightarrow \infty }
\left( e^{-a \cdot \tau} \cdot e^{(-\jmath \cdot \omega_{0} -\jmath \cdot \omega) \cdot \tau} \right)-\lim_{\tau \rightarrow \infty }1\right)=\\
&=\frac{1}{2 \cdot \jmath \cdot (-a + \jmath \cdot \omega_{0} -\jmath \cdot \omega)} \cdot \left(\lim_{\tau \rightarrow \infty }
( e^{-a \cdot \tau})\cdot \lim_{\tau \rightarrow \infty } e^{(\jmath \cdot \omega_{0} -\jmath \cdot \omega) \cdot \tau}-1\right)+\\
&-\frac{1}{2 \cdot \jmath \cdot (-a - \jmath \cdot \omega_{0} -\jmath \cdot \omega)} \cdot \left(\lim_{\tau \rightarrow \infty }
( e^{-a \cdot \tau})\cdot \lim_{\tau \rightarrow \infty } e^{(-\jmath \cdot \omega_{0} -\jmath \cdot \omega) \cdot \tau}-1\right)=\\
&=\frac{1}{2 \cdot \jmath \cdot (-a + \jmath \cdot \omega_{0} -\jmath \cdot \omega)} \cdot \left(0 \cdot \lim_{\tau \rightarrow \infty } e^{(\jmath \cdot \omega_{0} -\jmath \cdot \omega) \cdot \tau}-1\right)+\\
&-\frac{1}{2 \cdot \jmath \cdot (-a - \jmath \cdot \omega_{0} -\jmath \cdot \omega)} \cdot \left(0 \cdot \lim_{\tau \rightarrow \infty } e^{(-\jmath \cdot \omega_{0} -\jmath \cdot \omega) \cdot \tau}-1\right)=\\
&=\frac{-1}{2 \cdot \jmath \cdot (-a + \jmath \cdot \omega_{0} -\jmath \cdot \omega)}+\frac{1}{2 \cdot \jmath \cdot (-a - \jmath \cdot \omega_{0} -\jmath \cdot \omega)}=\\
&=\frac{-(2 \cdot \jmath \cdot (-a - \jmath \cdot \omega_{0} -\jmath \cdot \omega))+2 \cdot \jmath \cdot (-a + \jmath \cdot \omega_{0} -\jmath \cdot \omega)}{2 \cdot \jmath \cdot (-a + \jmath \cdot \omega_{0} -\jmath \cdot \omega) \cdot 2 \cdot \jmath \cdot (-a - \jmath \cdot \omega_{0} -\jmath \cdot \omega)}=\\
&=\frac{2 \cdot \jmath \cdot a + 2 \cdot \jmath^{2} \cdot \omega_{0} + 2 \cdot \jmath^{2} \cdot \omega - 2 \cdot \jmath \cdot a + 2 \cdot \jmath^{2} \cdot \omega_{0} - 2 \cdot \jmath^{2} \cdot \omega}
{4 \cdot \jmath^{2} \cdot 
	(a^{2}+ a \cdot \jmath \cdot \omega_{0}+ a \cdot \jmath \cdot \omega
-a \cdot \jmath \cdot \omega_{0}- \jmath^{2} \cdot \omega_{0}^{2}- \jmath^{2} \cdot \omega_{0} \cdot \omega
+a \cdot \jmath \cdot \omega+ \jmath^{2} \cdot \omega_{0} \cdot \omega + \jmath^{2} \cdot \omega^{2})}=\\
&=\frac{4 \cdot \jmath^{2} \cdot \omega_{0}}
{4 \cdot \jmath^{2} \cdot (a^{2} + 2 \cdot a \cdot \jmath \cdot \omega - \jmath^{2} \cdot \omega_{0}^{2} + \jmath^{2} \cdot \omega^{2})}=\\
&=\frac{\omega_{0}}{a^{2} + 2 \cdot a \cdot \jmath \cdot \omega + \omega_{0}^{2} - \omega^{2}}=\\
&=\frac{\omega_{0}}{\omega_{0}^{2} + (a^{2} + 2 \cdot a \cdot \jmath \cdot \omega - \omega^{2})}=\\
&=\frac{\omega_{0}}{\omega_{0}^{2} + (a + \jmath \cdot \omega)^{2}}
\end{align*}

\TT{Transformata sygnału $f(t)$ to $F(\jmath \omega)=\frac{\omega_{0}}{\omega_{0}^{2} + (a + \jmath \cdot \omega)^{2}}$.}{The Fourier transform of the $f(t)$ is equal to $F(\jmath \omega)=\frac{\omega_{0}}{\omega_{0}^{2} + (a + \jmath \cdot \omega)^{2}}$.}

\TT{Widmo amplitudowe obliczamy ze wzoru:}{The magnitude spectrum is defined as:}

\begin{align*}
M(\omega)&=\left| F(j \omega) \right|=\\
&=\left| \frac{\omega_{0}}{\omega_{0}^{2} + (a + \jmath \cdot \omega)^{2}} \right|=\\
&=\left| \frac{\omega_{0}}{\omega_{0}^{2} + a^2 + 2\cdot a \cdot \jmath \cdot \omega + \left(\jmath \cdot \omega\right)^{2}}\right|=\\
&=\left| \frac{\omega_{0}}{\omega_{0}^{2} + a^2 + 2\cdot a \cdot \jmath \cdot \omega - \omega^{2}}\right|=\\
&=\left| \frac{\omega_{0}}{\omega_{0}^{2}- \omega^{2} + a^2 + \jmath \cdot 2\cdot a \cdot \omega }\right|=\\
&=\begin{Bmatrix}
\left|\frac{z_1}{z_2}\right| = \frac{\left|z_1\right|}{\left|z_2\right|}
\end{Bmatrix}=\\
&=\frac{\left|\omega_{0}\right|}{\left|\omega_{0}^{2}- \omega^{2} + a^2 + \jmath \cdot 2\cdot a \cdot \omega \right|}=\\
&=\begin{Bmatrix}
\left|a+\jmath \cdot b\right| = \sqrt{a^2+b^2}
\end{Bmatrix}=\\
&=\frac{\omega_{0}}{\sqrt{\left(\omega_{0}^{2}- \omega^{2} + a^2 \right)^2+ \left( 2\cdot a \cdot \omega \right)^2}}
\end{align*}

\begin{figure}[H]
  \centering
  \begin{tikzpicture}
  \draw[->] (-5.0,+0.0) -- (+5.0,+0.0) node[right] {$\omega$};
  \draw[->] (+0.0,-1.5) -- (+0.0,+4.0) node[above] {$M(\omega)$};
  
  \draw[scale=1.0,domain=-4.0:4.0,samples=2000,smooth,variable=\x,red,thick] plot ({\x},{4*2/sqrt((2^2-(\x)^2+(0.6)^2)^2+(2*0.6*\x)^2)});
  
  \draw[-] (-2.0-0.1,-0.1)--(-2.0+0.1,0.1) node[midway, below, outer sep=5pt] {$-\omega_{0}$};
  \draw[-] (+2 .0-0.1,-0.1)--(+2.0+0.1,0.1) node[midway, below, outer sep=5pt] {$\omega_{0}$};
  \pgfmathsetlengthmacro{\radius}{4*2/sqrt((0.6)^4+4*(0.6)^2*2^2)*1cm}
  \draw[-] (-0.1,\radius-0.1)--(+0.1,\radius+0.1) node[midway, left] {$\frac{\omega_{0}}{\sqrt{a^4 + 4\cdot a^2 \omega_0^2}}$};

  \draw[-,dashed] (-2.0,\radius)--(+2.0,\radius);
  
  \draw[-,dashed] (+2.0,+0.0)--(+2.0,\radius);
  \draw[-,dashed] (-2.0,+0.0)--(-2.0,\radius);
  \end{tikzpicture}
\end{figure}

\TT{Widmo aplitudowe sygnału rzeczywistego jest zawsze parzyste.}{The magnitude spectrum of a \underline{real signal} is an even-symmetric function of $k$.}

\TT{Widmo fazowe obliczamy ze wzoru:}{The phase spectrum is defined as:}

\begin{align*}
\Phi ( \omega )&=arg\left( \frac{\omega_{0}}{\omega_{0}^{2} + (a + \jmath \cdot \omega)^{2}} \right)=\\
&=arg\left( \frac{\omega_{0}}{\omega_{0}^{2} + a^2 + 2\cdot a \cdot \jmath \cdot \omega + \left(\jmath \cdot \omega\right)^{2}}\right)=\\
&=arg\left( \frac{\omega_{0}}{\omega_{0}^{2} + a^2 + 2\cdot a \cdot \jmath \cdot \omega - \omega^{2}}\right)=\\
&=arg\left( \frac{\omega_{0}}{\omega_{0}^{2}- \omega^{2} + a^2 + \jmath \cdot 2\cdot a \cdot \omega }\right)=\\
&=\begin{Bmatrix}
arg\left(\frac{z_1}{z_2}\right) = arg\left(z_1\right) -arg\left(z_2\right)
\end{Bmatrix}=\\
&=arg\left( \omega_{0}\right) - arg\left(\omega_{0}^{2}- \omega^{2} + a^2 + \jmath \cdot 2\cdot a \cdot \omega \right)=\\
&=\begin{Bmatrix}
arg\left(a + \jmath \cdot b\right) = \TT{arctg}{arctan}(\frac{b}{a})\\
\end{Bmatrix}=\\
&=\TT{arctg}{arctan}\left(\frac{0}{\omega_0}\right) - \TT{arctg}{arctan}\left(\frac{2\cdot a \cdot \omega}{\omega_0^2-\omega^2+a^2}\right)=\\
&=\TT{arctg}{arctan}\left(0\right) - \TT{arctg}{arctan}\left(\frac{2\cdot a \cdot \omega}{\omega_0^2-\omega^2+a^2}\right)=\\
&=0 - \TT{arctg}{arctan}\left(\frac{2\cdot a \cdot \omega}{\omega_0^2-\omega^2+a^2}\right)=\\
&=- \TT{arctg}{arctan}\left(\frac{2\cdot a \cdot \omega}{\omega_0^2-\omega^2+a^2}\right)
\end{align*}

\begin{figure}[H]
  \centering
  \begin{tikzpicture}
  \draw[->] (-5.0,+0.0) -- (+5.0,+0.0) node[right] {$\omega$};
  \draw[->] (+0.0,-1.5) -- (+0.0,+2.0) node[above] {$\Phi(\omega)$};
  
  \draw[scale=1.0,domain=-4.0:4.0,samples=2000,smooth,variable=\x,red,thick] plot ({\x},{-atan(\x*2*0.6/(2^2-(\x)^2+(0.6)^2))/90});
  
  
  \draw[-] (-2.0-0.1,-0.1)--(-2.0+0.1,0.1) node[midway, below, outer sep=5pt] {$-\omega_0$};
  \draw[-] (+2.0-0.1,-0.1)--(+2.0+0.1,0.1) node[midway, below, outer sep=5pt] {$\omega_0$};
  \draw[-] (-0.1,+1.0-0.1)--(+0.1,+1.0+0.1) node[midway, above left] {$\frac{\pi}{2}$};
  \draw[-] (-0.1,-1.0-0.1)--(+0.1,-1.0+0.1) node[midway, above left] {$-\frac{\pi}{2}$};
  
  \end{tikzpicture}
\end{figure}

\TT{Widmo fazowe sygnału rzeczywistego jest zawsze nieparzyste.}{The phase spectrum of a \underline{real signal} is an odd-symmetric function of $k$.}

\end{task}
