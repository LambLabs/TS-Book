\begin{task}
\TT{Oblicz transformatę Fouriera sygnału $f(t)$ przedstawionego na rysunku wykorzystując twierdzenia opisujące własciwości transformacji Fouriera.
Wykorzystaj informację o tym, że $\mathcal F\{\Pi(t)\}=Sa\left(\frac{\omega}{2}\right)$. Podaj co najmniej 2 sposoby opisu sygnału $f(t)$.}{Compute the Fourier transform of the $f(t)$ signal shown below using theorems describing the properties of Fourier transformation. Exploit the following transform $\mathcal F\{\Pi(t)\}=Sa\left(\frac{\omega}{2}\right)$. Give solutions for at least two different descriptions of the $f(t)$ signal.}

\begin{figure}[H]
  \centering
  \begin{tikzpicture}
  \draw[->] (-4.0,+0.0) -- (+4.0,+0.0) node[right] {$t$};
  \draw[->] (+0.0,-1.0) -- (+0.0,+2.5) node[above] {$f(t)$};

  \draw[-,red, thick] (-3.0,+0.0) -- (-2.0,+0.0) -- (-2.0,+2.0) -- (-1.0,+2.0) -- (-1.0,1.0) -- (1.0,1.0) -- (1.0,2.0) -- (2.0,2.0) -- (2.0,0.0) -- (3.0,0.0);
  \draw[-] (-2.0-0.1,-0.1)--(-2.0+0.1,0.1) node[midway, below, outer sep=5pt,align=center] {$-2$};
  \draw[-] (-1.0-0.1,-0.1)--(-1.0+0.1,0.1) node[midway, below, outer sep=5pt,align=center] {$-1$};
  \draw[-] (+1.0-0.1,-0.1)--(+1.0+0.1,0.1) node[midway, below, outer sep=5pt] {$1$};
  \draw[-] (+2.0-0.1,-0.1)--(+2.0+0.1,0.1) node[midway, below, outer sep=5pt] {$2$};
  \draw[-] (-0.1,1.0-0.1)--(+0.1,1.0+0.1) node[midway, above left] {$1$};
  \draw[-] (-0.1,2.0-0.1)--(+0.1,2.0+0.1) node[midway, above left] {$2$};
  \end{tikzpicture}
\end{figure}

\TT{W pierwszej kolejności opiszmy sygnał za pomocą sygnałów elementarnych:}{First of all, describe the $f(t)$ signal using elementary signals:}

\begin{equation}
f(t) = 2 \cdot \Pi\left(\frac{t}{4}\right) - \Pi\left(\frac{t}{2}\right)
\end{equation}

\TT{Ponieważ transformacja Fouriera jest przekształceniem liniowym, dlatego można wyznaczyć osobno transformaty poszczególnych sygnałów elementarnych, czyli:}{Based on linearity of the Fourier transformation, we can calculate transforms for elementary signals separately:}

\begin{equation}
f(t) = 2\cdot f_{1}(t) - f_{2}(t)
\end{equation}
\TT{gdzie:}{where:}
\begin{align*}
f_{1}(t) = \Pi\left(\frac{t}{4}\right)\\
f_{2}(t) = \Pi\left(\frac{t}{2}\right)
\end{align*}

\TT{Wyznaczmy transformtę sygnału $f_{1}(t)$, czyli $F_{1}(\jmath \omega)$.}{Calculate the Fourier transform $F_{1}(\jmath \omega)$ for the first signal $f_{1}(t)$.}

\TT{Z treści zadania wiemy, że:}{We know that:}
$\mathcal F \{\Pi(t)\} = Sa\left(\frac{\omega}{2}\right)$.

\TT{Wykorzystując twierdzenie o zmianie skali mamy:}{Based on the scaling theorem:}

\begin{align*}
\TimeScalingTeorem{g}{G}{f}{F}
\end{align*}

\TT{otrzymujemy:}{we get:}

\begin{align*}
\Pi(t) \xrightarrow{\mathcal F} & Sa\left(\frac{\omega}{2}\right)\\
\Pi(\frac{t}{4}) \xrightarrow{\mathcal F} & \frac{1}{\left|\frac{1}{4}\right|} \cdot Sa\left(\frac{ \frac{\omega}{ \frac{1}{4} }}{2}\right)\\
\Pi(\frac{t}{4}) \xrightarrow{\mathcal F} & 4 \cdot Sa\left(\frac{\omega \cdot 4}{2}\right)\\
\Pi(\frac{t}{4}) \xrightarrow{\mathcal F} & 4 \cdot Sa\left(2 \cdot \omega \right)\\
\end{align*}

\TT{Transformata sygnału $f_{1}(t)$ to:}{The Fourier transform of $f_{1}(t)$ signal is equal to:}

\begin{equation}
F_{1}(\jmath \omega) = \mathcal F\{f_{1}(t)\} = 4 \cdot Sa\left(2 \cdot \omega\right)
\end{equation}

\TT{Wyznaczmy transformtę sygnału $f_{2}(t)$, czyli $F_{2}(\jmath \omega)$.}{Now, we calculate the Fourier transform $F_{2}(\jmath \omega)$ for the second signal $f_{2}(t)$.}

\TT{Z treści zadania wiemy, że:}{We know that:}
$\mathcal F \{\Pi(t)\} = Sa\left(\frac{\omega}{2}\right)$.

\TT{Wykorzystując twierdzenie o zmianie skali:}{Based on the scaling theorem:}

\begin{align*}
\TimeScalingTeorem{g}{G}{f}{F}
\end{align*}

\TT{otrzymujemy:}{we get:}

\begin{align*}
\Pi(t) \xrightarrow{\mathcal F} & Sa\left(\frac{\omega}{2}\right)\\
\Pi(\frac{t}{2}) \xrightarrow{\mathcal F} & \frac{1}{\left|\frac{1}{2}\right|} \cdot Sa\left(\frac{ \frac{\omega}{ \frac{1}{2} }}{2}\right)\\
\Pi(\frac{t}{2}) \xrightarrow{\mathcal F} & 2 \cdot Sa\left(\frac{\omega \cdot 2}{2}\right)\\
\Pi(\frac{t}{2}) \xrightarrow{\mathcal F} & 2 \cdot Sa\left(\omega \right)\\
\end{align*}

\TT{Transformata sygnału $f_{2}(t)$ to:}{The Fourier transform of $f_{2}(t)$ signal is equal to:}
\begin{equation}
F_{2}(\jmath \omega) = \mathcal F\{f_{2}(t)\} = 2 \cdot Sa\left(\omega\right)
\end{equation}

\TT{Czyli transformata sygnału $f(t)$ to:}{Finally, the Fourier transform for $f(t)$ signal is equal to:}

\begin{align*}
F(\jmath \omega) = \mathcal F\{f(t)\} &= 2 \cdot \left( 4 \cdot Sa\left(2 \cdot \omega\right)\right) - 2 \cdot Sa\left(\omega\right) = 8 \cdot Sa \left(2 \cdot \omega\right) - 2 \cdot Sa\left(\omega\right)
\end{align*}

\TT{Transformata sygnału $f(t) = 2 \cdot \Pi\left(\frac{t}{4}\right) - \Pi\left(\frac{t}{2}\right)$ to $F(\jmath \omega)=8 \cdot Sa \left(2 \cdot \omega\right) - 2 \cdot Sa\left(\omega\right)$.}{The Fourier transform of $f(t) = 2 \cdot \Pi\left(\frac{t}{4}\right) - \Pi\left(\frac{t}{2}\right)$ is equal to $F(\jmath \omega)=8 \cdot Sa \left(2 \cdot \omega\right) - 2 \cdot Sa\left(\omega\right)$.}

\TT{Inne możliwości opisu sygnał $f(t)$ za pomocą sygnałów elementarnych:}{Different ways to describe the $f(t)$ signal using elementary signals:}

\begin{align*}
f(t) &= \Pi\left(\frac{t}{4}\right) + \Pi\left(t - \frac{3}{2}\right) + \Pi\left(t + \frac{3}{2}\right)\\
f(t) &= \Pi\left(\frac{t}{2}\right) + 2 \cdot \Pi\left(t - \frac{3}{2}\right) + 2 \cdot \Pi\left(t + \frac{3}{2}\right)
\end{align*}

Rozważmy następujący opis sygnału $f(t)$ za pomocą sygnałów elementarnych

\begin{align*}
f(t) &= \Pi\left(\frac{t}{4}\right) + \Pi\left(t - \frac{3}{2}\right) + \Pi\left(t + \frac{3}{2}\right)
\end{align*}

Ponieważ transformacja Fouriera jest przekształceniem liniowym, dlatego można wyznaczyć osobno transformaty poszczególnych sygnałów elementarnych, czyli:

\begin{align*}
f(t) &= f_1(t) + f_2(t) + f_3(t)
\end{align*}

gdzie:

\begin{align*}
f_1(t) &= \Pi\left(\frac{t}{4}\right)\\
f_2(t) &= \Pi\left(t - \frac{3}{2}\right)\\
f_3(t) &= \Pi\left(t + \frac{3}{2}\right)
\end{align*}

Wyznaczmy transformatę sygnału $f_1(t)$, czyli $F_1\left(\jmath\omega\right)$. 

Z treści zadania wiemy że: $\mathcal F \{\Pi(t)\} = Sa\left(\frac{\omega}{2}\right)$.

Wykorzystując twierdzenie o zmianie skali mamy:

\begin{align*}
\TimeScalingTeorem{g}{G}{f_1}{F_1}
\end{align*}

\TT{otrzymujemy:}{we get:}

\begin{align*}
\Pi(t) \xrightarrow{\mathcal F} & Sa\left(\frac{\omega}{2}\right)\\
\Pi(\frac{t}{4}) \xrightarrow{\mathcal F} & \frac{1}{\left|\frac{1}{4}\right|} \cdot Sa\left(\frac{ \frac{\omega}{ \frac{1}{4} }}{2}\right)\\
\Pi(\frac{t}{4}) \xrightarrow{\mathcal F} & 4 \cdot Sa\left(\frac{\omega \cdot 4}{2}\right)\\
\Pi(\frac{t}{4}) \xrightarrow{\mathcal F} & 4 \cdot Sa\left(2 \cdot \omega \right)\\
\end{align*}

\TT{Transformata sygnału $f_{1}(t)$ to:}{The Fourier transform of $f_{1}(t)$ signal is equal to:}

\begin{equation}
F_{1}(\jmath \omega) = \mathcal F\{f_{1}(t)\} = 4 \cdot Sa\left(2 \cdot \omega\right)
\end{equation}

Wyznaczmy transformatę sygnału $f_2(t)$, czyli $F_2\left(\jmath\omega\right)$.
Z treści zadania wiemy że: $\mathcal F \{\Pi(t)\} = Sa\left(\frac{\omega}{2}\right)$.
Wykorzystując twierdzenie o przesunięciu w dziedzinie czasu:

\begin{align*}
\TimeShiftTeorem{g}{G}{f_2}{F_2}
\end{align*}

otrzymujemy:

\begin{align*}
\Pi(t) \xrightarrow{\mathcal F} & Sa\left(\frac{\omega}{2}\right)\\
\Pi\left(t-\frac{3}{2}\right) \xrightarrow{\mathcal F} & Sa\left(\frac{\omega}{2}\right) \cdot e^{-\jmath \cdot \omega \cdot \frac{3}{2}}
\end{align*}

\TT{Transformata sygnału $f_{2}(t)$ to:}{The Fourier transform of $f_{2}(t)$ signal is equal to:}

\begin{equation}
F_{2}(\jmath \omega) = \mathcal F\{f_{2}(t)\} = Sa\left(\frac{\omega}{2}\right) \cdot e^{-\jmath \cdot \omega \cdot \frac{3}{2}}
\end{equation}

Wyznaczmy transformatę sygnału $f_3(t)$, czyli $F_3\left(\jmath\omega\right)$.
Z treści zadania wiemy że: $\mathcal F \{\Pi(t)\} = Sa\left(\frac{\omega}{2}\right)$.
Wykorzystując twierdzenie o przesunięciu w dziedzinie czasu:

\begin{align*}
\TimeShiftTeorem{g}{G}{f_3}{F_3}
\end{align*}

otrzymujemy:

\begin{align*}
\Pi(t) \xrightarrow{\mathcal F} & Sa\left(\frac{\omega}{2}\right)\\
\Pi\left(t+\frac{3}{2}\right) \xrightarrow{\mathcal F} & Sa\left(\frac{\omega}{2}\right) \cdot e^{-\jmath \cdot \omega \cdot \left(-\frac{3}{2}\right)}\\
\Pi\left(t+\frac{3}{2}\right) \xrightarrow{\mathcal F} & Sa\left(\frac{\omega}{2}\right) \cdot e^{\jmath \cdot \omega \cdot \frac{3}{2}}
\end{align*}

\TT{Transformata sygnału $f_{3}(t)$ to:}{The Fourier transform of $f_{3}(t)$ signal is equal to:}

\begin{equation}
F_{3}(\jmath \omega) = \mathcal F\{f_{3}(t)\} = Sa\left(\frac{\omega}{2}\right) \cdot e^{\jmath \cdot \omega \cdot \frac{3}{2}}
\end{equation}

\TT{Czyli transformata sygnału $f(t)$ to:}{Finally, the Fourier transform for $f(t)$ signal is equal to:}

\begin{align*}
F(\jmath \omega) = \mathcal F\{f(t)\} &= 4 \cdot Sa\left(2 \cdot \omega\right) + Sa\left(\frac{\omega}{2}\right) \cdot e^{-\jmath \cdot \omega \cdot \frac{3}{2}} + Sa\left(\frac{\omega}{2}\right) \cdot e^{\jmath \cdot \omega \cdot \frac{3}{2}} =\\
&= 4 \cdot Sa\left(2 \cdot \omega\right) + Sa\left(\frac{\omega}{2}\right) \cdot \left( e^{-\jmath \cdot \omega \cdot \frac{3}{2}} + e^{\jmath \cdot \omega \cdot \frac{3}{2}}\right) =\\
&= 4 \cdot Sa\left(2 \cdot \omega\right) + Sa\left(\frac{\omega}{2}\right) \cdot 2 \cdot \frac{ e^{-\jmath \cdot \omega \cdot \frac{3}{2}} + e^{\jmath \cdot \omega \cdot \frac{3}{2}}}{2} =\\
&=\begin{Bmatrix}
\EulerCos
\end{Bmatrix}=\\
&= 4 \cdot Sa\left(2 \cdot \omega\right) + Sa\left(\frac{\omega}{2}\right) \cdot 2 \cdot cos\left(\omega \cdot \frac{3}{2}\right)=\\
&= 4 \cdot Sa\left(2 \cdot \omega\right) + 2 \cdot Sa\left(\frac{\omega}{2}\right) \cdot cos\left(\omega \cdot \frac{3}{2}\right)
\end{align*}

\TT{Transformata sygnału $f(t) = \Pi\left(\frac{t}{4}\right) + \Pi\left(t - \frac{3}{2}\right) + \Pi\left(t + \frac{3}{2}\right)$ to $F(\jmath \omega)=4 \cdot Sa\left(2 \cdot \omega\right) + 2 \cdot Sa\left(\frac{\omega}{2}\right) \cdot cos\left(\omega \cdot \frac{3}{2}\right)$.}{The Fourier transform of $f(t) = \Pi\left(\frac{t}{4}\right) + \Pi\left(t - \frac{3}{2}\right) + \Pi\left(t + \frac{3}{2}\right)$ is equal to $F(\jmath \omega)=4 \cdot Sa\left(2 \cdot \omega\right) + 2 \cdot Sa\left(\frac{\omega}{2}\right) \cdot cos\left(\omega \cdot \frac{3}{2}\right)$.}

\end{task}
