\begin{task}
Oblicz transformatę Fouriera sygnału $f(t)=\frac{1}{1+t^2}$ za pomocą twierdzeń.

\begin{figure}[H]
\centering
\begin{tikzpicture}
  %\draw (0,0) circle (1in);
  \draw[->] (-4.0,+0.0) -- (+4.0,+0.0) node[right] {$t$};
  \draw[->] (+0.0,-2.5) -- (+0.0,+2.5) node[above] {$f(t)$};
  
  \draw[scale=1.0,domain=-3.5:3.5,samples=1000,smooth,variable=\x,red,thick] plot ({\x},{2/(1+\x*\x)});%
  
  \draw[-,red, dashed] (-1.0,+1.0) -- (+1.0,+1.0);
  \draw[-,red, dashed] (-1.0,+0.0) -- (-1.0,+1.0);
  \draw[-,red, dashed] (+1.0,+0.0) -- (+1.0,+1.0);
  
  \draw[-] (-0.1,+2.0-0.1)--(+0.1,+2.0+0.1) node[midway, above left] {$1$};
  \draw[-] (-0.1,+1.0-0.1)--(+0.1,+1.0+0.1) node[midway, above left] {$0.5$};
  \draw[-] (-1.0-0.1,-0.1)--(-1.0+0.1,0.1) node[midway, below, outer sep=5pt,align=center] {$-1$};
  \draw[-] (+1.0-0.1,-0.1)--(+1.0+0.1,0.1) node[midway, below, outer sep=5pt,align=center] {$1$};
\end{tikzpicture}
\end{figure}

Załóżmy sygnał $g(t)=e^{-\left|t\right|}$ i wyznaczmy jego transformatę.

\begin{align*}
G(\jmath \omega)&=\int_{-\infty}^{\infty} g(t) \cdot e^{-\jmath \cdot \omega \cdot t} \cdot dt\\
&=\int_{-\infty}^{\infty} e^{-\left|t\right|} \cdot e^{-\jmath \cdot \omega \cdot t}\cdot dt\\
&=\int_{-\infty}^{0} e^{t} \cdot e^{-\jmath \cdot \omega \cdot t}\cdot dt + \int_{0}^{\infty} e^{-t} \cdot e^{-\jmath \cdot \omega \cdot t}\cdot dt\\
&=\int_{-\infty}^{0} e^{t -\jmath \cdot \omega \cdot t}\cdot dt + \int_{0}^{\infty} e^{-t -\jmath \cdot \omega \cdot t}\cdot dt\\
&=\lim_{\tau \rightarrow \infty} \int_{-\tau}^{0} e^{\left(1-\jmath \cdot \omega \right)\cdot t}\cdot dt + \lim_{\tau \rightarrow \infty} \int_{0}^{\tau} e^{-\left(1 +\jmath \cdot \omega \right)\cdot t}\cdot dt\\
&=\begin{Bmatrix}
z_1=-\left(1+\jmath \cdot \omega \right)\cdot t & z_2 = \left(1-\jmath \cdot \omega \right)\cdot t\\
dz_1=-\left(1+\jmath \cdot \omega \right)\cdot dt & dz_2 = \left(1-\jmath \cdot \omega \right)\cdot dt\\
dt=\frac{1}{-\left(1+\jmath \cdot \omega \right)}\cdot dz_1 & dt = \frac{1}{1-\jmath \cdot \omega}\cdot dz_2\\
\end{Bmatrix}\\
&=\lim_{\tau \rightarrow \infty} \int_{-\tau}^{0} e^{z_2}\cdot \frac{1}{1-\jmath \cdot \omega }\cdot dz_2 + \lim_{\tau \rightarrow \infty} \int_{0}^{\tau} e^{z_1}\cdot \frac{1}{-\left(1+\jmath \cdot \omega\right)}\cdot dz_1\\
&=\frac{1}{1-\jmath \cdot \omega }\cdot \lim_{\tau \rightarrow \infty} \int_{-\tau}^{0} e^{z_2}\cdot dz_2 + \frac{1}{-\left(1+\jmath \cdot \omega\right)}\cdot \lim_{\tau \rightarrow \infty} \int_{0}^{\tau} e^{z_1}\cdot  dz_1\\
&=\frac{1}{1-\jmath \cdot \omega}\cdot \lim_{\tau \rightarrow \infty} \left. e^{z_2}\right|_{-\tau}^{0} + \frac{1}{-\left(1+\jmath \cdot \omega\right)}\cdot \lim_{\tau \rightarrow \infty} \left. e^{z_1}\right|_{0}^{\tau}\\
&=\frac{1}{1-\jmath \cdot \omega }\cdot \lim_{\tau \rightarrow \infty} \left. e^{\left(1-\jmath \cdot \omega \right)\cdot t}\right|_{-\tau}^{0} + \frac{1}{-\left(1+\jmath \cdot \omega\right)}\cdot \lim_{\tau \rightarrow \infty} \left. e^{-\left(1+\jmath \cdot \omega \right)\cdot t}\right|_{0}^{\tau}\\
&=\frac{1}{1-\jmath \cdot \omega }\cdot \lim_{\tau \rightarrow \infty} \left( e^{\left(1-\jmath \cdot \omega \right)\cdot 0} - e^{\left(1-\jmath \cdot \omega \right)\cdot (-\tau)}\right) + \frac{1}{-\left(1+\jmath \cdot \omega\right)}\cdot \lim_{\tau \rightarrow \infty} \left( e^{-\left(1+\jmath \cdot \omega \right)\cdot \tau} - e^{-\left(1+\jmath \cdot \omega \right)\cdot 0}\right)\\
&=\frac{1}{1-\jmath \cdot \omega }\cdot \lim_{\tau \rightarrow \infty} \left( e^{0} - e^{-\left(1-\jmath \cdot \omega \right)\cdot \tau}\right) + \frac{1}{-\left(1+\jmath \cdot \omega\right)}\cdot \lim_{\tau \rightarrow \infty} \left( e^{-\left(1+\jmath \cdot \omega \right)\cdot \tau} - e^{0}\right)\\
&=\frac{1}{1-\jmath \cdot \omega }\cdot \lim_{\tau \rightarrow \infty} \left( 1 - e^{-\tau +\jmath \cdot \omega \cdot \tau}\right) + \frac{1}{-\left(1+\jmath \cdot \omega\right)}\cdot \lim_{\tau \rightarrow \infty} \left( e^{-\tau -\jmath \cdot \omega \cdot \tau} - 1\right)\\
&=\frac{1}{1-\jmath \cdot \omega }\cdot \left( \lim_{\tau \rightarrow \infty} 1 - \lim_{\tau \rightarrow \infty} e^{-\tau} \cdot e^{\jmath \cdot \omega \cdot \tau}\right) + \frac{1}{-\left(1+\jmath \cdot \omega\right)}\cdot \left(\lim_{\tau \rightarrow \infty} e^{-\tau}\cdot e^{-\jmath \cdot \omega \cdot \tau} - \lim_{\tau \rightarrow \infty} 1\right)\\
&=\frac{1}{1-\jmath \cdot \omega }\cdot \left( 1 - \lim_{\tau \rightarrow \infty} e^{-\tau} \cdot \lim_{\tau \rightarrow \infty} e^{\jmath \cdot \omega \cdot \tau}\right) + \frac{1}{-\left(1+\jmath \cdot \omega\right)}\cdot \left(\lim_{\tau \rightarrow \infty} e^{-\tau}\cdot \lim_{\tau \rightarrow \infty} e^{-\jmath \cdot \omega \cdot \tau} - 1\right)\\
&=\frac{1}{1-\jmath \cdot \omega }\cdot \left( 1 - 0 \cdot \lim_{\tau \rightarrow \infty} e^{\jmath \cdot \omega \cdot \tau}\right) + \frac{1}{-\left(1+\jmath \cdot \omega\right)}\cdot \left(0\cdot \lim_{\tau \rightarrow \infty} e^{-\jmath \cdot \omega \cdot \tau} - 1\right)\\
&=\frac{1}{1-\jmath \cdot \omega }\cdot \left( 1 - 0 \right) + \frac{1}{-\left(1+\jmath \cdot \omega\right)}\cdot \left(0 - 1\right)\\
&=\frac{1}{1-\jmath \cdot \omega } + \frac{1}{-\left(1+\jmath \cdot \omega\right)} \cdot (-1)\\
&=\frac{1}{1-\jmath \cdot \omega } + \frac{1}{1+\jmath \cdot \omega}\\
&=\frac{\left(1+\jmath \cdot \omega\right)}{\left(1+\jmath \cdot \omega\right) \cdot \left(1-\jmath \cdot \omega\right) } + \frac{\left(1-\jmath \cdot \omega\right)}{\left(1+\jmath \cdot \omega\right)\cdot \left(1-\jmath \cdot \omega\right)}\\
&=\frac{\left(1+\jmath \cdot \omega\right) + \left(1-\jmath \cdot \omega\right)}{\left(1+\jmath \cdot \omega\right) \cdot \left(1-\jmath \cdot \omega\right) }\\
&=\frac{2}{1+\omega^2}\\
\end{align*}

Transformata sygnału $g(t) = e^{-\left|t\right|}$ jest równa $G(\jmath \omega)=\frac{2}{1+\omega^2}$. Postać funkcji $G(\jmath \omega)=\frac{2}{1+\omega^2}$ nie jest identyczna z postacią funkcji $f(t)$, funkcja różni się o współczynnik $2$. 

Z twierdzenia o liniowości transformaty 

\begin{align*}
g(t) &\xrightarrow{\mathcal F} G(\jmath \omega)\\
h(t)=\alpha \cdot g(t) &\xrightarrow{\mathcal F} H(\jmath \omega) = \alpha \cdot G(\jmath \omega)\
\end{align*}

otrzymujemy

\begin{align*}
h(t) &= \frac{1}{2} \cdot e^{-\left|t\right|}\\
H(\jmath \omega)&=\frac{1}{2} \cdot \frac{2}{1+\omega^2}\\
&=\frac{1}{1+\omega^2}
\end{align*}

Na podstawie sygnału $h(t)$ i korzystając z twierdzenia o symetrii możemy wyznaczyć transformatę sygnału $f(t)$.

\begin{align*}
\SymetryTeorem{h}{H}{f}{F}
\end{align*}

\begin{align*}
F(\jmath \omega)&=2\pi \cdot h(-\omega)\\
&=2\pi \cdot \frac{1}{2} \cdot e^{-\left|-\omega\right|}\\
&=\pi \cdot e^{-\left|\omega\right|}\\
\end{align*}

Transformata Fouriera sygnału $f(t)=\frac{1}{1+t^2}$ jest równa $F(\jmath \omega)=\pi \cdot e^{-\left|\omega\right|}$

\end{task}

