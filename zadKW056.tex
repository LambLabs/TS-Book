\begin{task}

\TT{Wyznacz okres syngału $f(t)=sin\left(\frac{2}{5}\cdot t\right) - sin\left(\frac{3}{4}\cdot t\right) - cos\left(\frac{t}{4}\right)$}{}

\TT{Rozważana funkcja składa się z (jest liniową kombinacją) trzech następujących funkcji trygonometrycznych:}{}
\begin{enumerate}
  \item $g_1(t)=sin\left(\frac{2}{5}\cdot t\right)$
  \item $g_2(t)=sin\left(\frac{3}{4}\cdot t\right)$
  \item $g_3(t)=cos\left(\frac{t}{4}\right)$
\end{enumerate}
tak że $f(t) = g_1(t) - g_2(t) - g_3(t)$.

Wyznaczmy dla każdej z funkcji składowych okres tej funkcji. Zacznijmy od $g_1(t)$ porównajmy wzór funkcji z domyślną postacią funkcji trygonometrycznej, możemy wyznaczyć okres funkcji $g_1(t)$. 
\begin{align*}
\left.\begin{matrix}
  sin\left(\frac{2}{5}\cdot t\right)
  \\ 
  sin\left(\frac{2\cdot \pi}{T_1} \cdot t\right)
\end{matrix}\right\}\Rightarrow \frac{2}{5}=\frac{2\cdot \pi}{T_1}\Rightarrow T_1=5\cdot \pi
\end{align*}
Nastepnie porównajmy wzór funkcji $g_2(t)$ z domyślną postacią funkcji trygonometrycznej, możemy wyznaczyć okres funkcji $g_2(t)$. 
\begin{align*}
\left.\begin{matrix}
  sin\left(\frac{3}{4}\cdot t\right)
  \\ 
  sin\left(\frac{2\cdot \pi}{T_1} \cdot t\right)
\end{matrix}\right\}\Rightarrow \frac{3}{4}=\frac{2\cdot \pi}{T_1}\Rightarrow T_1=\frac{2}{3}\cdot \pi
\end{align*}
I na koniec porównajmy wzór funkcji $g_3(t)$ z domyślną postacią funkcji trygonometrycznej, możemy wyznaczyć okres funkcji $g_3(t)$. 
\begin{align*}
 \left.\begin{matrix}
  cos\left(\frac{1}{4}\cdot t\right)
  \\ 
  cos\left(\frac{2\cdot \pi}{T_1} \cdot t\right)
\end{matrix}\right\}\Rightarrow \frac{1}{4}=\frac{2\cdot \pi}{T_1}\Rightarrow T_1=2\cdot \pi
\end{align*}

Okresem rozważanej funkcji $f(t)$ jest najmniejsza wspólna wielokrotnośc okresów funkcji składowych. 
\begin{align*}
T=NWW\left(T_1,T_2,T_3\right)
\end{align*}

Rozważmy kolejne wielokrotności poszczególnych okresów

\begin{tabular}{|c|c|c|c|c|c|c|c|c|c|c|c|c|c|c|c|}
\hline
$T_1$ & $5\pi$ &\framebox{$10\pi$} & $15\pi$ & $20\pi$ & $25\pi$ & & & & &\\ \hline
$T_2$ & $\frac{2}{3}\pi$ & $\frac{4}{3}\pi$ & $\frac{6}{3}\pi= 2\pi$ & $\frac{8}{3}\pi$ & $\frac{10}{3}\pi$ & $\cdots$ & $\frac{24}{3}\pi = 8\pi$ & $\frac{26}{3}\pi$ & $\frac{28}{3}\pi$ &\framebox{$\frac{30}{3}\pi=10\pi$}  \\ \hline
$T_3$ & $2\pi$ & $4\pi$ & $6\pi$ & $8\pi$ & \framebox{$10\pi$} & & & & &\\
\hline
\end{tabular}
\\
\\
Najmniejszą wspólną wielokrotnością okresów $T_1$, $T_2$ i $T_3$ jest $10\pi$, a więc okrese rozważanej funkcji to $T=10\pi$.

\end{task}