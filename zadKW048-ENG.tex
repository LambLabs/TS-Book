\begin{task}
\TT{Wyznacz współczynniki zespolonego szeregu Fouriera dla okresowego sygnału $g(t)$ przedstawionego na rysunku. Wykorzystaj własności szeregu Fouriera oraz współczynniki zespolonego szeregu Fouriera wyznaczone w zadaniu \ref{TaskKW015}}{Calculate coefficients of the periodic signal $f(t)$ shown below for the expansion into a complex exponential Fourier series. Draw magnitude and phase spectra.}

\begin{figure}[H]
  \centering
  \begin{tikzpicture}
  %\draw (0,0) circle (1in);
  \draw[->] (-3.0,+0.0) -- (+5.0,+0.0) node[right] {$t$};
  \draw[->] (+0.0,-1.5) -- (+0.0,+2.0) node[above] {$g(t)$};
  \draw[scale=1.0,domain=-1.3:-0.8,samples=100,smooth,variable=\x,red,thick] plot ({\x},{0.0+1*sin((\x-0.2)*180.0/3.141592*1*3.141592/1.0)});
  \draw[-,red, thick] (-0.8,0.0) -- (0.2,0.0);
  \draw[scale=1.0,domain=0.2:1.2,samples=100,smooth,variable=\x,red,thick] plot ({\x},{0.0+1*sin((\x-0.2)*180.0/3.141592*1*3.141592/1.0)});
  \draw[-,red, thick] (1.2,0.0) -- (2.2,0.0);
  \draw[scale=1.0,domain=2.2:3.2,samples=100,smooth,variable=\x,red,thick] plot ({\x},{0.0+1*sin((\x-0.2)*180.0/3.141592*1*3.141592/1.0)});
  \draw[-,red, thick] (3.2,0.0) -- (3.7,0.0);
  \draw[-,red, dashed] (-2.0,1.0) -- (4.0,1.0);
  %\draw[-] (-1.0-0.1,-0.1)--(-1.0+0.1,0.1) node[midway, below, outer sep=10pt,align=center] {$-\frac{T}{2}$};
  \draw[-] (-0.8-0.1,-0.1)--(-0.8+0.1,0.1) node[midway, below, outer sep=5pt] {$D-\frac{T}{2}$};
  \draw[-] (+0.2-0.1,-0.1)--(+0.2+0.1,0.1) node[midway, below, outer sep=5pt] {$D$};
  \draw[-] (+1.2-0.1,-0.1)--(+1.2+0.1,0.1) node[midway, below, outer sep=5pt] {$D+\frac{T}{2}$};
  \draw[-] (+2.0-0.1,-0.1)--(+2.0+0.1,0.1) node[midway, below, outer sep=5pt] {$T$};
  \draw[-] (-0.1,+1.0-0.1)--(+0.1,+1.0+0.1) node[midway, left] {$A$};
  
  \end{tikzpicture}
\end{figure}

% To be done

\TT{W pierwszej kolejności należy opisać sygnał za pomocą wzoru.}{Periodic signal $f(t)$, as a piecewise linear function, is given by:}

\begin{equation}
   g(x)=\begin{cases}
   A\cdot sin\left(\frac{2\pi}{T}\cdot \left(t-D\right)\right) & t \in \left (  D+k \cdot T; D+\frac{T}{2}+k \cdot T \right ) \\
   0 & t \in \left ( D+\frac{T}{2}+k \cdot T; D+T +k \cdot T\right )
   \end{cases} \wedge k \in \TT{C}{Z}
\end{equation}

\TT{Można zauważyć iż sygnał $g(t)$ jest z przesuniętą w czasie wersją sygnału $f(t)$ z zadania \ref{TaskKW015}}{}
\begin{align*}
g(t) &= f\left(t - D\right)
\end{align*}

\TT{Współczynniki zespolonego szeregu Fouriera $F_k$ dla sygnału $f(t)$ wyznaczone w zadaniu \ref{TaskKW015} wynoszą:}{}
\begin{align*}
F_0&=\frac{A}{\pi}\\
F_{-1}&=\jmath \cdot \frac{A}{4}\\
F_{1}&=-\jmath \cdot \frac{A}{4}\\
F_k&=\frac{A}{2 \cdot \pi} \cdot \left(\frac{(-1)^{k}+1}{1-k^2}\right)\\
\end{align*}

\TT{Korzystając z twierdzenia o przesunięciu w dziedzinie czasu można wyznaczyć współczynniki $G_k$ na podstawie współczynników $F_k$ sygnału $f(t)$ jako:}{}
\begin{align*}
g(t) &= f(t-t_0)\\
G_k &= F_{k} \cdot e^{-\jmath \cdot \frac{2\pi}{T} \cdot k \cdot t_0}
\end{align*}
  
\TT{Wstawiając wartości współczynników $F_k$ otrzymujemy}{}
\begin{align*}
G_k &= F_{k} \cdot e^{-\jmath \cdot \frac{2\pi}{T} \cdot k \cdot t_0} = \\
& = \frac{A}{2 \cdot \pi} \cdot \left(\frac{(-1)^{k}+1}{1-k^2}\right) \cdot e^{-\jmath \cdot \frac{2\pi}{T} \cdot k \cdot t_0} = \\
&=\left\{\begin{array}{ll}
t_0 = D
\end{array}\right\}=\\
& = \frac{A}{2 \cdot \pi} \cdot \left(\frac{(-1)^{k}+1}{1-k^2}\right) \cdot e^{-\jmath \cdot \frac{2\pi}{T} \cdot k \cdot D} \\
\end{align*}

\TT{A wiec współczynniki $G_k$ dla sygnału $g(t)$ są równe $\frac{A}{2 \cdot \pi} \cdot \left(\frac{(-1)^{k}+1}{1-k^2}\right) \cdot e^{-\jmath \cdot \frac{2\pi}{T} \cdot k \cdot D}$, dla $k \neq 1 \wedge k\neq -1$. Oznacza to iż współczynnik dla $k=1$ i $k=-1$ musimy wyznaczyć jeszcze raz analizując dokładnie co podstawiamy. Zacznijmy od wyznaczenia $G_1$ }{}

\begin{align*}
G_1 &= F_{1} \cdot e^{-\jmath \cdot \frac{2\pi}{T} \cdot t_0} = \\
& = -\jmath \cdot \frac{A}{4} \cdot e^{-\jmath \cdot \frac{2\pi}{T} \cdot t_0} = \\
&=\left\{\begin{array}{ll}
t_0 = D
\end{array}\right\}=\\
& = -\jmath \cdot \frac{A}{4} \cdot e^{-\jmath \cdot \frac{2\pi}{T} \cdot D}  \\
\end{align*}

\TT{Podobnie wyznaczmy $G_{-1}$:}{}

\begin{align*}
G_{-1} &= F_{-1} \cdot e^{-\jmath \cdot \frac{2\pi}{T} \cdot (-1) \cdot t_0} = \\
& = \jmath \cdot \frac{A}{4} \cdot e^{\jmath \cdot \frac{2\pi}{T} \cdot t_0} = \\
&=\left\{\begin{array}{ll}
t_0 = D
\end{array}\right\}=\\
& = \jmath \cdot \frac{A}{4} \cdot e^{\jmath \cdot \frac{2\pi}{T} \cdot D}
\end{align*}

\TT{Ostatecznie współczynniki zespolonego szeregu Fouriera dla funkcji przedstawionej na rysunku przyjmują wartości.}{To sum up, coefficients for the expansion into a complex exponential Fourier series are given by:}
\begin{align*}
G_{-1}&=\jmath \cdot \frac{A}{4} \cdot e^{\jmath \cdot \frac{2\pi}{T} \cdot D}\\
G_{1}&=-\jmath \cdot \frac{A}{4} \cdot e^{-\jmath \cdot \frac{2\pi}{T} \cdot D}\\
G_k&=\frac{A}{2 \cdot \pi} \cdot \left(\frac{(-1)^{k}+1}{1-k^2}\right) \cdot e^{-\jmath \cdot \frac{2\pi}{T} \cdot k \cdot D}
\end{align*}

\end{task}