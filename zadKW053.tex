\begin{task}
Oblicz transformatę Fouriera sygnału $f(t)$ przedstawionego na rysunku za pomocą twierdzeń, wiedząc że transformata sygnału prostokątnego $g(t)=\Pi\left(t\right)$ jest równa $G(\jmath\omega)=Sa\left(\frac{\omega}{2}\right)$.

\begin{figure}[H]
\centering
\begin{tikzpicture}
  %\draw (0,0) circle (1in);
  \draw[->] (-5.0,+0.0) -- (+5.0,+0.0) node[right] {$t$};
  \draw[->] (+0.0,-1.0) -- (+0.0,+3.0) node[above] {$f(t)$};
 
  \draw[scale=1.0,domain=-3.1415:3.1415,smooth,variable=\x,red,thick] plot ({\x},{2.0*(cos(1.0*\x*180/3.14))^2});
  
  \draw[-,red,dashed] (-3.1415,0.0) -- (-3.1415,2.0);
  \draw[-,red,dashed] (+3.1415,0.0) -- (+3.1415,2.0);
  \draw[-,red,thick] (-4.0,0) --(-3.1415,0);
  \draw[-,red,thick] (4.0,0) --(3.1415,0);
  
  
  \draw[-] (-0.1,+2.0-0.1)--(+0.1,+2.0+0.1) node[midway, left] {$A$};
  \draw[-] (+3.1415/2-0.1,-0.1)--(+3.1415/2+0.1,0.1) node[midway, below, outer sep=5pt,align=center] {$\frac{\pi}{2\omega_0}$};
  \draw[-] (-3.1415/2-0.1,-0.1)--(-3.1415/2+0.1,0.1) node[midway, below, outer sep=5pt,align=center] {$-\frac{\pi}{2\omega_0}$};
  
  \draw[-] (+3.1415/1-0.1,-0.1)--(+3.1415/1+0.1,0.1) node[midway, below, outer sep=5pt,align=center] {$\frac{\pi}{\omega_0}$};
  \draw[-] (-3.1415/1-0.1,-0.1)--(-3.1415/1+0.1,0.1) node[midway, below, outer sep=5pt,align=center] {$-\frac{\pi}{\omega_0}$};
  
\end{tikzpicture}
\end{figure}

Zacznijmy od napisania wzoru sygnału przedstawionego na rysunku

\begin{align*}
f(t)=\begin{cases}
0 & t \in \left( -\infty; -\frac{\pi}{\omega_0} \right ) \\
A\cdot cos^2\left(\omega_0 \cdot t\right) & t \in \left( -\frac{\pi}{\omega_0}; \frac{\pi}{\omega_0} \right ) \\
0 & t \in \left( \frac{\pi}{\omega_0}; \infty \right )
\end{cases} 
\end{align*}

Co możemy zapisać za pomocą sygnałów elementarnych jako

\begin{align*}
f(t) = A\cdot \Pi\left(\frac{t \cdot \omega_0}{2\pi}\right) \cdot cos^2\left(\omega_0 \cdot t\right)
\end{align*}

Nasz sygnał jest iloczynem pewnej funkcji $h(t)$ oraz cosinusów

\begin{align*}
f(t) &= A\cdot \Pi\left(\frac{t \cdot \omega_0}{2\pi}\right) \cdot cos^2\left(\omega_0 \cdot t\right) = \\
&= A\cdot h(t) \cdot cos^2\left(\omega_0 \cdot t\right) = \\
\end{align*}

gdzie

\begin{align*}
h(t)&= \Pi\left(\frac{t \cdot \omega_0}{2\pi}\right)
\end{align*}

Wyznaczmy transformatę sygnału $h(t)$. Z treści zadania wiemy że transformata sygnału $g(t)=\Pi\left(t\right)$ jest równa $G(\jmath\omega)=Sa\left(\frac{\omega}{2}\right)$. 

Postać funkcji $g(t)$ nie jest identyczna z postacią funkcji $h(t)$, funkcja różni się skalą. 
%Dokończyć
Wyznaczanym transformaty funkcji przeskalowanej $h(t)=\Pi\left(\frac{t \cdot \omega_0}{2\pi}\right)$

Z twierdzenia o zmianie skali mamy 

\begin{align*}
\TimeScalingTeorem{g}{G}{h}{H}
\end{align*}

a wiec otrzymujemy

\begin{align*}
h(t) &= \Pi\left(\frac{t}{T}\right)=\\
&=\Pi\left(\frac{t \cdot \omega_0}{2\pi}\right)=\\
&=\Pi\left(t \cdot \frac{\omega_0}{2\pi}\right)=\\
&=g\left(t \cdot \frac{\omega_0}{2\pi}\right)
\end{align*}

gdzie

\begin{align*}
\alpha=\frac{\omega_0}{2\pi}
\end{align*}

a więc

\begin{align*}
H(\jmath \omega)&=\frac{1}{\frac{\omega_0}{2\pi}} \cdot G\left(\frac{\jmath \omega}{\frac{\omega_0}{2\pi}}\right)=\\
&=\frac{2\pi}{\omega_0} \cdot G\left(\frac{\jmath \omega \cdot 2 \pi}{\omega_0}\right)=\\
&=\frac{2\pi}{\omega_0} \cdot Sa\left(\frac{\frac{\omega \cdot 2 \pi}{\omega_0}}{2}\right)=\\
&=\frac{2\pi}{\omega_0} \cdot Sa\left(\omega \cdot \frac{\pi}{\omega_0}\right)
\end{align*}

A więc transformata sygnału $h(t)$ jest równa $H(\jmath \omega)=\frac{2\pi}{\omega_0} \cdot Sa\left(\omega \cdot \frac{\pi}{\omega_0}\right)$

Wróćmy do wzoru sygnału i przedstawmy go następująco

\begin{align*}
f(t) &= A\cdot \Pi\left(\frac{t \cdot \omega_0}{2\pi}\right) \cdot cos^2\left(\omega_0 \cdot t\right) = \\
&= A\cdot h(t) \cdot cos^2\left(\omega_0 \cdot t\right) = \\
&= A\cdot h(t) \cdot cos\left(\omega_0 \cdot t\right) \cdot cos\left(\omega_0 \cdot t\right) = \\
&= A\cdot k(t) \cdot cos\left(\omega_0 \cdot t\right) = \\
\end{align*}

gdzie

\begin{align*}
k(t) &= h(t) \cdot cos\left(\omega_0 \cdot t\right) = \\
&=\begin{Bmatrix}
\EulerCos
\end{Bmatrix}=\\
& = h(t) \cdot \frac{e^{\jmath \cdot \omega_0 \cdot t} + e^{-\jmath \cdot \omega_0 \cdot t}}{2} = \\
& = \frac{1}{2} \left( h(t) \cdot e^{\jmath \cdot \omega_0 \cdot t} + h(t) \cdot e^{-\jmath \cdot \omega_0 \cdot t} \right)
\end{align*}

Wyznaczmy transformatę sygnału $k(t)$. Korzystając z twierdzenia o modulacji mamy

\begin{align*}
\FrequencyShiftTeorem{h}{H}{k}{K}
\end{align*}

a wiec transformata sygnału $k(t)$ wynosi:

\begin{align*}
K(\jmath \omega)&=\frac{1}{2} \left( H(\jmath \left(\omega -\omega_0)\right) + H(\jmath \left(\omega + \omega_0)\right) \right) = \\
&=\frac{1}{2} \left( \frac{2\pi}{\omega_0} \cdot Sa\left(\left(\omega - \omega_0 \right)\cdot \frac{\pi}{\omega_0}\right) +\frac{2\pi}{\omega_0} \cdot Sa\left(\left(\omega+\omega_0\right) \cdot \frac{\pi}{\omega_0}\right) \right) = \\
&=\frac{\pi}{\omega_0} \cdot Sa\left(\left(\omega - \omega_0 \right)\cdot \frac{\pi}{\omega_0}\right) +\frac{\pi}{\omega_0} \cdot Sa\left(\left(\omega+\omega_0\right) \cdot \frac{\pi}{\omega_0}\right)
\end{align*}

Wróćmy do wzoru sygnału $f(t)$

\begin{align*}
f(t) &= A\cdot \Pi\left(\frac{t \cdot \omega_0}{2\pi}\right) \cdot cos^2\left(\omega_0 \cdot t\right) = \\
&= A\cdot h(t) \cdot cos^2\left(\omega_0 \cdot t\right) = \\
&= A\cdot h(t) \cdot cos\left(\omega_0 \cdot t\right) \cdot cos\left(\omega_0 \cdot t\right) = \\
&= A\cdot k(t) \cdot cos\left(\omega_0 \cdot t\right) = \\
&=\begin{Bmatrix}
\EulerCos
\end{Bmatrix}=\\
& = A\cdot k(t) \cdot \frac{e^{\jmath \cdot \omega_0 \cdot t} + e^{-\jmath \cdot \omega_0 \cdot t}}{2} = \\
& = \frac{A}{2} \left( k(t) \cdot e^{\jmath \cdot \omega_0 \cdot t} + k(t) \cdot e^{-\jmath \cdot \omega_0 \cdot t} \right)
\end{align*}

Znając transformatę sygnału $k(t)$ i korzystająć z twierdzenia o modulacji możemy wyznaczyć transformatę sygnału $f(t)$.

\begin{align*}
\FrequencyShiftTeorem{k}{K}{f}{F}
\end{align*}

a więc transformata sygnału $f(t)$ wynosi

\begin{align*}
F(\jmath \omega)&=\frac{A}{2} \left( K(\jmath \left(\omega -\omega_0)\right) + K(\jmath \left(\omega + \omega_0)\right) \right) = \\
&=\frac{A}{2} \left( \frac{\pi}{\omega_0} \cdot Sa\left(\left(\omega -\omega_0 - \omega_0 \right)\cdot \frac{\pi}{\omega_0}\right) +\frac{\pi}{\omega_0} \cdot Sa\left(\left(\omega-\omega_0+\omega_0\right) \cdot \frac{\pi}{\omega_0}\right)\right. + \\
&+ \left.\frac{\pi}{\omega_0} \cdot Sa\left(\left(\omega+\omega_0 - \omega_0 \right)\cdot \frac{\pi}{\omega_0}\right) +\frac{\pi}{\omega_0} \cdot Sa\left(\left(\omega+\omega_0+\omega_0\right) \cdot \frac{\pi}{\omega_0}\right) \right) = \\
&=\frac{A}{2} \left( \frac{\pi}{\omega_0} \cdot Sa\left(\left(\omega -2 \cdot\omega_0 \right)\cdot \frac{\pi}{\omega_0}\right) +\frac{\pi}{\omega_0} \cdot Sa\left(\omega \cdot \frac{\pi}{\omega_0}\right)\right. + \\
&+ \left.\frac{\pi}{\omega_0} \cdot Sa\left(\omega \cdot \frac{\pi}{\omega_0}\right) +\frac{\pi}{\omega_0} \cdot Sa\left(\left(\omega+2\cdot\omega_0\right) \cdot \frac{\pi}{\omega_0}\right) \right) = \\
&=\frac{A}{2} \left( \frac{\pi}{\omega_0} \cdot Sa\left(\left(\omega -2 \cdot\omega_0 \right)\cdot \frac{\pi}{\omega_0}\right) +2 \cdot \frac{\pi}{\omega_0} \cdot Sa\left(\omega \cdot \frac{\pi}{\omega_0}\right)\right. + \\
&+ \left. \frac{\pi}{\omega_0} \cdot Sa\left(\left(\omega+2\cdot\omega_0\right) \cdot \frac{\pi}{\omega_0}\right) \right) = \\
&=\frac{A \cdot \pi}{2 \cdot \omega_0} \left( Sa\left(\left(\omega -2 \cdot\omega_0 \right)\cdot \frac{\pi}{\omega_0}\right) +2 \cdot Sa\left(\omega \cdot \frac{\pi}{\omega_0}\right) + Sa\left(\left(\omega+2\cdot\omega_0\right) \cdot \frac{\pi}{\omega_0}\right) \right)
\end{align*}

Ostatecznie transformata sygnału $f(t)$ wynosi $F(\jmath \omega)=\frac{A \cdot \pi}{2 \cdot \omega_0} \left( Sa\left(\left(\omega -2 \cdot\omega_0 \right)\cdot \frac{\pi}{\omega_0}\right) +2 \cdot Sa\left(\omega \cdot \frac{\pi}{\omega_0}\right) + Sa\left(\left(\omega+2\cdot\omega_0\right) \cdot \frac{\pi}{\omega_0}\right) \right)$.

\end{task}

