\begin{task}
Oblicz transformatę Fouriera sygnału $f(t)$ przedstawionego na rysunku za pomocą twierdzeń.

\begin{figure}[H]
\centering
\begin{tikzpicture}
  %\draw (0,0) circle (1in);
  \draw[->] (-4.0,+0.0) -- (+4.0,+0.0) node[right] {$t$};
  \draw[->] (+0.0,-1.5) -- (+0.0,+1.5) node[above] {$f(t)$};
  \draw[-,red, thick] (-3.5,+0.0) -- (-1.0,+0.0);
  \draw[scale=1.0,domain=-1.0:1.0,samples=200,smooth,variable=\x,red,thick] plot ({\x},{1-(\x*\x)});
  \draw[-,red, thick] (+1.0,+0.0) -- (+3.0,0.0);
  
  \draw[-] (-1.0-0.1,-0.1)--(-1.0+0.1,0.1) node[midway, below, outer sep=5pt,align=center] {$-1$};
  \draw[-] (+1.0-0.1,-0.1)--(+1.0+0.1,0.1) node[midway, below, outer sep=5pt] {$1$};
  \draw[-] (-0.1,+1.0-0.1)--(+0.1,+1.0+0.1) node[midway, above right] {$1$};
\end{tikzpicture}
\end{figure}

Sygnal $f(t)$ możemy opisać jako:

\begin{equation}
f(t)=\begin{cases}
0 & t \in \left( -\infty; -1 \right ) \\
1-t^2\ & t \in \left( -1; 1 \right ) \\
0 & t \in \left( 1; \infty \right )
\end{cases} 
\end{equation}

W pierwszej kolejności wyznaczamy pochodną sygnału $f(t)$

\begin{equation}
g(t)=f'(t)=\begin{cases}
0 & t \in \left( -\infty; -1 \right ) \\
-2 \cdot t\ & t \in \left( -1; 1 \right ) \\
0 & t \in \left( 1; \infty \right )
\end{cases} 
\end{equation}

\begin{figure}[H]
	\centering
	\begin{tikzpicture}
	%\draw (0,0) circle (1in);
	\draw[->] (-4.0,+0.0) -- (+4.0,+0.0) node[right] {$t$};
	\draw[->] (+0.0,-2.5) -- (+0.0,+2.5) node[above] {$g(t)$};
	\draw[-,red, thick] (-3.5,+0.0) -- (-1.0,+0.0);
	\draw[-,red, thick] (+1.0,+0.0) -- (+3.5,+0.0);
	\draw[-,red, thick] (-1.0,+0.0) -- (-1.0,+2.0);
	\draw[-,red, thick] (-1.0,+2.0) -- (+1.0,-2.0);
	\draw[-,red, thick] (+1.0,+0.0) -- (+1.0,-2.0);
	
	\draw[-] (-1.0-0.1,-0.1)--(-1.0+0.1,0.1) node[midway, below, outer sep=5pt,align=center] {$-1$};
	\draw[-] (+1.0-0.1,-0.1)--(+1.0+0.1,0.1) node[midway, above, outer sep=5pt] {$1$};
	\draw[-] (-0.1,+2.0-0.1)--(+0.1,+2.0+0.1) node[midway, left] {$2$};
	\draw[-] (-0.1,-2.0-0.1)--(+0.1,-2.0+0.1) node[midway, left] {$-2$};
	\end{tikzpicture}
\end{figure}

Można sprawdzić, że całkując sygnał $g(t)$ otrzymamy sygnał $f(t)$, czyli:

\begin{equation}
f(t) = \int_{-\infty}^{t} g(x) \cdot dx
\end{equation}

Skoro tak jest, to transformatę sygnału $f(t)$ mozna wyznaczyć z twierdzenia o całkowaniu sygnału, w tym przypadku całkować będziemy sygnał $g(t)$:

\begin{equation}
F(\jmath \omega) = \frac{1}{\jmath \cdot \omega} \cdot G(\jmath \omega) + \pi \cdot \delta(\omega) \cdot G(0)
\end{equation}

Pytanie, czy można dalej uprościc sygnał $g(t)$ dokonując jego rózniczkowania. Wyznaczmy pochodną sygnału $g(t)$, czyli drugą pochodną sygnału $f(t)$:
\begin{equation}
h(t)= \frac{\partial}{\partial t}g(t) = \frac{\partial^2}{\partial t^2}f(t) 
\end{equation}

\begin{equation}
h(t)=g'(t)=\begin{cases}
0 & t \in \left( -\infty; -1 \right ) \\
-2& t \in \left( -1; 1 \right ) \\
0 & t \in \left( 1; \infty \right )
\end{cases} + 2 \cdot \delta(t+1) + 2 \cdot \delta(t-1)
\end{equation}

\begin{figure}[H]
  \centering
  \begin{tikzpicture}
  %\draw (0,0) circle (1in);
  \draw[->] (-4.0,+0.0) -- (+4.0,+0.0) node[right] {$t$};
  \draw[->] (+0.0,-2.5) -- (+0.0,+2.5) node[above] {$h(t)$};
  \draw[-,red, thick] (-3.5,+0.0) -- (-1.0,+0.0);
  \draw[-,red, thick] (-1.0,+0.0) -- (-1.0,-2.0);
  \draw[-,red, thick] (-1.0,-2.0) -- (+1.0,-2.0);
  \draw[-,red, thick] (+1.0,-2.0) -- (+1.0,+0.0);
  \draw[-,red, thick] (+1.0,+0.0) -- (+3.5,+0.0);
  \draw[->,red, thick] (+1.0,+0.0) -- (+1.0,+2.0);
  \draw[->,red, thick] (-1.0,+0.0) -- (-1.0,+2.0);
  
  \draw[-] (-1.0-0.1,-0.1)--(-1.0+0.1,0.1) node[midway, below left, outer sep=5pt,align=center] {$-1$};
  \draw[-] (+1.0-0.1,-0.1)--(+1.0+0.1,0.1) node[midway, below right, outer sep=5pt] {$1$};
  \draw[-] (-0.1,-2.0-0.1)--(+0.1,-2.0+0.1) node[midway, below left] {$-2$};
  \draw[] (-1.0,+2.0)--(-1.0,+2.0) node[midway, above] {($2$)};
  \draw[] (1.0,+2.0)--(1.0,+2.0) node[midway, above] {($2$)};

  \end{tikzpicture}
\end{figure}

Funkcja $h(t)$ składa się z dwóch sygnałów $h_1(t)$ i $h_2(t)$

\begin{equation}
h(t)=h_1(t)+h_2(t)
\end{equation}

\begin{equation}
h_1(t)=\begin{cases}
0 & t \in \left( -\infty; -1 \right ) \\
-2 & t \in \left( -1; 1 \right ) \\
0 & t \in \left( 1; \infty \right )
\end{cases}
\end{equation}

\begin{equation}
h_2(t)= 2 \cdot \delta(t+1) + 2 \cdot \delta(t-1)
\end{equation}

\begin{figure}[H]
  \centering
  \begin{tikzpicture}
  %\draw (0,0) circle (1in);
  \draw[->] (-4.0,+0.0) -- (+4.0,+0.0) node[right] {$t$};
  \draw[->] (+0.0,-2.5) -- (+0.0,+2.0) node[above] {$h_1(t)$};
  \draw[-,red, thick] (-3.5,+0.0) -- (-1.0,+0.0);
  \draw[-,red, thick] (-1.0,+0.0) -- (-1.0,-2.0);
  \draw[-,red, thick] (-1.0,-2.0) -- (+1.0,-2.0);
  \draw[-,red, thick] (+1.0,-2.0) -- (+1.0,+0.0);
  \draw[-,red, thick] (+1.0,+0.0) -- (+3.5,+0.0);
  
  \draw[-] (-1.0-0.1,-0.1)--(-1.0+0.1,0.1) node[midway, above, outer sep=5pt,align=center] {$-1$};
  \draw[-] (+1.0-0.1,-0.1)--(+1.0+0.1,0.1) node[midway, above, outer sep=5pt] {$1$};
  \draw[-] (-0.1,-2.0-0.1)--(+0.1,-2.0+0.1) node[midway, below left] {$-2$};
  \end{tikzpicture}
\end{figure}

\begin{figure}[H]
	\centering
	\begin{tikzpicture}
	%\draw (0,0) circle (1in);
	\draw[->] (-4.0,+0.0) -- (+4.0,+0.0) node[right] {$t$};
	\draw[->] (+0.0,-1.5) -- (+0.0,+2.5) node[above] {$h_2(t)$};
	\draw[-,red, thick] (-3.5,+0.0) -- (3.5,+0.0);
	\draw[->,red, thick] (+1.0,+0.0) -- (+1.0,+2.0);
	\draw[->,red, thick] (-1.0,+0.0) -- (-1.0,+2.0);
	
	\draw[-] (-1.0-0.1,-0.1)--(-1.0+0.1,0.1) node[midway, below left, outer sep=5pt,align=center] {$-1$};
	\draw[-] (+1.0-0.1,-0.1)--(+1.0+0.1,0.1) node[midway, below right, outer sep=5pt] {$1$};
	\draw[] (-1.0,+2.0)--(-1.0,+2.0) node[midway, above] {($2$)};
	\draw[] (1.0,+2.0)--(1.0,+2.0) node[midway, above] {($2$)};
	
	\end{tikzpicture}
\end{figure}

Wyznaczenie transformaty sygnału $h_2(t)$ złożonego z delt Diracka jest znacznie prostsze.

\begin{equation}
H_2(\jmath \omega )=\int_{-\infty }^{\infty}h_2(t) \cdot e^{-\jmath \cdot \omega \cdot t}\cdot dt
\end{equation}

Podstawiamy do wzoru na transformatę wzór naszej funkcji

\begin{align*}
H_2(\jmath \omega )&=\int_{-\infty }^{\infty}h_2(t) \cdot e^{-\jmath \cdot \omega \cdot t}\cdot dt\\
&=\int_{-\infty }^{\infty}\left( 2 \cdot \delta(t+1) + 2 \cdot \delta(t-1)\right) \cdot e^{-\jmath \cdot \omega \cdot t}\cdot dt\\
&=2 \cdot \int_{-\infty }^{\infty} \left( \delta(t+1) + \delta(t-1) \right) \cdot e^{-\jmath \cdot \omega \cdot t}\cdot dt\\
&=2 \cdot \left( \int_{-\infty }^{\infty} \delta(t+1) \cdot e^{-\jmath \cdot \omega \cdot t}\cdot dt + \int_{-\infty }^{\infty} \delta(t-1) \cdot e^{-\jmath \cdot \omega \cdot t}\cdot dt \right)\\
&=\begin{Bmatrix}
\SamplingPropertyOfDelta
\end{Bmatrix}\\
&= 2 \cdot \left( e^{-\jmath \cdot \omega \cdot (-1)} + e^{-\jmath \cdot \omega \cdot 1} \right)\\
&= 2 \cdot \left( e^{\jmath \cdot \omega} + e^{-\jmath \cdot \omega} \right)\\
&= 2 \cdot \left( e^{\jmath \cdot \omega} + e^{-\jmath \cdot \omega} \right) \cdot \frac{2}{2}\\
&= 4 \cdot \frac{ e^{\jmath \cdot \omega} + e^{-\jmath \cdot \omega} }{2} \\
&=\begin{Bmatrix}
\EulerCos
\end{Bmatrix}\\
&= 4 \cdot cos\left( \omega\right) \\
\end{align*}

Transformata sygnału $h_2(t)$ to $G_2(\jmath \omega)=4 \cdot cos\left( \omega\right)$

Funkcja $h_1(t)$ jest jeszcze zbyt złożona, więc wyznaczamy pochodną raz jeszcze 

\begin{equation}
i(t)=h_1'(t)=\begin{cases}
0 & t \in \left( -\infty; -1 \right ) \\
0 & t \in \left( -1; 1 \right ) \\
0 & t \in \left( 1; \infty \right )
\end{cases} - 2 \delta(t+1) + 2 \delta(t-1)
\end{equation}

,czyli po prostu:

\begin{equation}
i(t)=h_1'(t)= -2 \delta(t+1) + 2 \delta(t-1)
\end{equation}

\begin{figure}[H]
  \centering
  \begin{tikzpicture}
  %\draw (0,0) circle (1in);
  \draw[->] (-4.0,+0.0) -- (+4.0,+0.0) node[right] {$t$};
  \draw[->] (+0.0,-2.5) -- (+0.0,+2.5) node[above] {$i(t)$};
  \draw[-,red, thick] (-3.5,+0.0) -- (+3.5,+0.0);
  \draw[->,red, thick] (-1.0,+0.0) -- (-1.0,-2.0);
  \draw[->,red, thick] (+1.0,+0.0) -- (+1.0,+2.0);
  
  \draw[-] (-1.0-0.1,-0.1)--(-1.0+0.1,0.1) node[midway, above, outer sep=5pt,align=center] {$-1$};
  \draw[-] (+1.0-0.1,-0.1)--(+1.0+0.1,0.1) node[midway, below, outer sep=5pt] {$1$};
  \draw[] (1.0,+2.0)--(+1.0,+2.0) node[midway, left] {$(2)$};
  \draw[] (-1.0,-2.0)--(-1.0,-2.0) node[midway, left] {$(-2)$};
  \end{tikzpicture}
\end{figure}

Wyznaczanie transformaty sygnału $i(t)$ złożonego z delt Diracka jest znacznie prostsze. 

\begin{equation}
I(\jmath \omega )=\int_{-\infty }^{\infty}i(t) \cdot e^{-\jmath \cdot \omega \cdot t}\cdot dt
\end{equation}

Podstawiamy do wzoru na transformatę wzór naszej funkcji


\begin{align*}
I(\jmath \omega )&=\int_{-\infty }^{\infty}i(t) \cdot e^{-\jmath \cdot \omega \cdot t}\cdot dt\\
&=\int_{-\infty }^{\infty}\left( -2 \cdot \delta(t+1) + 2 \cdot \delta(t-1)\right) \cdot e^{-\jmath \cdot \omega \cdot t}\cdot dt\\
&=2 \cdot \int_{-\infty }^{\infty} \left( -\delta(t+1) + \delta(t-1) \right) \cdot e^{-\jmath \cdot \omega \cdot t}\cdot dt\\
&=2 \cdot \left( \int_{-\infty }^{\infty} -\delta(t+1) \cdot e^{-\jmath \cdot \omega \cdot t}\cdot dt + \int_{-\infty }^{\infty} \delta(t-1) \cdot e^{-\jmath \cdot \omega \cdot t}\cdot dt \right)\\
&=\begin{Bmatrix}
\SamplingPropertyOfDelta
\end{Bmatrix}\\
&= 2 \cdot \left( -e^{-\jmath \cdot \omega \cdot (-1)} + e^{-\jmath \cdot \omega \cdot 1} \right)\\
&= 2 \cdot \left( -e^{\jmath \cdot \omega} + e^{-\jmath \cdot \omega} \right)\\
&= -2 \cdot \left( e^{\jmath \cdot \omega} - e^{-\jmath \cdot \omega} \right) \cdot \frac{2 \cdot \jmath}{2 \cdot \jmath}\\
&= -4 \cdot \jmath \cdot \frac{ e^{\jmath \cdot \omega} - e^{-\jmath \cdot \omega} }{2 \cdot \jmath} \\
&=\begin{Bmatrix}
\EulerSin
\end{Bmatrix}\\
&= -4 \cdot \jmath \cdot sin\left( \omega\right) \\
\end{align*}

Transformata sygnału $i(t)$ to $I(\jmath \omega)=-4 \cdot \jmath \cdot sin\left( \omega\right)$
\\

Następnie możemy wykorzystać twierdzenie o całkowaniu, aby wyznaczyć transformatę sygnału $h_1(t)$ na podstawie transformaty sygnału $i(t)=h_1'(t)$
\begin{align*}
\IntegralTeorem{i}{I}{h_1}{H_1}
\end{align*}

Podstawiając obliczoną wcześniej transformatę $I(\jmath \omega)$ sygnału $i(t)$ otrzymujemy transformatę $H_1(\jmath \omega)$ sygnału $h_1(t)$

\begin{align*}
H_1(\jmath \omega) &= \frac{1}{\jmath \cdot \omega} \cdot I(\jmath \omega) + \pi \cdot \delta(\omega) \cdot I(0)\\
&\begin{Bmatrix*}[l]
I(0)=-4 \cdot \jmath \cdot sin\left( 0 \right)\\
I(0)=-4 \cdot \jmath \cdot 0\\
I(0)=0\\
\end{Bmatrix*}\\
&=\frac{1}{\jmath \cdot \omega} \cdot \left(-4 \cdot \jmath \cdot sin\left( \omega \right)\right) + 0\\
&=-4\cdot \frac{sin\left( \omega \right)}{\omega}\\
&=\begin{Bmatrix}
\SaDef
\end{Bmatrix}\\
&=-4\cdot Sa\left( \omega\right)\\
\end{align*}

Ostatecznie transformata sygnału $h_1(t)$ jest równa $H_1(\jmath \omega)=- 4 \cdot Sa\left( \omega\right)$.

Korzystając z liniowości transformacji Fouriera 
\begin{align*}
\HomogeneousTeorem{h}{H}
\end{align*}

można wyznaczyć transformatę Fouriera $H(\jmath \omega)$ funkcji $h(t)$
\begin{align*}
H(\jmath \omega) &= H_1(\jmath \omega)+H_2(\jmath \omega)\\
&= -4\cdot Sa\left( \omega\right) + 4 \cdot cos\left( \omega \right)\\
&= 4\cdot \left(cos\left( \omega\right) - Sa\left( \omega\right)\right)
\end{align*}

Znając transformatę $H(\jmath \omega)$ i korzystając z twierdzenia o całkowaniu można wyznaczyć transformatę $G(\jmath \omega)$ funkcji $g(t)$
\begin{align*}
\IntegralTeorem{h}{H}{g}{G}
\end{align*}
 
Podstawiając odpowiednie dane otrzymujemy:
 \begin{align*}
G(\jmath \omega) &= \frac{1}{\jmath \cdot \omega} \cdot H(\jmath \omega) + \pi \cdot \delta(\omega) \cdot H(0)\\
&\begin{Bmatrix*}[l]
H(0)=4\cdot \left(cos\left( 0\right) - Sa\left( 0\right)\right)\\
H(0)=4\cdot \left(1 -1 \right)\\
H(0)=4\cdot 0\\
H(0)=0\\
\end{Bmatrix*}\\
&=\frac{1}{\jmath \cdot \omega} \cdot \left( 4\cdot \left(cos\left( \omega\right) - Sa\left( \omega\right)\right) \right) + 0\\
&=\frac{4}{\jmath \cdot \omega} \cdot \left(cos\left( \omega\right) - Sa\left( \omega\right)\right)
\end{align*}

Ostatecznie transformata sygnału $g(t)$ jest równa $G(\jmath \omega)=\frac{4}{\jmath \cdot \omega} \cdot \left(cos\left( \omega\right) - Sa\left( \omega\right)\right)$.

Znając transformatę $G(\jmath \omega)$ i kolejny raz korzystając z twierdzenia o całkowaniu można wyznaczyć transformatę $F(\jmath \omega)$ funkcji $f(t)$
\begin{align*}
\IntegralTeorem{g}{G}{f}{F}
\end{align*}

Podstawiając odpowiednie dane otrzymujemy:
\begin{align*}
F(\jmath \omega) &= \frac{1}{\jmath \cdot \omega} \cdot G(\jmath \omega) + \pi \cdot \delta(\omega) \cdot G(0)\\
&\begin{Bmatrix*}[l]
G(0)=\frac{4}{\jmath \cdot 0} \cdot \left(cos\left( 0\right) - Sa\left(0\right)\right)\\
G(0)=\frac{0}{0}!!!\\
G(0)=\int_{-\infty}^{\infty} g(t) \cdot dt=\int_{-1}^{1} (-2) \cdot t \cdot dt = \left. (-2)\cdot \frac{t^2}{2} \right|_{-1}^{1}\\
G(0)=(-2) \cdot \left(\frac{1}{2} - \frac{1}{2}\right)= (-2) \cdot 0\\
G(0)=0\\
\end{Bmatrix*}\\
&=\frac{1}{\jmath \cdot \omega} \cdot \frac{4}{\jmath \cdot \omega} \cdot \left(cos\left( \omega\right) - Sa\left( \omega\right)\right) + 0\\
&=\frac{4}{\jmath^2 \cdot \omega^2} \cdot \left(cos\left( \omega\right) - Sa\left( \omega\right)\right)\\
&=\frac{4}{(-1) \cdot \omega^2} \cdot \left(cos\left( \omega\right) - Sa\left( \omega\right)\right)\\
&=\frac{4}{\omega^2} \cdot \left(Sa\left( \omega\right) - cos\left( \omega\right)\right)
\end{align*}

Ostatecznie transformata sygnału $f(t)$ jest równa $F(\jmath \omega)=\frac{4}{\omega^2} \cdot \left(Sa\left( \omega\right) - cos\left( \omega\right)\right)$.
\end{task}

