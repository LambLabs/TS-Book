\begin{task}
Oblicz transformatę Fouriera sygnału $f(t)$ przedstawionego na rysunku oraz narysuj jego widmo amplitudowe i fazowe

\begin{figure}[H]
\centering
\begin{tikzpicture}
  %\draw (0,0) circle (1in);
  \draw[->] (-3.0,+0.0) -- (+5.0,+0.0) node[right] {$t$};
  \draw[->] (+0.0,-1.0) -- (+0.0,+2.0) node[above] {$f(t)$};

  \draw[-] (-0.1,+1.5-0.1)--(+0.1,+1.5+0.1) node[midway, left] {$A$};
  
  \draw[scale=1.0,domain=-2.5:0.0,smooth,variable=\x,red,thick] plot ({\x},{0});
  \draw[scale=1.0,domain=0.0:4.0,smooth,variable=\x,red,thick] plot ({\x},{1.5*exp(-\x)});
  
\end{tikzpicture}
\end{figure}

\begin{equation}
f(t) = \left\{\begin{matrix*}[l]
0 & dla & t \in \left(-\infty; 0\right)\\
A \cdot e^{-a\cdot t} & dla & t \in \left(0; \infty\right)
\end{matrix*}\right.
\end{equation}

Transformatę Fouriera obliczamy ze wzoru:

\begin{equation}
F(\jmath \omega )=\int_{-\infty }^{\infty}f(t) \cdot e^{-\jmath \cdot \omega \cdot t}\cdot dt
\end{equation}

Podstawiamy do wzoru na transformatę wzór naszej funkcji

\begin{align*}
F(\jmath \omega )&=\int_{-\infty }^{\infty}f(t) \cdot e^{-\jmath \cdot \omega \cdot t}\cdot dt\\
&=\int_{-\infty}^{0} 0 \cdot e^{-\jmath \cdot \omega \cdot t}\cdot dt
+\int_{0}^{\infty} A \cdot e^{-a \cdot t} \cdot e^{-\jmath \cdot \omega \cdot t}\cdot dt\\
&=\int_{-\infty}^{0} 0 \cdot dt
+\int_{0}^{\infty} A \cdot e^{-a \cdot t} \cdot e^{-\jmath \cdot \omega \cdot t}\cdot dt\\
&=0 +\int_{0}^{\infty} A \cdot e^{-a \cdot t} \cdot e^{-\jmath \cdot \omega \cdot t}\cdot dt\\
&=\int_{0}^{\infty} A \cdot e^{-a \cdot t} \cdot e^{-\jmath \cdot \omega \cdot t}\cdot dt\\
&=A\cdot \int_{0}^{\infty} e^{-( a + \jmath \cdot \omega) \cdot t}\cdot dt\\
&=\lim_{\tau \rightarrow \infty} A\cdot \int_{0}^{\tau} e^{-( a + \jmath \cdot \omega) \cdot t}\cdot dt\\
&=\begin{Bmatrix}
z&=-( a + \jmath \cdot \omega) \cdot t\\
dz&=-( a + \jmath \cdot \omega) \cdot dt\\
dt&=\frac{1}{-( a + \jmath \cdot \omega)} \cdot dz\\
\end{Bmatrix}\\
&=\lim_{\tau \rightarrow \infty} A\cdot \int_{0}^{\tau} e^{z}\cdot \frac{1}{-( a + \jmath \cdot \omega)} \cdot dz\\
&=A\cdot \frac{1}{-( a + \jmath \cdot \omega)}\cdot \lim_{\tau \rightarrow \infty} \int_{0}^{\tau} e^{z} \cdot dz\\
&=A\cdot \frac{1}{-( a + \jmath \cdot \omega)}\cdot \lim_{\tau \rightarrow \infty} \left. e^{z} \right|_{0}^{\tau}\\
&=\frac{A}{-( a + \jmath \cdot \omega)}\cdot \lim_{\tau \rightarrow \infty} \left. e^{-( a + \jmath \cdot \omega) \cdot t} \right|_{0}^{\tau}\\
%&=A\cdot \left(\left. \frac{e^{-( a + \jmath \cdot \omega) \cdot t}}{-( a + \jmath \cdot \omega)}\right |_{0}^{\infty} \right )\\
&=\frac{A}{-( a + \jmath \cdot \omega)} \cdot \lim_{\tau \rightarrow \infty} \left(e^{-( a + \jmath \cdot \omega) \cdot \tau} - e^{-( a + \jmath \cdot \omega) \cdot 0}\right)\\
&=\frac{A}{-( a + \jmath \cdot \omega)} \cdot \lim_{\tau \rightarrow \infty} \left(e^{-( a + \jmath \cdot \omega) \cdot \tau} - e^{0}\right)\\
&=\frac{A}{-( a + \jmath \cdot \omega)} \cdot \lim_{\tau \rightarrow \infty} \left(e^{-( a + \jmath \cdot \omega) \cdot \tau} - 1\right)\\
&=\frac{A}{-( a + \jmath \cdot \omega)} \cdot \left(\lim_{\tau \rightarrow \infty} e^{-( a + \jmath \cdot \omega) \cdot \tau} - 1\right)\\
&=\frac{A}{-( a + \jmath \cdot \omega)} \cdot \left(\lim_{\tau \rightarrow \infty} e^{-a \cdot \tau + \jmath \cdot \omega \cdot \tau } - 1\right)\\
&=\frac{A}{-( a + \jmath \cdot \omega)} \cdot \left(\lim_{\tau \rightarrow \infty} e^{-a \cdot \tau} \cdot e^{\jmath \cdot \omega \cdot \tau } - 1\right)\\
&=\frac{A}{-( a + \jmath \cdot \omega)} \cdot \left(\lim_{\tau \rightarrow \infty} e^{-a \cdot \tau} \cdot \lim_{\tau \rightarrow \infty} e^{\jmath \cdot \omega \cdot \tau } - 1\right)\\
&=\frac{A}{-( a + \jmath \cdot \omega)} \cdot \left(0 \cdot \lim_{\tau \rightarrow \infty} e^{\jmath \cdot \omega \cdot \tau } - 1\right)\\
&=\frac{A}{-( a + \jmath \cdot \omega)} \cdot \left(0 - 1\right)\\
&=\frac{A}{a + \jmath \cdot \omega}
\end{align*}

Transformata sygnału $f(t)$ to $F(\jmath \omega)=\frac{A}{ a + \jmath \cdot \omega}$

Wyznaczmy jawnie część rzeczywistą i urojoną transformaty:
\begin{equation}
\begin{aligned}
F(\jmath \omega )&=\frac{A}{( a + \jmath \cdot \omega)}\\
&=\frac{A}{( a + \jmath \cdot \omega)} \cdot \frac{(a - \jmath \cdot \omega)}{(a - \jmath \cdot \omega)}\\
&=\frac{A \cdot ( a - \jmath \cdot \omega)}{( a^2 + \omega^2)}\\
&=\frac{A \cdot a}{( a^2 + \omega^2)} - \jmath \cdot \frac{A \cdot \omega}{( a^2 + \omega^2)}
\end{aligned}
\end{equation}


Widmo amplitudowe obliczamy ze wzoru:
\begin{equation}
\begin{aligned}
M(\omega)&=\left| F(j \omega) \right|\\
&=\sqrt{\left(\frac{A \cdot a}{( a^2 + \omega^2)}\right)^2 + \left(\frac{-A \cdot \omega}{( a^2 + \omega^2)}\right)^2}\\
&=\sqrt{\frac{A^2 \cdot (a^2+\omega^2)}{(a^2 + \omega^2)^2}}\\
&=\sqrt{\frac{A^2}{(a^2 + \omega^2)}}\\
&=\frac{A}{\sqrt{a^2 + \omega^2}}
\end{aligned}
\end{equation}

\begin{figure}[H]
  \centering
  \begin{tikzpicture}
  \draw[->] (-5.0,+0.0) -- (+5.0,+0.0) node[right] {$\omega$};
  \draw[->] (+0.0,-1.5) -- (+0.0,+4.0) node[above] {$M(\omega)$};
  
  \draw[scale=1.0,domain=-4.0:4.0,samples=2000,smooth,variable=\x,red,thick] plot ({\x},{3/sqrt(1+(\x)^2)});
  
  \draw[-] (-1.0-0.1,-0.1)--(-1.0+0.1,0.1) node[midway, below, outer sep=5pt] {$-a$};
  \draw[-] (+1.0-0.1,-0.1)--(+1.0+0.1,0.1) node[midway, below, outer sep=5pt] {$a$};
  \draw[-] (-2.0-0.1,-0.1)--(-2.0+0.1,0.1) node[midway, below, outer sep=5pt] {$-2 \cdot a$};
  \draw[-] (+2.0-0.1,-0.1)--(+2.0+0.1,0.1) node[midway, below, outer sep=5pt] {$2 \cdot a$};
  \draw[-] (-0.1,+3.0-0.1)--(+0.1,+3.0+0.1) node[midway, left] {$\frac{A}{a}$};
  \draw[-] (-0.1,+3.0/1.4142-0.1)--(+0.1,+3.0/1.4142+0.1) node[midway, left] {$\frac{A}{\sqrt{2}\cdot a}$};
  \draw[-] (-0.1,+3.0/2.2360-0.1)--(+0.1,+3.0/2.2360+0.1) node[midway, left] {$\frac{A}{\sqrt{5}\cdot a}$};
  \draw[-,dashed] (-1.0,+3.0/1.4142)--(+1.0,+3.0/1.4142);
  \draw[-,dashed] (-1.0,+0.0)--(-1.0,+3.0/1.4142);
  \draw[-,dashed] (+1.0,+0.0)--(+1.0,+3.0/1.4142);
  \draw[-,dashed] (-2.0,+3.0/2.2360)--(+2.0,+3.0/2.2360);
  \draw[-,dashed] (-2.0,+0.0)--(-2.0,+3.0/2.2360);
  \draw[-,dashed] (+2.0,+0.0)--(+2.0,+3.0/2.2360);
  \end{tikzpicture}
\end{figure}

Widmo fazowe obliczamy ze wzoru:
\begin{equation}
\begin{aligned}
\Phi ( \omega )&=arctg(\frac{Im\{F(\jmath \cdot \omega )\}}{Re\{F(\jmath \cdot \omega )\}})\\
&=arctg\left(\frac{\left(\frac{-A \cdot \omega}{( a^2 + \omega^2)}\right)}{\left(\frac{A \cdot a}{( a^2 + \omega^2)}\right)}\right)\\
&=arctg\left(\frac{-A \cdot \omega}{( a^2 + \omega^2)} \cdot \frac{( a^2 + \omega^2)}{A \cdot a}\right)\\
&=arctg\left(-\frac{\omega}{a}\right)\\
\end{aligned}
\end{equation}

\begin{figure}[H]
  \centering
  \begin{tikzpicture}
  \draw[->] (-5.0,+0.0) -- (+5.0,+0.0) node[right] {$\omega$};
  \draw[->] (+0.0,-1.5) -- (+0.0,+2.0) node[above] {$\Phi(\omega)$};
  
  \draw[scale=1.0,domain=-4.0:4.0,samples=2000,smooth,variable=\x,red,thick] plot ({\x},{atan(-\x)/90});
  
  
  \draw[-] (-1.0-0.1,-0.1)--(-1.0+0.1,0.1) node[midway, below, outer sep=5pt] {-$a$};
  \draw[-] (+1.0-0.1,-0.1)--(+1.0+0.1,0.1) node[midway, below, outer sep=5pt] {$a$};
  \draw[-] (-0.1,+1.0-0.1)--(+0.1,+1.0+0.1) node[midway, above left] {$\pi$};
  \draw[-] (-0.1,-1.0-0.1)--(+0.1,-1.0+0.1) node[midway, above left] {-$\pi$};
  
  \end{tikzpicture}
\end{figure}


\end{task}
