\begin{task}
Oblicz transformatę Fouriera sygnału $f(t)$ przedstawionego na rysunku oraz narysuj jego widmo amplitudowe i fazowe

\begin{figure}[H]
\centering
\begin{tikzpicture}
  %\draw (0,0) circle (1in);
  \draw[->] (-3.0,+0.0) -- (+5.0,+0.0) node[right] {$t$};
  \draw[->] (+0.0,-1.0) -- (+0.0,+2.0) node[above] {$f(t)$};

  \draw[-] (-0.1,+1.5-0.1)--(+0.1,+1.5+0.1) node[midway, left] {$A$};
  
  \draw[scale=1.0,domain=-2.5:0.0,smooth,variable=\x,red,thick] plot ({\x},{0});
  \draw[scale=1.0,domain=0.0:4.0,smooth,variable=\x,red,thick] plot ({\x},{1.5*exp(-\x)});
  
\end{tikzpicture}
\end{figure}

\begin{equation}
f(t) = \left\{\begin{matrix*}[l]
0 & dla & t \in \left(-\infty; 0\right)\\
A \cdot e^{-a\cdot t} & dla & t \in \left(0; \infty\right)
\end{matrix*}\right.
\end{equation}

Transformatę Fouriera obliczamy ze wzoru:

\begin{equation}
F(\jmath \omega )=\int_{-\infty }^{\infty}f(t) \cdot e^{-\jmath \cdot \omega \cdot t}\cdot dt
\end{equation}

Podstawiamy do wzoru na transformatę wzór naszej funkcji

\begin{equation}
\begin{aligned}
F(\jmath \omega )&=\int_{-\infty }^{\infty}f(t) \cdot e^{-\jmath \cdot \omega \cdot t}\cdot dt\\
&=\int_{0}^{\infty} A \cdot e^{-a \cdot t} \cdot e^{-\jmath \cdot \omega \cdot t}\cdot dt\\
&=A\cdot \int_{0}^{\infty} e^{-( a + \jmath \cdot \omega) \cdot t}\cdot dt\\
&=A\cdot \left(\left. \frac{e^{-( a + \jmath \cdot \omega) \cdot t}}{-( a + \jmath \cdot \omega)}\right |_{0}^{\infty} \right )\\
&=\frac{A}{-( a + \jmath \cdot \omega)} \cdot \left(e^{-( a + \jmath \cdot \omega) \cdot \infty} - e^{-( a + \jmath \cdot \omega) \cdot 0}\right)\\
&=\frac{A}{-( a + \jmath \cdot \omega)} \cdot \left(0 - 1\right)\\
&=\frac{A}{( a + \jmath \cdot \omega)}
\end{aligned}
\end{equation}

Transformata sygnału $f(t)$ to $F(\jmath \omega)=\frac{A}{( a + \jmath \cdot \omega)}$

Wyznaczmy jawnie część rzeczywistą i urojoną transformaty:
\begin{equation}
\begin{aligned}
F(\jmath \omega )&=\frac{A}{( a + \jmath \cdot \omega)}\\
&=\frac{A}{( a + \jmath \cdot \omega)} \cdot \frac{(a - \jmath \cdot \omega)}{(a - \jmath \cdot \omega)}\\
&=\frac{A \cdot ( a - \jmath \cdot \omega)}{( a^2 + \omega^2)}\\
&=\frac{A \cdot a}{( a^2 + \omega^2)} - \jmath \cdot \frac{A \cdot \omega}{( a^2 + \omega^2)}
\end{aligned}
\end{equation}


Widmo amplitudowe obliczamy ze wzoru:
\begin{equation}
\begin{aligned}
M(\omega)&=\left| F(j \cdot \omega) \right|\\
&=\sqrt{\left(\frac{A \cdot a}{( a^2 + \omega^2)}\right)^2 + \left(\frac{-A \cdot \omega}{( a^2 + \omega^2)}\right)^2}\\
&=\sqrt{\frac{A^2 \cdot (a^2+\omega^2)}{(a^2 + \omega^2)^2}}\\
&=\sqrt{\frac{A^2}{(a^2 + \omega^2)}}\\
&=\frac{A}{\sqrt{a^2 + \omega^2}}
\end{aligned}
\end{equation}


Widmo fazowe obliczamy ze wzoru:
\begin{equation}
\begin{aligned}
\Phi ( \omega )&=arctg(\frac{Im\{F(\jmath \cdot \omega )\}}{Re\{F(\jmath \cdot \omega )\}})\\
&=arctg\left(\frac{\left(\frac{-A \cdot \omega}{( a^2 + \omega^2)}\right)}{\left(\frac{A \cdot a}{( a^2 + \omega^2)}\right)}\right)\\
&=arctg\left(\frac{-A \cdot \omega}{( a^2 + \omega^2)} \cdot \frac{( a^2 + \omega^2)}{A \cdot a}\right)\\
&=arctg\left(-\frac{\omega}{a}\right)\\
\end{aligned}
\end{equation}

\end{task}
